%%%%%%%%%%%%%%%%%%%%%%%%%%%%%%%%%%%%%%%%%%%

\documentclass[11pt, a4paper, oneside]{article}
\usepackage[utf8]{inputenc}
\usepackage[T1]{fontenc}
\usepackage[francais]{babel}
\usepackage{lmodern}
\usepackage[boldsans]{concmath}
\usepackage{utf8math}
\usepackage{enumerate} 
\usepackage{amsmath}
\usepackage{amssymb}
\usepackage{amsthm}
\usepackage{mathtools}
\usepackage{comment}
\usepackage{faktor}
\usepackage[dvipsnames]{xcolor}
\newcommand{\smat}[1]{ \big(\begin{smallmatrix} #1 \end{smallmatrix}\big)}

\newcommand{\indice}[1]{{\scriptsize \color{RubineRed} {#1}}}

%%%%%%%%%%%%%%%%%%%%%%%%%%%%%%%%%%%%%%%%%%%

\begin{document}
\title{Solutions Examen Algèbre Linéaire Avancée 2, 2022}
\maketitle

\noindent
\textbf{Attention:} Les solutions présentées sont très détaillées, les commentaires en \indice{ petits caractères} ne sont pas  nécessaires dans vos solutions et servent à faciliter la compréhension si certaines étapes du raisonnement ne sont pas claires

\subsection*{Question 22}
\begin{enumerate}[a)]
\item
\begin{itemize}
\item On veut montrer que les matrices $A$ et $B$ sont semblables sur $\mathbb{R}$
\item Il existe une matrice $P \in \mathbb{C}^{n \times n}$ inversible tel que
\begin{align}
\label{eq1}
A = PBP^{-1} \Longleftrightarrow AP = PB
\end{align}
\indice{Car $A$ et $B$ sont semblables sur $\mathbb{C}$}
\item On écrit $P=P_1 + i P_2$ avec $P_1,P_2 \in \mathbb{R}^{n \times n}$ \\
\indice{Cela est possible car chaque élément de la matrice s'écrit comme $(P)_{ij} = p_{ij} + i\Tilde{p}_{ij}$ avec $p_{ij},\Tilde{p}_{ij} \in \mathbb{R}$ et on peut donc rassembler les $p_{ij}$ dans la matrice $P_1$ et les $\Tilde{p}_{ij}$ dans la matrice $P_2$}
\item L'équation (\ref{eq1}) donne les deux équations $AP_1=P_1B$ et $AP_2=P_2B$ \textcolor{blue}{(2 points pour ces relations)}
\item On considère la fonction $f(x) \coloneqq \det(P_1 + xP_2)$, comme $f$ est définie par un déterminant on sait du cours Algèbre Linéaire Avancée 1 que $f$ est un polynôme réel de degré au plus $n$ \\
\indice{On peut voir $f$ aussi comme un polynôme complexe et comme $P$ est une matrice inversible on sait que $f(i) \neq 0$}
\item $f$ n'est pas le polynôme identiquement nul sur $\mathbb{C}$ et par conséquent $f$ n'est pas le polynôme identiquement nul sur $\mathbb{R}$ \textcolor{blue}{(2 points pour remarquer que $f$ est un polynôme non identiquement nul)}
\item Il existe un élément $y \in \mathbb{R}$ tel que $f(y) \neq 0$ car $f$ possède au plus $n$ racines réelles distinctes \\
\indice{$\mathbb{R}$ possède une infinité d'éléments}
\item Posons $Q \coloneqq P_1 + yP_2 \in \mathbb{R}^{n \times n}$, alors $Q$ est inversible et on obtient
$$AQ = AP_1 + yAP_2 = P_1B + yP_2B = QB$$ 
\textcolor{blue}{(2 points pour montrer l'existence de $y$ et terminer l'argument)}
\end{itemize}

\item
\indice{On montre les deux direction du si et seulement si}
\begin{itemize}
    \item [$\Longrightarrow$] 
    \begin{itemize} 
    \item Quand deux matrices réelles $A$ et $B$ sont semblables sur $\mathbb{R}$ alors $A$ est semblable à sa forme normale de Jordan mais aussi à la forme normale de Jordan de la matrice $B$ \\
    \indice{Car \flqq être semblable\frqq est une relation d'équivalence, on utilise la transitivité}
    \item $A$ est semblable à deux formes normales de Jordan
    \item Par l'unicité on obtient que $A$ et $B$ admettent la même forme normale de Jordan \textcolor{blue}{(2 points pour cette direction de l'équivalence)}
    \end{itemize}    

    \item [$\Longleftarrow$] 
    \begin{itemize} 
    \item Quand $A$ et $B$ admettent la même forme normale de Jordan alors les deux matrices sont semblable sur $\mathbb{C}$ \\
    \indice{On utilise la transitivité}
    \item Le point a) donne que $A$ et $B$ sont aussi semblable sur $\mathbb{R}$ \textcolor{blue}{(2 points pour cette direction de l'équivalence)} \\
\end{itemize}
\end{itemize}
\textcolor{blue}{dans la correction on enlève un point dans la direction $\Longleftarrow$ de b) si l'étudiant affirme que la matrice de la forme normale de Jordan et les matrices de changement de base sont forcément réelles}
\end{enumerate}

\subsection*{Question 23}
\begin{enumerate}[a)]
\item
\begin{itemize}
\item On veut montrer que $E$ est un espace vectoriel sur $F$ en vérifiant l'associativité, la distributivité et la neutralité de $1_E = 1_F$
\item Prenons des éléments $x,y \in F$ et $z,w \in E$
\begin{itemize}
    \item [\textbf{Associativité}:] $$x\cdot(y\cdot z) = x\cdot(yz) = x(yz) = (xy)z = (xy) \cdot z$$ \indice{Où on a utilisé l'associativité de la multiplication sur $E$}
    \item [\textbf{Distributivité}:] $$x\cdot(z+w) = x(z+w) = xz+xw = x\cdot z + x\cdot w$$
    et
    $$(x+y)\cdot z = (x+y)z = xz + yz = x\cdot z + y\cdot z$$
    \indice{Où on a utilisé la distributivité sur $E$}
    \item [\textbf{Neutralité}:] $$1_E\cdot z = 1_Fz = z$$
\end{itemize}
\end{itemize}
\textcolor{blue}{1 point par propriété, attention pour la distributivité il faut vérifier les deux égalités}

\item
\begin{itemize}
\item $E$ est un espace vectoriel de dimension finie sur $F$
\item Prenons $e\in E \backslash \{0\}$, on veut montrer qu'il existe un polynôme $f(x) \in F[x]$ tel que $f(x) \neq 0$ et $f(e)=0$
\item Soit $n$ la dimension de $E$ sur $F$, alors $1,e,e^2,\ldots,e^{n+1}$ sont $n+1$ éléments de $E$ 
\item Il existe une combinaison linéaire non-triviale de ces éléments avec des coefficients dans $F$ qui est égal à zéro
$$\sum_{i=0}^n f_ie^i = 0$$
avec $f_i \in F \; \forall i\in \{0,1,\ldots,n\}$
\item Le polynôme à définir est $f(x) = \sum_{i=0}^n f_ix^i$ \\
\end{itemize}
\textcolor{blue}{3 points, il s'agit d'une question difficile où on donne dans pour la plupart des solutions soit 0 points soit 3}

\item
\begin{itemize}
\item On suppose l'existence de deux polynômes $g(x),h(x) \in F[x]$ qui satisfont les conditions
\item Considérons le polynôme $g-h$ qui est de degré strictement inférieur au degré de $g$ et de $h$ et qui s'annule en $e$ \textcolor{blue}{(1 point pour l'idée de considérer la différence)}
\item Par la minimalité des polynômes $g$ et $h$ on obtient que $g-h$ doit être le polynôme identiquement nul et on a donc $g=h$ \textcolor{blue}{(1 point pour la conclusion à l'aide de la minimalité)}
\end{itemize}


\item
\begin{itemize}
\item On suppose que $p_e(x)\in F[x]$ est un polynôme réductible
\item Ainsi $p_e=g\cdot h$ avec $g(x),h(x) \in F[x]$ des polynômes non-constantes \textcolor{blue}{(1 point pour appliquer la définition d'irréductible/réductible)} 
\item $g(e) = 0$ ou $h(e) = 0$ \\ \indice{Car un corps est intègre}
\item Comme $\deg(g),\deg(h) < \deg(p_e)$ on a une contradiction avec la minimalité de $p_e$ \textcolor{blue}{(1 point pour la conclusion à l'aide de la minimalité)} \\
\indice{On a donc montré que $p_e$ est irréductible sur $F$} \\
\end{itemize}
\textcolor{blue}{Attention le choix $f(x)=x-e$ n'est pas correcte pour la question b) car il ne s'agit pas d'un polynôme avec coefficients dans $F$}
\end{enumerate}


\subsection*{Question 24}
\begin{itemize}
%soit $G=\begin{pmatrix} a & b \\ b & c\end{pmatrix} \in \mathbb{Z}^{n \times n}$ une matrice symétrique, unimodulaire et définie positive \\
\item On veut montrer qu'il existe une matrice unimodulaire $U$ telle que $G = UU^T$
\item Idée: appliquer des opérations unimodulaires sur les lignes et les colonnes de $G$ pour arriver à la matrice identité \textcolor{blue}{(1 point pour cette idée)}
\item Pour $U \in \mathbb{Z}^{n \times n}$ unimodulaire, la matrice $U^TGU$ reste symétrique, unimodulaire et définie positive \textcolor{blue}{(1 point pour remarquer ces propriétés)}
\item Comme la matrice $G$ est définie positive on a que $a,c \geq 0$ \textcolor{blue}{(1 point pour cette implication)} \\
\indice{À une permutation des deux lignes près on peut supposer qu'on est dans le cas $a \leq c$}
\item Quand $|b|<a$ on a $$1=\det(G)=ac-b^2\leq(|b|+1)^2-b^2=2|b|+1$$ Ceci implique $b=0$ et $a=c=1$, c'est-à-dire $G=I$ \textcolor{blue}{(2 points pour le cas $|b|<a$)}
\item Le cas $|b|\geq a$ \\
Il existe $\lambda \in \mathbb{Z}$ tel que $b=\lambda a + b'$ avec $|b'|<a$ \\
\indice{Division euclidienne dans l'anneau euclidien $\mathbb{Z}$ de $b$ par $a$} \\
Pour $U = \begin{pmatrix} 1 & -\lambda \\ 0 & 1 \end{pmatrix}$ on calcule 
$$U^TGU = \begin{pmatrix} 1 & 0 \\ -\lambda & 1 \end{pmatrix} \begin{pmatrix} a & b \\ b & c\end{pmatrix} \begin{pmatrix} 1 & -\lambda \\ 0 & 1 \end{pmatrix} = \begin{pmatrix} a & b-\lambda a \\ b-\lambda a & c-2\lambda b + \lambda^2a \end{pmatrix}$$
Notons cette dernière matrice $\begin{pmatrix} a & b' \\ b' & c' \end{pmatrix}$ \\
\indice{On est dans le cas $|b'|<a$ mais on ne peut pas conclure immédiatement car le coefficient $c'$ a changé} \\
On sait que $|b'|<|b|$ et on peut donc une nouvelle fois permuter $a$ et $c'$ pour avoir $a\leq c'$ \\
En répétant ce processus on obtient une suite strictement décroissante de nombres entiers positifs $|b|>|b'|>|b''|> \ldots$ qui se termine \\
\indice{Par la méthode de descente infinie} \textcolor{blue}{(4 points pour le cas $|b|\leq a$)}
\item La condition $|b|<a$ doit être vrai \indice{après au plus $|b|$ étapes} et on obtient la matrice identité à cette étape
\item Avec toutes les matrices unimodulaire $U_1,\ldots,U_N$ utilisées pour arriver à cette étape on obtient
$$U_N^T\cdots U_1^TGU_1 \cdots U_N = \left(U_1\cdots U_N\right)^TG\left(U_1\cdots U_N\right) = I$$
\end{itemize}
\textcolor{blue}{en plus on a donné 1 point pour mentionner ou appliquer clairement le fait que le produit de matrices unimodulaires reste unimodulaire \\
on enlève des points si \begin{itemize} 
\item l'étudiant suppose $b>0$
\item l'étudiant termine le processus avec $b=0$ sans explication supplémentaire
\item l'étudiant fait des erreurs de calcul notamment au niveau des matrices
\end{itemize}}

\end{document}