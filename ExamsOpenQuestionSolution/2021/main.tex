%%%%%%%%%%%%%%%%%%%%%%%%%%%%%%%%%%%%%%%%%%%

\documentclass[11pt, a4paper, oneside]{article}
\usepackage[utf8]{inputenc}
\usepackage[T1]{fontenc}
\usepackage[francais]{babel}
\usepackage{lmodern}
\usepackage[boldsans]{concmath}
\usepackage{utf8math}
\usepackage{enumerate} 
\usepackage{amsmath}
\usepackage{amssymb}
\usepackage{amsthm}
\usepackage{mathtools}
\usepackage{comment}
\usepackage{faktor}
\usepackage[dvipsnames]{xcolor}
\newcommand{\smat}[1]{ \big(\begin{smallmatrix} #1 \end{smallmatrix}\big)}

\newcommand{\indice}[1]{{\scriptsize \color{RubineRed} {#1}}}

%%%%%%%%%%%%%%%%%%%%%%%%%%%%%%%%%%%%%%%%%%%

\begin{document}
\title{Solutions Examen Algèbre Linéaire Avancée 2, 2021}
\maketitle

\noindent 
\textbf{Attention:} Les solutions présentées sont très détaillées, les commentaires en \indice{ petits caractères} ne sont pas  nécessaires dans vos solutions et servent à faciliter la compréhension si certaines étapes du raisonnement ne sont pas claires

%\textbf{Work in progress!} 

\subsection*{Question 23}
\begin{enumerate}[i)] 
\item
  \begin{itemize}
  \item À montrer:  $\det(A) = \prod_{i=1}^r \lambda_i^{m_i}$
  \item   Si on écrit $p_A(x) = \sum_{j=0}^n \alpha_jx^j$, $α_i ∈K$, alors
    \begin{displaymath}
      \alpha_0 = p_A(x) = \det(A-0\cdot I) = \det(A)
    \end{displaymath}
    
\indice{Par définition on a $p_A(x) = \det(A-xI)$ et donc une évaluation en $x=0$ donne $\alpha_0 = \det(A-0\cdot I) = \det(A)$}  
\item Par supposition: Pour tout $i$:  $(x- λ_i)^{m_i} \mid p_A(x)$
\item Factorisation en irréductibles uniques en $K[x]$ $⟹$   $ \prod_{i=1}^r (x-\lambda_i)^{m_i} \mid p_A(x) $  
\item $n = ∑_i m_i =  ∑_i \deg(x- λ_i)^{m_i} $  $⟹$ $p_A(x) = \alpha_n \prod_{i=1}^r (x-\lambda_i)^{m_i}$ 
\item $α_n = (-1)^n$
  
\indice{$\alpha_n = (-1)^n$: Discuté plusieurs fois en classe, par la formule de Leibniz on peut écrire $p_A(x) = \sum_{\sigma \in S_n} q_{\sigma}(x)$ pour des polynômes $q_{\sigma}(x) = \text{sgn}(\sigma)\prod_{i=1}^n(A-xI)_{i,\sigma(i)}$ \\
On remarque que le degré de $q_{\sigma}(x)$ est égal au nombre de points fixés par la permutation $\sigma$ \\
Ainsi seulement la permutation $\sigma = id$ contribue au coefficient $\alpha_n$ de $p_A(x)$ \\
Comme $q_{id}(x) = \prod_{i=1}^n(a_{ii}-x)$ on obtient directement que $\alpha_n = (-1)^n$}
\item On a donc $α_0 = p_A(0) = (-1)^n \prod_{i=1}^r (0-\lambda_i)^{m_i} = \prod_{i=1}^r \lambda_i^{m_i}$   \\

\end{itemize}
\textcolor{blue}{4 points en total \\ on enlève 2 points si l'étudiant suppose que la matrice $A$ et diagonalisable, on enlève 2 points si le signe est mal placé dans une des expressions, on enlève 1 point pour tout autre erreur}

\item
  \begin{itemize}
    \item 
  On veut montrer que $\alpha_{n-1} = (-1)^{n-1}\text{Tr}(A)$ 
\item Formule de Leibniz: $ p_A(x) = \sum_{\sigma \in S_n} q_{\sigma}(x)$ où
  \begin{displaymath}
      q_{\sigma}(x) = \text{sgn}(\sigma)\prod_{i=1}^n(A-xI)_{i,\sigma(i)}
  \end{displaymath}
 
\item Pour $σ ≠ \mathrm{id}$, $\deg(q_{\sigma}(x)) ≤ n-2$
\item  Comme $q_{id}(x) = \prod_{i=1}^n(a_{ii}-x)$ on a  $$\alpha_{n-1} = (-1)^{n-1}a_{11} + \ldots + (-1)^{n-1}a_{nn} = (-1)^{n-1}\text{Tr}(A)$$
\end{itemize} 
\textcolor{blue}{3 points en total \\ on enlève entre 0.5 et 1 points pour des erreurs dans la formule de Leibniz, on enlève entre 0.5 et 1 points si la restriction à la permutation $\sigma = id$ n'est pas justifée, si l'étudiant écrit simplement la formule de Leibniz avec la justification sur la restriction à $\sigma = id$ mais sans conclure l'étudiant obtient 2 points}

\item
  \begin{itemize}
  \item 
On veut montrer que $\text{Tr}(A) = \sum_{i=1}^r m_i\lambda_i$  

\indice {Dans le point ii) on a montré que $\alpha_{n-1} = (-1)^{n-1}\text{Tr}(A)$}
\item $p_A(x) = α_0+ \cdots  + α_n x^n =   (-1)^n \prod_{i=1}^r (x-\lambda_i)^{m_i}$, alors: 
\begin{align*}
\alpha_{n-1} &= \underbrace{(-1)^{n+1}\lambda_1 + \ldots + (-1)^{n+1}\lambda_1}_{m_1 \text{fois}} + \ldots + \underbrace{(-1)^{n+1}\lambda_r + \ldots + (-1)^{n+1}\lambda_r}_{m_r \text{fois}} \\
&= (-1)^{n-1} \sum_{i=1}^r m_i\lambda_i
\end{align*}
\item
En utilisant le résultat du point ii) on conclut que $\text{Tr}(A) = \sum_{i=1}^r m_i\lambda_i$ \\ 
\end{itemize}
\textcolor{blue}{3 points en total \\ on enlève 2 points si l'étudiant suppose que $A$ est diagonalisable, on enlève 1 point pour des erreurs de signe} 
\end{enumerate}


\subsection*{Question 26}
% {\tiny toutes les normes sur l'espace vectoriel $\mathbb{C}^{n \times n}$ sont équivalentes (car il s'agit d'un espace vectoriel de dimension finie) mais pour fixer nos idées on considère la convergence par rapport à la norme $\lVert \cdot \rVert_2$}
\begin{enumerate}[i)]
\item
  \begin{itemize}
  \item Supposons $\rho(A) \geq 1$ et $A^k$ converge vers la matrice $0$
  \item Soit  $\lambda ∈ℂ$, $|λ|≥1$  valeur propre et  $x \in \mathbb{C}^n \setminus \{0\}$ vecteur propre  associé à $\lambda$
  \item On passe par $x := x / \|x\|$, alors $\|x\| = 1$, où $\| \cdot\|$
    dénote la norme associée aux produit Hermitien standard    
  % \item Pour tout $ε>0$ il existe $N$ tel que pour tout $i,j$:   $|(A^k)_{i,j}| < ε$
  %   pour tout $k \in \mathbb{N}$ $k≥ N$ 
  % \item Alors  pour tout $k \in \mathbb{N}$ $k≥ N$ et $i ∈ \{1,\dots,n\}$: 
  %   \begin{displaymath}
  %     \|(A^k x)_i \| < ε n 
  %   \end{displaymath}
  \item La suite  $A^k x ∈ ℂ^n$ converge vers $0 ∈ ℂ^n$
    
    \indice{Chaque composante de $A^k$ est une suite convergeant vers $0$ et chaque composante de $A^k x$ est une somme finie de suites convergeant vers $0$} 
  \item Mais $\|A^k x\|  = \|λ^k x\|  = |λ|^k \|x\| ≥ 1$
  \item Contradiction! \\
  \end{itemize}
\textcolor{blue}{3 points en total \\ 1 point pour l'idée et le début de la preuve par contradiction, 1 point pour le calcul  $\|A^k x\|  = \|λ^k x\|  = |λ|^k \|x\| ≥ 1$  et 1 point pour conclure l'argument} 

\item 
\begin{itemize}
\item On veut montrer que si $|\lambda|<1$ alors $\lim_{n\rightarrow \infty}B^n = 0$ \\
\indice{Comme $N$ est nilpotente il existe $k \in \mathbb{N}$ tel que $N^k = 0$, en plus comme $\lambda I$ est une matrice scalaire elle commute avec la matrice $N$,} 
\item On applique la formule du binôme de Newton, on suppose $n>k$ et on calcule
\begin{align*}
    B^n &= (\lambda I + N)^n \\
    &= \sum_{i=0}^n \begin{pmatrix} n \\ i \end{pmatrix} \lambda^{n-i}I \cdot N^i \\
    &= \sum_{i=0}^{k-1} \begin{pmatrix} n \\ i \end{pmatrix} \lambda^{n-i}I \cdot N^i + \underbrace{\sum_{i=k}^n \begin{pmatrix} n \\ i \end{pmatrix} \lambda^{n-i}I \cdot N^i}_{=0} \\
    &= \sum_{i=0}^{k-1} \begin{pmatrix} n \\ i \end{pmatrix} \lambda^{n-i}I \cdot N^i
\end{align*}
\indice{Un résultat d'Analyse nous donne que} $$\lim_{n \rightarrow \infty} \begin{pmatrix} n \\ i \end{pmatrix} \lambda^{n-i} = 0$$ pour $i \in \{0,\ldots,k-1\}$ fixé et $|\lambda|<1$ \\
\indice{On donc montré que $B^n$ est égal à une somme finie dont chaque terme converge vers 0 quand $n \rightarrow \infty$} 
\item On obtient $\lim_{n\rightarrow \infty}B^n = 0$ \\
\end{itemize}
\textcolor{blue}{4 points en total \\ 1 point pour l'utilisation de la formule du binôme de Newton, 2 points pour simplifier l'expression correctement à l'aide de la nilpotence de $N$ et 1 point pour conclure l'argument à l'aide de la limite $\lim_{n \rightarrow \infty} \begin{pmatrix} n \\ i \end{pmatrix} \lambda^{n-i}$} 

\item 
\indice{On montre les deux direction du si et seulement si}
\begin{itemize}
    \item [$\Longrightarrow$] 
    \begin{itemize}
    \item Contraposée: si $\rho(A) \geq 1$ alors la suite $A^n$ ne converge pas vers la matrice 0
    \item Conclusion par le point a) de l'exercice 
    \end{itemize}
    \item [$\Longleftarrow$] 
    \begin{itemize}
    %{\tiny on prend d'abord une valeur propre $\mu$ de $A$ associé au vecteur propre $y \in \mathbb{C}^n$, alors on obtient par un calcul direct la relation $A^ny = \mu^n y$} \\
    \item Prenons un vecteur propre unitaire $x \in \mathbb{C}^n$ associé à la valeur propre $\lambda$ qui vérifie $|\lambda| = \rho(A)$
    \item On calcule
    \begin{align*}
        0 &\leq \rho(A)^n \\
        &= |\lambda|^n \\
        &= \lVert \lambda^nx \rVert \\
        &= \lVert A^nx \rVert \\
        &\leq \lVert A^n \rVert \lVert x \rVert \\
        &= \lVert A^n \rVert
    \end{align*}
    \item Comme on a $\lVert A^n \rVert \rightarrow 0$ par supposition on voit que $\rho(A)^n \rightarrow 0$ \\
    \indice{Par le théorème des deux gendarmes}
    \item Ainsi $\rho(A)<1$ \\
    \end{itemize}
\end{itemize}
\textcolor{blue}{3 points en total \\ 1 point pour la direction $\Longrightarrow$ de l'équivalence et 2 points pour l'autre direction (1 point pour borner $\rho(A)^n$ supérieurement et 1 point pour conlure à l'aide du théorème des deux gendarmes)} 

\end{enumerate}

\subsection*{Question 27}
\begin{enumerate}[i)]
\item 
\begin{itemize}
\item 
%{soit $H\subset\mathbb{R}^n$ un sous-espace vectoriel de $\mathbb{R}^n$ et soit $v\in \mathbb{R}^n$, soit $l\in H$ tel que $\langle v-l, h\rangle = 0$ pour tout $h \in H$ \\
On veut montrer que $\lVert v-l \rVert \leq \lVert v \rVert$
\item Comme on a $\langle v-l,l \rangle = 0$ car $l \in H$, on applique le théorème de Pythagore dans le calcul suivant
$$\lVert v \rVert^2 = \lVert v-l + l \rVert^2 = \lVert v-l \rVert^2 + \lVert l \rVert^2$$
\item Ainsi $\lVert v \rVert^2 \geq \lVert v-l \rVert^2$ et donc $\lVert v \rVert \geq \lVert v-l \rVert$ \\
\end{itemize}
\textcolor{blue}{3 points en total \\ on donne 0 points si $\langle v,l \rangle \geq 0$ est utilisé sans justification, on donne 0 points pour $\lVert v-l \rVert^2 = \langle v,v \rangle - \langle v,l \rangle$, on enlève 1 point si par exemple l'application de Pythagore n'est pas justifiée}

\item 
\begin{itemize}
%soit $A\in \mathbb{R}^{n \times n}$ de rang plein et soit $A=A^*R$ une factorisation selon la méthode de Gram-Schmidt {\tiny (les colonnes de $A^* \in \mathbb{R}^{n \times n}$ sont deux-à-deux orthogonaux et $R \in \mathbb{R}^{n \times n}$ est une matrice triangulaire supérieure avec que de 1 sur la diagonale)} \\
\item On veut montrer que $\det(A^TA) = \prod_{i=1}^n \lVert a_i^* \rVert^2$ \\
\indice{Où $a_i^*$ est la $i$-ème colonne de la matrice $A^*$}
\item Il est clair que $\det(R)=1$ \\
\indice{Car $R$ est une matrice triangulaire supérieure avec que de 1 sur la diagonale}
\item On calcule
\begin{align*}
    \det(A^TA) &= \det((A^*R)^T(A^*R)) \\
    &= \det(R^TA^{*T}A^*R) \\
    &= \underbrace{\det(R^T)}_{=1}\underbrace{\det(R)}_{=1}\det(A^{*T}A^*) \\
    &= \det(A^{*T}A^*)
\end{align*}
\item Par définition du produit matriciel on voit que
$$(A^{*T}A^*)_{ij} = a_i^{*T}a_j^* = \begin{cases} \lVert a_i^* \rVert^2 & \text{si} \; i=j \\ 0 & \text{si} \; i\neq j \end{cases}$$
\indice{On sait que les colonnes de $A^*$ sont deux-à-deux orthogonaux}
\item On obtient $A^{*T}A^* = \text{diag}(\lVert a_1^* \rVert^2, \ldots, \lVert a_n^* \rVert^2)$ et $\det(A^{*T}A^*) = \prod_{i=1}^n \lVert a_i^* \rVert^2$
\item Par le calcul qui précède on obtient
$\det(A^TA) = \prod_{i=1}^n \lVert a_i^* \rVert^2$ \\
\end{itemize}
\textcolor{blue}{4 points en total \\ on donne 2 points pour montrer $\det(A^TA) = \det(A^{*T}A^*)$ et 2 points pour la preuve de $\det(A^{*T}A^*) = \prod_{i=1}^n \lVert a_i^* \rVert^2$}

\item 
\begin{itemize} 
\item On veut montrer que $|\det(A)| \leq \prod_{i=1}^n \lVert a_i \rVert$ \\
\indice{Où $a_i$ est la $i$-ième colonne de la matrice $A$}
%{\tiny par le point b) on $\det(A^TA) = \prod_{i=1}^n \lVert a_i^* \rVert^2$}
\item On a $\det(A^TA) = \det(A^T)\det(A) = \det(A)^2$ \\
\indice {On obtient donc $|\det(A)| = \prod_{i=1}^n \lVert a_i^* \rVert$ par la partie ii)} \\
\indice{Comme on a utilisé la méthode de Gram-Schmidt pour définir $A^*$ on a la relation $a_i^* = a_i -l$ où $l \in \text{span}\{a_1^*,\ldots,a_{i-1}^*\}$, en plus on a aussi $a_i^* \perp \text{span}\{a_1^*,\ldots,a_{i-1}^*\}$}
\item Posons $H = \text{span}\{a_1^*,\ldots,a_{i-1}^*\}$ et on conclut par le point i) que $\lVert a_i^* \rVert = \lVert a_i-l \rVert \leq \lVert a_i \rVert$
\item On en déduit que $$|\det(A)| = \prod_{i=1}^n \lVert a_i^* \rVert \leq \prod_{i=1}^n \lVert a_i \rVert$$ \\
\end{itemize}
\textcolor{blue}{3 points en total \\ on donne 1 point pour montrer $|\det(A)| = \prod_{i=1}^n \lVert a_i^* \rVert$ et 2 points pour $\lVert a_i^* \rVert \leq \lVert a_i \rVert$ en utilisant i) et après avoir verifié les conditions}
\end{enumerate}
\end{document}
%%% Local Variables:
%%% mode: latex
%%% TeX-master: t
%%% End: