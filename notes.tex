\documentclass[a4paper,11pt,french]{scrbook} 
 
\usepackage[utf8]{inputenc} 
\usepackage{utf8math}

%\usepackage[euler-digits,euler-hat-accent]{eulervm}
\usepackage[boldsans]{concmath}

\usepackage{mathrsfs}
\usepackage[french]{babel}
\usepackage{mathtools} % includes amsmath
\usepackage{amssymb}
\usepackage{amsthm}
\usepackage{amscd}
\usepackage{todonotes}

%\usepackage{multirow}
\usepackage{enumerate}

%\usepackage{tikz}
\usepackage{framed}
%\usepackage[colorlinks]{hyperref}
%\usepackage{showlabels}  


\newcommand{\smat}[1]{ \big(\begin{smallmatrix} #1 \end{smallmatrix}\big)}

\newcommand{\E}{\mathbb{E}}
\newcommand{\N}{\mathbb{N}}
\newcommand{\Q}{\mathbb{Q}}
\newcommand{\R}{\mathbb{R}}
\newcommand{\Z}{\mathbb{Z}}
\newcommand{\C}{\mathbb{C}}
\newcommand{\K}{\mathbb{K}}
\newcommand{\x}{\mathbf{x}}
\newcommand{\y}{\mathbf{y}}
\newcommand{\X}{\mathscr{X}}
\newcommand{\cA}{\mathcal{A}}
\newcommand{\cD}{\mathcal{D}}
\newcommand{\cI}{\mathcal{I}}
\newcommand{\cP}{\mathcal{P}}
\newcommand{\cV}{\mathcal{V}}
\newcommand{\pscal}[1]{\langle {#1} \rangle}
\newcommand{\wt}[1]{\widetilde{#1}}
\newcommand{\car}{\mathrm{Char}}

\providecommand{\one}{\mathbf{1}}
\DeclareMathOperator{\vol}{vol}
\DeclareMathOperator{\rank}{rang}
\DeclareMathOperator{\noy}{noyau}
\DeclareMathOperator{\cone}{cone}
\DeclareMathOperator{\tcone}{tcone}
\DeclareMathOperator{\conv}{conv}
\DeclareMathOperator{\spec}{spec}
\DeclareMathOperator{\cof}{cof}
\DeclareMathOperator{\diam}{diam}
\DeclareMathOperator{\sign}{sgn}
\DeclareMathOperator{\poly}{poly}
\DeclareMathOperator{\Ker}{Ker}
\DeclareMathOperator{\Var}{Var}
\DeclareMathOperator{\Id}{Id}
\DeclareMathOperator{\tr}{tr}
\DeclareMathOperator{\relint}{relint}
\DeclareMathOperator{\spa}{span}
\DeclareMathOperator{\Tr}{Tr}
\DeclareMathOperator{\diag}{diag}
\DeclarePairedDelimiter\mnorm{\lvert\lvert\lvert}{\rvert\rvert\rvert} % matrix norm



\theoremstyle{plain}
\newtheorem{theorem}{Théorème}[chapter]
\newtheorem{lemma}[theorem]{Lemme}
\newtheorem{proposition}[theorem]{Proposition}
\newtheorem{claim}[theorem]{Claim}
\newtheorem{corollary}[theorem]{Corollaire}

\theoremstyle{definition}
\newtheorem{definition}{Définition}[chapter]
\newtheorem*{notation}{Notation}
\newtheorem{example}{Exemple}[chapter]
\newtheorem{remark}[theorem]{Remarque}
\newtheorem{problem}{Problème}[chapter]
\newtheorem{exercise}{Exercice}[chapter]
\newtheorem{algorithm}{Algorithme}[chapter]



\newcommand{\iunit}{\mathrm{i}}
\newcommand{\CC}{{\mathbb C}}
\newcommand{\EE}{{\mathbb E}}
\newcommand{\FF}{{\mathbb F}}
\newcommand{\KK}{{\mathbb K}}
\newcommand{\NN}{{\mathbb N}}
\newcommand{\QQ}{{\mathbb Q}}
\newcommand{\RR}{{\mathbb R}}
\newcommand{\ZZ}{{\mathbb Z}}

\newcommand{\calB}{{\mathcal B}}


%opening

\title{Algèbre linéaire avancée II}
\author{Friedrich Eisenbrand}

%\includeonly{chapter02-theoreme-spectral}

\begin{document}

\maketitle
  
\section*{Préface}
\noindent Ceci sont mes notes du cours \emph{Algèbre Linéaire Avancée II}.
La qualité de ce texte dépend fortement de la participation  des étudiants. 
Ces  sources sont gérées sur \emph{GitHub}, une plateforme importante de collaboration. Si vous trouvez des fautes, des erreurs typographiques, ou même des démonstrations plus élégantes, ou des exemples qui vous aident à comprendre la matière, je vous invite à créer une \emph{branch} des fichiers en question, où dedans vous éditez le texte. Après, vous publiez (\emph{publish}) cette \emph{branch} et vos collègues peuvent discuter vos modifications. Si vous êtes satisfait avec vos modifications, vous me demandez, avec un \emph{Pull Request}, d'accepter vos modifications et finalement, le document peut être changé. Je me réjouis en avance de votre participation. 

  
\section*{Contributions}

Des corrections et modifications ont été implémentées par: 
\begin{itemize}
\item Orane Jecker 
\item Natalia Karaskova
\item Dylan Samuelian
\item Aziz Benmosbah
\item Djian Post
\item Robin Mamie
\item Alfonso Cevallos
\item Kévin Jorand
\item Charles Dufour 
\item Christoph Hunkenschröder
\item Adam Cierniak
\item Mann-Tchi Dang
\item Yasmine Bennis
\item Corentin 
\item Lucas Gehrt
\item Jooyoung Kim 
\item Léo Navarro
\item Arthur Serres
\item Matthieu Haeberle
\item Chady Bensaid
\item Cedric Lehr
\item Orso Renucci
\item mqultra 
\end{itemize}

Le deuxième   chapitre est basé, en partie, sur les notes du cours de Daniel Kressner.  

\tableofcontents

\chapter{Polynômes}
\label{cha:polynomes}

\section{Notions fondamentales}
\label{sec:noti-fond}


Soit $R$ un anneau. On se souvient que cela signifie que  $R$ est un ensemble,  muni des opérations binaires  $+ : R × R → R$  et $⋅: R × R → R$ telles que: 
\begin{enumerate}[(R1)]
\item $a+ b  = b+a$ pour tout     $a,b ∈ R$. \label{R1}
\item $a + (b+c) = (a + b) +c$, pour tout $a ,b,c ∈ R$. \label{R2}
\item Il existe un élément $0 ∈R$ tel que $0+a =a$ pour chaque $a ∈R$. \label{R3}
\item Pour chaque $a ∈R$ il existe un élément $-a ∈R$ tel que $a + (-a) = 0$. \label{R4}
\item $a(bc) = (ab) c$ , pour tout $a,b,c ∈ R$. \label{R5}
\item Il existe un élément $1 ∈R$ tel que $a ⋅ 1 = 1 ⋅a = a$ pour chaque $a ∈R$. \label{R6}
\item $a (b+c) = ab + ac$ et $(b+c) a =ba +ca$  pour tout $a,b,c ∈R$.\label{R7} 
\end{enumerate}
Si, de plus, on a 
\begin{enumerate}[(R1)]
  \setcounter{enumi}{7}
\item  $a b = ba$ pour tous $a,b ∈R$ \label{R8}. 
\end{enumerate}
alors l'anneau $R$ est appelé \emph{anneau commutatif}. Le \emph{centre} de $R$ est l'ensemble $Z(R) = \{ r ∈ R : ra = ar \text{ pour tout } a ∈ R\}$. 
Un élément  $a ≠0$ de $R$ est un \emph{diviseur de zéro} s'il existe un $b≠0$ tel que $ab = 0$ ou $ba = 0$. Un anneau commutatif sans diviseurs de zéro est appelé \emph{anneau intègre}.  



\begin{exercise}
  Soit $R$ un anneau, alors l'élément $1$ est unique. 
\end{exercise}



\begin{example}
  \label{exe:32}
  \begin{enumerate}[i)]
  \item Les nombres entiers $ℤ$ avec l'addition et la multiplication standard forment un anneau commutatif sans diviseurs de zéro. $ℤ$ est donc un anneau intègre. 
  \item $5 ⋅ℤ = \{ 5 z :z ∈ ℤ\}$ avec l'addition et la multiplication standard n'est pas un anneau. L'axiome (R\ref{R6}) n'est pas satisfait.
  \item L'ensemble des matrices  $ℤ^{2 ×2}$ avec l'addition et multiplication des matrices est un anneau non commutatif avec des diviseurs de zéro. 
    \begin{eqnarray*}
      \begin{pmatrix}
        1 & 0 \\
        1 & 1 
      \end{pmatrix}
       \begin{pmatrix}
        1 & 1 \\
         0& 1 
       \end{pmatrix}   & = &
                            \begin{pmatrix}
                               1 & 1 \\
                               1 & 2 \\
                             \end{pmatrix}  \\      
      \begin{pmatrix}
        1 & 1 \\
        0 & 1 
      \end{pmatrix}
       \begin{pmatrix}
        1 & 0 \\
         1 & 1 
       \end{pmatrix}   & = &
                            \begin{pmatrix}
                               2 & 1 \\
                               1 & 1 \\
                             \end{pmatrix}       
    \end{eqnarray*}
     \begin{eqnarray*}
      \begin{pmatrix}
        0 & 0 \\
        0 & 1 
      \end{pmatrix}
       \begin{pmatrix}
        1 & 0 \\
         0& 0 
       \end{pmatrix}   & = &
                            \begin{pmatrix}
                               0 & 0 \\
                               0 & 0 \\
                             \end{pmatrix}  
     \end{eqnarray*}
   \end{enumerate}
\end{example}



La démonstration du théorème suivant est un exercice.  
\begin{theorem}
  \label{thr:58}
  Soit $R$ un anneau et $S ⊆R$. Si les deux conditions suivantes sont vérifiées 
  \begin{enumerate}[a)]
  \item $1 ∈ S$ \label{item:27}
  \item Pour $s,t ∈S$ alors  $s⋅t∈S$   et $s-t ∈S$. \label{item:28}
  \end{enumerate} 
  alors $S$ est aussi un anneau. 
\end{theorem}

Un élément $r ∈R$ est \emph{inversible} s'il existe un élément $r^{-1} ∈ R$ tel que
\begin{displaymath}
  r ⋅ r^{-1} = r^{-1} ⋅r = 1. 
\end{displaymath}
On dénote l'ensemble des éléments inversibles par $R^*$. On se rappelle que $(R^*, ⋅)$ est un groupe, le \emph{groupe des éléments inversibles}. Un anneau intègre tel que $R \setminus \{0\} = R^*$ est un \emph{corps}. 

\begin{exercise}
  \label{exe:39}
  Soit $R$ un anneau et $r ∈ R^*$. Alors $r$ n'est pas un diviseur de zéro. 
\end{exercise}

\begin{exercise}
  \label{exe:42}
  Soit $R$ un anneau et $R^{n ×n}$ l'anneau des matrices $n ×n$ sur $R$. Montrer que le centre de $R^{n ×n}$ est $Z(R^{n ×n}) = \{ a I_n :a ∈Z(R)\}$. 
\end{exercise}

\begin{theorem}
  \label{thr:50}
  Soit $R$ un anneau, alors il existe un anneau $S ⊇R$ ($R$ est un sous-anneau de $S$) et un élément $x ∈ S \setminus R$ tel que
  \begin{enumerate}[(i)]
  \item $a x = x a $ pour chaque $a ∈ R$.
  \item Si 
    $  a_0+ a_1x + \cdots + a_n x^n =0$  et $a_i ∈R$ pour tout $i$,  
    alors  $a_i = 0$ pour tout $i$.
  \end{enumerate}
\end{theorem}
Ce théorème est démontré dans le cours \emph{anneaux et corps}.   Il nous donne une manière formelle d'introduire le concept d'une \emph{indéterminée} (ou  \emph{variable}) qui n'est autre que ledit élément $x ∈ S \setminus R$.  

\begin{definition}
  \label{def:51}
  Un polynôme sur $R$ est une expression de la forme
$p(x) = a_0 + a_1 x + \cdots + a_n x^n$, où $n ∈ℕ$,  $a_i ∈R$ pour tout $i$. L'élément $a_i$ est le $i$-ème \emph{coefficient} de $p(x)$. L'ensemble des polynômes sur $R$ est dénoté par $R[x]$. 
\end{definition}


\begin{example}
  \begin{enumerate}[i)]
  \item  $p(x) = 3 + x^2 + 5x^4 ∈ ℤ[x]$.
  \item $p(x) = \smat{1 &0 \\ 1 & 1} +   \smat{3 &3 \\ 2 & 1} x^3 ∈ ℤ^{2×2}[x]$. 
  \end{enumerate}
\end{example}



\begin{proposition}
  \label{prop:8}
  Deux polynômes 
\begin{equation}
  \label{eq:34}
  p(x) = a_0 + a_1x + a_2x^2 + \cdots \,\,\text{ et }  \,\, q(x) = b_0 + b_1x + b_2x^2 + \cdots
\end{equation}
sont \emph{égaux} si et seulement si $a_i =b_i$ pour tout $i ∈ℕ$. Dans ce
cas, on écrit $p(x) = q(x)$.
\end{proposition}

\begin{exercise}
  \label{exe:34}
  Démontrez  la proposition~\ref{prop:8}. 
\end{exercise}


\begin{theorem}
  \label{thr:51}
  $R[x]$ est un anneau. Si $R$ est commutatif, alors $R[x]$ est commutatif. Si $R$ est anneau sans diviseur de zéro    alors $R[x]$ est un  anneau sans diviseur de zéro. 
\end{theorem}
\begin{proof}
  Pour montrer que $R[x]$ est un anneau on  invoque  Theorème~\ref{thr:58}. Clairement $1 ∈ R[x]$ ce qui implique  \ref{item:27}). Pour \ref{item:28})
  il faut montrer que
  pour deux polynômes $f(x) = a_0 + a_1 x + \cdots + a_n x^n$ et $g(x) = b_0 + b_1x + \cdots + b_m x^m$ sur $R$, on a : 
  \begin{enumerate}[i)]
  \item $f- g ∈ R[x]$ et
  \item $f ⋅ g ∈R[x]$. 
  \end{enumerate}
  La somme s'écrit comme
  \begin{equation}
    \label{eq:47}
    f(x) + g(x) = ∑_{i=0}^{\max\{m,n\}} (a_i +  b_i) \, x^i  ∈R[x], 
  \end{equation}
  où $a_i = 0$ pour $i>n$ et $b_i = 0$ pour $i>m$. 
  % Alors   $R[x]$ est stable pour $+$. 
  % L'inverse de $g(x)$ est $-g(x) = -b_0 - b_1 x  \cdots - b_m x^m ∈ R[x]$.
  Alors
  on a $f-g ∈ R[x]$. 
  
  Le    produit de $f$ et $g$ s'écrit comme

  \begin{equation}
  \label{ceq:21}
  p(x) ⋅q(x) = a_0 b_0 + (a_0b_1 +a_1b_0) x + (a_0b_2+ a_1b_1 + a_2b_0)x^2 + \cdots .
\end{equation}
Bref, on a la formule 
  \begin{equation}
    \label{eq:8}
    f(x) ⋅g(x) = ∑_{i=0}^{m+n} ( ∑_{j+k = i} a_j b_k) \, x^i ∈ R[x]. 
  \end{equation}
  C'est à dire que $R[x]$ est stable pour l'opération $⋅$ de $S$.  Alors, $R[x]$ est un sous-anneau de $S$ et 
  donc $R[x]$ est un anneau. La formule~\eqref{eq:8} implique que
  $R[x]$ est commutatif, si $R$ est commutatif.

  Finalement, si $f(x) = a_0 + a_1 x + \cdots ≠ 0$ et
  $g(x) = b_0 + b_1 x + \cdots ≠ 0 $ sont deux polynômes non-nuls et si $R$ est un 
  anneau sans diviseur de zéro, il faut montrer que $f(x) ⋅ g(x) ≠
  0$. Soit $n = \max\{ i :a_i ≠0 \}$ et $m = \max\{ i :b_i ≠0 \}$. Le coefficient de $x^{m+n}$ du polynôme $f⋅g$ est $a_n ⋅b_m$. Ce coefficient n'est pas nul dès que $R$  est un 
  anneau sans diviseur de zéro. 
\end{proof}

\begin{example}
  \label{exe:33}
  \begin{eqnarray*}
    f(x) & = &  3 \, x^{3} + x + 2 \\
    g(x) & = & 2 \, x^{4} + 2 \, x^{2} + 1 \\
    f(x) ⋅g(x) & = & 6 \, x^{7} + 8 \, x^{5} + 4 \, x^{4} + 5 \, x^{3} + 4 \,  x^{2} + x + 2 \\ 
  \end{eqnarray*}
\end{example}



% \begin{theorem}
%   \label{thr:43}
%   L'ensemble des polynômes $K[x]$ sur un corps $K$ est un anneau intègre.
% \end{theorem}

\begin{definition}
  \label{def:52}
Le \emph{degré} de $p(x) = a_0 + a_1x + a_2x^2 + \cdots \neq 0$ est 
\begin{displaymath}
  \deg(p) = \max\{i \colon  a_i \neq 0\}
\end{displaymath}
et $\deg(0) = -\infty$. 
Si $p \neq 0$, le coefficient $a_{\deg(p)}$ est le \emph{coefficient dominant} de $p$. 
Un polynôme de degré zéro est une \emph{constante}.
\end{definition}

\begin{example}
  \label{exe:35}
  Soit $f(x) = 2 + 3x + 5x^3 ∈ ℤ[x]$, alors $\deg(f) =3$  et le coefficient dominant de $f$ est $5$. 
\end{example}

\begin{theorem}
  \label{thr:34} Soit $R$ un anneau,   $f,g \in R[x] \setminus \{0\}$ tel que le coefficient dominant de $f$ ou de $g$ n'est pas un diviseur de zéro.  Alors, $\deg(f \cdot g) = \deg(f) + \deg(g)$. 
\end{theorem}
\begin{proof}
  
  Soient $f(x) = a_0 + \cdots + a_n x^n$ et $g(x) = b_0+ \cdots b_m x^m$ tels que $a_n, b_m  \neq 0$. Le coefficient de $x^{n+m}$  est $a_n \cdot  b_m \neq 0$. Les coefficients de $x^k$, $k> n+m$ sont tous nuls.  
\end{proof}

Un polynôme $p(x) = a_0 + a_1 x + \cdots + a_n x^n ∈ R[x]$ induit une application $f_p:  R ⟶ R$, $f_p(r) = a_0+ a_1 r+ \cdots + a_n r^n$. Nous écrivons aussi $p(r)$ pour $f_p(r)$ et on parle de l'\emph{évaluation} de $p$ sur $r$. 

\begin{example}
  \label{exe:43}
  Soit $p(x) = x ∈ R[x]$ et $q(x) = (x + a) ∈ R[x]$, alors $p(x) q(x) = x^2 + ax$. Quand est-ce que l'on a $p(r) q(r) = (p ⋅q)(r)$? C'est le cas si et seulement si
  \begin{displaymath}
     r^2 + r a = r^2 + a r  
   \end{displaymath}
   c'est-à-dire si et seulement si $ra = ar$. 
   On attire l'attention sur le fait que cet exemple traite de l'évaluation en $r \in R$ de polynômes. 
   Ici, $p(r)q(r)$ dénote l'évaluation de $p$ et $q$ en $r$ \emph{puis} d'en multiplier les résultats.
   Tandis que $(p⋅q)(r)$ représente la multiplication de $p$ et $q$ \emph{puis} on évalue ce produit en $r$.
\end{example}


\begin{theorem}
  \label{thr:53}
  Soit $R$ un anneau et $\alpha \in Z(R)$ un élément du centre de $R$. L'application 
  \begin{displaymath}   
    \begin{array}{rcl}
    \Phi \colon R[x] &\rightarrow & R \\
           f(x) &\mapsto &f(\alpha)          
    \end{array}
  \end{displaymath}
est un morphisme d'anneaux surjectif. 
\end{theorem}


\begin{exercise}
  \label{exe:44}
  Démontrer le théorème~\ref{thr:53}. 
\end{exercise}


\section{Interpolation}
\label{sec:interpolation}


Deux polynômes différents $p$ et $q$ peuvent induire la même application $f_p$ et $f_q$, même s'ils sont des polynômes sur un corps $K$.

\begin{example}
  \label{exe:36}
  Soit $K = ℤ_2$, $p(x) = x + x^2 $ et $q(x) = x^2 + x^3$ deux
  polynômes sur $K$. Il est clair que $p ≠ q$. Mais $f_p:\, K →K$ et
  $f_p:\, K →K$ sont les mêmes applications car pour tout
  $x ∈ℤ_2$ on a $p(x) = q(x) = 0$.
\end{example}

Par contre, si $K$ est un corps infini, deux polynômes différents induisent deux applications différentes. Nous allons voir les détails de ce fait maintenant.

\begin{theorem}
  \label{thr:52}
  Soit $K$ un corps, $r_0,\dots,r_n ∈K$ des éléments distincts (c.-à-d. $r_i ≠ r_j$ pour $i≠j$) et $y_0,\dots,y_n ∈K$. Il existe  exactement un seul polynôme $f(x) ∈ K[x]$ de degré au plus $n$ tel que
  \begin{displaymath}
    f(r_i) = y_i \text{ pour tout } i ∈ \{0,\dots,n\}. 
  \end{displaymath}
\end{theorem}

\begin{proof}
  Un polynôme  $f(x) = a_0 + a_1x + \cdots + a_n x^n$ satisfait $f(r_i) = y_i$ pour tout $i$ si et seulement si le coefficients $a_0,\dots,a_n$ satisfont
  \begin{equation}
    \label{eq:48}
    \begin{pmatrix}
      1 & r_0& \cdots & r_0^n \\
      1 & r_1& \cdots & r_1^n \\
       & &  \cdots  &  \\
       1 & r_n& \cdots & r_n^n \\
     \end{pmatrix}
     \begin{pmatrix}
       a_0\\ a_1 \\ \vdots \\ a_n
     \end{pmatrix} =
     \begin{pmatrix}
       y_0\\ y_1 \\ \vdots \\ y_n
     \end{pmatrix}
   \end{equation}

   La matrice $V_{r_0,\dots,r_n} ∈K^{(n+1)×(n+1)}$  à gauche de~\eqref{eq:48} est connue sous le nom de \emph{matrice de Vandermonde}
   des éléments $r_0,\dots,r_n$. Le théorème sera prouvé une fois que nous aurons démontré que $\det(V_{r_0,\dots,r_n}) ≠0$.

   On démontre  $\det(V_{r_0,\dots,r_n}) ≠0$ par récurrence sur $n$. Pour $n = 0$, le déterminant est $1$. Pour $n>0$, on soustrait $r_0$ fois la colonne $n$ de la colonne $n+1$. Après, on soustrait $r_0$ fois la colonne $n-1$ de la colonne $n$ etc. Le résultat est la matrice
   \begin{equation}
     \label{eq:49}
     \begin{pmatrix}
       1 & 0  &      \hdotsfor{2} &  0 \\
       1 & r_1- r_0&  r_1 (r_1-r_0)  & \cdots & r_1^{n-1} (r_1 - r_0) \\
       & & &  & \cdots  &  \\
       1 & r_n -r_0&  r_n (r_n-r_0) & \cdots & r_n^{n-1} (r_n - r_0) \\
     \end{pmatrix}
   \end{equation}
   Le déterminant de la matrice~\eqref{eq:49} est celle de
   $V_{r_0,\dots,r_n}$. Développement du déterminant le long de la
   première ligne donne le déterminant de la matrice
 \begin{equation}
\label{eq:50}
    \begin{pmatrix}
       r_1- r_0&  r_1 (r_1-r_0) \cdots & r_1^{n-1} (r_1 - r_0) \\
        & &  \cdots  &  \\
        r_n -r_0&  r_n (r_n-r_0) \cdots & r_n^{n-1} (r_n - r_0) \\
     \end{pmatrix}
   \end{equation}
   Alors
   \begin{displaymath}
     \det(V_{r_0,\dots,r_n})  = (r_n-r_0) \cdots (r_1-r_0)  \det(V_{r_1,\dots,r_n})
   \end{displaymath}      
Comme les $r_i$ sont distincts, le produit   $(r_n-r_0) \cdots (r_1-r_0) $ n'est pas zéro. Par l'hypothèse de récurrence, $\det(V_{r_1,\dots,r_n}) ≠0$ et donc $\det(V_{r_0,r_1,\dots,r_n}) ≠0$.  
\end{proof}

Le théorème~\ref{thr:52} ne contredit pas l'exemple~\ref{exe:36} car le corps $ℤ_2$ n'admet que $2$ éléments. Il existe bel et bien un unique polynôme $p \in  ℤ_2[x]$ de degré inférieur ou égal à 1 tel que $p(0) = p(1) = 0$. Une énumération des polynômes possibles ou la résolution du système linéaire donne $p(x)=0$.

\begin{exercise}
  \label{exe:37}
  Montrer que le déterminant de $V_{r_0,\dots,r_n}$ est
  \begin{displaymath}
    \det (V_{r_0,\dots,r_n}) = ∏_{0 ≤ i<j ≤n} (r_j - r_i). 
  \end{displaymath}
\end{exercise}



\begin{example}
  \label{exe:38}
  On cherche le polynôme $f(x) ∈ ℤ_5[x]$ de degré au plus $3$ tel que
  $f(0) = 2$, $f(1)=2$, $f(2) = 2$ et $f(3) = 3$.

  En trouvant l'unique solution du système
  \begin{displaymath}
    \left(\begin{array}{rrrr}
1 & 0 & 0 & 0 \\
1 & 1 & 1 & 1 \\
1 & 2 & 4 & 3 \\
1 & 3 & 4 & 2
\end{array}\right)
\begin{pmatrix}
  a_0 \\ a_1 \\ a_2 \\ a_3
\end{pmatrix}
=
\begin{pmatrix}
  2 \\ 2 \\ 2 \\3
\end{pmatrix},
\end{displaymath}on obtient
\begin{displaymath}
  \begin{pmatrix}
  a_0 \\ a_1 \\ a_2 \\ a_3
\end{pmatrix} =
\begin{pmatrix}
  2 \\ 2 \\ 2 \\ 1
\end{pmatrix}.
\end{displaymath}
Ainsi, $f(x) = 2 + 2x + 2x^2 + x^3$ est le polynôme recherché. 

\end{example}


\begin{corollary}
  \label{thr:42}
  Soit $K$ un corps infini. Deux polynômes $p(x),q(x) ∈ K[x]$ sont égaux, si et seulement si les applications $f_p$ et $f_q$ sont les mêmes.
\end{corollary}
\begin{proof}
  Si $p(x) = q(x)$, alors $p(r) = q(r)$ pour tout $r ∈K$ et donc $f_p = f_q$. D'autre part, si $f_p = f_q$, on peut supposer qu'un des deux polynômes n'est pas le polynôme $0$ (s'ils sont tous les deux nuls, alors $p=q$).  Comme $p(r) = q(r)$ pour tout $r ∈K$ et le corps $K$ est infini, les polynômes prennent les mêmes valeurs sur $\max\{\deg(p),\deg(q)\}+1$ éléments distincts de $K$. Il en résulte que $p(x) = q(x)$ par le Theorème~\ref{thr:52}. 
\end{proof}




\section{Division avec reste et  racines} 
\label{sec:divisibilite}

Soit $R$ un anneau et $K$ un corps pour ce chapitre. La \emph{division avec reste} est l'opération suivante. 

\begin{theorem}
  \label{thr:33}
  Soient $f,g \in R[x]$, $\deg(g) >0$ et le coefficient dominant de $g$ un élément  inversible de $R$.   Il existe $q,r \in R[x]$ uniques tels que 
  \begin{displaymath}
    f(x) = q(x) g(x) + r(x) 
  \end{displaymath}
  et $\deg(r) < \deg(g)$. 
\end{theorem}


\begin{proof}
  La preuve se fait par récurrence sur $\deg(f)$. Si $\deg(f) < \deg(g)$, alors on pose $q = 0$ et $r = f$.

Soit alors $\deg(f) = n \geq \deg(g)=m$ et 
\begin{displaymath}
  f(x) = a_0+ \cdots +a_n x^n \text{ et } g(x) = b_0 + \cdots + b_m x^m 
\end{displaymath}
où $a_n$ et $b_m$ sont les coefficients dominants de $f$ et $g$ respectivement. 
Clairement 
\begin{displaymath}
  \deg\left( f(x) - \frac{a_n}{ b_m } x^{n-m} g(x) \right) < \deg(f(x))
\end{displaymath}
et par hypothèse de récurrence il existe $q_0,r ∈R[x]$ tel que 
\begin{displaymath}
  f(x) - \frac{a_n}{ b_m } x^{n-m} g(x)  = q_0(x) g(x) + r(x) 
\end{displaymath}
et $\deg(r(x)) < \deg(g(x))$. On  obtient alors
\begin{eqnarray*}
  f(x) & = &  \left(q_0(x) + \frac{a_n}{ b_m } x^{n-m} \right) g(x) + r(x) \\
       & = & = q(x)g(x) + r(x),
\end{eqnarray*}
 où $q(x) = q_0(x) + \frac{a_n}{ b_m } x^{n-m}$. 
%
Supposons maintenant qu'il existe deux autres polynômes $q'(x) \neq q(x)$ et $r'(x) \neq r(x)$ tels que 
\begin{displaymath}
    f(x) = q'(x) g(x) + r'(x) 
  \end{displaymath}
  et $\deg(r') < \deg(g)$. 
Alors 
\begin{displaymath}
   r(x) - r'(x) = (q'(x) - q(x)) ⋅ g(x). 
\end{displaymath}
Par le théorème~\ref{thr:34}
\begin{displaymath}
 \deg( r - r')  = \deg(q' - q) + \deg(g) \geq \deg(g), 
\end{displaymath}
ce qui contredit le fait que $\deg(r) < \deg(g)$ et $\deg(r') < \deg(g)$. 
\end{proof}


% \begin{definition}
%   Pour $f(x)  = a_0 + \cdots + a_n x^n \in K[x]$ et $\alpha \in K$, l'évaluation $f(\alpha)$ est $ a_0 + a_1 \alpha + \cdots + a_n \alpha^n \in K$. 
% \end{definition}


\begin{example}
  \label{exe:24}
  La division avec reste du polynôme $x^5+2x^2+1$ par $2x^3+x+1$ de $ℤ_3[x]$ donne

  \begin{displaymath}
    x^5+2x^2+1 = (2x^2 +2) (2x^3 + x +1) + (x+2). 
  \end{displaymath}
  
\end{example}

%\begin{proposition}
%  \label{prop:5}
%  Pour $\alpha \in K$, $\phi_\alpha: \, K[x] \longrightarrow K$, %$\phi_\alpha(f(x)) = f(\alpha)$ est un homomorphisme.  
%\end{proposition}
%
On peut formuler la division avec reste comme un procédé récursif comme suivant. On suppose ici  $\deg(g(x)>0$. 

\begin{algorithm}
  \caption{Division avec reste}\label{euclid}
  \begin{algorithmic}[1]
    \Procedure{Div}{$f(x), g(x)$ }
    \State{Soient $n= \deg(f)$ et $m = \deg(g)$}   
  \If{$n < m$}
    \State\Return{$0$,$f(x)$}
    \Else
    \State{Soient $a,b ∈K$ coefficients dominants de $f$ et $g$ respectivement}  
    \State{$(q_0(x),r(x)) \leftarrow  $ {\Call{Div}{$f(x) - (a/b) x^{n-m} g(x)$, $g(x)$}}}
    \State \Return $(a/b) x^{n-m} q_o(x),r(x))$  
  \EndIf
  \EndProcedure
\end{algorithmic}
\end{algorithm}



\begin{definition}
  \label{def:32}
  Un polynôme  $q(x)$ \emph{divise} un  polynôme $f(x)$ s'il existe un polynôme $g(x)$ tel que $f(x) = g(x) \cdot q(x)$. On dit que $q(x)$ est un \emph{diviseur} de $f(x)$ et on écrit $q(x) \mid f(x)$. 
\end{definition}


\begin{example}
  \label{exe:41}
  Soient $q(x) = x^2 +1 ∈ℤ_2[x]$ et $ f(x) = x^3 + x^2 - x+1 ∈ ℤ_2[x]$. On a
  $f(x) = q(x) (x+1)$ et donc 
   $q(x) \mid f(x)$. 
\end{example}


\begin{definition}
  \label{def:31}
  Soit $p(x) \in K[x] \setminus\{0\}$. Un $\alpha \in K$ tel que $f_p(\alpha) = 0$ est une  \emph{racine} de $f(x)$.  
\end{definition}


\begin{example}
  \label{exe:40}
  Soit $p(x) = x^4 + x^3 + x^2 + x + 1 ∈ ℤ_5(x)$, alors $α = 1$ est une racine de $p(x)$. 
\end{example}


\begin{theorem}[Théorème fondamental de l'algèbre]
  \label{thr:44}
  Tout polynôme $p(x) ∈ℂ[x] ⧹\{0\}$ non constant admet au moins une racine complexe.
\end{theorem}




\begin{theorem}
  \label{thr:35}
  Soit $f(x)∈ K[x] \setminus \{0\}$ un polynôme  et $\alpha \in K$, alors $\alpha$ est une racine de $f$ si et seulement si $(x- \alpha)  \mid f(x)$. 
\end{theorem}

\begin{proof}
  Si $f(x) = q(x) \cdot (x - \alpha)$, alors $f(\alpha) = 0$. 

%Autrement, si $f$ est une constante, $f = 0$ et $(x - \alpha)$ divise $f$.
Pour l'autre sens, il existe $q(x)$ et $r(x)$ tels que
\begin{displaymath}
  f(x) = q(x) \cdot (x - \alpha) + r(x)
\end{displaymath}
%$\deg(r) = 0$. Alors $f(\alpha) = 0$ implique $r=0$. 
avec $\deg(r) \leq 0$. Alors $f(\alpha) = 0$ implique $r=0$. 
\end{proof}


\begin{definition}
  \label{def:41}
  La \emph{multiplicité} d'une racine $α$ de $p(x) ∈ K[x] ⧹\{0\}$ est le plus grand entier $i≥1$ tel que $ (x-α)^i \mid p(x)$. Si $p(x)$ est le polynôme caractéristique d'un endomorphisme d'un espace vectoriel, on appelle la multiplicité de $α$ la \emph{multiplicité algébrique}. 
\end{definition} 


\begin{example}[Suite de l'exemple 1.10]  
  \label{exe:45}
  Le polynôme  $p(x) = x^4 + x^3 + x^2 + x + 1  ∈ℤ_5$  est divisible par $x-1$, et
  \begin{displaymath}
    p(x)  = (x^3 + 2x^2 + 3x + 4)  (x-1). 
  \end{displaymath}
  De plus, $1$ est aussi une racine de $x^3 + 2x^2 + 3x + 4$. En fait 
  \begin{displaymath}
    x^4 + x^3 + x^2 + x + 1 = (x-1)^4,
  \end{displaymath}
  alors la multiplicité de la racine $1$ de $p(x)$ est $4$ parce que $(x-1)^4$ divise $p$ mais pas $(x-1)^5$, en tant que polynôme de degré 5 (théorème~\ref{thr:34}).   
\end{example}




\section{Le plus grand diviseur commun}
\label{sec:fact-de-polyn}

Soit $K$ un corps dans ce chapitre. 

\begin{theorem}
  \label{thr:54}
  Soient $f(x)$ et $g(x)$ deux polynômes sur $K$ non tous deux nuls et soit 
  \begin{displaymath}
    I = \{ u ⋅ f + v ⋅ g : u,v ∈ K[x]\}.
  \end{displaymath}
  Il existe un  polynôme $d(x)∈K[x]$ tel que
  \begin{equation}
    \label{eq:51}
    I = \{ h ⋅ d : h ∈ K[x]\}. 
  \end{equation}
\end{theorem}

\begin{proof}
  Remarquons que $I$ contient des polynômes non nuls (notamment $f$ ou $g$). Soit $d ∈ I \setminus \{0\}$ de degré minimal et $u',v' ∈ K[x]$ tels que
  \begin{displaymath}
    u' ⋅ f + v' ⋅g = d. 
  \end{displaymath}
  Soit $u⋅f + v⋅ g ∈ I$. La division avec reste donne $u⋅f + v⋅ g = qd +r$, avec $\deg(r) < \deg(d)$. Alors
  \begin{displaymath}
    r = (u - qu') ⋅ f + (v - qv') ⋅ g ∈ I
  \end{displaymath}
  et par minimalité de $d ∈ I \setminus\{0\}$, $r=0$. Ainsi il existe un $h∈K[x]$ tel que $h⋅d =  u⋅f + v⋅ g$. Il est clair que $h⋅d ∈I$ pour tous les $h ∈K[x]$ et l'assertion est démontrée.   
\end{proof}


\begin{definition}
  \label{def:54}
  Un polynôme $f(x) ∈ K[x] \setminus \{0\}$  dont le coefficient dominant est $1$ est appelé \emph{polynôme unitaire}. 
\end{definition}


\begin{definition}
  \label{def:55}
  Soient $f,g ∈ K[x]$ non tous deux nuls. Un \emph{diviseur commun} de $f$ et $g$  est un diviseur de $f$ et $g$. 
\end{definition}

\begin{theorem}
  \label{thr:55}
  Soient $f, g$ et $d$  comme dans le  théorème~\ref{thr:54}.
  \begin{enumerate}[i)]
  \item $d$ est un diviseur commun de $f$ et $g$. \label{item:24}
  \item Chaque diviseur commun de $f$ et $g$ est un diviseur de $d$. \label{item:25}
  \item Si $d$ est unitaire, alors $d$ est unique.  \label{item:26}
  \end{enumerate}
\end{theorem}

\begin{proof}
  L'assertion \ref{item:24}) suit du fait que $f,g ∈ I$ et de~\eqref{eq:51}. Soient $u,v ∈K[x]$  tels que
  $    d = u⋅f  + v ⋅ g $
  et soit $w$ un diviseur commun de $f$ et $g$. Alors il existe $f',g' ∈K[x]$ tels que  $f = f' w$ et $g = g' w$. Par conséquent
  \begin{displaymath}
    d = (u⋅f'  + v ⋅ g') w,
  \end{displaymath}
  ce que montre que $w \mid d$ et \ref{item:25}). 
  Soient $d$ et $d'$ deux polynômes unitaires satisfaisant~\eqref{eq:51}. \ref{item:24}) et \ref{item:25}) impliquent que $d \mid d'$ et $d' \mid d$.
  Alors il existe $z,z' ∈K[x]$ tel que $d = d' z'$ et $d' = dz$. Par suite, $d = d ⋅z ⋅ z'$. Le theorème~\ref{thr:34} implique que $z,z' ∈ K$. Et comme $d$ et $d'$ sont unitaires, $z=z'=1$, ce que démontre \ref{item:26}). 
\end{proof}

\begin{definition}
  \label{def:56}
  L'unique polynôme unitaire $d ∈ K[x]$ satisfaisant \eqref{eq:51} est appelé  le \emph{plus grand commun diviseur} de $f$ et $g$. Il est noté $\gcd(f,g)$ ou pgcd$(f,g)$.
\end{definition}

\section{L'algorithme d'Euclide}
\label{sec:lalg-de-eucl}



Pour calculer le plus grand diviseur commun de $f(x)$ et $g(x)$  on peut utiliser l'algorithme d'Euclide. Soient $f_0,f_1 ∈K[x]$ pas tous les deux nuls et $\deg(f_0) ≥ \deg(f_1)$. Si $f_1 = 0$, alors
\begin{displaymath}
\gcd(f_0,f_1) =   f_0. 
\end{displaymath}
Autrement, on applique la division avec reste
\begin{displaymath}
  f_0 = q_1 f_1 + f_2, 
\end{displaymath}
où $q_1,f_2 ∈K[x]$ et $ \deg(f_2)< \deg(f_1)$. Un polynôme  $d ∈K[x]$ est un diviseur commun de $f_0$ et $f_1$ si et seulement si $d$ est un diviseur commun de $f_1$ et $f_2$. L'algorithme d'Euclide est le procédé de calculer la suite $f_0,f_1,f_2,\dots,f_{k-1},f_k ∈K[x]$  où $\deg(f_{k-1})≥0$, $f_k=0$ et 
\begin{displaymath}
  f_{i-1} = q_i f_i + f_{i+1} 
\end{displaymath}
est le résultat de la division avec reste de $f_{i-1} $ par $f_i$. Le procédé se termine toujours car la suite des degrés $\{\deg(f_i)\}$ est entière et strictement décroissante (méthode de descente infinie de Fermat). Le dernier reste non nul $f_{k-1}$ est un multiple constant de $\gcd(f_0,f_1)$ : il suffit de diviser par le coefficient dominant pour le rendre unitaire.

%% maybe prove this fact: f_{k-1} is in I and divides both f_0 and f_1, so it divides d and d divides it. The unicity of {thr:55}iii) gives the result.

\begin{example}
  \label{exe:25-b}
  On calcule le plus grand diviseur commun de
$f_0 =  4 x^{6} + x^{4} + 2 x^{2} + 2 ∈ℤ_5[x]$  et 
$f_1 =  3 x^{4} + x^{3} + 2 x^{2} + 2 x + 2 ∈ℤ_5[x]$. 
  \begin{displaymath}
   q_1 =  3 x^{2} + 4 x + 2, \, 
   f_2 =  4 x^{3} + 4 x^{2} + 3 x + 3, 
 \end{displaymath}

 \begin{displaymath}
   q_2 =  2 x + 2, \, 
f_3 =  3 x^{2} + 1, 
\end{displaymath}

\begin{displaymath}
  q_3 =  3 x + 3, \, 
  f_4 =  0. 
\end{displaymath}
Alors tout diviseur commun de $f_0$ et $f_1$ divise $f_3 =  3 x^{2} + 1$ et $f_3$ est aussi un diviseur commun. On divise par $3$ (ou multiplie par $2$) et on obtient  $\gcd(f_0,f_1) = x^2 +2$. 

\end{example}

Le calcul des suites $f_i$ et $q_i$ donne aussi une représentation $\gcd(f_0,f_1) = u ⋅f_0 + v ⋅f_1$, $u,v ∈K[x]$. En effet


\begin{displaymath}
  \begin{pmatrix}
    f_{i} \\  f_{i+1} 
  \end{pmatrix}
  =
  \begin{pmatrix}
    0 & 1 \\
    1 & -q_i
  \end{pmatrix}
  \begin{pmatrix}
     f_{i-1} \\  f_{i}
   \end{pmatrix}
\end{displaymath}
et alors
\begin{displaymath}
  \begin{pmatrix}
    f_{k-1} \\ f_k
  \end{pmatrix} =
  \begin{pmatrix}
    0 & 1 \\
    1 & -q_{k-1}
  \end{pmatrix} \cdots
  \begin{pmatrix}
    0 & 1 \\
    1 & -q_{1}
  \end{pmatrix}
  \begin{pmatrix}
    f_0 \\ f_1
  \end{pmatrix}.
\end{displaymath}
\begin{example}
  \label{exe:26}
  On continue l'exemple~\ref{exe:25-b}.


  \begin{displaymath}
    \left(\begin{array}{rr}
3 x + 3 & x^{3} + 4 x^{2} + 2 x \\
x^{2} + 2 x + 2 & 2 x^{4} + 4 x^{2} + 3
\end{array}\right) = 
    \left(\begin{array}{rr}
0 & 1 \\
1 & 2 x + 2
\end{array}\right) \left(\begin{array}{rr}
0 & 1 \\
1 & 3 x + 3
\end{array}\right) \left(\begin{array}{rr}
0 & 1 \\
1 & 2 x^{2} + x + 3
                         \end{array}\right)                                 
                     \end{displaymath}
                     et
\begin{displaymath}
\left(\begin{array}{r}
3 x^{2} + 1 \\
0
\end{array}\right) = \left(\begin{array}{rr}
3 x + 3 & x^{3} + 4 x^{2} + 2 x \\
x^{2} + 2 x + 2 & 2 x^{4} + 4 x^{2} + 3
\end{array}\right)
\left(\begin{array}{r}
4 x^{6} + x^{4} + 2 x^{2} + 2 \\
3 x^{4} + x^{3} + 2 x^{2} + 2 x + 2
\end{array}\right).   
\end{displaymath}

\begin{displaymath}
 \left(\begin{array}{r}
x^{2} + 2 \\
0
\end{array}\right) =
\left(\begin{array}{rr}
x + 1 & 2 x^{3} + 3 x^{2} + 4 x \\
2 x^{2} + 4 x + 4 & 4 x^{4} + 3 x^{2} + 1
\end{array}\right)
\left(\begin{array}{r}
4 x^{6} + x^{4} + 2 x^{2} + 2 \\
3 x^{4} + x^{3} + 2 x^{2} + 2 x + 2
\end{array}\right)
\end{displaymath}

Alors

\begin{displaymath}
  x^{2} + 2 = \gcd(f_0,f_1) = (x + 1)f_0 + (2 x^{3} + 3 x^{2} + 4 x) f_1. 
\end{displaymath}
\end{example}



\section{Factorisation en irréductibles}
\label{sec:fact-en-irred}

On fixe un corps $K$ dans ce chapitre. 

\begin{definition}
  \label{def:53}
  Un polynôme $p(x) ∈ K[x] \setminus \{0\}$ est \emph{irréductible}, si
  \begin{enumerate}[i)]
  \item $\deg(p) ≥1$ et
  \item  si $p(x) = f(x) g(x)$ alors $\deg(f) = 0$ ou $\deg(g) = 0$. 
  \end{enumerate}
\end{definition}

\begin{example}
  \label{exe:46}
  \begin{enumerate}
  \item   Chaque polynôme linéaire $p(x) = ax + b ∈ K[x]$, $a ∈ K \setminus\{0\}$ est irréductible. En effet, si $p(x) = f(x) g(x)$ et $\deg(f)>0$ et $\deg(g)>0$, alors le théorème~\ref{thr:34} implique que $\deg(p) >1$.
  \item $x^2 +1 ∈ℝ[x]$ est irréductible. Autrement, il existe un polynôme linéaire $f(x) =  x - α ∈ ℝ[x]$ qui divise $f$ ce qui implique que $α$ est une racine de $x^2 +1$. Cependant, aucun nombre $α$ réel ne satisfait $α^2 = -1$.
  \end{enumerate}
\end{example}


Clairement, tout polynôme $f(x) ∈ K[x] \setminus \{0\}$    peut être factorisé comme
\begin{equation}
  \label{eq:33}
  f(x) = a ⋅ ∏_i p_i(x),
\end{equation}
dont les $p_i$ sont irréductibles est unitaires et $a ∈ K$. On va voir maintenant que cette factorisation est unique. 

\begin{theorem}
\label{thr:56}
  Soit $p(x) ∈K[x]$ irréductible et supposons que $p(x)$ divise un produit $f_1(x) \cdots f_k(x)$ de polynômes non nul. Alors $p(x)$ divise un polynôme $f_i(x)$. 
\end{theorem}


\begin{proof}
  Par récurrence il suffit de démontrer l'assertion pour $k=2$. Ainsi, supposons $p \mid fg$, $f,g ∈K[x] \setminus \{0\}$. Si $p$ ne divise pas $f$, alors $\gcd(p,f) = 1$ car les seuls diviseurs de $p$ sont des multiples constants de $1$ et lui-même. Soient donc $u,v ∈ K[x]$ t.q.  $up + vf = 1$. Alors $upg + vfg = g$, et donc $p \mid g$. 
\end{proof}


\begin{theorem}
  \label{thr:57}
  La factorisation~\eqref{eq:33} est unique à l'ordre près des $p_i$. 
\end{theorem}

\begin{proof}
  Pour une  factorisation $f(x) = a ∏_i q_i(x)$, où les $q_i$ sont irréductibles et unitaires on utilise le théorème~\ref{thr:56} pour déduire qu'il  existe $j$ tel que  $p_1 \mid q_j$. Comme $p_1$ et $q_j$ sont  irréductibles et unitaires, $p_1 = q_j$. En divisant par $p_1$, l'assertion suit par récurrence. 
\end{proof}


\begin{corollary}
  \label{co:11}
  Soient $f(x) ∈ K[x] \setminus \{0\}$ et $α_1,\dots, α_ℓ$ des racines de $f$ de multiplicité $k_1,\dots,k_ℓ$ respectivement. Alors il existe $g(x)∈ K[x]$ tel que  
  \begin{displaymath}
    f(x) = g(x) ⋅ ∏_{i=1}^ℓ (x - α_i)^{k_i} .
  \end{displaymath}
\end{corollary}

\begin{exercise}
  Démontre le Corollaire~\ref{co:11}. 
\end{exercise}


%%% Local Variables:
%%% mode: latex
%%% TeX-master: "notes"
%%% End:

\chapter{Valeurs propres}
\label{cha:valeurs-propres-et}


\section{Valeurs propres et vecteurs propres}
\label{sec:valeurs-propres-et}

\begin{definition}
  \label{def:16}
  Soit $V$ un espace vectoriel sur un corps $K$ et $f \colon V ⟶V$ un endomorphisme. Un \emph{vecteur propre} de $f$  associé à la \emph{valeur propre} $λ ∈K$ est un vecteur $v ≠ 0$ de $V$ tel que $f(v) = λ\,v$.
\end{definition}

\begin{example}
  \label{exe:23}
  Soit $f:V⟶V$, l'endomorphisme $f(v) = 0$ pour tous $v ∈V$. Alors tous $0≠v ∈V$ est un vecteur propre associé à $λ=0$. 
\end{example}


\begin{lemma}
  \label{lem:4}
  Soit $B = \{v_1,\dots,v_n\}$ une base de $V$ et $A ∈ K^{n×n}$ la matrice de l'endomorphisme $f : V ⟶V$ relatif à $B$. La matrice $A$ est une matrice diagonale, c'est à dire $A$ est de la forme
  \begin{displaymath}
    A =
    \begin{pmatrix}
      λ_1  \\
         & \ddots \\
         & & λ_n
    \end{pmatrix},
  \end{displaymath}
si et seulement si $v_i$ est un vecteur propre associé à la valeur propre $λ_i$ pour tout $i=1,\dots,n$.
\end{lemma}

\begin{proof}
  Pour $v ∈V$ soit $[v]_B ∈K^n$ le vecteur des coordonnées de $v$ relatif à $B$. L'application $φ: V ⟶ K^n$, $φ(x) = [x]_B$ est un isomorphisme.  On a $[f(v_i)]_B = A \,  [v_i]_B$ pour $i=1,\dots,n$.
Supposons que $\{v_1,\dots,v_n\}$ est une base de vecteurs propres. 
  Des que $[v_i]_B = e_i$, et $f(v_i) = λ_i v_i$ alors
    \begin{displaymath}
      λ_i ⋅ e_i = A \, e_i, \text{ pour } i ∈\{1,\dots,n\},
    \end{displaymath}
    c.à.d. que $A$ est une matrice diagonale.

La direction d'inverse est analogue. 
\end{proof}


\begin{definition}
  \label{def:39}
  Un endomorphisme $f :V ⟶ V$ pour lequel existe une base de $V$ composée de vecteurs propres est \emph{diagonalisable}. 
\end{definition}


\begin{definition}
  \label{def:40}
  Soit $A ∈ K^{n ×n}$ une matrice.
  Un \emph{vecteur propre} de $A$ associé à la \emph{valeur propre} $λ ∈K$ est un vecteur propre de l'endormophisme  $f(x) = Ax$ de $K^n$.
\end{definition}




\begin{example}
 \begin{enumerate}
 \item Soit $A = \left(\begin{array}{cc}
1 & 0 \\
0 & 0
\end{array}
\right) \in ℝ^{2 ×2}$. Alors
\begin{itemize}
 \item $v_1 = \left( \begin{array}{c} 1 \\ 0 \end{array} \right)$ est un vecteur propre associ\'e \`a la valeur propre $\lambda_1 = 1$,
 \item $v_2 = \left( \begin{array}{c} 0 \\ 1 \end{array} \right)$ est un vecteur propre associ\'e \`a la valeur propre $\lambda_2 = 0$,
 \item $v_3 = \left( \begin{array}{c} 1 \\ 1 \end{array} \right)$ n'est pas un vecteur propre.
\end{itemize}
\item Soit $A = \left(\begin{array}{cc}
\cos \phi & \sin \phi \\
-\sin\phi & \cos \phi
\end{array}
\right) \in ℝ^{2 ×2}$ pour $\phi \in ℝ$. \begin{itemize}
\item
Si $\phi\not=k\pi$, $k\inℕ$, alors $A$ n'a pas de valeur propre (r\'eelle).
\item Si $\phi = (2k+1)\pi$, $k\in ℕ$, alors $A =
\left(\begin{array}{cc}
-1 & 0 \\
0 & -1
\end{array}
\right)$ a une valeur propre $\lambda = -1$ et tous les vecteurs non-nuls $x\in ℝ^2$ sont des vecteurs propres associ\'es \`a $\lambda$.
\item Si $\phi = 2k \pi$, $k\in ℕ$, alors $A =
\left(\begin{array}{cc}
1 & 0 \\
0 & 1
\end{array}
\right)$ a une valeur propre $\lambda = 1$ et encore tous les vecteurs non-nuls $x\in ℝ^2$ sont des vecteurs propres associ\'es \`a $\lambda$.
\end{itemize}
On va voir que si on consid\`ere $A$ comme une matrice complexe, alors on a toujours les valeurs propres $\cos \phi + \iunit \sin \phi$ et $\cos \phi - \iunit \sin \phi$.
\end{enumerate}
\end{example}

\begin{lemma}
  \label{lem:21}
  Un vecteur $v ∈ V ⧹\{0\}$ est un vecteur propre de $f:V ⟶V$  associé à la valeur propre $λ ∈ K$ si et seulement si $v ∈ \ker(f - λ ⋅ \Id)$.
\end{lemma}



Rappel: L'endomophisme $\Id : V ⟶V$ défini comme $\Id(v) = v$ pour tous $v∈ V$  est appelé l'\emph{identité}.

\begin{definition}
  \label{def:1}
  Soit $λ$ une valeur propre de l'endomorphisme $f:V ⟶V$. Le sous espace $E_λ$ de $V$, défini comme
  \begin{displaymath}
    E_λ = \ker(f - λ ⋅ \Id)
  \end{displaymath}
  est l'espace propre de $f$ associé à $λ$. La dimension de $E_λ$ est la multiplicité géométrique de $λ$.
\end{definition}

\begin{lemma}
  \label{elem:1}
  Soient $v_1,\dots,v_r ∈V$ des vecteurs propres, associés aux valeurs propres $λ_1,\dots,λ_r$ distinctes (c'est à dire $λ_i ≠ λ_j$ pour $i≠j$), alors  $\{v_1,\dots,v_r\}$ est un ensemble libre.
\end{lemma}

\begin{proof}
  Supposons que le théorème soit faux et soit $r≥1$ minimal, tel qu'il existent des vecteurs propres $v_1,\dots,v_r ∈V$  associés aux valeurs propres $λ_1,\dots,λ_r$ qui sont linéairement dépendants. Des que $v_i ≠0$ alors $r>1$.
  Considérons une combinaison linéaire non triviale
  \begin{equation}
    \label{eq:35}
    α_1 v_1 + \cdots + α_r v_r = 0.
  \end{equation}
  Puisque \eqref{eq:35} est un contre exemple minimal, on a $α_i≠0$ pour tous $i$. Nous pouvons supposer que $λ_r ≠0$. Autrement, on réarrange~\eqref{eq:35}. 

  Si on applique $f$ à l'expression \eqref{eq:35} on obtient
  \begin{displaymath}
    λ_1 α_1 v_1 + \cdots + λ_rα_r v_r = 0
  \end{displaymath}
  et en divisant par $λ_r$
  \begin{equation}
    \label{eq:37}
    (λ_1/λ_r) α_1 v_1 + \cdots + α_r v_r = 0.
  \end{equation}

  On soustrait \eqref{eq:37} de \eqref{eq:35} et on obtient

  \begin{displaymath}
    (1- λ_1/λ_n)α_1 v_1 + \cdots + (1- λ_{r-1}/λ_r)α_{r-1} v_{r-1}
  \end{displaymath} Ceci est en contradiction avec la minimalité de $r$.
\end{proof}


\begin{corollary}
  \label{eco:1}
   Soit $f :V ⟶V$ un endomorphisme d'un espace vectoriel $V$ sur $K$ de dimension $n ∈ ℕ$ et soient $λ_1,\dots,λ_r$
  les valeurs propres différentes de $f$
  et soient $n_1,\dots,n_r$
  leurs multiplicités géométriques respectives. Soient
  $B_i= \{v_1^{(i)},\dots,v_{n_i}^{(i)}\}$
  des bases de $E_{λ_i}$ 
  respectivement, pour $i=1,\dots,r$. Alors
  \begin{displaymath}
    \{ v_1^{(1)},\dots,v_{n_1}^{(1)},v_1^{(2)},\dots,v_{n_2}^{(2)},\cdots,v_1^{(r)},\dots,v_{n_r}^{(r)} \}
  \end{displaymath}
est un ensemble libre. L'application $f$ est diagonalisable si et seulement si
\begin{displaymath}
  n_1 + \cdots + n_r =n.
\end{displaymath}
\end{corollary}
\begin{proof}
  Soit la combinaison linéaire 
  \begin{displaymath}
    ∑_{i=1}^r ∑_{j=1}^{n_i} α_{ij} v^{(i)}_j = 0.
  \end{displaymath}
  Remarquons que les vecteurs $ ∑_{j=1}^{n_i} α_{ij} v^{(i)}_j $ appartiennent à $E_{\lambda_i}$ pour tout $i$. Autrement dit, ce sont des vecteurs propres associés à des valeurs propres distinctes et dont la somme est nulle. Le lemme~\ref{elem:1} garantit donc que tous les vecteurs soient nuls. Par suite, $ ∑_{j=1}^{n_i} α_{ij} v^{(i)}_j $ et les $α_{ij}$ sont tous égaux à zéro car les $v_1^{(i)},\dots,v_{n_i}^{(i)}$ sont linéairement indépendants. Ça démontre que
   \begin{displaymath}
    \{ v_1^{(1)},\dots,v_{n_1}^{(1)},v_1^{(2)},\dots,v_{n_2}^{(2)},\cdots,v_1^{(r)},\dots,v_{n_r}^{(r)} \}	
  \end{displaymath} est un ensemble libre. En plus, si $n_1+\cdots+n_r=n$, $f$ est diagonalisable par définition car l'ensemble forme une base de $K^n$.

  À l'inverse, si $f$ est diagonalisable, et si $m_i$ dénote le nombre vecteurs propres en $E_{λ_i}$ dans la base consistant de vecteurs propres, alors $m_i ≤ n_i$, et on a
  \begin{displaymath}
    n = m_1 + \cdots + m_r ≤ n_1+ \cdots + n_r ≤n,
  \end{displaymath}
  et donc $n_1+\cdots + n_r =n$.
\end{proof}




Voici une marche à suivre afin de déterminer si $f:V ⟶V$ est diagonalisable ou non.

\begin{enumerate}
\item Déterminer les différentes $λ_1,\dots,λ_r ∈K$ tel que $\ker(f - λ \Id) ≠ \{ 0 \}$
\item Pour chaque $λ_i$ calculer une base $\{v_1^{(i)},\dots,v_{n_i}^{(i)}\}$ de $E_{λ_i}$.
\item $f$ est diagonalisable si et seulement si  $n_1+\cdots+n_r =n$.
\end{enumerate}





\subsection*{Exercices}

\begin{enumerate}
\item Une matrice $A ∈ K^{n ×n}$ est appelée \emph{diagonalisable},  si endomorphisme $φ:K^n ⟶K^n$ défini comme $φ(x) = Ax$ est diagonalisable. Démontrer que $A$ est diagonalisable, si et seulement s'il existe $U ∈ K^{n×n}$ inversible tel que $U^{-1} A U$ est une matrice diagonale. \label{item:18}
\end{enumerate}

\section{Le polynôme caractéristique}
\label{sec:le-polyn-caract}

Durant ce chapitre nous allons étudier les endomorphismes $f:V⟶V$ d'un espace vectoriel de dimension fini $n ∈ ℕ$.  Si  $B = \{v_1,\dots,v_n\}$ est une base de $V$, on a
\begin{displaymath}
  f(x) = \phi_B^{-1} (A_B \phi_B(x)),
\end{displaymath}
où $\phi_B$ est l'ismomorphisme $\phi_B \colon V \longrightarrow K^n$, $\phi_B(x) = [x]_B$ sont les coordonnées de $x$ par rapport à la base $B$. On a le diagramme suivant
\begin{displaymath}
  {
  \begin{CD}
    V     @>f>>  V\\
    @VV \phi_B V        @VV \phi_B V\\
    K^n     @>A \cdot x>>  K^n
  \end{CD}}
\end{displaymath}
Les colonnes de la matrice $A_B$ sont les coordonnées de $f(v_1),\dots,f(v_n)$ dans la base $B$. Si $B'$ est une autre base de $V$ on a
\begin{displaymath}
  [x]_{B'} = P_{BB'}[x]_B,
\end{displaymath}
où $P_{BB'}$ est la matrice de changement de base de $B$ en $B'$. Comme on a
\begin{displaymath}
  [f(v)]_{B'} = A_{B'} [v]_{B'} = A_{B'} P_{BB'}[v]_B
\end{displaymath}
et
\begin{displaymath}
  [f(v)]_{B'} =  P_{BB'}[f(v)]_B,
\end{displaymath}
on trouve
\begin{displaymath}
  [f(v)]_B =  P_{BB'}^{-1}A_{B'} P_{BB'}[v]_B \,\,\,\text{ pour tous } v ∈V.
\end{displaymath}
Et ça implique 
\begin{equation}
  \label{eq:36}
  A_{B} =  P_{BB'}^{-1} A_{B'}  P_{BB'}
\end{equation}
En particulier,
\begin{displaymath}
  \det(A_{B'}) = \det(A_B) 
\end{displaymath}
ce qui laisse nous définir le \emph{déterminant d'un endomorphisme} $f$ comme $\det(f)= \det(A_B)$. 



Clairement, $λ$ est une valeur propre de $f$ si et seulement si $λ$ est une valeur propre de $A_B$ et c'est le cas si et seulement si
\begin{equation}
  \label{eq:30}
  \det(A_B - λ I_n) = 0.
\end{equation}

Rappelons la formule de Leibniz pour le déterminant d'une matrice $B ∈ K^{n ×n}$
\begin{equation}
  \label{eq:31}
  \det(B)  = ∑_{π ∈S_n} \sign(π) ∏_{i=1}^n b_{iπ(i)}
\end{equation}
et si on regroupe les puissances de $λ$, on a
\begin{equation}
  \label{eq:32}
  \det(A_B - λI_n) = a_n λ^n + a_{n-1} λ^{n-1}+ \dots + a_1 λ+ a_0
\end{equation}
où $a_n,\dots,a_0 ∈K$. 

\begin{definition}
  \label{def:50}
  Le polynôme $\det(A_B - λI_n) ∈ K[λ]$ est le \emph{polynôme caractéristique} de $f$.   
\end{definition}

Remarquons que $\det(A_B) = \det(A - 0 ⋅ I_n)$, d'où $a_0 = \det(A_B)$.
L'expression \eqref{eq:32} est un polynôme avec indéterminée $λ$ et comme polynôme formel, est défini par la formule de Leibniz
\begin{displaymath}
p_A(λ) = \det(A - λ I_n) = ∑_{π ∈S_n} \sign(π) ∏_{i=1}^n (A - λ I_n)_{i,π(i)}.
\end{displaymath}
En tant que somme des polynômes $ \sign(π) ∏_{i=1}^n (A - λ I_n)_{i,π(i)}$, son degré est au plus $n$. Considérons la permutation triviale $\pi = \Id$ donnant le produit de degré $n$
\begin{displaymath}
  \sign(\Id) ∏_{i=1}^n (A - λ I_n)_{i\Id(i)} =  ∏_{i=1}^n (A_{ii} - λ),
\end{displaymath}
et en remarquons que toutes les autres permutations aboutissent à un produit au degré inférieur à $n-2$. Cela signifie en particulier que $a_n = (-1)^n$, et donc que le polynôme caractéristique est de degré $n$.

\begin{lemma}
  \label{lem:22}
Soit $p_A(λ) = a_0 + a_1 λ + \cdots + a_n λ^n$ le polynôme caractéristique de la matrice $A ∈ K^{n ×n}$. Alors, $a_0 = \det(A)$ et $a_n = (-1)^n$.
\end{lemma}


\begin{corollary}
  \label{co:10}
  Soit $V ≠ \{0\}$ un espace vectoriel de dimension fini sur $K = ℂ$, et $f: V → V$ un endomorphisme. Alors $f$ possède une valeur propre. 
\end{corollary}
\begin{proof}
  Soit $f(λ) ∈ ℂ[λ]$ le polynôme caractéristique de $f$ et $n$ la dimension de $V$. Le degré de $f$ est égal à $n≥1$, et donc $p(x)$ possède une racine $λ^* ∈ℂ$ (théorème fondamental de l'algèbre). Cette racine $λ^*$  est une valeur propre de $f$. 
\end{proof}


\begin{remark}
  \label{rem:6}
  Pour deux bases $B$ et $B'$, comme on a $A_{B} =  P_{BB'}^{-1} A_{B'}  P_{BB'}$, alors
  \begin{eqnarray*}
    \det(A_B - λI_n) & = & \det(P_{BB'}^{-1} A_{B'}  P_{BB'} - λP_{BB'}^{-1}I_n  P_{BB'}) \\
                     & = & \det(P_{BB'}^{-1}) \det(A_{B' }- λI_n) \det(P_{BB'}) \\
     & = & \det(A_{B' }- λI_n).
  \end{eqnarray*}
  La définition~\ref{def:50} ne dépend ainsi pas de la base choisie et a donc un sens.
\end{remark}




\begin{definition}
  \label{def:57}
 
  Soit  $λ ∈K$ une valeur propre de l'endomorphisme  $f: V ⟶V$. La \emph{multiplicité algébrique} de $λ$ est la multiplicité de $λ$ comme racine de $\det(f - λ \Id)$. 
\end{definition}


  \begin{proposition}
    \label{prop:6}
    Soit $f: V → V$ un endomorphisme et soit $λ ∈K$ une valeur propre de $f$. La multiplicité géométrique de $λ$ est au plus la multiplicité algébrique de $λ$. 
  \end{proposition}

  \begin{proof}
    Soit $m$ la multiplicité géométrique de $λ$ et soit  $\{v_1,\dots,v_m\}$  une base de $E_{λ}$. On la complète en une base
    \begin{displaymath}
B =     \{ v_1,\dots,v_m, w_1,\dots,w_{n-m}\} 
    \end{displaymath}
    de $V$.

    La matrice $A_B$ de l'endomorphisme $f$ dans la base $B$ est alors de la forme
    \begin{displaymath}
      A_B =
      \begin{pmatrix}
        λ I_m & C \\
        0 & D
      \end{pmatrix}
    \end{displaymath}
    où $C ∈ K^{m × n-m}$ et $D ∈ K^{(n-m) ×(n-m)}$. En effet, on a par définition $A_B[v_i]_{B} = [f(v_i)]_{B}$, et par conséquent $A_B e_i = \lambda e_i$, où $e_i$ est le $i$-ème vecteur canonique de dimension $n$.
    
    Lorsqu'on développe le déterminant d'une matrice en blocs comme $A_B$ grâce à la formule de Leibniz, seules les permutations envoyant $\{1, \dots, m\}$ et $\{m+1, \dots, n\}$ sur eux-même donnent un produit non nul. On peut alors diviser la somme en deux pour obtenir que le déterminant est exactement le produit des déterminants des blocs diagonaux.
    Le polynôme caractéristique $p(x) ∈ K[x]$ de $f$ est alors
    \begin{eqnarray*} p(x) & = &  
      \det   \begin{pmatrix}
        (λ -x) I_m & C \\
        0 & D-xI_{n-m}
      \end{pmatrix} \\
          & = & (λ -x)^m \det \left(D-xI_{n-m}\right). 
    \end{eqnarray*}
    La multiplicité algébrique de $λ$ est donc au moins $m$. 
  \end{proof}



\begin{theorem}[Théorème de diagonalisation]
  \label{thr:45}
  Soit $V$ un espace vectoriel sur $K$ de dimension $n$, $f:V⟶V$  un endomorphisme et $λ_1,\dots,λ_r ∈ K$ les valeurs propres distinctes de $f$.    Alors $f$ est diagonalisable si et seulement si 
  \begin{enumerate}[i)]
  \item le polynôme caractéristique $p_f(x)$ de $f$ décompose en facteurs linéaires, c'est-à-dire, \label{item:19} 
    \begin{displaymath}
      p_f(x) = (-1)^n ∏_{i=1}^r (x - λ_i )^{a_i}
    \end{displaymath}
    où $a_i$ est la multiplicité algébrique de $λ_i∈K$ pour tous $i$. 
    \item  $\dim(E_{λ_i}) = a_i$, pour tous $i=1,\dots,r$. C'est à dire, les multiplicités algébriques et géométriques sont les mêmes. \label{item:20}
  \end{enumerate}
\end{theorem}

\begin{proof}
  Supposons $f$ diagonalisable. Soit $B$ une base composée de vecteurs propres de $f$ et $A$ la matrice de $f$ associée à la base $B$. Le lemme~\ref{lem:4} implique que $A$ est diagonale et alors  $p_f(x) = \det(A - x \Id) = (-1)^n ∏_{i=1}^r (x - λ_i)^{a_i}$. La dimension de $E_{λ_i}$ est celle du noyau $\ker(A - λ_i I_n)$. Clairement $\dim(\ker(A - λ_i I_n)) = a_i$, et on a alors montré \ref{item:19}) et \ref{item:20}).


  Supposons maintenant que \ref{item:19}) et \ref{item:20}) tiennent. Soient $g_i$ les multiplicités géométriques des valeurs propres $λ_i$, $i=1,\dots,r$. Comme on a
  \begin{displaymath}
    \deg((-1)^n∏_{i=1}^r(x-λ_i)^{a_i}) =n,
  \end{displaymath}
  alors $g_1+ \cdots + g_r = n$ et $f$ est diagonalisable grâce au corollaire~\ref{eco:1}. 
\end{proof}



\begin{example}
  \label{exe:48}
  Le polynôme caractéristique de la matrice 
  \begin{displaymath}
    A =
    \begin{pmatrix}
      1 & 0 \\
      1 & 1 
    \end{pmatrix}
  \end{displaymath}
  est $(x - 1)^2$. La multiplicité géométrique de $λ = 1$ est $1$ et la multiplicité algébrique est $2$. La matrice n'est pas diagonalisable. 
\end{example}

\begin{example}
  \label{exe:30}
  Soit $f: ℝ^3 → ℝ^3$ donnée par
  \begin{displaymath}
      f(x) = Ax, \, \text{ où }      A =  \begin{pmatrix}
        0 & -1 & 1 \\
        -3 & -2 & 3 \\
        -2 & -2 & 3
      \end{pmatrix}      
    \end{displaymath}
    Pour la base canonique  $B = \{e_1,e_2,e_3\}$ de $ℝ^3$, on a $A_B =A$. Le polynôme caractéristique de $f$ est
    \begin{displaymath}
      p(x) = -x^3 + x^2 + x -1 = - (x-1)^2 (x+1). 
    \end{displaymath}
    Les valeurs propres de $f$ sont $λ_1 = 1$ et $λ_2 = -1$ et
    \begin{displaymath}
      \left\{
          \begin{pmatrix}
            1 \\ 0 \\1
          \end{pmatrix},
          \begin{pmatrix}
            0 \\ 1 \\ 1
          \end{pmatrix} \right\} \text{ et }  \left\{
          \begin{pmatrix}
            1 \\ 3 \\2
          \end{pmatrix} \right\}
    \end{displaymath}
    sont des bases de $E_{λ_1}$ et  $E_{λ_2}$  respectivement. Alors $f$ est diagonalisable et pour la base
    \begin{displaymath}
      B' =  \left\{
          \begin{pmatrix}
            1 \\ 0 \\1
          \end{pmatrix},
          \begin{pmatrix}
            0 \\ 1 \\ 1
          \end{pmatrix}, 
          \begin{pmatrix}
            1 \\ 3 \\2
          \end{pmatrix} \right\}   
    \end{displaymath}
    on a
    \begin{displaymath}
      A_{B'} =
      \begin{pmatrix}
        1 & 0 & 0 \\
        0 & 1 & 0 \\
        0 & 0 & -1 
      \end{pmatrix}
    \end{displaymath}
    et
    \begin{displaymath}
      P_{BB'} = \left(\begin{matrix}1 & 0 & 1\\0 & 1 & 3\\1 & 1 & 2\end{matrix}\right)^{-1}.
    \end{displaymath}
    On peut vérifier qu'on a bien
\begin{equation}
  A_{B} =  P_{BB'}^{-1} A_{B'}  P_{BB'}. 
\end{equation}

    
  \end{example}

  
\subsection*{Exercices}

\begin{enumerate}
\item Donner un exemple d'un corps fini $K$ et deux polynomes $p(x) ≠q(x) ∈ K[x]$ tel que $f_p=f_q$.
\end{enumerate}

\section{Matrices semblables}
\label{sec:matrices-semblables}

\begin{definition}
  \label{def:42}
  Deux matrices $A,B ∈K^{n×n}$ sont \emph{semblables}, s'il existe une matrice inversible $P ∈ K^{n ×n}$ tel que $A = P^{-1} ⋅B ⋅P$.  
\end{definition}
L'équation~\eqref{eq:36} montre que les matrices $A_B$ et $A_{B'}$ d'un endomorphisme $f:V⟶V$ sont semblables, pour $B$ et $B'$ deux bases de $V$.

\begin{definition}
  \label{def:43}
  L'ensemble des valeurs propres d'une matrice $A ∈ K^{n ×n}$ (resp. d'un endomorphisme $f:V ⟶V$) est appelé le \emph{spectre} de $A$ (resp. de $f$), noté $\spec(A)$ (resp. $\spec(f)$). 
\end{definition}

\begin{theorem}
  \label{thr:46}
  Soit $A ∈ K^{n ×n}$ une matrice et $P ∈ K^{n ×n}$ une matrice inversible.
  \begin{enumerate}[i)]
  \item Le spectre de $A$ et celui de $P^{-1}AP$ sont les mêmes.
  \item $v ∈ K^n$ est un vecteur propre de $A$ si et seulement si $P^{-1} v$  est un vecteur propre de $P^{-1}AP$.
  \item Les polynômes caractéristiques $p_A(x)$ et $p_{P^{-1}AP}(x)$ sont identiques. 
  \end{enumerate}
\end{theorem}


\section{Théorème de Hamilton-Cayley}
\label{sec:theoreme-de-hamilton}


Soit $A ∈ K^{n ×n}$ et $p(x) = a_0 + a_1x + \cdots + a_n x^n ∈ K[x] ⧹\{0\}$ un polynôme. On peut évaluer le polynôme en la matrice $A$ comme suit :
\begin{displaymath}
  p(A) = a_0 ⋅I_n + a_1 A + \cdots + a_n A^n ∈ K^{n ×n}. 
\end{displaymath}
Maintenant, soit $p_A(x)$ le polynôme caractéristique de $A$ et $v$ un vecteur propre de $A$ associé à la valeur propre $λ$. On voit
\begin{displaymath}
  p_A(A) ⋅v = a_0 v + a_1 λv + \cdots + a_n λ^n v = p_A(λ)v = 0 ⋅v = 0. 
\end{displaymath}
Dans le cas où $A$ est diagonalisable, il existe une base de vecteurs propres $\{ v_1,\dots,v_n \}$. On a alors $p_A(A) ⋅v_i = 0$ pour tous $i$, et donc $p_A(A) =0$. 

\begin{theorem}[Hamilton-Cayley] 
  \label{thr:47}
  Soit $A ∈ K^{n ×n}$ et $p_A(λ)$ le polynôme caractéristique de $A$, alors
  \begin{displaymath}
    p_A(A) =0.
  \end{displaymath}
\end{theorem}
\begin{proof}
 On écrit
  \begin{displaymath}
    \det(A - λI_n) I_n = \cof(A - λI_n)^T (A - λI_n),
  \end{displaymath}
  où $\cof(A - λI_n)$ est la comatrice de $(A - λI_n)$.
  
  En regroupant les coefficients de $λ^i$ dans $\cof(A - λI_n)^T$ on obtient
  \begin{displaymath}
    \cof(A - λI_n)^T = ∑_{i=0}^{n-1} λ^i B_i
  \end{displaymath}
  avec certaines matrices $B_i ∈K^{n ×n}$. Alors
  \begin{displaymath}
    a_0 I_n + a_1 λ I_n + \cdots + a_nλ^n I_n  = B_0A + ∑_{i=1}^{n-1}λ^i (B_iA - B_{i-1}) - λ^n B_{n-1},
  \end{displaymath}
  où $p_A(λ) = a_0+ \cdots + a_n λ^n$. Ceci implique
  \begin{equation}
    \label{eq:38}
    \begin{array}{rcl}
      a_0 I_n & = &  B_0A \\
      a_iI_n & = &  B_iA - B_{i-1} \text{ pour } i ∈  \{1,\dots,n-1\}\\
      a_n I_n & = & - B_{n-1}
    \end{array}
  \end{equation}
  ce que sont des équations de matrices en $K^{n ×n}$. Si on multiplie les matrices indicées par $i$ à droite par $A^i$ et qu'on somme les équations, on obtient $p_A(A)$ à gauche du signe d'égalité. À droite, on obtient une somme téléscopique égale à la matrice nulle.
\end{proof}





\begin{example}
Le polyn\^ome caract\'eristique de $A = \left( \begin{array}{cc} 1 & 1 \\ 0 & 2
\end{array} \right)$ est $p_A(t) = (1-t)(2-t)$. On a 
\[
 p_A(A) = (I_n-A)(2I_n-A) = 0 
\]
Pour la matrices $A = \left( \begin{array}{cc} 2 & 0 \\ 0 & 2 \end{array}
\right)$, on a bien sûr que $p_A(A) = 0$ pour $p_A(t) = (2-t)^2$. Cependant, il existe un polyn\^ome unitaire de degr\'e strictement inférieur tel que $q(A) = 0$, à savoir $q(t) = t-2$. 

\begin{definition}
  \label{def:44}
  Le polyn\^ome unitaire de degr\'e minimal parmi ceux qui annulent $A$ est appelé \emph{polyn\^ome minimal} de $A$.
\end{definition}

%Nous examinerons de plus près le polyn{\^o}me minimal du chapitre~\ref{sec:polyn-les-lalg}.
\end{example}

Les résultats suivants donnent des utilisations typiques du théorème~\ref{thr:47}
\begin{corollary} Soit $A \in K^{n ×n}$.
\begin{enumerate}
\item[(i)] Toute puissance $A^k$ avec $k∈ℕ$ peut s'\'ecrire comme une combinaison lin\'eaire des puissances $I,A,A^2,\ldots,A^{n-1}$.
\item[(ii)] Si $A$ est inversible, alors l'inverse $A^{-1}$ peut s'\'ecrire comme une combinaison lin\'eaire des puissances $I,A,A^2,\ldots,A^{n-1}$.
\end{enumerate}
\end{corollary}

\begin{proof}
 (i). Trivialement, l'assertion est vraie pour $k = 0,1,\ldots,n-1$.
On montre le cas $k = n$. Par le théorème~\ref{thr:47}:
\[
 0 = p_A(A) = \alpha_0 I + \alpha_1 A  \cdots + \alpha_{n-1} A^{n-1} + A^n \quad
\Rightarrow \quad A^n = 
-\alpha_0 I - \alpha_1 A  \cdots - \alpha_{n-1} A^{n-1}.
\]
De fa\c{c}on similaire, on montre le cas $k>n$ par r\'ecurrence, utilisant $0 = A^{k-n} p_A(A)$.

(ii). Si $A$ est inversible alors $\alpha_0 = \det(A)$ est inversible. De $0 = p_A(A)$ on obtient que
\[
  I = -\frac{\alpha_1}{\alpha_0} A  \cdots -\frac{\alpha_{n-1}}{\alpha_0}
A^{n-1} -\frac{1}{\alpha_0} A^n = A \Big( -\frac{\alpha_1}{\alpha_0} I  \cdots
-\frac{\alpha_{n-1}}{\alpha_0} A^{n-2} -\frac{1}{\alpha_0} A^{n-1} \Big)
\]
et donc $A^{-1} = -\frac{\alpha_1}{\alpha_0} I  \cdots
-\frac{\alpha_{n-1}}{\alpha_0} A^{n-2} -\frac{1}{\alpha_0} A^{n-1}$.
\end{proof}



%%% Local Variables:
%%% mode: latex
%%% TeX-master: "notes"
%%% End:
 
\chapter{Bilinéarité   et produits scalaires}  
\label{cha:prod-scal-et}


\noindent 
Dans le chapitre~\ref{cha:valeurs-propres-et} nous avons vu le concept de similarité de deux matrices $A , B ∈K ^{n ×n}$. Les matrices $A$ et $B$ sont semblables, s'il existe une matrice  inversible $P ∈K^{n ×n}$ tel que
\begin{displaymath}
  A = P^{-1} ⋅B ⋅P. 
\end{displaymath}
Le Théorème~\ref{thr:45} explique, quand une matrice $A ∈ K^{n ×n}$ est semblable à une matrice diagonale. C'est la cas si et seulement si le polynôme caractéristique de $A$ décompose en facteurs linéaires et le multiplicités algébriques et géométriques   de ces racines sont les mêmes.

Dans ce chapitre on ce concentre sur une autre relation d'équivalence.  Deux matrices $A,B \in K^{n\times n}$ sont dites \emph{congruentes} s'il existe une matrice $P \in K^{n \times n}$ inversible telle que 
\begin{displaymath}
  A = P^T B P.
\end{displaymath}
Nous écrivons dans ce cas $A \cong B$ et on se demande quand est-ce que une matrice $A ∈ K^{n ×n}$ est congruente à une matrice diagonale. $A \cong \diag(a_1,\dots,a_n)$. Dans ce cas, on a certainement
\begin{displaymath}
  A^T = A, 
\end{displaymath}
c.à.d. que $A$ est une matrice \emph{symétrique}.

Considérons un exemple. Soit $A ∈ ℤ_3^{ n × n}$ la matrice  
\begin{displaymath}
  A =
  \begin{pmatrix}
    0 & 1 \\
    1 & 0 
  \end{pmatrix}.
\end{displaymath}
L'effet de la multiplication de $A$ avec
\begin{displaymath}
  P_1 =  \begin{pmatrix}
    1 & 0\\
    1 & 1 
  \end{pmatrix} 
\end{displaymath}
à droite est que la nouvelle première colonne de $A$ est la somme des deux colonnes. L'effet de la multiplication de $A$ avec $P^T$ à gauche est que la nouvelle première ligne de $A$ est la somme des deux lignes. Alors on obtient
\begin{displaymath}
  P_1^T A  P_1 =
  \begin{pmatrix}
    2 & 1 \\
    1 & 0 
  \end{pmatrix}.
\end{displaymath}
D'ici on peut additionner la première colonne de $A$ sur la deuxième et additionner la  première ligne de $A$ sur la deuxième et on obtient avec 
\begin{displaymath}
  P_2 =  \begin{pmatrix}
    1 & 1\\
    0 & 1 
  \end{pmatrix} 
\end{displaymath}
\begin{displaymath}
  (P_1 P_2)^T A P_1P_2 =  \begin{pmatrix}
    2 & 0 \\
    0 & 1 
  \end{pmatrix}.
\end{displaymath}
%
Maintenant, si  $A ∈ ℤ_2^{ n × n}$  est encore la  matrice  
\begin{displaymath}
  A =
  \begin{pmatrix}
    0 & 1 \\
    1 & 0 
  \end{pmatrix}, 
\end{displaymath}
on observe que 
\begin{displaymath}
  P_1^T A  P_1 =
  \begin{pmatrix}
    0 & 1 \\
    1 & 0 
  \end{pmatrix}.
\end{displaymath}
En fait, on peut montrer que, sur $ℤ_2$, la matrice $A$ n'est pas congruente à une matrice diagonale.

Dans ce chapitre, nous allons voir que toute matrice symétrique $A ∈ K^{ n ×n}$ est congruente à une matrice diagonale si $1_K + 1_K ≠ 0$. 

\section{Formes bililnéaires}
\label{sec:formes-bililneaires}

\begin{definition}
\label{def:58}
  Soit $V$
  un espace vectoriel sur un corps $K$.
  Une \emph{forme bilinéaire} sur $V$
  est une correspondance qui à tout couple $(v,w)$
  d'éléments de $V$ 
  associe un scalaire, noté $\pscal{v,w} ∈ K$,
   satisfaisant aux deux propriétés suivantes:
  \begin{enumerate}[BL 1]
  \item Si $u,v$ et $w$ sont des éléments de $V$, et $α ∈ K$ est un scalaire,  \label{ps:2}
    \begin{displaymath}
      \pscal{u,v+w} = \pscal{u,v}+\pscal{u,w} \quad \text{ et } \pscal{u,α ⋅ w} = α ⋅\pscal{u,w}. 
    \end{displaymath}
  \item 
    Si $u,v$ et $w$ sont des éléments de $V$, et $α ∈ K$ est un scalaire,  \label{ps:3}
    \begin{displaymath}
      \pscal{v+w,u} = \pscal{v,u}+\pscal{w,u} \quad \text{ et } \pscal{α ⋅ u, w} = α ⋅\pscal{u,w}. 
    \end{displaymath}     
  \end{enumerate}
La  forme bilinéaire  est dite \emph{symétrique} si 
pour tout $v,w \in V$ 
\begin{displaymath}
    \pscal{v,w} = \pscal{w,v}. 
\end{displaymath}
\noindent
On dit que la forme bilinéaire est \emph{non dégénérée à gauche} (respectivement \emph{à droite}) si la condition suivante est vérifiée:
\begin{quote}
  Si $v \in V$, et si $\pscal{v,w}=0$ pour tout $w \in V$, alors $v = 0$. 
\end{quote}
Si la forme bilinéaire est non dégénérée à gauche et à droite, on dit qu'elle est \emph{non dégénérée}.
\end{definition}

\begin{example}
 \label{ex:1}
  Soit $V = K^n$, l'application  
  \begin{displaymath}
    \begin{array}{rcl}
      \pscal{\,}\colon    \,  V\times V& \longrightarrow &K \\
                              (u,v)   & \longmapsto & \sum_{i=1}^n u_i v_i
    \end{array}  
  \end{displaymath}
est une forme bilinéaire. 
Vérifions (BL~\ref{ps:2}). Pour tous  $u,v,w ∈ K^n$ et $α ∈ K$:  
\begin{eqnarray*}
  \pscal{u,v+w} & = & \sum_{i=1}^n u_i \, (v_i+w_i) \\
  & = &  \sum_{i=1}^n \left( u_i v_i + u_i w_i \right) \\
  & = &  \sum_{i=1}^n  u_iv_i +\sum_{i=1}^n u_iw_i \\
  &= &\pscal{u,v} + \pscal{u,w}
\end{eqnarray*}
 et 
 \begin{displaymath}
   \pscal{u,α ⋅w} = \sum_{i=1}^n u_i α w_i = α ∑_{i=1}^n u_i w_i = α \pscal{u,w}. 
 \end{displaymath}
On appelle cette forme bilinéaire la \emph{forme bilinéaire standard} de $K^n$.   On vérifie aussi très facilement que la forme bilinéaire standard est symétrique et non dégénérée.
\end{example}



\begin{example}
  \label{exe:31}
  Soit $V = ℝ^3$ et
  \begin{displaymath}
    \pscal{u,v} = u^T
    \begin{pmatrix}
      1 & 0 & 1\\
      2 & 0 & 2 \\
      1 & 1 & 0\\
    \end{pmatrix}
    v \text{ pour } u,v ∈ ℝ^3,    
  \end{displaymath}
  est une forme bilinéaire non-symmetrique, dégénérée à droite  et à gauche. 
\end{example}


\begin{example}
  \label{ex:2}
  Soit $V$ l'espace des fonctions continues à valeurs réelles, définies sur l'intervalle $[0,2 \cdot \pi]$. Si $f,g \in V$ on pose 
  \begin{displaymath}
    \pscal{f,g} = \int_0^{2\pi} f(x)g(x) \, dx.
  \end{displaymath}
Clairement, $\pscal{,}$ est une forme bilinéaire symétrique sur $V$ non dégénérée. 
\end{example}

\begin{exercise}
  \label{ex:3}
  Montrer que les  formes bilinéaires  des exemples~\ref{ex:1} et \ref{ex:2} sont non dégénérées. 
\end{exercise}



Soit $V$ un espace vectoriel de dimension finie et $B = \{v_1,\dots,v_n\}$ une base de $V$. Pour une forme bilinéaire $f: V × V ⟶ K$ 
 et $x = \sum_i\alpha_i v_i$ et $y = \sum_j \beta_j v_j$ on a 
 \begin{eqnarray*}
   f(x,y) & = & f \left( ∑_{i=1}^n \alpha_i v_i ,   ∑_{j=1}^n \beta_j v_j \right) \\
          & = & ∑_{i=1}^n \alpha_i f \left(  v_i ,   ∑_{j=1}^n \beta_j v_j \right) \\
          & = & ∑_{i,j=1}^n \alpha_i β_j f (  v_i ,  v_j )
    \end{eqnarray*}
alors pour la matrice $A_B^f \in K^{n\times n}$, ayant comme composantes $f(v_i,v_j)$, on a 
\begin{displaymath}
  f(x,y) = [x]_B^T \,A_B^f \,  [y]_B. 
\end{displaymath}
\begin{exercise} Soit $V$ de dimension finie et $B$ une base de $V$. 
  Deux formes bilinéaires $f,g : V × V ⟶ K$
  sont différentes si et seulement si $A_B^f \neq A_B^g$. 
\end{exercise}


\begin{example}
  \label{exe:9}
  Soit $V = \{ p(x) : p ∈ ℝ[x], \, \deg(p)≤ 2\}$ l'espace vectoriel des polynômes réelles de degré au plus $2$ et $B = \{ 1,x,x^2\}$ une base de $V$ et $f(p,q) = \int_0^1 p(x) ⋅ q(x) dx$. Il est facile de vérifier que $f$ est une forme bilinéaire sur $V$. La matrice $A_B^f$ est 
  \begin{displaymath}
    A_B^f =
    \begin{pmatrix}
      1 & 1/2 & 1/3 \\
      1/2 & 1/3 & 1/4\\
      1/3 & 1/4 & 1/5
    \end{pmatrix}. 
  \end{displaymath}
 Pour $p(x) = 2+3\,x - 5\,x^2$ et $q(x) = 2\,x + 3\,x^2$ on obtient 
 \begin{displaymath}
   \int_0^1 f(x) p(x) dx =
   \begin{pmatrix}
     2 & 3 & -5
   \end{pmatrix}
 \begin{pmatrix}
      1 & 1/2 & 1/3 \\
      1/2 & 1/3 & 1/4\\
      1/3 & 1/4 & 1/5
    \end{pmatrix}
   \begin{pmatrix}
     0\\2\\3
   \end{pmatrix}
 \end{displaymath}
\end{example}


Pour mémoire, pour deux bases $B,B'$ et étant donné $[x]_{B'}$,  on trouve les coordonnées de $x$ dans la base $B$,  $[x]_B$,  à l'aide de la matrice de changement de base $P_{B'B}$ comme 
\begin{displaymath}
  [x]_B = P_{B'B} [x]_{B'}. 
\end{displaymath}
Cette formule nous montre que 
\begin{equation}
\label{eq:28}
 A_{B'}^f =  P_{B'B}^T \,A_B^f \,  P_{B'B}. 
\end{equation}

\begin{exercise}
  Soit $V$ un $K$-espace vectoriel de dimension finie et $B$ une base de $V$. 
  Une forme bilinéaire $f : V ×V ⟶ K$ 
  est symétrique si et seulement si $A_B^f$ est symétrique.
\end{exercise}

\begin{proposition}
  \label{prop:7}
  Soit $V$ un $K$-espace vectoriel de dimension finie, $B= \{b_1, \dots , b_n\}$ une base de $V$ et $f : V × V ⟶K$ une forme bilinéaire.   Les conditions suivantes sont équivalentes. 
  \begin{enumerate}[i)]
  \item $\rank(A_B^f) = n$ \label{item:12}
  \item $f$ est non dégénérée à gauche, i.e. si $v \in V$, et si $\pscal{v,w}=0$ pour tout $w \in V$, alors $v = 0$ \label{item:13}
  \item $f$ est non dégénérée à droite, i.e. si $v \in V$, et si $\pscal{w, v}=0$ pour tout $w \in V$, alors $v = 0$ \label{item:14}
  \end{enumerate}
\end{proposition}

\begin{proof} 

Nous montrons \ref{item:12}) et \ref{item:13}) sont équivalentes.  De la même manière, on démontre aussi que \ref{item:12}) et \ref{item:14}) sont équivalentes.

\ref{item:12}) $\Rightarrow$ \ref{item:13}) : Supposons que $\rank(A_B^f) = n$ et soit $v \in V$, $v \neq 0$. Pour $w \in V$ on a 
\begin{displaymath}
  f(v,w) = [v]_B^T A_B^f [w]_B. 
\end{displaymath}
Dès que $[v]_B \neq 0$, on a $[v]_B^T A_B^f \neq 0^T$ (car $\noy(A_B^f) = \{0\} \iff \rank(A_B^f) = n$). Supposons que la $i$-ème composante de $[v]_B^T A_B^f$ n'est pas égale a $0$. Alors $[v]_B^T A_B^f e_i \neq 0$ où toutes les composantes de $e_i$ sont $0$ sauf la $i$-ème composante, 
qui est égale a $1$. Alors $f(v,b_i) \neq 0$. Donc $f$ est non dégénérée à gauche. \newline

\ref{item:13}) $\Rightarrow$ \ref{item:12}) : Si $f$ est non dégénérée à gauche, alors $x^T A_B^f \neq 0$ pour tout $x \in K^n$ tel que $x \neq 0$ (sinon, on aurait trouvé un $x$ tel que $x^T A_B^f y = 0$ pour tout $y \in K^n$). Ceci implique que les lignes de $A_B^f$ sont linéairement indépendantes. Alors $\rank(A_B^f) = n$. 
\end{proof}






\section{Orthogonalité} 
\label{sec:orthogonalite}

\begin{framed}\noindent 
  Pour ce paragraphe~\ref{sec:orthogonalite}, s'il n'est pas spécifié autrement,  $V$
  est toujours un espace vectoriel sur $K$
  muni d'une forme bilinéaire symétrique $\pscal{,}$. 
\end{framed}



\begin{definition}
  \label{def:2}
 Deux éléments $u,v \in V$ sont \emph{orthogonaux} ou \emph{perpendiculaires} si $\pscal{u,v} = 0$, et l'on écrit $u \perp v$. 
\end{definition}

\begin{proposition}
  \label{prop:1}
  Soit $E \subseteq V$ une partie de $V$, alors $E^\perp = \{ v \in V \colon v \perp e \text{ pour tout } e \in E \}$   est un sous-espace vectoriel de $V$. 
\end{proposition}

\begin{proof}
  Pour mémoire: $\emptyset \neq W\subseteq V$ est un sous-espace si les conditions suivantes sont vérifiées. 
  \begin{enumerate}[i)]
  \item Si $u,v \in W$ on a $u+v \in W$.  
  \item Si $c \in K$ et $u \in W$ on a $c \cdot u \in W$. 
  \end{enumerate}
  
  \noindent
  Des que $0 ∈ E^\perp  $, on a que $E^\perp≠ ∅$.  %ici E^\perp \neq \varnothing peur résoudre le problème a la compilation : l'ensemble vide est affiché deux fois ...
  Si $u,v \in E^\perp$ alors pour tout $e \in E$ 
  \begin{displaymath}
%    \pscal{e,u+v} = \pscal{e,u} + \pscal{e,-v} = 0 + 0 = 0,
    \pscal{e,u+v} = \pscal{e,u} + \pscal{e,v} = 0 + 0 = 0,
  \end{displaymath}  
  et pour $c \in K$ 
  \begin{displaymath}
    \pscal{e,c\cdot v} = c \,\pscal{e,v} = c \cdot 0 = 0. 
  \end{displaymath}
\end{proof}



\begin{exercise}
  \label{exe:1}
  Soit $E \subseteq V$ et $E^*$ le sous-espace de $V$ engendré par les éléments de $E$. Montrer $E^\perp = {E^*}^\perp$. 
\end{exercise}


\begin{example}
  \label{exe:2}
  Soient $K$ un corps et $(a_{ij}) \in K^{m\times n}$ une matrice à $m$ lignes et $n$ colonnes. Le système homogène linéaire 
  \begin{equation}
    \label{eq:1}
    A\,X = 0,
  \end{equation}
  peut s'écrire sous la forme 
  \begin{displaymath}
    \pscal{A_1,X}=0, \dots ,\pscal{A_m,X}=0, 
  \end{displaymath}
  où les $A_i$  sont les vecteurs lignes de la matrice $A$ et $\pscal{,}$ dénote la forme bilinéaire standard de $K^n$. Soit $W$ le sous-espace de $K^n$ engendré par les $A_i$ et $U$ le sous-espace de $K^n$ des solutions du système~\eqref{eq:1}. Alors on a  $U = W^\perp$ et $\dim(W^\perp) = \dim(U) = n - \rank(A) = \dim(\noy(A))$. 
  
 \end{example}




\begin{definition}
  \label{def:8}
  La caractéristique d'un anneau (unitaire) $R$, $\mathrm{Char}(R)$ 
  est l'ordre de $1_R$
  comme élément du groupe abélien $(R,+)$.
  En d'autres mots, c'est le nombre 
  \begin{displaymath}
    \min_{k \in \N_+} \underbrace{1+ \cdots + 1}_{k \text{ fois }} = 0
  \end{displaymath}
  Si cet ordre est infini, la caractéristique de $R$ est $0$.
\end{definition}

\begin{notation}
  Pour $n \in \N_+$ l'anneau des classes des restes est dénoté comme
  $\Z / n \Z$ ou plus brièvement $\Z_n$ (parfois aussi noté $\Bbb F_n$). Ceci est un corps si et seulement si $n$ est un nombre premier. 
\end{notation}

\begin{example}
\label{exe:10}
   Soit $n \in \N_+$. Alors la  caractéristique de $\Z_n$ est $n$. 
   La caractéristique de $\Q,\R$ et $\C$ est zéro.    
\end{example}

\begin{lemma}
  \label{lem:1}
 Soit $\mathrm{Char}(K)\neq 2$. Si $\pscal{u,u}=0$ pour tout $u \in V$ alors 
  \begin{displaymath}
    \pscal{u,v}=0 \text{ pour tous }u,v \in V
  \end{displaymath}
On dit que la forme bilinéaire symétrique $\pscal{,}$ est  \emph{nulle}. 
\end{lemma}

\begin{proof}
  Soient $u,v \in V$. 
  On peut écrire
  \begin{displaymath}
   2 \cdot  \pscal{u,v} = \pscal{u+v,u+v} - \pscal{u,u} - \pscal{v,v} 
  \end{displaymath}

et comme $2 \neq 0$ on a $\pscal{u,v} = 0$. 
\end{proof}


\begin{definition}
  \label{def:38}
  Une base $\{v_1,\dots,v_n\}$
  de l'espace vectoriel $V$
  est une \emph{base orthogonale} si $\pscal{v_i,v_j}=0$
  pour $i\neq j$.
\end{definition}



\begin{remark}
  \label{rem:7}
  Pour une forme bilinéaire symétrique $\langle , \rangle$ et une base $B = \{v_1,\dots,v_n\}$. On se rappelle que 
  \begin{displaymath}
    〈v_i,v_j〉 = (A_B^{〈, 〉})_{ij}
  \end{displaymath}
  Alors $B$ est une base orthogonale, si et seulement si, $A_B^{\langle , \rangle}$ est une matrice diagonale.  
\end{remark}


\begin{theorem}
  \label{thr:5}
  Soit $\mathrm{Char}(K)\neq 2$
  et supposons que $V$
  est de dimension finie. Alors $V$ possède une base orthogonale.
\end{theorem}


\begin{proof}
  On montre le théorème par induction. Si $\dim(V) = 1$ alors toute base contient seulement un élément et alors est orthogonale. 

Soit $\dim(V) >1$.  Si $\pscal{u,u} = 0$ pour tout $u$, le lemme~\ref{lem:1} implique que la forme bilinéaire symétrique est nulle et toute base de $V$  est orthogonale.  
Autrement, soit $u \in V$ tel que $\pscal{u,u} \neq 0$ et soit $V_1 = \spa\{u\}$. Pour  $x \in V$ le vecteur 
\begin{displaymath}
  x - \pscal{x,u}/\pscal{u,u} \cdot  u \in V_1^\perp
\end{displaymath}
et alors $V = V_1 + V_1^\perp$. Cette somme est directe parce que chaque élément de $V_1 \cap V_1^\perp$ s'écrit comme $\beta \cdot u$ pour $\beta \in K$. Et $\pscal{u,\beta u} = \beta \pscal{u,u} = 0$ implique $\beta = 0$. 

Alors $\dim(V_1^\perp) < \dim(V)$, et par induction, $V_1^\perp$ possède une base orthogonale $\{v_2,\dots,v_n\}$. Alors $\{u,v_2,\dots,v_n\}$ est une base orthogonale de $V$. 
\end{proof}  



\begin{example}
  \label{exe:30}
  Soit $V = ℤ_5^3$ et $〈,〉: ℤ_5^3 × ℤ_5^3 → ℤ_5$ défini comme
  \begin{displaymath}
    〈x,y〉 = x^T A y,
  \end{displaymath}
  où
  \begin{displaymath}
    A = \left(\begin{array}{rrr}
                0 & 2 & 1 \\
                2 & 0 & 4 \\
                1 & 4 & 0
              \end{array}\right)
  \end{displaymath}
  Le but est de trouver une base orthogonale de $ℤ_5^3$. On va trouver une matrice inversible $P ∈ ℤ_5^{3×3}$ tel que $P^T A P$  est une matrice diagonale. Si  $p_1,p_2,p_3$ sont les colonnes de $P$, alors
  \begin{displaymath}
    \{ p_1,p_2,p_3 \}
  \end{displaymath}
  est une base de $ℤ_5^3$ et c'est une base orthogonale, des que
  \begin{displaymath}
    〈p_i,p_j〉 =0 \text{ si } i ≠j, \, 1 ≤i,j ≤3. 
  \end{displaymath}
  Nous allons additionner la $2$-ème colonne de $A$  sur la $1$-ère colonne et la $2$-ème ligne  de $A$  sur la $1$-ère ligne  de $A$. C'est à dire on calcule
  \begin{displaymath}
    P^T A P  = \left(\begin{array}{rrr}
4 & 2 & 0 \\
2 & 0 & 4 \\
0 & 4 & 0
\end{array}\right)
  \end{displaymath}
  où
  \begin{displaymath}
P  =  \left(\begin{array}{rrr}
1 & 0 & 0 \\
1 & 1 & 0 \\
0 & 0 & 1
\end{array}\right)
  \end{displaymath}
  Après on va additionner $2⋅$ la $1$-ère colonne sur la deuxième, et l'opération correspondante de lignes et on obtient 
  \begin{displaymath}
    P^T A P  =
    \left(\begin{array}{rrr}
            4 & 0 & 0 \\
            0 & 4 & 4 \\
            0 & 4 & 0
          \end{array}\right)
      \end{displaymath}
      où
\begin{displaymath}
P = \left(\begin{array}{rrr}
1 & 2 & 0 \\
1 & 3 & 0 \\
0 & 0 & 1
\end{array}\right)        
\end{displaymath}
 Après on va additionner $4⋅$ la $2$-ère colonne sur la $3$-ème, et l'opération correspondante de lignes et on obtient 
  \begin{displaymath}
    P^T A P  =\left(\begin{array}{rrr}
4 & 0 & 0 \\
0 & 4 & 0 \\
0 & 0 & 1
\end{array}\right)    
  \end{displaymath}
      où
\begin{displaymath}
  P =\left(\begin{array}{rrr}
1 & 2 & 3 \\
1 & 3 & 2 \\
0 & 0 & 1
\end{array}\right).   
\end{displaymath}
Nous avons trouvé une base orthogonale
\begin{displaymath}
  \left\{
    \begin{pmatrix}
      1 \\
1  \\
0 
    \end{pmatrix},
    \begin{pmatrix}
      2 \\
 3  \\
 0 
    \end{pmatrix},
    \begin{pmatrix}
      3 \\
      2 \\
      1
    \end{pmatrix}
    \right\}.  
\end{displaymath}      
\end{example}

\subsection*{Exercices} 

\begin{enumerate}
\item Soit $K$ un corps. Si la caractéristique de $K$ est différente de zéro, alors elle est un nombre premier. 
\item Soit $K$ un corps fini. Montrer que $|K| = q^\ell$ pour un nombre premier $q$ et un nombre naturel $\ell ∈ \N$. \emph{Indication: $K$ est un espace vectoriel de dimension finie sur $\Z_q$ pour un $q$ premier.}
\item On considère les vecteurs  \label{item:1}
  \begin{displaymath}
    v_1 =
    \begin{pmatrix}
      1\\1\\0\\0
    \end{pmatrix}, 
 v_2 =
    \begin{pmatrix}
      0\\1\\1\\0
    \end{pmatrix}, 
\text{ et }
 v_3 =
    \begin{pmatrix}
      0\\0\\1\\1
    \end{pmatrix} \in \Z_2^4. 
  \end{displaymath}
Est-ce que $\spa\{v_1,v_2,v_3\}$ possède une base orthogonale par rapport à la forme bilinéaire symétrique standard de l'exemple~\ref{ex:1}? 

\item En considérant le forme bilinéaire symétrique standard de l'exemple~\ref{ex:1}, trouver une base orthogonale du sous-espace de $\Z_3^4$ engendré par 
  \begin{displaymath}
    v_1 =
    \begin{pmatrix}
      1\\1\\1\\0
    \end{pmatrix}, 
 v_2 =
    \begin{pmatrix}
      0\\1\\1\\1
    \end{pmatrix}, 
\text{ et }
 v_3 =
    \begin{pmatrix}
      1\\0\\1\\1
    \end{pmatrix} \in \Z_3^4. 
  \end{displaymath}



\end{enumerate}

\section{Matrices congruentes} 
\label{sec:class-des-matr}

% \begin{framed}\noindent 
%   Pour ce paragraphe~\ref{sec:class-des-matr}, s'il n'est pas spécifié autrement,  $V$
%   est toujours un espace vectoriel sur $K$
%   muni d'une forme bilinéaire symétrique $\pscal{,}$. 
% \end{framed}


\begin{definition}
  \label{def:13}
  Deux matrices $A,B \in K^{n\times n}$ sont dites \emph{congruentes} s'il existe une matrice $P \in K^{n \times n}$ inversible telle que 
\begin{displaymath}
  A = P^T B P.
\end{displaymath}
Nous écrivons dans ce cas $A \cong B$. 
\end{definition}


\begin{example}
  \label{exe:22}
  Si $V$ est de dimension finie et $B,B'$ sont deux bases de V, la relation \eqref{eq:28} montre que $A_B^{\pscal{,}} \cong A_{B'}^{\pscal{,}}$. 
\end{example}

\begin{lemma}
  \label{lem:6}
  La relation $\cong$ est une relation d'équivalence. 
\end{lemma}

\begin{proof}
  Voir exercice. 
\end{proof}


Le relation entre $\cong$ et le concept de l'orthogonalité   est précisée dans le lemme suivant. 
\begin{lemma}
\label{thr:13}
  Soit $V$ un espace vectoriel de dimension finie et $B = \{v_1,\dots,v_n\}$ une base quelconque. Alors $V$ possède une base orthogonale si et seulement s'il existe une matrice diagonale $D$ 
telle que 
 $A_B^{\pscal{.}} \cong D$. 
\end{lemma}
\begin{proof}
Si $B'$ est une base orthogonale de $V$, alors $A_{B'}^{〈,〉}$ est une matrice diagonale. Grâce à la relation \eqref{eq:28}, $A_{B}^{〈,〉}$ est congruente à une matrice diagonale. 


Si $A_B^{\pscal{.}} \cong D$ où $D ∈ K^{n×n}$ est une matrice diagonale, alors 
il existe une matrice $P ∈K^{n ×n}$ inversible, telle que 
\begin{displaymath}
  P^T A_B^{\pscal{.}} P = D. 
\end{displaymath}
La base  $B' = \{w_1,\dots,w_n\}$ donnée par les colonnes de $P$ (en tant que coordonées dans la base $B$) :
  \begin{displaymath}
     [w_j]_B =
  \begin{pmatrix}
    p_{1j}\\ \vdots \\ p_{nj} 
  \end{pmatrix} \iff w_j = \sum_{i=1}^n p_{ij} v_i, \quad \forall j=1, \dots, n,
  \end{displaymath}
 est donc une base orthogonale. 
\end{proof}




 

\begin{corollary}
  \label{co:4}
  Soit $K$ un corps de caractéristique différente de $2$. Toute matrice symétrique  $A \in K^{n \times n}$ est congruente à une matrice diagonale. 
\end{corollary}
\begin{proof}
Ceci est un corollaire du  lemme~\ref{thr:5} et du théorème~\ref{thr:13} 
parce que  $K^n$ muni de la forme bilinéaire symétrique $\pscal{u,v} = u^TAv$ possède une base orthogonale. 
\end{proof}

%The example below is uncomplete and thus makes no sense. Since I didn't know what was its purpose, I couldn't complete it.
%\begin{example}
%  \label{exe:14}
%  Soient $V$ un espace vectoriel sur $\R$ et $\{v_1,v_2,v_3\}$ une base de V. 
%\\ \todo[inline]{Finish or delete the preceding example}
%\end{example}







Maintenant, nous allons formaliser la procédé appliquée dans example~\ref{exe:30}. 
\begin{algorithm}
  \label{alg:1}
  Cet algorithme trouve une matrice diagonale congruente à la matrice symétrique $A \in K^{n \times n}$ où $K$ est un corps tel que $\car(K) \neq 2$.  L'algorithme procède en $n$ itérations. Après la $(i-1)$-ème itération, $i \geq 1$, (aussi après  la $0$-ème itération) l'algorithme a transformé $A$ en une matrice congruente 
  \begin{equation}
    \label{eq:7}
    \begin{pmatrix}
      c_1 \\
      & c_2 \\
      & & \ddots & &&\\
      & & & c_{i-1} \\
      & & & &  b_{i,i} & \dots & b_{i,n} \\
%      & & & &  b_{i+1,i} & \dots & b_{i+1,n} \\
      & & & &     \vdots       &  & \vdots \\
      & & & &  b_{n,i} & \dots & b_{n,n} \\      
    \end{pmatrix}
  \end{equation}
où les composantes des premières $(i-1)$ lignes et colonnes sont nulles sauf éventuellement sur la diagonale. 

\medskip 
\noindent 
Pour $1 \leq i \leq n$, la \emph{$i$-ème itération} procède comme suit. 
\begin{itemize}
 \item Soit l'indice $k$ minimal tel que $k \geq i$ et $b_{kk} \neq 0$. On échange la $i$-ème ligne et la $k$-ème ligne puis la $i$-ème colonne et la $k$-ème colonne. Ceci permet (entre autres) d'échanger les coefficients $b_{ii}$ et $b_{kk}$ de la diagonale, s'assurant ainsi d'avoir un coefficient non nul. 
 
 \item Si l'indice $k$ de l'étape précédente n'existe pas (tous les coefficients diagonaux après $c_{i-1}$ sont nuls), soit $j \in \{i+1,\dots,n\}$ un indice vérifiant $b_{ij} \neq 0$. On ajoute la $j$-ème ligne à la $i$-ème ligne puis la $j$-ème colonne à la $i$-ème colonne. Le $i$-ème coefficient de la diagonale devient alors $2 b_{ij} + b_{jj} = 2 b_{ij} \neq 0$.
 
 \item Si, à son tour, un tel indice $j$ n'existe pas, on peut procéder à la $i+1$-ème itération car la matrice est déjà de la forme \eqref{eq:7} (avec $i+1$ à la place de $i$).
  
\item Pour chaque $j \in \{i+1,\dots,n\}$:  on additionne $-b_{ij}/b_{ii}$ fois la $i$-ème ligne sur la $j$-ème ligne et on additionne $-b_{ij}/b_{ii}$ fois la $i$-ème colonne sur la $j$-ème colonne. Ceci permet d'annuler les coefficients à droite et sous le coefficient $b_{ii}$. On peut alors poursuivre à l'étape $i+1$.
\end{itemize}    

Remarquons que chaque opération est faite à la fois sur les lignes et sur les colonnes. Ceci garantit que la matrice résultante reste symétrique. De plus, les opérations sont faites sur les lignes et colonnes d'indices $j \geq i$, laissant intacte la forme de la matrice \eqref{eq:7}.

\end{algorithm}


\begin{example}
  \label{exe:15}
  Soit $V$ une espace vectoriel sur $\Q$ de dimension $3$ muni d'une forme bilinéaire symétrique $〈,〉$. Soit $B = \{v_1,v_2,v_3\}$ une base de $V$ et 
  \begin{displaymath}
    A^{\pscal{.}}_B =
    \begin{pmatrix}
      1 & 0 & 2\\
      0 & 3 & 4\\
      2 & 4 & 0 
    \end{pmatrix}
  \end{displaymath}
Le but est de trouver une  base orthogonale de $V$. 

En utilisant notre algorithme on trouve 
\begin{displaymath}
  P = 
  \begin{pmatrix}
    
1 & 0 & -2 \\
0 & 1 & -\frac{4}{3} \\
0 & 0 & 1
  \end{pmatrix}
\end{displaymath}
telle que 
\begin{displaymath}
  P^T \cdot A^{\pscal{.}}_B \cdot P =
  \begin{pmatrix}
    1 & 0 & 0 \\
0 & 3 & 0 \\
0 & 0 & -\frac{28}{3}
  \end{pmatrix}. 
\end{displaymath}
Alors $B' = \{v_1,v_2,-2v_1 -(4/3) v_2 + v_3\}$ est une base orthogonale de $V$. 
\end{example}





\subsection*{Exercices}

\begin{enumerate}
\item Montrer que $\cong$ est une relation d'équivalence sur l'ensemble des matrices $K^{n\times n}$. 
\item Est-ce que la matrice 
  \begin{displaymath}
    \begin{pmatrix}
      0 & 1 & 0 \\
      1 & 0 & 1\\
      0 & 1 & 0 
    \end{pmatrix} \in \Z_2^{3\times 3} 
  \end{displaymath}
  est congruente à une matrice diagonale? \emph{Indice : voir l'exercice~\ref{item:1}. de la section~\ref{sec:orthogonalite}.}  
\item Soit $V$ un espace vectoriel  sur un corps $K$ de dimension finie muni d'une forme bilinéaire symétrique $\pscal{.}$. Soit $B = \{v_1,\dots,v_n\}$ une base de $V$. Montrer que $A_B^{\pscal{.}} \in K^{n \times n}$ est congruente à une matrice diagonale si  et seulement si $V$ possède une base orthogonale. 

\item Soit $K$ un corps de caractéristique $2$ et soit $V$ un espace vectoriel sur $K$ de dimension finie muni d'une forme bilinéaire symétrique non-nulle.  Soit 
  \begin{displaymath}
    C =
    \begin{pmatrix}
      0 & 1 \\
      1 & 0 
    \end{pmatrix}.
  \end{displaymath}
  \begin{enumerate}[a)]
  \item Soit $\dim(V) = 2$. Montrer que $V$ ne possède pas de base orthogonale si et seulement s'il existe une base $B$ de $V$ telle que $A_B^{\pscal{.}} = C$. 
  \item Soit $\dim(V) = n$. Montrer que $V$ ne possède pas de base orthogonale si et seulement s'il existe une base $B$ de $V$ telle que 
    \begin{displaymath} A_B^{\pscal{.}} = 
      \begin{pmatrix}
        d_1 \\
        & d_2 \\
        & & \ddots \\
        & &   & d_k \\
        & &   &  & C \\
        & &   &  & & \ddots \\
        & &   &  & & & C \\
      \end{pmatrix}
    \end{displaymath}
    et $d_1,\dots,d_k = 0$, et le nombre de $C$ n'est pas égal à zéro. 
  \end{enumerate}
\item Modifier l'algorithme~\ref{alg:1} tel qu'il soit aussi correct pour des corps de caractéristique $2$.  Soit l'algorithme découvre que la matrice symétrique $A \in K^{n \times n}$ n'est pas congruente à une matrice diagonale, soit l'algorithme calcule une matrice diagonale congruente à $A$. 
\item Comment peut-on déterminer si un espace vectoriel de dimension finie muni d'une forme bilinéaire symétrique  possède une base orthogonale? Décrire très brièvement une méthode. 
 
\item Soit $V$ un espace euclidien de dimension $n$. Montrer que $V$ possède une base $B$ telle que pour tout $x,y \in V$
  \begin{displaymath}
    \pscal{x,y} = [x]_B \cdot [y]_B, 
  \end{displaymath}
où $ [x]_B \cdot [y]_B$  dénote la forme bilinéaire standard de $\R^n$ entre $[x]_B$ et $ [y]_B$ .  

\end{enumerate}




\section{Le théorème de Sylvester}
\label{sec:le-theoreme-de}




Soit $V$ un espace vectoriel de dimension finie sur un corps $K$, $\mathrm{Char}(K) \neq 2$, muni d'une forme bilinéaire symétrique. Nous avons vu (théorème~\ref{thr:5}) que $V$ possède une base orthogonale. Supposons que cette base est $B = \{v_1,\dots,v_n\}$ et considérons $x = \sum_i \alpha_i v_i \in V$ et $y = \sum_i \beta_i v_i \in V$. La forme bilinéaire  s'écrit 
\begin{eqnarray*}
  \pscal{x,y} & =  & \sum_{i, j} \alpha_i \beta_j \pscal{v_i,v_j} \\
              & = & \sum_i \alpha_i \beta_i \pscal{v_i,v_i} \\
               & = & [x]_B^T 
                    \begin{pmatrix}
                      c_1 & & \\
                                     & \ddots & \\
                                     & & c_n
                    \end{pmatrix} [y]_B
\end{eqnarray*}
où $c_i = \pscal{v_i,v_i}$ pour tout $i$. 
Si $K = \R$ on peut ordonner la base afin d'avoir $c_1,\dots,c_r>0$, $c_{r+1},\dots,c_s < 0$ et $c_{s+1},\dots,c_n = 0$.



Maintenant soit $K = \R$  et $A \in \R^{n \times n}$ symétrique. Le Corollaire~\ref{co:4} implique qu'il existe une matrice inversible $P ∈ ℝ^{n \times n}$ tel que $P^T A P = D$ où $D$ est une matrice diagonale. Si on échange deux colonnes de $P$ et note $P'$ la nouvelle matrice obtenue, alors $P'^T A P' = D'$, où $D'$ est obtenue de $D$ en échangeant les éléments diagonaux correspondants. Alors on peut trouver une matrice inversible $P ∈ ℝ^{n ×n}$ telle que 
\begin{equation}
  \label{eq:29}
  P^T A {P} =
    \begin{pmatrix}
      c_1\\
      & \ddots \\
      && c_n
    \end{pmatrix}. 
\end{equation}
où les  $c_i$ sont ordonnés de sorte que  $c_1,\dots, c_r >0$, $c_{r+1},\dots ,c_s<0$ et $c_{s+1},\dots , c_n = 0$. En multipliant les premières $s$ colonnes de $P$ par $1/ \sqrt{|c_i|}$ on obtient en fait une factorisation~\eqref{eq:29} telle que 
 $c_1,\dots, c_r =1$, $c_{r+1},\dots ,c_s=-1$ et $c_{s+1},\dots , c_n = 0$. 

Alors on trouve  $P ∈ ℝ^{n ×n}$ inversible telle que 
\begin{equation}
  \label{eq:5}
  P^T A P =
  \begin{pmatrix}
    1 &   \\
      & \ddots &  \\
      &        & 1 \\
      &        &  & -1 \\
      &        &  &    & \ddots \\
      &        &  &    &        & -1 \\
      &        &  &    &        &    & 0 \\
      &        &  &    &        &    &   & \ddots  \\
      &        &  &    &        &    &   &        & 0  \\

  \end{pmatrix}.
\end{equation}


\begin{definition}
  \label{def:37}
  Pour un espace vectoriel sur $ℝ$ de dimension finie, on appelle une base $B$ de $V$ telle que $A_B^{\pscal{,}}$ a la forme décrite en~\eqref{eq:5} une \emph{base de Sylvester}. 
\end{definition}

Nous allons maintenant démontrer, que les nombres $r$ et $s$ sont invariants par rapport au choix de la base $B$ de $V$. 


\begin{definition}
  \label{def:11}
  Le sous espace $V_0 = \{v \in V \colon \pscal{v,x} = 0 \text{ pour tout } x \in V\}$ est appelé l'\emph{espace de nullité} de la forme bilinéaire symétrique $\pscal{.}$. 
\end{definition}

\begin{theorem}
  \label{thr:9}
  Soit $V$
  un espace vectoriel de dimension finie sur un corps $K$
  de caractéristique $\neq 2$
  et soit $V$
  muni d'une forme bilinéaire symétrique. Soit $B = \{v_1,\dots,v_n\}$
  une base orthogonale de $V$.
  La dimension $\dim(V_0)$
  est égale au nombre d'indices $i$ tel que $\pscal{v_i,v_i}=0$.
\end{theorem}

\begin{proof}
  Nous utilisons la notation d'au-dessus et écrivons 
  \begin{displaymath}
    \pscal{v,x} =   [v]_B^T
                    \begin{pmatrix}
                      c_1 & & \\
                                     & \ddots & \\
                                     & & c_n
                    \end{pmatrix} [x]_B. 
  \end{displaymath}
Cette expression est égale à zéro pour tout $x ∈V$  si et seulement si $\left([v]_B\right)_i = 0$ pour tout $i$ tel que $c_i \neq 0$. Ceci démontre que $\{ v_i \colon \pscal{v_i,v_i}=0\}$ est une base de l'espace de nullité. 
\end{proof}

\begin{definition}
  La dimension de l'espace de nullité $\dim(V_0)$ est appelé l'\emph{indice de nullité} de la forme bilinéaire symétrique. 
\end{definition}

\begin{theorem}[Théorème de Sylvester] 
\label{thr:10}
Soit $V$
un espace vectoriel de dimension finie sur $\R$
muni d'une forme bilinéaire symétrique.
Il existe un nombre entier $r ≥0$ tel que, pour chaque base orthogonale 
  $B = \{v_1,\dots,v_n\}$ de $V$, 
 exactement $r$ des indices $i$ satisfont $\pscal{v_i,v_i}>0$.
\end{theorem}


\begin{proof}
  Soient $\{v_1,\dots,v_n\}$
  et $\{w_1,\dots,w_n\}$ des
  bases orthogonales de $V$ ordonnées 
  telles que $\pscal{v_i,v_i} >0 $
  si $1 \leq i \leq r$,
  $\pscal{v_i,v_i} <0 $
  si $r+1 \leq i \leq s$
  et $\pscal{v_i,v_i} =0 $
  si $s+1 \leq i \leq n$.
  De même $\pscal{w_i,w_i} >0 $
  si $1 \leq i \leq r'$,
  $\pscal{w_i,w_i} <0 $
  si $r'+1 \leq i \leq s'$
  et $\pscal{w_i,w_i} =0 $ si $s'+1 \leq i \leq n$.



  On démontre que $v_1,\dots,v_r,w_{r'+1},\dots w_n$
  est linéairement indépendant. Ça implique que $r + n-r' \leq n$
  et alors $r \leq r'$.
  Parce que l'argument est symétrique on peut conclure que $r = r'$.

  Si $v_1,\dots,v_r,w_{r'+1},\dots w_n$ est linéairement dépendant, il existe des scalaires $x_1,\dots x_r$ et $y_{r'+1},\dots y_n$ respectivement  non tous égaux à zéro tels que 
  \begin{displaymath}
    x_1 v_1 + \cdots + x_r v_r = y_{r'+1} w_{r'+1} + \cdots y_n w_n 
  \end{displaymath}
et ça implique, car les $v_i$ et respectivement les $w_i$  sont orthogonaux entre eux, 
\begin{displaymath}
   x_1^2\pscal{v_1,v_1} + \cdots + x_r^2 \pscal{v_r,v_r} = y_{r'+1}^2 \pscal{w_{r'+1},w_{r'+1}} + \cdots y_n^2 \pscal{w_n,w_n} 
\end{displaymath}
Les $\pscal{v_i,v_i}$ à gauche sont strictement positifs. Les $\pscal{w_i,w_i} $ à droite sont négatifs ou nuls. Il suit que $x_1=0,\dots,x_r = 0$ et, comme les $w_i$ sont linéairement indépendants, on a également $y_{r'+1}=0,\dots,y_n=0$. 
\end{proof}


\begin{definition}
  \label{def:12}
  L'entier $r$ du théorème de Sylvester est appelé l'\emph{indice de positivité} de la forme bilinéaire symétrique. 
\end{definition}



\begin{example}
  \label{exe:11}
  Trouver une base de Sylvester de $ℝ^4$ et les indices de nullité et de positivité de la forme bilinéaire symétrique $x^TAy$ où  
  \begin{displaymath}
A = 
    \begin{pmatrix}
      2 & 4 & 6 \\
      4 & 4 & 3 \\
      6 & 3 & 1 
    \end{pmatrix}
  \end{displaymath}
On utilise des transformations élémentaires sur les colonnes et les mêmes sur les lignes tour à tour en alternant. 

\noindent Les transformations élémentaires sur les colonnes sont représentées par 
\begin{displaymath}
P_1 = 
  \left[\begin{matrix}1 & -2 & -3\\0 & 1 & 0\\0 & 0 & 1\end{matrix}\right]
\end{displaymath}
et transforment la matrice $A$ en 
\begin{displaymath}
   \begin{pmatrix}
      2 & 4 & 6 \\
      4 & 4 & 3 \\
      6 & 3 & 1 
    \end{pmatrix} \cdot  \left[\begin{matrix}1 & -2 & -3\\0 & 1 & 0\\0 & 0 & 1\end{matrix}\right] = \left[\begin{matrix}2 & 0 & 0\\4 & -4 & -9\\6 & -9 & -17\end{matrix}\right]. 
\end{displaymath}
Alors 
\begin{displaymath}
  P_1^T\cdot A \cdot P = \left[\begin{matrix}2 & 0 & 0\\0 & -4 & -9\\0 & -9 & -17\end{matrix}\right]
\end{displaymath}
Avec 
\begin{displaymath}
  P_2 = \left[\begin{matrix}1 & 0 & 0\\0 & 1 & -\frac{9}{4}\\0 & 0 &1\end{matrix}\right]
\end{displaymath}
on obtient 
\begin{displaymath}
  P_2^TP_1^T A P_1 P_2 = \left[\begin{matrix}2 & 0 & 0\\0 & -4 & 0\\0 & 0 & \frac{13}{4} \end{matrix}\right]
\end{displaymath}
L'indice de nullité est zéro et l'indice de positivité est $2$. Le produit $P_1\cdot P_2$ est égal à 
\begin{displaymath}
  P_1 \cdot P_2 = \left[\begin{matrix}1 & -2 & \frac{3}{2}\\0 & 1 & -\frac{9}{4}\\0 & 0 & 1\end{matrix}\right]
\end{displaymath}
En divisant les colonnes de $P$ par $\sqrt{2},\sqrt{4}$ et $\sqrt{13/4}$ respectivement, on obtient une transformation $P$ telle que $P^TAP =
\begin{pmatrix}
  1& \\
  & 1 & \\
  & & -1
\end{pmatrix}$. 
\end{example}
Les colonnes de $P$ sont une base de Sylvester. 



\subsection*{Exercices} 

\begin{enumerate}
\item Démontrer, à l'aide des théorèmes \ref{thr:9} et \ref{thr:10}, que l'indice de négativité (l'entier $s$ de l'équation \eqref{eq:29}) ne dépend lui aussi pas de la base choisie.
\item Déterminer l'indice de nullité et l'indice de positivité des 
formes bilinéaire symétriques 
 définies par les matrices suivantes
  \begin{displaymath}
    \begin{pmatrix}
      1 & 2 \\
      2 & -1
    \end{pmatrix}
    , \,
    \begin{pmatrix}
      1 & 1 \\
      1 & 1
    \end{pmatrix}, \, 
    \begin{pmatrix}
      2 & 4 & 2 \\
      4 & 3 &  1 \\ 
      2 & 1 &1
    \end{pmatrix}. 
  \end{displaymath}

\item Soit $V$ un espace vectoriel de dimension finie sur $\R$ et soit $\pscal{.}$ une forme bilinéaire symétrique  sur $V$.  Montrer que $V$ admet une décomposition en somme directe 
  \begin{displaymath}
    V_0 \oplus V^+ \oplus V^-
  \end{displaymath}
où $V_0$ est l'espace de nullité et $V^+$ et $V^-$ sont des sous-espaces tels que 
\begin{displaymath}
  \pscal{v,v} >0 \text{ pour tout } v \in V^+ \setminus\{0\}
\end{displaymath}
et  
\begin{displaymath}
  \pscal{v,v} <0 \text{ pour tout } v \in V^- \setminus\{0\}. 
\end{displaymath}
\end{enumerate}






\section{Le cas réel, défini positif}
\label{sec:le-case-reel}



\begin{definition}
  \label{def:4}
  Soit $V$ un espace vectoriel sur $\R$ muni 
  d'une forme bilinéaire symétrique. 
  La forme bilinéaire symétrique  est définie positive si $\pscal{v,v} \geq 0$ pour tout $v \in V$, et si $\pscal{v,v}>0$ lorsque $v \neq 0$. Une forme bilinéaire symétrique définie positive est un \emph{produit scalaire}. 
\end{definition}

\begin{example}
  \label{exe:3}
  Soit $V = \R^n$. La forme bilinéaire symétrique
  \begin{displaymath}
    \pscal{u,v} = \sum_{i=1}^n u_iv_i 
  \end{displaymath}
  est un produit scalaire, appelé le \emph{produit scalaire ordinaire}. 
  Aussi, la forme bilinéaire  de l'exemple~\ref{ex:2} est un produit scalaire. 
\end{example}

\begin{definition}
  \label{def:5}
  Soit $\pscal{,}$ un produit scalaire. La \emph{longueur} ou la \emph{norme} d'un élément $v \in V$ est le nombre 
  \begin{displaymath}
    \| v \| = \sqrt{\pscal{v,v}}.
  \end{displaymath}
  Un élément $v \in V$ est un \emph{vecteur unitaire} si $\|v\| = 1$. 
\end{definition}


\begin{framed}\noindent 
  Pour le reste de ce paragraphe~\ref{sec:le-case-reel}, s'il n'est pas spécifié autrement,  $V$
  est toujours un espace vectoriel sur $\R$
  muni d'un produit scalaire. 
On appelle un  espace vectoriel sur $\R$ muni d'un produit scalaire  un \emph{espace euclidien}.  
\end{framed}



\begin{proposition}
  \label{prop:2}
  Pour $v \in V$ et $\alpha \in \R$ on a
  \begin{displaymath}
    \| \alpha \,v \| = |\alpha| \, \|v\|. 
  \end{displaymath}
\end{proposition}



\begin{proof}
  \begin{eqnarray*}    
   \| \alpha \,v \| & = &  \sqrt{\pscal{\alpha \, v, \alpha \, v} } \\
                    & = & \sqrt{\alpha^2 \pscal{v,v}} \\
                    & = & |\alpha | \, \|v\|. 
  \end{eqnarray*}
\end{proof}

\begin{proposition}[Théorème de Pythagore]
\label{prop:4}
Si $v$ et $w$ sont perpendiculaires 
\begin{displaymath}
  \|v+w\|^2 = \|v\|^2 + \|w\|^2. 
\end{displaymath}  
\end{proposition}


\begin{proof}
  \begin{eqnarray*}
     \|v+w\|^2 &= & \pscal{v+w,v+w} \\
               & = & \pscal{v,v+w} + \pscal{w,v+w} \\
               & = & \pscal{v,v} + \pscal{v,w} + \pscal{w,v} + \pscal{w,w} \\
               & = & \|v\|^2 + \|w\|^2
  \end{eqnarray*}
\end{proof}

\begin{proposition}[Règle du parallélogramme] Pour tous $v$ et $w$, on a 
  \begin{displaymath}
    \|v+w\|^2 + \|v-w\|^2 = 2 \|v\|^2 + 2 \|w\|^2.  
  \end{displaymath}  
\end{proposition}



Soit $V$ un espace vectoriel sur un corps $K$ muni d'une forme bilinéaire symétrique  $\pscal{,}$. 
Si $w$ est un élément de $V$ tel que $\pscal{w,w} \neq 0$, pour  tout $v \in V$, il existe un élément unique $\alpha \in K$ tel que $\pscal{w,v -\alpha\, w} = 0$. 

En fait,
\begin{displaymath}
  \pscal{w,v -\alpha\, w} = \pscal{w,v} - \alpha \pscal{w,w}. 
\end{displaymath}
Alors $\pscal{w,v -\alpha\, w} = 0$ si et seulement si $\alpha = \pscal{v,w} / \pscal{w,w}$. 


\begin{definition}
  \label{def:6}
  Soit $V$ un espace euclidien. 
  Soit $w \in V \setminus \{0\}$. Pour $v \in V$, soit $\alpha = \pscal{v,w} / \pscal{w,w}$.  Le nombre $\alpha$ est la \emph{composante} de $v$ sur $w$, ou \emph{le coefficient de Fourier de $v$ relativement à $w$}. Le vecteur $\alpha \, w$ s'appelle la \emph{projection} de $v$ sur $w$. 
\end{definition}


\begin{example}
  \label{exe:5}
  Soit $V$ l'espace vectoriel de l'exemple~\ref{ex:2} et $f(x) = \sin kx$, où $k \in \N_{>0}$. Alors
  \begin{displaymath}
   \|f\| = \sqrt{\pscal{f,f}} =  \sqrt{\int_0^{2\pi} \sin^2 kx \, dx} = \sqrt{\pi}
  \end{displaymath}
Si $g$ est une fonction quelconque, continue sur $[0,2\,\pi]$, le coefficient de Fourier de $g$ relativement à $f$ est 
\begin{displaymath}
  \pscal{f,g} /   \pscal{f,f}  = \frac{1}{\pi} \int_0^{2\, \pi} g(x) \sin kx \,dx. 
\end{displaymath}
\end{example}



\begin{theorem}[Inégalité de Cauchy-Schwarz]
  Pour tous $v,w \in V$, on a  
  \begin{displaymath}
    |\pscal{v,w}| \leq \|v\| \, \|w\|.
  \end{displaymath}
\end{theorem}


\begin{proof}
  Si $w=0$, les deux termes de cette inégalité sont nuls et elle devient évidente. Supposons maintenant que $w \neq 0$. Si $\alpha = \pscal{v,w} / \pscal{w,w}$ est la composante de $v$ sur $w$, $v - \alpha w$ est perpendiculaire à $w$, donc aussi à $\alpha\,w$. D'après le théorème de Pythagore, on trouve 
  \begin{eqnarray*}
    \|v\|^2 & = & \|v - \alpha \,w\|^2 + \|\alpha \, w \|^2 \\
            & ≥ &   \alpha^2 \|w\| ^2 \\ 
            & = &     \pscal{v,w}^2    / \|w\| ^2. 
  \end{eqnarray*}
Cela implique 
\begin{displaymath}
   \left| \pscal{v, w } \right| \leq \|v\| \|w\|.
\end{displaymath}
\end{proof}




\begin{theorem}[Inégalité triangulaire]
  \label{thr:1}
  Si $v,w \in V$. 
  \begin{displaymath}
    \|v+w\| \leq \|v\| + \|w\|. 
  \end{displaymath}
\end{theorem}


\begin{proof}
  \begin{eqnarray*}
    \|v+w\|^2 & =     & \|v\|^2 + 2 \pscal{v,w} + \|w\|^2 \\
              & \leq & \|v\|^2 + 2 \|v\|\,\|w\| + \|w\|^2 \\
              & = & (\|v\| + \|w\|)^2,
  \end{eqnarray*}
en recourant à l'inégalité de Cauchy-Schwarz. 
\end{proof}


\begin{lemma}
  \label{lem:2}
  Soit $V$ un espace euclidien  et 
  soient $v_1,\dots,v_n$ des éléments de V, deux à deux orthogonaux, tels que $\pscal{v_i,v_i}\neq 0$ pour tout $i$, et soit $a_1,\dots,a_n \in \R$.
  %Soit $\alpha_i = \pscal{v,v_i}/\pscal{v_i,v_i}$ la composante de $v$ sur $v_i$, alors le vecteur 
  Le vecteur 
\begin{displaymath}
  v - a_1v_1- \cdots - a_n v_n\, % \text{ où  }\, a_i \in ℝ 
\end{displaymath}
est perpendiculaire à tous les $v_1,\dots,v_n$ si et seulement si $a_i$ est la composante de $v$ sur $v_i$, c'est-à-dire $a_i=  \pscal{v,v_i}/ \pscal{v_i,v_i}$ pour tout $i$. 
\end{lemma}

\begin{proof}
  Pour le vérifier, il suffit d'en faire le produit scalaire avec $v_j$ pour tout $j$. Tous les termes $\pscal{v_i,v_j}$ donnent zéro si $i\neq j$. Le reste
  \begin{displaymath}
    \pscal{v,v_j} - a_j\pscal{v_jv_j}
  \end{displaymath}
s'annule si et seulement si $a_j = \pscal{v,v_j}/\pscal{v_jv_j}$. 
\end{proof}


 % \begin{definition}
 %   \label{def:3}
 %   Soit $V$ un espace vectoriel muni d'un produit scalaire. Soit $\{v_1,\dots,v_n\}$ une base de $V$. On dira que cette base est \emph{orthogonale} si $\pscal{v_i,v_j}=0$ pour tout $i \neq j$. 
 % \end{definition}

\begin{notation}
  Soient $V$ un espace vectoriel et $v_1,\dots,v_n \in V$. Le sous-espace engendré par $v_1,\dots,v_n$ est dénoté par  $\spa\{v_1,\dots,v_n\}$. 
\end{notation}



\begin{theorem}[Le procédé d'orthogonalisation de Gram-Schmidt]
  \label{thr:2}
  Soient $V$ un espace euclidien et  $\{v_1,\dots,v_n\} \subseteq V$
  un ensemble libre.  
  Il existe un ensemble libre orthogonal $\{u_1,\dots,u_n\}$
  de $V$
  tel que pour tout $i$, 
  $\{v_1,\dots,v_i\}$
  et $\{u_1,\dots,u_i\}$ engendrent le même sous-espace de $V$.
\end{theorem}


\begin{proof}
  On montre le théorème par induction.  On met $u_1 = v_1$
  et on suppose qu'on a construit $\{u_1,\dots,u_{i-1}\}$
  pour $i \geq 2$.
  L'ensemble $\{u_1,\dots,u_{i-1},v_i\}$
  est libre et une base du sous-espace engendré
  par $\{v_1,\dots,v_i\}$.
  On met  
  \begin{displaymath}
    u_i = v_i - \alpha_{1,i}  u_1 - \cdots - \alpha_{i-1,i} u_{i-1}
  \end{displaymath}
  où les $\alpha_{j,i}$
  sont les composantes de $v_i$
  sur $u_j$.
  Comme ça 
  \begin{eqnarray*}
    \spa \{u_1,\dots,u_i\} & = & \spa\{u_1,\dots,u_{i-1},v_i\} \\
                           &=  &\spa\{v_1,\dots,v_i\}.
  \end{eqnarray*}
  Surtout $\{u_1,\dots,u_i\}$ est un ensemble orthogonal.  
\end{proof}



\begin{exercise}
  \label{exe:6}
  Est-ce qu'il faut vraiment supposer que le produit scalaire
  $\pscal{.}$
  soit réel et défini positif et sur $\R$ pour ce procédé? Peux-tu imaginer une condition plus
  faible et satisfaite par le produit scalaire qui permet
  le procédé de Gram-Schmidt? 
\end{exercise}



\begin{definition}
  \label{def:7}
  Une base $\{u_1,\dots,u_n\}$
  d'un espace euclidien est \emph{orthonormale} si elle est
  orthogonale et se compose de vecteurs tous unitaires.
\end{definition}




\begin{corollary}
  \label{co:1}
  Soit $V$
  un espace euclidien de dimension finie. Supposons $V \neq
  \{0\}$. $V$ possède alors une base orthonormale.
\end{corollary}

\begin{proof}
  Soient $\{v_1,\dots,v_n\}$ une base de $V$ et $\{u_1,\dots,u_n\}$ le résultat du procédé Gram-Schmidt appliqué à $\{v_1,\dots,v_n\}$. Alors $\{u_1/ \|u_1\|,\dots,u_n/\|u_n\|\}$ est une base orthonormale de $V$. 
\end{proof}


\begin{example}
\label{exe:7}
Trouver une base orthonormale de l'espace vectoriel engendré par 
\begin{displaymath}
  \begin{pmatrix}
    1\\1\\0\\1
  \end{pmatrix}, 
  \begin{pmatrix}
    1\\-2\\0\\0
  \end{pmatrix}
  \text{ et } 
  \begin{pmatrix}
    1\\0\\-1\\2
  \end{pmatrix}
\end{displaymath}

Notons $A,B$ et $C$ les vecteurs. Soit $A'=A$ et 
\begin{displaymath}
  B' = B - \frac{A'\cdot B}{A'\cdot A'} \cdot A'
\end{displaymath}
On trouve 
\begin{displaymath}
  B' = \frac{1}{3}
  \begin{pmatrix}
    4\\-5\\0\\1
  \end{pmatrix}
\end{displaymath}
On calcule 
\begin{displaymath}
  C' = C - \frac{A'\cdot C}{A'\cdot A'} \cdot A' - \frac{B'\cdot C}{B'\cdot B'} \cdot B'
\end{displaymath}
et on trouve 
\begin{displaymath}
  C' = \frac{1}{7}
  \begin{pmatrix}
    -4\\-2\\-7\\6
  \end{pmatrix}  
\end{displaymath}
La base orthonormale est
\begin{displaymath}
  A' / \|A'\| = \frac{1}{\sqrt{3}}
  \begin{pmatrix}
    1\\1\\0\\1
  \end{pmatrix},
    B'/ \|B'\| = \frac{1}{\sqrt{42}}
  \begin{pmatrix}
    4\\-5\\0\\1
  \end{pmatrix}
\text{ et }
C'/ \|C'\| = \frac{1}{\sqrt{105}}
  \begin{pmatrix}
    -4\\-2\\-7\\6
  \end{pmatrix}  
\end{displaymath}
\end{example}

\begin{corollary}
  \label{co:2}
  Soit $A \in \R^{m\times n}$ une matrice de rang (colonne) plein. On peut factoriser $A$ comme 
  \begin{displaymath}
    A = A^* \cdot R
  \end{displaymath}
où les colonnes de  $A^* \in \R^{m\times n}$ sont deux à deux orthonormales et $R \in \R^{n \times n}$ est une matrice triangulaire supérieure dont les valeurs diagonales sont positives. 
\end{corollary}

\begin{proof}
Comme $\rank(A) = n$, les colonnes de $A$ sont libres ; dès lors on peut appliquer le procédé de Gram-Schmidt à $\{a_1, \dots , a_n \} $ où $a_j$ désigne la j-ième colonne de $A$. On obtient alors une base orthogonale $B=  \{a'_1, \dots , a'_n\}$ avec la relation :
\begin{displaymath}
a_1=a'_1
~~,~~
a_j = \sum_{i=1}^{j-1} \alpha_{i,j} a'_i +a'_j
\end{displaymath}
pour tout $j \in \{2, \dots,n\}$, et où $\alpha_{i,j} $ est le coefficient de Fourier de $a_j$ relativement à $a'_i$.
Grâce à ce procédé, on a pu écrire $a_j$ comme une combinaison linéaire de $\{a'_1, \dots , a'_n\} $. On peut représenter cela avec un produit matrice-vecteur:
\begin{displaymath}
a_j= (a'_1 ~ \dots ~ a'_n)
\begin{pmatrix}
\alpha_{1,j}\\\alpha_{2,j}\\\vdots\\\alpha_{j-1,j}\\1\\0\\\vdots\\0\\
\end{pmatrix}.  
%~~~~~~A' \in \R^{m\times n}
%~~~~~~s_j \in \R^{n}
\end{displaymath}

En posant 
\begin{displaymath}
S = %:= (s_1 ~ \dots ~ s_n)= 
\begin{pmatrix}
1 & \alpha_{1,2} & \alpha_{1,3} & \cdots & \alpha_{1,n-1} & \alpha_{1,n}\\
0 & 1 & \alpha_{2,3} & \cdots & \alpha_{2,n-1} & \alpha_{2,n} \\
\vdots && \ddots &&& \vdots \\
\vdots &&& \ddots &&\vdots\\
0 & \cdots & \cdots&0& 1 & \alpha_{n-1,n}\\
0 & 0 & \cdots &\cdots & 0 & 1
\end{pmatrix} \in \R^{n \times n}, 
\end{displaymath}
on a, par les propriétés du produit matriciel,
\begin{displaymath}
A = A'S.  %=(A'_1s_1 ~\dots ~ A'_1s_n)=(a_1 ~ \dots ~ a_n)=A
\end{displaymath}
Il nous faut encore normaliser les colonnes de la matrice $A'$. Pour cela, on définit les deux matrices diagonales suivantes:
\begin{displaymath}
D=\begin{pmatrix}
1/\|a'_1\| & & \\
& \ddots & \\
& & 1/\|a'_n\|
\end{pmatrix} , \quad 
D^{-1}=\begin{pmatrix}
\|a'_1\| & & \\
& \ddots & \\
& & \|a'_n\|
\end{pmatrix} 
~~ D,D^{-1} \in \R^{n \times n}
\end{displaymath}
En posant $A^*=A'_1D$ et $R=D^{-1}S$ on obtient:
\begin{displaymath}
A=A^* R
\end{displaymath}
où $A^* \in \R^{m\times n}$ et $R \in \R^{n \times n}$ sont des matrices qui vérifient les propriétés de l'énoncé. 
\end{proof}


\begin{example}
  \label{exe:8}
  Trouver une  factorisation $Q,R$ du Corollaire~\ref{co:2} de la matrice 
  \begin{displaymath}
    \left(\begin{matrix}1 & 1 & 1\\1 & -2 & 0\\0 & 0 & -1\\1 & 0 & 2\end{matrix}\right)
  \end{displaymath}

On trouve 
\begin{displaymath}
  \left[\begin{matrix}1 & 1 & 1\\1 & -2 & 0\\0 & 0 & -1\\1 & 0 & 2\end{matrix}\right]
= \left[\begin{matrix}1 & \frac{4}{3} & - \frac{4}{7}\\1 & - \frac{5}{3} & - \frac{2}{7}\\0 & 0 & -1\\1 & \frac{1}{3} & \frac{6}{7}\end{matrix}\right] 
\left[\begin{matrix}1 & - \frac{1}{3} & 1\\0 & 1 & \frac{3}{7}\\0 & 0 & 1\end{matrix}\right]
\end{displaymath}
et alors
\begin{displaymath}
   \left[\begin{matrix}1 & 1 & 1\\1 & -2 & 0\\0 & 0 & -1\\1 & 0 & 2\end{matrix}\right] = 
\left[\begin{matrix}\frac{1}{\sqrt{3}} & \frac{2 \sqrt{42}}{21} & - \frac{4 }{\sqrt{105}}\\\frac{1}{\sqrt{3}} & - \frac{5}{\sqrt{42}} & - \frac{2 }{\sqrt{105}}\\0 & 0 & -\frac{\sqrt{105}}{15}\\\frac{1}{\sqrt{3}} & \frac{1}{\sqrt{42}} & \frac{2 \sqrt{105}}{35}\end{matrix}\right] 
\left[\begin{matrix}\sqrt{3} & - \frac{\sqrt{3}}{3} & \sqrt{3}\\0 &\frac{\sqrt{42}}{3} & \frac{\sqrt{42}}{7}\\0 & 0 & \frac{\sqrt{105}}{7}\end{matrix}\right]
\end{displaymath}
\end{example} 




\begin{theorem}[Inégalité de Bessel]
  \label{thr:4}
  Si $v_1,\dots,v_n$ sont des vecteurs unitaires deux à deux orthogonaux et si $\alpha_i = \pscal{v,v_i}$ sont les coefficients de Fourier de $v$ relativement à $v_i$ alors 
  \begin{displaymath}
    \sum_{i=1}^n \alpha_i^2 \leq \|v\|^2. 
  \end{displaymath}
\end{theorem}

\begin{proof}
  \begin{eqnarray*}
    0 & \leq & \pscal{v - \sum_{i=1}^n \alpha_i v_i , v - \sum_{i=1}^n \alpha_i v_i} \\
     & = & \pscal{v,v} - 2 \cdot \sum \alpha_i \pscal{v,v_i} + \sum \alpha_i^2 \\
    & = & \pscal{v,v}  - \sum \alpha_i^2 \\
  \end{eqnarray*}
\end{proof}



\subsection*{Exercices}

\begin{enumerate}

\item Soit $V$ un espace vectoriel sur $ℝ$ de dimension fini, muni d'une forme bilinéaire $\pscal{.}$. Soit $B = \{b_1,\dots,b_n\}$ une base orthogonale et  $U = \spa \{ b_i :  i=1,\dots,n, \, 〈b_i,b_i〉 >0\}$. Montrer que $\pscal{.}$ restreint à $U$ est un produit scalaire du sous-espace $U$. 
\item Soient $V$ un espace vectoriel muni d'une forme bilinéaire symétrique $\pscal{.}$ et $\{v_1,\dots,v_n\} \subseteq V$ un ensemble de
vecteurs deux à deux orthogonaux. 
  \begin{enumerate}[a)]
  \item   Montrer que $\{v_1,\dots,v_n\}$ est un ensemble libre si pour tout $i$, 
 $\pscal{v_i,v_i} \neq  0$.  
\item  Donner un contre-exemple ou une démonstration de la réciproque. 
\end{enumerate}


\item Considérant l'exemple~\ref{ex:2}, montrer que l'ensemble 
  \begin{displaymath}
    \{1,\sin x, \cos x, \sin(2x), \cos(2x), \sin(3x), \cos(3x), \dots\}
  \end{displaymath}
 est un ensemble de 
vecteurs deux à deux orthogonaux. 
\item Trouver la factorisation $Q\cdot R$  du Corollaire~\ref{co:2} de la matrice 
  \begin{displaymath}
    \begin{pmatrix}
      1 & 1 & 0 &0 \\
      0& 1 & 1 & 0\\
      0 & 0 & 1 & 1\\
      1 & 0 & 0 & 1\\
    \end{pmatrix}
  \end{displaymath}
  Trouver la factorisation de la matrice $n\times n$ 
  \begin{displaymath}
    \begin{pmatrix}
      1 & 1 & 0 & 0 & \cdots & 0\\
      0 & 1 & 1 & 0 & \cdots & 0 \\
      && \vdots &&\\
      0 & \cdots & \cdots&0& 1 & 1\\
      1 & 0 & \cdots &\cdots & 0 & 1
    \end{pmatrix}
  \end{displaymath}
\item Trouver une forme bilinéaire symétrique de $\R^n$ telle qu'il existe des vecteurs $u,v \in \R^n$ avec $\pscal{u,u}<0$ et $\pscal{v,v}>0$. 
\item Soit $V$ un espace vectoriel sur $\R$ muni d'une forme bilinéaire symétrique. S'il existe des vecteurs $u,v \in V$ tels que $\pscal{u,u}<0$ et $\pscal{v,v}>0$, il existe un vecteur $w \neq 0$ tel que $\pscal{w,w}=0$. 
\item Montrer que l'inégalité de Bessel (Théorème \ref{thr:4}) est une égalité si $v$ est dans le sous-espace engendré par les $v_1,\dots,v_n$. 
\item On considère l'espace euclidien des fonctions continues sur l'intervalle $[0,1]$ muni de la forme bilinéaire symétrique 
  \begin{displaymath}
    \pscal{f,g} = \int_0^1 f(x)g(x) \, dx. 
  \end{displaymath}
  \begin{enumerate}[i)]
  \item Soit $V$ le sous-espace engendré par $f(x) = x$ et $g(x) = x^2$. Trouver une base orthonormale de $V$. 
  \item Soit $V$ le sous-espace engendré par $\{1,x,x^2\}$. Trouver une base orthonormale de $V$. 
  \end{enumerate}
\item Soient $V$ un espace euclidien, $\{u_1,\dots,u_n\}$ un ensemble orthonormal et $f,g \in \spa\{u_1,\dots,u_n\}$. Montrer l'\emph{identité de Parseval}
  \begin{displaymath}
    \pscal{f,g} = \sum_i \pscal{f,u_i}\pscal{u_i,g}. 
  \end{displaymath}
\end{enumerate}


\section{La méthode des moindres carrées} 
\label{sec:le-methode-des}

Soient $A \in \R^{m \times n} $ et $b \in \R^m$ et supposons  que le système des équations linéaires 
\begin{equation}
  \label{eq:2}
  Ax = b
\end{equation}
n'a pas de solution. Dans beaucoup d'applications, on cherche un $x \in \R^n$ tel que la distance entre  $Ax$ et $b$ est \emph{minimale}. On aimerait alors résoudre le problème d'optimisation suivant
\begin{equation}
  \label{eq:3}
  \min_{x \in \R^n} \|Ax - b\|. 
\end{equation}

\begin{framed}\noindent 
  Pour le reste de ce paragraphe~\ref{sec:le-methode-des}, s'il n'est pas spécifié autrement, $V$ est toujours un espace euclidien.
\end{framed} 


\begin{theorem}
  \label{thr:3}
  Soient $v_1,\dots,v_n$ des vecteurs deux à deux orthogonaux et tels que $\|v_i\|>0$ pour tout $i$. Soit $v$ un élément de $V$ et soit $\alpha_i = \pscal{v,v_i}/\pscal{v_i,v_i}$ la composante de $v$ sur $v_i$. Pour $a_1,\dots,a_n \in \R$ alors 
  \begin{displaymath}
    \left\| v - \sum_{i=1}^n \alpha_iv_i \right\|  \leq \left\| v - \sum_{i=1}^n a_iv_i \right\|.
  \end{displaymath}
De plus, l'inégalité au-dessus est une égalité si et seulement si $a_i = \alpha_i$ pour tout $i$. 
Alors $\sum_{i=1}^n \alpha_iv_i$ est l'unique  meilleure approximation de $v$ par un vecteur du sous-espace engendré par les $v_1,\dots,v_n$. 
\end{theorem}


\begin{proof}
  \begin{eqnarray*}
    \|v- \sum_{i=1}^n a_iv_i \|^2 & = &  \|v - \sum_{i=1}^n \alpha_iv_i  - \sum_{i=1}^n (a_i - \alpha_i)v_i \|^2 \\
                           & = & \|v - \sum_{i=1}^n \alpha_iv_i \|^2 + \| \sum_{i=1}^n (a_i - \alpha_i)v_i \|^2
  \end{eqnarray*}
en utilisant le lemme~\ref{lem:2} et le théorème de Pythagore. 
\end{proof}

\noindent 
Maintenant nous pouvons décrire un \emph{algorithme} pour résoudre le problème suivant. 
\begin{framed}
  \noindent 
  Soient $v,v_1,\dots,v_n \in V$, trouver $u \in \spa \{v_1,\dots,v_n\}$ tel que la distance 
  \begin{displaymath}
    \|v - u\|
  \end{displaymath}
  est minimale. 
\end{framed}

\begin{algorithm}
\label{alg:2}

~\\
\begin{enumerate}[i)]
\item Trouver une base orthonormale $\{u_1,\dots,u_k\}$ du sous-espace $\spa\{v_1,\dots,v_n\}$ 
  avec le procédé de Gram-Schmidt. 
\item Retourner $ u = \sum_{i=1}^k \pscal{v,u_i} u_i$. 
\end{enumerate}
\end{algorithm}






\begin{theorem}
  \label{thr:6}
  Soient $A \in \R^{m\times n}$ et $b \in \R^m$. Les solutions du système
  \begin{equation}
    \label{eq:4}    
    A^TAx = A^T b
  \end{equation}
  sont les solutions  optimales du problème \eqref{eq:3} (où l'on considère la norme euclidienne sur $\R^m$)
\end{theorem}

\begin{proof}
  Soit $\{a^*_1,\dots,a^*_k\}$ une base orthonormale du sous-espace $\mathrm{Col}(A)=\{Ax \colon x \in \R^n\}$ engendré par les colonnes de $A$. Le théorème~\ref{thr:3} implique que  les solutions du problème \eqref{eq:3} sont les solutions du système 
  \begin{displaymath}
    A x = \sum_i \pscal{b,a^*_i} \cdot a^*_i. 
  \end{displaymath}
Le lemme~\ref{lem:2} implique que 
\begin{displaymath}
  b - \sum_i \pscal{b,a^*_i} \cdot a^*_i
\end{displaymath}
est perpendiculaire à tout $a_i^*$ et dès que les $a^*_i$ engendrent 
$\mathrm{Col}(A)$ 
on a 
\begin{displaymath}
  A^T (Ax - b) = 0
\end{displaymath}
pour toute solution optimale $x$ de \eqref{eq:3}. 


Maintenant, soit $x$ une solution du système~\eqref{eq:4}. 
Alors $ A x -b$ est 
perpendiculaire à tous les $a_i^*$. Le seul vecteur $v \in \spa\{a_1^*,\dots,a_k^*\} = \mathrm{Col}(A)$ tel que $\pscal{b-v,v}=0$  est $v = \sum_i\pscal{a_i^*,b} \cdot a_i^*$. Ceci démontre le théorème.  
\end{proof}

\begin{remark}
Pour une norme $\lvert\lvert \cdot \rvert\rvert$ quelconque engendrée par un produit scalaire $\langle \cdot , \cdot \rangle$), une preuve similaire montre que les solutions du système
  \begin{equation*}  
    A^TF^{\langle \cdot , \cdot \rangle}Ax = A^T F^{\langle \cdot , \cdot \rangle} b
  \end{equation*}
  sont les solutions  optimales du problème \eqref{eq:3}, où $F^{\langle \cdot , \cdot \rangle}$ est la matrice du produit scalaire $\langle \cdot , \cdot \rangle$ selon la base canonique (i.e. $(F^{\langle \cdot , \cdot \rangle})_{i,j} = \langle e_i , e_j \rangle$).
\end{remark}

\begin{example}
\label{exe:4}
Trouver une solution de moindre carrées sur les données 
\begin{displaymath}
  A =
  \begin{pmatrix}
    4 & 0 \\ 0 &2 \\ 1 & 1
  \end{pmatrix}
\text{ et }  b =
\begin{pmatrix}
  2\\0\\11
\end{pmatrix}
\end{displaymath}  

\begin{displaymath}
  A^T A =
  \begin{pmatrix}
    17 & 1 \\
    1 & 5
  \end{pmatrix} \text{ et } A^T b =
  \begin{pmatrix}
    19 \\ 11
  \end{pmatrix}. 
\end{displaymath}
La solution du système 
\begin{displaymath}
   \begin{pmatrix}
    17 & 1 \\
    1 & 5
  \end{pmatrix}
  \begin{pmatrix}
    x_1\\x_2
  \end{pmatrix}
= \begin{pmatrix}
    19 \\ 11
  \end{pmatrix}.
\end{displaymath}
est $x^* = (1,2)^T$. 
\end{example}








\section{Formes linéaires, bilinéaires et l'espace dual}
\label{sec:lespace-dual}

Soient $V$ un espace vectoriel sur un corps $K$ et $V^*$ l'ensemble  des  applications linéaires de $V$ dans $K$, où on considère $K$ comme espace vectoriel de dimension $1$ sur lui-même. Clairement, $V^*$ est un espace vectoriel lui-même. 
\begin{definition}
\label{def:9}
  L'ensemble des  applications linéaires  $φ : V ⟶ K$ est noté $V^*$ et, muni de l'addition et de la multiplication scalaire usuelles, est appelé l'\emph{espace dual} de $V$. Les éléments de $V^*$ sont appelés \emph{formes linéaires}.
\end{definition}

\begin{remark}
  \label{def:10}
  Soit $V$ un espace vectoriel sur un corps $K$. Une application 
  \begin{displaymath}
    f\colon V \times V \longrightarrow K
  \end{displaymath}
  est une forme bilinéaire si et seulement si pour tout $v \in V$,
  les applications $g,h: V \longrightarrow K$
  telles que $g(x) = f(v,x)$
  et $h(x) = f(x,v)$ sont des formes linéaires.
\end{remark}

Si $V$ est de dimension finie et si $B = \{v_1,\dots,v_n\}$ est une base de $V$, l'image d'un vecteur $x = \sum_i\alpha_i v_i$ par une forme linéaire $f$ est 
\begin{eqnarray*}
    f(x) & = &  f\left(\sum_i\alpha_i v_i\right) \\
         & = & \sum_i \alpha_i f(v_i)  \\
         & = & (f(v_1),\dots,f(v_n)) [x]_B,               
\end{eqnarray*}
où $[x]_B = (\alpha_1,\dots,\alpha_n)^T$ sont les coordonnées de $x$ dans la base $B$.  





\begin{lemma}
  \label{lem:3} Supposons que $V$ est de dimension finie et $\{v_1,\dots,v_n\}$ est une base de $V$. 
  La fonction $\phi_j\colon V \longrightarrow K$ 
    \begin{displaymath}
  \phi_j \left(\sum_i \alpha_i v_i \right) =  \alpha_j 
\end{displaymath}
est une forme linéaire. 
\end{lemma}
\begin{proof}
  Immédiate. 
\end{proof}

\begin{theorem}
  \label{thr:7}
  Les formes linéaires $\{\phi_j \in V^* \colon j=1,\dots,n\}$ du lemme~\ref{lem:3} précédent  forment une base de $V^*$.
\end{theorem}

\begin{proof}
  Si, pour $\beta_i \in K$, on a $\sum_i \beta_i \phi_i = 0$, alors 
  \begin{displaymath}
 0 =    \left(\sum_i \beta_i \phi_i\right) (v_j) =  \beta_j,
  \end{displaymath}
c'est-à-dire  les $\phi_j$ sont linéairement indépendantes. Les $\phi_j$ engendrent $V^*$ puisque pour $f \in V^*$, $f = \sum_i  f(v_i)  \, \phi_i$. 
\end{proof}

\begin{definition}
  \label{def:14}
  La base $\{\phi_1,\dots,\phi_n\}$ est la \emph{base duale} de la base $\{v_1,\dots,v_n\}$. 
\end{definition}




\begin{lemma}
  \label{lem:5}
  Soit $V$
  un espace vectoriel sur le corps $K$
  de dimension finie muni d'une forme bilinéaire  non dégénérée et soit
  $f\colon V \longrightarrow K$
  une forme linéaire. Il existe un $v \in V$
  tel que $f(x) = \pscal{v,x}$ pour tout $x \in V$.
\end{lemma}

\begin{proof}
  Soient $B = \{v_1,\dots,v_n\}$ une base  de $V$ et $A_B^{\pscal{}} \in K^{n\times n}$ la matrice dans la base $B$ associée à la forme bilinéaire. Soit $a \in K^n$ tel que  $f(x) = a^T[x]_B$ pour tout $x \in V$. Dès que $A_B^{\pscal{}}$ est de rang plein (Proposition~\ref{prop:7}), alors il existe $v \in V$ tel que $[v]_B^TA_B^{\pscal{}} = a^T$. Ceci revient à résoudre un système d'équations linéaires (cf semestre 1) et comme la matrice $A_B^{\pscal{}}$ est de rang plein, on a l'existence (et même l'unicité) d'une solution, i.e.,  $[v]_B^TA_B^{\pscal{}} = a^T$. Ainsi 
$f(x) = \pscal{v,x}$ pour tout $x \in V$. 
\end{proof}




\begin{theorem}[Supplémentaire orthogonal]
  \label{thr:8}
  Soient $V$ un espace vectoriel de dimension finie sur corps $K$ et $W$ un sous-espace de $V$. Soit $\pscal{.}$ une forme bilinéaire symétrique tel que, si restreint sur $W\times W$, elle est non dégénérée. Alors $V = W \oplus W^\perp$.  
\end{theorem}

\begin{proof}
  Pour un élément $u \in W \cap W^\perp$ on a $\pscal{u,w} = 0$ pour tout $w \in W$. Dès que $\pscal{.}$ est non dégénéré sur $W$ on a $u=0$, alors $W \cap W^\perp = \{0\}$. 

Il reste à démontrer que $V = W + W^\perp$. Pour $v \in V$ le lemme~\ref{lem:5} implique qu'il existe un $w_0 \in W$ tel que pour tout $u \in W$,  $\pscal{u,v} = \pscal{u,w_0}$ et ça démontre   $v - w_0 \in W^\perp$ et alors $v \in W + W^\perp$. 
\end{proof}

\subsection*{Exercices} 

\begin{enumerate}
\item Soient $V$ un espace vectoriel de dimension finie,  $f: V \longrightarrow K$ une forme linéaire et $B,B'$ des bases de $V$. Soit 
  \begin{displaymath}
    f(x) = a^T [x]_B 
  \end{displaymath}
  où $a \in K^n$. 
  Décrire $f(x)$ en termes de $P_{B'B}$ et $[x]_{B'}$. 
\item Soient $V$ un espace vectoriel de dimension finie,  $f: V\times V \longrightarrow K$ une forme bilinéaire et $B,B'$ des bases de $V$. Soit 
  \begin{displaymath}
    f(x,y) = [y]_B ^T A_B^f [y]_B.  
  \end{displaymath}
  Décrire $f(x,y)$ en termes de $P_{B'B}$, $[x]_{B'}$, et $[y]_{B'}$. 
\item On considère les vecteurs 
  \begin{displaymath}
    v_1 =
    \begin{pmatrix}
      1\\1\\0\\0
    \end{pmatrix} \text{ et }
 v_2 =
    \begin{pmatrix}
      0\\1\\1\\0
    \end{pmatrix} \in \Z_2^4
  \end{displaymath}
et la forme bilinéaire standard. Trouver une base du $\spa\{v_1,v_2\}^\perp$. Est-ce que $\Z_2^4 = \spa\{v_1,v_2\} \oplus \spa\{v_1,v_2\}^\perp$? 
\item Soit $V \subseteq \R[x]$ l'espace euclidien des polynômes de degré au plus $n$ muni du produit scalaire  $\pscal{p,q} = \int_0^1p(x)q(x) \, dx$.  Décrire la matrice $A_B^{\pscal{}}$ pour $B = \{1,x,\dots,x^n\}$. 
\item Soit $V$ un espace vectoriel sur un corps $K$ et soient $f,g \in V^* \setminus \{0\}$ linéairement indépendants. Montrer que
  \begin{displaymath}
    \ker{f} \cap \ker{g} 
  \end{displaymath}
est de dimension $n-2$. 
\item Soit $V$ un espace vectoriel de dimension finie sur un corps $K$ et soit $\pscal{.}$ une forme bilinéaire symétrique. Exprimez $(W_1 +W_2)^\perp$ et $(W_1\cap W_2)^\perp$ en fonction de $W_1^\perp$ et $W_2^\perp$. 


\item Donner un exemple d'un espace vectoriel  $V$ muni d'un produit scalaire dégénéré et d'un sous-espace $W \subseteq V$  tel que $V$ n'est pas la somme directe de $W$ et $W^\perp$. 
\end{enumerate}








\section{Formes sesquilinéaires et produits hermitiens}
\label{sec:form-sesq-et}

Maintenant nous considérons le cas $K = ℂ$.
Il faut un peu modifier la définition d'un produit scalaire pour
obtenir des résultats similaires à ceux des sections précédentes.  Le
carré de la longueur d'un vecteur
\begin{displaymath}
  v =
  \begin{pmatrix}
    a_1 + i\cdot  b_1 \\
    \vdots \\
    a_n + i \cdot b_n
  \end{pmatrix} \in \C^n
\end{displaymath}
où $a_i,b_i \in \R$, est égal à 
\begin{displaymath}
  \sum_i (a_i^2 + b_i^2) = \sum_i (a_i + i b_i) \cdot (a_i - i b_i) = \sum_i v_i \cdot \overline{v_i}, 
\end{displaymath}
où $v_i = a_i + i \cdot b_i$ et $\bar{v_i}$ est la conjugaison de $v_i$. Ceci suggère la définition suivante. 

\begin{definition}
  \label{def:15}
  Soit $V$
  un espace vectoriel sur un corps $\C$ et 
  \begin{displaymath}
    \pscal{,} : V × V ⟶ ℂ 
  \end{displaymath}
  une correspondance qui à tout couple $(v,w)$
  d'éléments de $V$
  associe un nombre complexe, noté $\pscal{v,w}$. 
  En considérant 
  les propriétés suivantes: \medskip
  \begin{enumerate}[PH 1]
  \item On a $\pscal{v,w} = \overline{\pscal{w,v}}$ pour tout $v,w \in V$.  \label{ph1}
  \item Si $u,v$ et $w$ sont des éléments de $V$,  \label{ph2}
    \begin{displaymath}
      \pscal{u,v+w} = \pscal{u,v}+\pscal{u,w} \, \text{ et } \,  \pscal{v+w,u} = \pscal{v,u}+\pscal{w,u}
    \end{displaymath}
  \item Si $x \in \C$ et $u,v \in V$,  \label{ph3}
    \begin{displaymath}
       \pscal{x \cdot u , v} = x \pscal{u,v} \, \text{ et } \,   \pscal{u , x \cdot v} = \overline{x}\cdot  \pscal{u,v}.
    \end{displaymath}  
  \end{enumerate}
  \noindent
on dit que $\pscal{,}$ est 
\begin{enumerate}[i)]
\item une \emph{forme sesquilinéaire}, si $\pscal{,}$ satisfait 
  PH~\ref{ph2} et PH~\ref{ph3}. 
\item une \emph{forme hermitienne}, si $\pscal{,}$ satisfait 
  PH~\ref{ph1}, PH~\ref{ph2} et PH~\ref{ph3}. 
\item un \emph{produit hermitien}, si   $\pscal{,}$ satisfait 
  PH~\ref{ph1}, PH~\ref{ph2} et PH~\ref{ph3} et 
  \begin{displaymath}
    \pscal{v,v} >0, \text{ pour tout } v \in V \setminus \{0\}. 
  \end{displaymath}
\end{enumerate}

Une forme  sesquilinéaire  est \emph{non dégénérée à gauche} si la condition suivante est vérifiée. 
\begin{quote}
  Si $v \in V$ et si $\pscal{v,w}=0$ pour tout $w \in V$, alors $v = 0$. 
\end{quote}

\end{definition}


\begin{remark}
Si $\pscal{,} $  est une forme hermitienne, 
pour tout $v \in V$, on a $\pscal{v,v} \in \R$ dès que $\pscal{v,v} = \overline{\pscal{v,v}}$ par PH~\ref{ph1}.  On dit que la forme hermitienne   est \emph{définie positif} si $\pscal{v,v} >0$ pour tout $v \in V \setminus\{0\}$. Alors un produit hermitien est une forme hermitienne définie positif. 
\end{remark}

\begin{example}
  \label{exe:12}
  Le produit \emph{hermitien standard} de $\C^n$ 
  \begin{displaymath}
    \pscal{u,v} = \sum_i u_i \overline{v_i}
  \end{displaymath}
  satisfait les condition PH~\ref{ph1}-\ref{ph3} et est défini positif.  
\end{example}


Les notions d'\emph{orthogonalité, de perpendicularité, de base orthogonale} et \emph{de supplémentaire orthogonal}  sont définies comme avant. Par contre, les notions de \emph{coefficients de Fourier} et la \emph{projection de $v$ sur $w$} doivent etre modifiés: les coefficients de Fourier dans le cadre d'un $\C$-espace vectoriel sont les conjugués complexes des coefficients de Fourier de base.

\begin{example}
  \label{exe:13}
  Soit $V$ l'espace vectoriel des fonctions $f\colon \R \longrightarrow \C$   continues sur l'intervalle $[0, 2\pi]$. Pour $f,g \in V$ on pose
  \begin{displaymath}
    \pscal{f,g} = \int_0^{2\pi} f(x) \overline{g(x)} \, dx.
  \end{displaymath}
C'est un produit hermitien défini positif. Les fonctions $f_n (x)=  e^{inx}$ pour $n \in \Z$ sont orthogonales dès que 
\begin{displaymath}
  f_n(x) \overline{f_m(x)} = e^{i(n-m)x} = \cos((n-m) x) + i \cdot \sin((n-m) x)
\end{displaymath}
et alors 
\begin{displaymath}
  \pscal{f_n,f_n} = \int_0^{2 \pi} 1 \, dx = 2 \pi 
\end{displaymath}
et pour $n \neq m$ 
\begin{displaymath}
  \pscal{f_n,f_m} = \int_0^{2 \pi} \cos( (n-m) x) \, dx  + i \cdot \int_0^{2 \pi} \sin( (n-m) x) \, dx = 0.
\end{displaymath}
Pour $f \in V$ la composante de $f$ sur $f_n$, ou le coefficient de Fourier de $f$ relativement à $f_n$, est 
\begin{displaymath}
  \frac{\pscal{f,f_n}}{\pscal{f_n,f_n}} = \frac{1}{2\pi} \cdot \int_0^{2\pi} f(x) e^{-inx} \, dx. 
\end{displaymath}
\end{example}





Soient $V$ un espace vectoriel sur $\C$ de dimension finie et $f: V \times V \longrightarrow \C$ une forme sesquilinéaire. Pour une base  $B = \{v_1,\dots,v_n\}$ de $V$ et $x = \sum_i \alpha_i v_i$ et $y = \sum_i \beta_i v_i$ on a 
\begin{displaymath}
  \pscal{x,y} = \sum_{ij} \alpha_i\overline{\beta_j} f(v_i,v_j)
\end{displaymath}
et avec la matrice $A_B^f = (f(v_i,v_j))_{1 \leq i,j \leq n}$   alors 
\begin{equation}
  \label{eq:9}
  \pscal{x,y} = [x]_B ^T A_B^{f} \overline{[y]_B} 
\end{equation}
où pour un vecteur $v \in \C^n$ le vecteur $\overline{v}$ est tel que $(\overline{v})_i = \overline{(v_i)}$ pour tout $i$. Pour une matrice $A \in \C^{m \times n}$, $\overline{A} \in \C^{m \times n}$ est la matrice telle que 
$\left(\overline{A}\right)_{ij} = \overline{(A_{ij})}$ pour out $i,j$. 
\begin{definition}
  \label{def:17}
  Une matrice $A \in \C^{n \times n}$ est appelée \emph{hermitienne} si on a 
  \begin{displaymath}
    A = \overline{A^T}. 
  \end{displaymath}
\end{definition}


\begin{proposition}
  \label{prop:3}
  Soit  $V$  un espace vectoriel sur $\C$ de dimension finie et soit $B$  une base de $V$. Une forme sesquilinéaire $f$ est une forme hermitienne si et seulement si $A_B^f$ est hermitienne.  
\end{proposition}


\begin{definition}
  \label{def:18}
  Deux matrices $A,B \in \C^{n \times n}$ sont \emph{congruentes complexes} s'il existe une matrice inversible $P \in \C^{n \times n}$ telle que $A = {P^T} \cdot B \cdot \overline{P}$. Nous écrivons $A \cong_\C B$.  
\end{definition}
$\cong_\C$ est aussi une relation d'équivalence. On peut aussi modifier l'algorithme~\ref{alg:1} tel qu'il calcule une matrice diagonale congruente complexe par rapport à une matrice hermitienne $A \in \C^{n \times n}$,  voir l'exercice~\ref{item:7}. Alors on a le théorème suivant. 

\begin{theorem}
  \label{thr:11}
  Soit $V$ un espace vectoriel sur $\C$ de dimension finie, muni d'une forme  hermitienne. Alors $V$ possède une base orthogonale. 
\end{theorem}
\begin{proof}
  Soit $B = \{v_1,\dots,v_n\}$  une base de $V$. Pour $x,y \in V$ on a 
  \begin{eqnarray*}
    \pscal{x,y} & = & [x]_B^T A_B^{\pscal{.}} \overline{[y]_B}. 
  \end{eqnarray*}
  Soit $P \in \C^{n\times n} $ inversible telle que
  \begin{displaymath}
    P^T A \overline{P} =
    \begin{pmatrix}
      c_1\\
      & \ddots \\
      && c_n
    \end{pmatrix}. 
  \end{displaymath}
  La base orthogonale est $w_1,\dots,w_n$ dont $ [w_j]_B$ est  la $j$-ème colonne de $P$, 
  \begin{displaymath}
     [w_j]_B =
  \begin{pmatrix}
    p_{1j}\\ \vdots \\ p_{nj}
  \end{pmatrix}, 
  \end{displaymath} 
alors $w_j = \sum_{i=1}^n p_{ij} v_i$. 

\end{proof}



\begin{example}
  \label{exe:16}
  On considère la matrice hermitienne 
  \begin{displaymath}
    A = \left[\begin{matrix}0 & - i & 3 + 4 i\\i & -2 & 12\\3 - 4 i & 12 & 5\end{matrix}\right]
  \end{displaymath}
et le but est de trouver une matrice inversible $P \in \C^{3 \times 3}$ telle que 
\begin{displaymath}
  P^T \cdot A \cdot \overline{P}
\end{displaymath}
est une matrice diagonale. Nous échangeons la première et la deuxième colonne ainsi que la première et la deuxième ligne et obtenons 
\begin{displaymath}
\left[\begin{matrix}-2 & i & 12\\- i & 0 & 3 + 4 i\\12 & 3 - 4 i & 5\end{matrix}\right]. 
\end{displaymath}
Après on transforme 
\begin{displaymath}
\left[\begin{matrix}1 & 0 & 0\\- 0.5 i & 1 & 0\\6 & 0 & 1\end{matrix}\right]\cdot  \left[\begin{matrix}-2 & i & 12\\- i & 0 & 3 + 4 i\\12 & 3 - 4 i & 5\end{matrix}\right]\cdot  
\left[\begin{matrix}1 & 0.5 i & 6\\0 & 1 & 0\\0 & 0 & 1\end{matrix}\right]
  = 
\left[\begin{matrix}-2 & 0 & 0\\0 & 0.5 & 3 - 2 i\\0 & 3 + 2 i & 77\end{matrix}\right]
\end{displaymath}
La prochaine transformation est 

\begin{displaymath}
  \left[\begin{matrix}1 & 0 & 0\\0 & 1 & 0\\0 & -6 - 4 i & 1\end{matrix}\right] \cdot 
\left[\begin{matrix}-2 & 0 & 0\\0 & 0.5 & 3 - 2 i\\0 & 3 + 2 i & 77\end{matrix}\right] \cdot 
\left[\begin{matrix}1 & 0 & 0\\0 & 1 & -6 + 4 i\\0 & 0 & 1\end{matrix}\right] = 
\left[\begin{matrix}-2 & 0 & 0\\0 & 0.5 & 0\\0 & 0 & 51\end{matrix}\right]. 
\end{displaymath}
Pour 
\begin{displaymath}
P =   \left[\begin{matrix}0 & 1 & 0\\1 & 0 & 0\\0 & 0 & 1\end{matrix}\right] \cdot 
\left[\begin{matrix}1 & - 0.5 i & 6\\0 & 1 & 0\\0 & 0 & 1\end{matrix}\right]
\cdot 
\left[\begin{matrix}1 & 0 & 0\\0 & 1 & -6 - 4 i\\0 & 0 & 1\end{matrix}\right]
\end{displaymath}
on obtient 
\begin{displaymath}
  P^T \cdot A \cdot \overline{P} = \left[\begin{matrix}-2 & 0 & 0\\0 & 0.5 & 0\\0 & 0 & 51\end{matrix}\right]. 
\end{displaymath}



\end{example}
 


\subsection*{Exercices}

\begin{enumerate}
\item Soit $V$ un espace vectoriel sur $\C$ et $f\colon V\times V \longrightarrow \C$ une application satisfaisant les axiomes 

  \begin{enumerate}[i)] 
  \item On a $f({v,w}) = \overline{f({w,v})}$ pour tout $v,w \in V$.  
  \item Si $u,v$ et $w$ sont des éléments de $V$,  
    \begin{displaymath}
      f({u,v+w}) = f({u,v}) +f({u,w}) 
    \end{displaymath}
  \item Si $x \in \C$ et $u,v \in V$, 
    \begin{displaymath}
      f({x \cdot u , v}) = x f({u,v}). 
    \end{displaymath}  
  \end{enumerate}
Montrer que $f$ est une forme hermitienne. 
\item Soit $V$ un espace vectoriel sur $\C$ de dimension finie et $f: V \times V \longrightarrow \C$ une forme sesquilinéaire. Pour une base  $B = \{v_1,\dots,v_n\}$ de $V$ et $x = \sum_i \alpha_i v_i$ et $y = \sum_i \beta_i v_i$ montrer en détail que 
  \begin{displaymath}
  \pscal{x,y} = [x]_B ^T A_B^{f} \overline{[y]_B} 
\end{displaymath}
avec la matrice $A_B^f = (f(v_i,v_j))_{1 \leq i,j \leq n}$. Indiquez l'application des axiomes PH~\ref{ph2}) et PH~\ref{ph3}) dans les pas correspondants. 
\item Démontrer la proposition~\ref{prop:3}. 
\item Soit $V$ un espace vectoriel sur $\C$ de dimension $n$ et soit $\pscal{.}$ une forme sesquilinéaire. Montrer que $\pscal{.}$ est non dégénéré si et seulement si $\rank(A_B^{\pscal{.}}) = n$  pour chaque base $B$ de $V$. 
\item Montrer que $\cong_\C$ est une relation d'équivalence. \label{item:6}
\item Modifier l'algorithme~\ref{alg:1} afin qu'il calcule une matrice diagonale congruente complexe par rapport à une matrice hermitienne $A \in \C^{n \times n}$. \label{item:7}
\item Soient $V$ un espace vectoriel sur $\C$ et $\dim(V) = 3$ et $B = \{v_1,v_2,v_3\}$ une base de $V$. Avec les matrices $A_i \in \C^{3\times 3}$ décrites en bas et les applications  $f_i(x,y) = [x]_B^T A_i \overline{[y]_B}$, cocher ce qui s'applique.  


\bigskip 

  \begin{center}
 
    \begin{tabular}{|c|c|c|c|}
      \hline 
      & $A_1$ & $A_2$ & $A_3$ \\\hline 
      forme sesquilinéaire & & & \\ \hline 
      forme hermitienne & & & \\ \hline 
    \end{tabular}
  \end{center}
  
  \medskip 


  \begin{displaymath}
    A_1 = 
    \begin{pmatrix}
      2 & 1 & 3 \\
      1 & 0 & 2 \\
      3 & 2 & 0
    \end{pmatrix}, \, 
    A_2 = 
 \begin{pmatrix}
      2 & 1+i & 3 \\
      1 & 0 & 2 \\
      3 & 2 & 0
    \end{pmatrix}, \, 
     A_3 = 
    \begin{pmatrix}
      2 & 1+ 2 \cdot i & 3 - i \\
      1 - 2 \cdot i & 0 & 2-i \\
      3-i & 2+i & 0
    \end{pmatrix}. 
  \end{displaymath}


  

\end{enumerate}





\section{Espaces hermitiens} 
\label{sec:espaces-hermitiens}



\begin{framed}\noindent 
  Pour le reste de ce paragraphe, s'il n'est pas spécifié autrement,  $V$
  est toujours un espace vectoriel sur $\C$
  muni d'un produit hermitien. Alors $V$ est un espace hermitien. 
\end{framed}




\begin{definition}
  \label{def:h5}
  Soit $\pscal{,}$ un produit hermitien défini positif. La \emph{longueur} ou la \emph{norme} d'un élément $v \in V$ est le nombre 
  \begin{displaymath}
    \| v \| = \sqrt{\pscal{v,v}}.
  \end{displaymath}
  Un élément $v \in V$ est un \emph{vecteur unitaire} si $\|v\| = 1$. 
\end{definition}

Aussi, l'inégalité de Cauchy-Schwarz est démontrée comme avant: 
\begin{equation}
  \label{eq:6}
  | \pscal{u,v}| \leq \|u\| \|v\|. 
\end{equation}


Les propriétés suivantes sont facilement vérifiées :
\begin{enumerate}[i)]
\item Pour tout $v \in V$, $\|v\|\geq 0$ et $\|v\| = 0$ si et seulement si $v = 0$. \label{item:8}
\item Pour $\alpha \in \C$ et $v \in V$ on a $\| \alpha \cdot v \| = |\alpha| \cdot \|v\|$. \label{item:9}
\item Pour chaque $u,v \in V$ $\|u+v\| \leq \|u\| + \|v\|$. \label{tr:2}
\end{enumerate}

Aussi, nous avons le théorème de Pythagore, l'inégalité de Bessel et la règle du parallélogramme.  
L'équivalent du procédé de Gram-Schmidt pour les espaces hermitiens est comme suit. 


\begin{theorem}[Le procédé d'orthogonalisation de Gram-Schmidt]
\label{thr:12}
  Soit $V$ un espace hermitien et  $\{v_1,\dots,v_n\} \subseteq V$
  un ensemble libre.  
  Il existe un ensemble libre orthogonal $\{u_1,\dots,u_n\}$
  de $V$
  tel que pour tout $i$,
  $\{v_1,\dots,v_i\}$
  et $\{u_1,\dots,u_i\}$ engendrent le même sous-espace de $V$.
\end{theorem}

Comme avant, une base orthonormale est une base orthogonale consistant
de vecteurs unitaires et le procédé de Gram-Schmidt nous donne le
corollaire suivant.

\begin{corollary}
  \label{co:6}
  Soit $V$
  un espace hermitien de dimension finie.  $V$
  possède alors une base orthonormale.
\end{corollary}



\subsection*{Exercices} 

\begin{enumerate}
\item (Composante de $u$
  sur $v$;
  indépendance de la direction) Soient $u,v \in V$
  tel que $\pscal{v,v} \neq 0$.
  Montrer qu'il existe un seul $α ∈ ℂ$ tel que
  \begin{displaymath}
    \pscal{u-α ⋅ v,v} = 0.
  \end{displaymath}
Pour ce $α$ on a 
\begin{displaymath}
  \pscal{v,u-α ⋅ v} =0. 
\end{displaymath}
\item Montrer l'inégalité de Cauchy-Schwarz. 
\item Montrer l'inégalité triangulaire~\ref{tr:2}). 
\item Montrer qu'un espace hermitien de dimension finie possède une base $B$ telle que $\pscal{x,y} = [x]_B^T \cdot \overline{[y]}_B$, où $\cdot$ est le produit hermitien standard. \label{item:4}
\end{enumerate}


 
%%% Local Variables:
%%% mode: latex
%%% TeX-master: "notes"
%%% End:

\chapter{Le théorème spectral et la décomposition en valeurs singulières}

\label{cha:appl-auto-adjo}

Dans ce chapitre, nous allons étudier les espaces euclidiens et hermitiens d'une manière plus profonde. Lorsque l'on parle de $\C$ ou de $\R$, nous allons utiliser la lettre $\K$ pour dénoter $\C$ ou $\R$. On va rappeler quelques notions importantes du cours du premier semestre. Un \emph{endomorphisme}  est une application linéaire $f \colon V \longrightarrow V$. Si $V$ est un espace vectoriel de dimension finie et si $B = \{v_1,\dots,v_n\}$ est une base de $V$, on a 
\begin{displaymath}
  f(x) = \phi_B^{-1} (A_B \phi_B(x)),
\end{displaymath}
où $\phi_B$ est l'ismomorphisme $\phi_B \colon V \longrightarrow K^n$, $\phi_B(x) = [x]_B$ sont les coordonnées de $x$ par rapport à la base $B$. On a le diagramme suivant 
\begin{displaymath}
  {
  \begin{CD}
    V     @>f>>  V\\
    @VV \phi_B V        @VV \phi_B V\\ 
    K^n     @>A_B \cdot x>>  K^n
  \end{CD}} 
\end{displaymath} 
Les colonnes de la matrice $A_B$ sont les coordonnées de $f(v_1),\dots,f(v_n)$ dans la base $B$. 
Une matrice $A \in K^{n \times n}$ est \emph{diagonalisable} s'il existe une matrice inversible $P \in K^{n \times n}$ telle que $P^{-1}\cdot A \cdot P$ est une matrice diagonale. 

\section{Les endomorphismes auto-adjoints} 
\label{sec:les-endom-auto}
\begin{framed}\noindent 
  Dans ce paragraphe~\ref{sec:les-endom-auto},  $V$
  est toujours un espace euclidien ou un espace hermitien de dimension finie. 
\end{framed}

\begin{definition}
\label{def:19}
Un endomorphisme $F$ est \emph{auto-adjoint} si 
\begin{displaymath}
  \pscal{F(v),w} = \pscal{v,F(w)} \text{ pour tous } v,w \in V. 
\end{displaymath}
\end{definition}

\begin{theorem}
  \label{thr:14}
  Soient $B = \{v_1,\dots,v_n\}$ une base orthonormale de $V$ et $F$ un endomorphisme. Alors $F$ est auto-adjoint si et seulement si sa matrice $A_B$ dans la base $B$ est symétrique ($\K = \R$) ou hermitienne ($\K = \C$). 
\end{theorem}

\begin{proof}
  On traite seulement le cas $\K=\C$. Le cas $\K = \R$ est démontré d'une manière analogue. Nous avons, où $\cdot$ dénote le produit hermitien standard, 
  \begin{displaymath}
    \pscal{F(v), w} = (A_B [v]_B ) \cdot [w]_B = [v]_B^T A_B^T \overline{[w]_B},
  \end{displaymath}
et 
\begin{displaymath}
  \pscal{v, F(w)}  = [v]_B^T \overline{A_B} \overline{[w]_B}. 
\end{displaymath}
Alors si $\overline{A_B} = A_B^T$, il est clair que $\pscal{F(v), w} = \pscal{v, F(w)}$ et donc $F$ est auto-adjoint. \\
Et si $F$ est auto-adjoint, en choisissant $v = v_i$, $w = v_j$, on obtient 		\begin{displaymath}
	\pscal{F(v_i), v_j} = e_i^TA_B^Te_j = (A_B^T)_{i, j} = \pscal{v_i, F(v_j)} = e_i^T \overline{A_B} e_j = (\overline{A_B})_{i, j},
	\end{displaymath}
donc $A_B^T = \overline{A_B}$
\end{proof}


\begin{lemma}
  \label{lem:8}
  Soit $A \in \C^{n \times n}$ une matrice hermitienne. Les valeurs propres de $A$ sont réelles. 
\end{lemma}
\begin{proof}
  Soient $\lambda \in \C$ une valeur propre et $v \neq 0$ son vecteur propre. Alors 
  \begin{displaymath}
    \lambda \, v^T \overline{v}  = v^T {A}^T \overline{v} = 
    v^T \overline{A \, v} = \overline{\lambda}  \, v^T \overline{v}. 
  \end{displaymath}
\end{proof}


\begin{corollary}
\label{co:3}
Soit $F$
un endomorphisme auto-adjoint, alors toutes ses valeurs propres sont
réelles.
\end{corollary}
\begin{proof}
  Soit $B = \{v_1,\dots,v_n\}$
  une base orthonormale. Les valeurs propres de $F$
  sont les valeurs propres de la matrice hermitienne $A_B$.
\end{proof}

\begin{corollary}
  \label{co:7}
  Une matrice symétrique $A \in \R^{n \times n}$ (hermitienne $A \in \C^{n \times n}$) a une valeur propre réelle. 
\end{corollary}

\begin{proof}
  Le polynôme caractéristique $p(x) = \det(A - x \cdot I_n)$ a une racine complexe selon le théorème fondamental de l'algèbre. Les valeurs propres de $A$ sont les racines de $p(x)$.  Mais toutes ces racines sont réelles selon le         corollaire~\ref{co:3}.  
\end{proof}

\begin{remark}
La même preuve, par le théorème fondamental de l'algèbre, montre en fait que le polynôme caractéristique de $A$ est scindé sur $\Bbb R$, i.e. $A$ possède $n$ valeurs propres réelles (en comptant les multiplicités algébriques).
\end{remark}

\begin{lemma}
  \label{lem:9}
  Soient $F$ un endomorphisme auto-adjoint et $u,v \neq 0$ deux vecteurs propres dont leurs valeurs propres  sont différentes, alors $\pscal{u,v}=0$. 
\end{lemma}

\begin{proof}
  Soient $\lambda \neq \gamma $ les valeurs propres correspondant aux vecteurs propres $u,v \neq 0$ respectivement. Puisque $\lambda,\gamma \in \R$ on a
  \begin{displaymath}
    \lambda \pscal{u,v} = \pscal{F(u),v} = \pscal{u,F(v)} = \gamma \pscal{u,v}
  \end{displaymath}
et alors $\pscal{u,v}=[u]_B \cdot [v]_B=0$, où $\cdot$ dénote le produit scalaire/hermitien standard et $B$ est une base orthonormale de $V$.  
\end{proof}



\begin{definition}
  \label{def:20}
  Une matrice inversible  $U \in \R^{n \times n}$ est \emph{orthogonale} si $ U^{-1}= U^T  $.  Une matrice inversible $U \in \C^{n \times n}$ est \emph{unitaire} si $U^{-1} = \overline{U}^T$. 
\end{definition}
\noindent 
Si $U$ est orthogonale (unitaire), les  colonnes de $U$ sont une base orthonormale de $\R^n$ ($\C^n$), où l'orthonormalité est entendue au sens du produit scalaire (hermitien) standard. 
\begin{notation}
Nous allons écrire $A^*$ pour dénoter $\overline{A}^T$ pour une matrice $A$.   
\end{notation}




\begin{theorem}[Théorème spectral]
\label{thr:16}
  Soit $A \in \K^{n\times n}$ une matrice symétrique (hermitienne), alors $A$ est diagonalisable avec une matrice orthogonale (unitaire) $P \in \K^{n \times n}$ telle que 
  \begin{equation}
    \label{diagonal}
    P^* \cdot A \cdot P =
    \begin{pmatrix}
      \lambda_1 \\
      & \ddots \\
      && \lambda_n
    \end{pmatrix}
  \end{equation}
  où $\lambda_1,\dots,\lambda_n \in \R$ sont les valeurs propres de $A$. 
\end{theorem}


\begin{proof}
  Soit $A \in \R^{n\times n}$
  une matrice symétrique. Le cas où $A \in \C^{n \times n}$
  est hermitienne est laissé en exercice.

  Le théorème est démontré par induction. Si $n=1$, l'assertion est triviale. \\
  Supposons le théorème vrai jusqu'à $n-1 \in \N, n \ge 2$. Montrons le pour $n$. \\
  Soit $\lambda_1 \in \R$
  une valeur propre de $A$
  et $v \neq 0 \in \R^n$
  un vecteur propre correspondant à $\lambda_1$.
  Avec la méthode de Gram-Schmidt, on peut trouver une base
  orthonormale $\{v,u_2,\dots,u_n\}$ dans $\R^n$. Soit $U \in \R^{n \times n-1}$ la matrice dont les colonnes sont $u_2,\dots,u_n$.  On considère la matrice
  \begin{displaymath}
    U^T \cdot A \cdot U \in \R^{n-1 \times n-1}.
  \end{displaymath}
  Cette matrice est symétrique. En effet :
  \begin{displaymath}
    (U^T \cdot A \cdot U)^T = U^T \cdot A^T \cdot (U^T)^T = U^T \cdot A \cdot U,
  \end{displaymath}
  car $A$ est symétrique. Par hypothèse d'induction, cette matrice peut être diagonalisée  avec une matrice orthogonale  $K \in \R^{ n-1 \times n-1}$, alors 
  \begin{displaymath}
    K^T \cdot  U^T \cdot A \cdot U \cdot K =
    \begin{pmatrix}
      \lambda_2 \\
      & \ddots \\
      && \lambda_n
    \end{pmatrix}. 
  \end{displaymath}
Maintenant, soit $P \in \R^{n \times n}$ la matrice 
\begin{displaymath}
  P = \left( v, U\,K\right) \in \R^{n \times n}
\end{displaymath} 
La matrice $P$ est orthogonale puisque 
\begin{displaymath}
  P^T \, P =
  \begin{pmatrix}
    v^T \, v & v^T U \, K \\
    (U\,K)^T \, v &  (U\,K)^T (U\,K)
  \end{pmatrix} = 
  \begin{pmatrix}
    1 \\
    & \ddots \\
    && 1 
  \end{pmatrix}. 
\end{displaymath}
car $v^T \cdot v = 1$, $U^T \cdot U = I_{n-1}$ et $v^T \cdot U = 0$. Et 
\begin{displaymath}
  P^T \, A \, P =
  \begin{pmatrix}
    v^T A \\
    K^T U^T A 
  \end{pmatrix}
  \begin{pmatrix}
    v & U\,K
  \end{pmatrix} =
  \begin{pmatrix}
    \lambda_1 \\
    & \ddots \\
    && \lambda_n
  \end{pmatrix}  
\end{displaymath}
\end{proof}


\begin{corollary}
  \label{co:8}
  Soit $V$ un espace euclidien (hermitien) de dimension finie et $F$ un endomorphisme auto-adjoint. Alors $V$ possède une base $\{v_1,\dots,v_n\}$ orthonormale de vecteurs propre de $F$. 
\end{corollary}

\begin{proof}
  Soit $B = \{u_1,\dots,u_n\}$ une base de $V$ tel que $\pscal{x,y} = [x]_B \cdot [y]_B$ où $\cdot $ dénote le produit hermitien standard (voir chapitre~\ref{sec:espaces-hermitiens}, exercice~\ref{item:4}) et soit $A_B^{(F)}$  la matrice symétrique (hermitienne) telle que $F(x)  = \phi_B^{-1}(A_B^{(F)} [x]_B)$. Selon théorème~\ref{thr:16}, $A_B^{(F)}$ est diagonalisable avec une matrice orthogonale (unitaire) $P$ 
  \begin{displaymath}
    P^* \cdot 
    A_B^{(F)} \cdot  P  = \begin{pmatrix}
      \lambda_1 \\
      & \ddots \\
      & & \lambda_n
    \end{pmatrix}
  \end{displaymath}
 Soient $p_1,\dots,p_n$  les colonnes de $P$. La base orthonormale de vecteurs propres de $F$ est $\{v_1,\dots,v_n\}$ où $v_i = \phi_B^{-1}(p_i)$. 
\end{proof}



Comment peut-on calculer la diagonalisation \eqref{diagonal}? Voici un procédé pour diagonaliser une matrice symétrique (hermitienne) $A \in \K^{n \times n}$. 

\begin{enumerate}[i)] 
\item Trouver les racines $\lambda_1,\dots,\lambda_k \in \R$ du polynôme caractéristique 
  \begin{displaymath}
    p(x) = \det(A - x\cdot I). 
  \end{displaymath}
\item Pour tout $j \in \{1,\dots,k\}$:
  \begin{enumerate}[a)] 
  \item Trouver une base $b^{(j)}_1,\dots,b^{(j)}_{d_j}$ du noyau de la matrice $A - \lambda_j \, I$, par exemple avec l'algorithme de Gauss. 
  \item Trouver une base orthonormale $p^{(j)}_1,\dots,p^{(j)}_{d_j}$ du $\spa\{b^{(j)}_1,\dots,b^{(j)}_{d_j}\}$, par exemple avec le procédé de Gram-Schmidt. 
  \end{enumerate}
  \item $P = \left(p^{(1)}_1,\dots,p^{(1)}_{d_1},\dots,p^{(k)}_1,\dots,p^{(k)}_{d_k}\right)$ 
\end{enumerate}




\begin{example}
  \label{exe:17}
  Soit $V$ un espace euclidien de dimension $3$ et soit $F$ un endomorphisme auto-adjoint de $V$. Soit $B=\{v_1,v_2,v_3\}$ une base orthonormale telle que 
  \begin{displaymath}
    A_B =
    \begin{pmatrix}
      3 & -2 & 4\\-2 & 6 & 2\\4 & 2 & 3
    \end{pmatrix}
  \end{displaymath}
Trouver une base orthonormale qui se compose des vecteurs propres. 


\bigskip 

Le polynôme caractéristique de $A_B$ est 
\begin{displaymath}
  p(x) = \det(A_B - x\, I) =  -x^3 + 12\, x^2 -21 x - 98 = -(x-7)^2(x+2)
\end{displaymath}
On trouve les bases des espaces propres
\begin{displaymath}
  \lambda_1 = 7: \, \,
b_1^{(1)} = \begin{pmatrix}
    1 \\ 0 \\ 1 
  \end{pmatrix}, \, 
b_2^{(1)} =   \begin{pmatrix}
    -1/2\\1\\0
  \end{pmatrix} \, \quad \text{ et } 
  \lambda_2 = -2: \, \, 
  b_1^{(2)} = 
  \begin{pmatrix}
    -1 \\ -1/2 \\1
  \end{pmatrix}  
\end{displaymath}

Les vecteurs $b_1^{(1)}$
et $b_2^{(1)}$
ne sont pas orthogonaux.  Le procédé de Gram-Schmidt produit
${b_2^{(1)}}^* = (-1/4,1,1/4)^T$.
Les vecteurs $b_1^{(1)},{b_2^{(1)}}^*,b_1^{(2)}$
sont une base orthogonale de vecteurs propres. Maintenant il reste à
les normaliser et on obtient
\begin{displaymath}
p_1^{(1)} = 
\left[\begin{matrix}\frac{\sqrt{2}}{2}\\0\\\frac{\sqrt{2}}{2}\end{matrix}\right], \, 
{p_2^{(1)}} = \left[\begin{matrix}- \frac{\sqrt{2}}{6}\\\frac{2 \sqrt{2}}{3}\\\frac{\sqrt{2}}{6}\end{matrix}\right], \, 
p_1^{(2)} = \left[\begin{matrix}-\frac{2}{3}\\-\frac{1}{3}\\\frac{2}{3}\end{matrix}\right]. 
\end{displaymath}

Alors $\left\{\frac{\sqrt{2}}{2}v_1 + \frac{\sqrt{2}}{2} v_3,- \frac{\sqrt{2}}{6}v_1 + \frac{2 \sqrt{2}}{3}v_2 + \frac{\sqrt{2}}{6} v_3 , -\frac{2}{3}v_1-\frac{1}{3}v_2+\frac{2}{3}v_3\right\}$ est une base orthonormale de vecteurs propre de $F$. 

\end{example}

\subsection*{Exercices}

\begin{enumerate}
\item Soit $z = x + iy \in \C^n$, où $x,y \in \R^n$. Montrer que $x$ et $y$ sont linéairement indépendants sur $\R$ si et seulement si $z$ et $\overline{z}$ sont linéairement indépendants sur  $\C$. 
\end{enumerate}


\section{Formes quadratiques réelles et matrices symétriques réelles}
\label{sec:form-quadr-reell}

Le lemme~\ref{lem:8} et le corollaire~\ref{co:7} démontrent qu'une matrice symétrique réelle possède une valeur propre réelle. Cette démonstration passe par les nombres complexes et utilise le théorème fondamental de l'algèbre. Pour le cas où $A = A^T \in \R^{n \times n}$, nous allons maintenant démontrer l'assertion du  corollaire~\ref{co:7}  d'une manière géométrique. L'ensemble 
\begin{displaymath}
  S^{n-1} = \{x \in \R^n \colon \|x\| = 1 \} 
\end{displaymath}
est appelé la \emph{$n$-sphère.} 

\begin{definition}
  \label{def:21}
  Une \emph{forme quadratique} est une fonction
  $f\colon \R^n \longrightarrow \R$,
  $f(x) = x^T A x$
  où $A \in \R^{n \times n}$ est une matrice symétrique.
\end{definition}
En fait, si $B \in \R^{n \times n}$
n'est pas symétrique, la matrice $A = 1/2 (B^T + B)$
est symétrique et $x^TBx = 1/2 (x^TB^T x + x^T B x) = x^T Ax$.
Alors la fonction $g(x) = x^TBx$ est aussi une forme quadratique.

Une {forme quadratique} est un polynôme de degré $2$
et une fonction continue. Puisque $S^{n-1}$
est compact et $f(x)$
est continue, $f(x)$
possède un maximum sur $S^{n-1}$.
Nous sommes prêts à démontrer le lemme d'une manière géométrique.

\begin{lemma}
  \label{lem:7}
  Soient $A \in \R^{n \times n}$ une matrice symétrique et $v \in S^{n-1}$ le maximum de la fonction $f(x) = x^TAx$ sur $S^{n-1}$. On a $Av = \lambda\, v$ pour  un $\lambda \in \R$. En particulier, $A$ possède une valeur propre réelle. 
\end{lemma}

\begin{proof}
  Supposons qu'il n'existe pas de $\lambda \in \R$  tel que $A\,v  =  \lambda v$. Alors,  en particulier, $A\, v \neq 0$ et on peut écrire 
  \begin{displaymath}
    A\,v = \alpha \, v + \beta \, w
  \end{displaymath}
où $w \in S^{n-1}$, $w \perp v$ et $\beta \neq 0$. Pour $x \in [-1,1]$ on a 
\begin{displaymath}
  \sqrt{(1 - x^2)} \,v + x\, w \in S^{n-1}.
\end{displaymath}On voit facilement que $\|\sqrt{(1 - x^2)} \,v + x\, w\|=1$ en utilisant le fait que $v\perp w$, et $v,w\in S^{n-1}$.
Nous considérons  la fonction $g\colon\; [-1,1] \rightarrow \R$ 
\begin{displaymath}
  g(x) = \left(\sqrt{(1 - x^2)} \,v + x\, w \right)^T A \left(\sqrt{(1 - x^2)} \,v + x\, w \right). 
\end{displaymath}
Notons que $g(0)=f(v)$, donc si $v$ maximise $f$ sur la $n$-sphère, en particulière $x=0$ doit maximiser $g(x)$ dans l'intervalle [-1,1]. Si on démontre que $g'(0) \neq 0$, nous avons déduit une contradiction et la démonstration est faite. 

Comme $w^TAv = v^TAw$, clairement 
\begin{displaymath}
  g(x) = (1-x^2) v^T A v + (2 \cdot \sqrt{(1 - x^2)} \cdot x) \, w^T A v + x^2 w^TAw.
\end{displaymath}
Ceci démontre que $g'(0) = 2 \cdot w^T A v = 2 \cdot \beta  \neq 0$. 

\end{proof}





\begin{definition}
  \label{def:f22}
  Une matrice symétrique $A \in \R^{n \times n}$ est 
  \begin{itemize}
  \item définie positive, si $x^TAx>0$ pour tout $x \in \R^n \setminus  \{0\}$ 
  \item définie négative, si $x^TAx<0$ pour tout $x \in \R^n \setminus  \{0\}$ 
  \item semi-définie positive, si $x^TAx\geq 0$ pour tout $x \in \R^n$ 
  \item semi-définie négative, si $x^TAx\leq 0$ pour tout $x \in \R^n$.  
  \end{itemize}
  La forme quadratique $x^TAx$ correspondante est appelée définie positive, définie négative, semi-définie positive ou semi-définie négative, en accord avec $A$. 
\end{definition}



\begin{theorem}
  \label{thr:15}
  Une matrice symétrique $A \in \R^{n \times n}$ est 
  \begin{enumerate}
  \item définie positive,
    si et seulement si toutes ses valeurs propres sont strictement positives. 
  \item définie négative, si et seulement si toutes ses valeurs propres sont strictement négatives. 
  \item semi-définie positive, si et seulement si toutes ses valeurs propres sont positives (ou zéro).  
  \item semi-définie négative, si et seulement si toutes ses valeurs propres sont negatives (ou zéro).  
  \end{enumerate}
\end{theorem}

\begin{proof}
  D'après le théorème~\ref{thr:16} il existe une 
  matrice orthogonale $U \in \R^{n\times n}$ telle que 
  \begin{equation}
    \label{eq:10}    
    A = U
    \begin{pmatrix}
      \lambda_1 \\
      & \ddots \\
      && \lambda_n
    \end{pmatrix} U^T. 
  \end{equation}
  où les $\lambda_i$ sont les valeurs propres de $A$. Les colonnes  $u_1,\dots,u_n$ de $U$ forment une base orthonormale de $\R^n$  de vecteurs propres de $A$. 
  Soit $x \in \R^n$, alors $x = \sum_i \alpha_i u_i$ et 
  \begin{displaymath}
    x^TAx = \sum_i (\alpha_i^2) \lambda_i.  
  \end{displaymath}
D'ici l'assertion suit directement. 
\end{proof}

Le \emph{$k$-mineur principal} d'une matrice $A$ est le déterminant de la matrice qui est construite en choisissant les premières $k$ lignes et colonnes de $A$; c'est-à-dire $\det(B_k)$ où $B_k \in \R^{k\times k}$ telle que $b_{ij} = a_{ij}$, $1 \leq i,j \leq k$. Soit $K  = \{l_1,\dots,l_k\}\subseteq \{1,\dots,n\}$ où $l_1<l_2<\dots<l_k$. 
La matrice $B_K \in \R^{k\times k}$  est définie par $b_{ij} = a_{l_il_j}$, pour $1 \leq i,j\leq k$. Un \emph{$k$-mineur symétrique} de $A$ est le déterminant d'une matrice $B_K$; c'est-à-dire $\det(B_K)$.  

\begin{theorem}
  \label{thr:17}
  Soit $A \in  \R^{n \times n}$ une matrice symétrique.   
  \begin{enumerate}[a)]
  \item $A$ \label{posdef:1}
    est définie positive si et seulement si tous ses mineurs
    principaux sont strictement positifs.
  \item $A$ \label{posdef:2}
    est semi-définie positive si et seulement si tous ses mineurs symétriques sont positifs (ou zéro).
  \end{enumerate}
  Ce théorème est connu sous le nom de "critère de Sylvestre".
\end{theorem}

\begin{proof}
%Nous   considérons la factorisation~\eqref{eq:10}. La 

On démontre~\ref{posdef:1}), tandis que \ref{posdef:2}) est un exercice. 
Soit $A$ une matrice définie positive, montrons que   
les matrices $B_k$, $ 1 \leq k \leq n$,  sont également définies positives : soit $z_k \in \mathbb R^k$ non-nul, 
on complète ce vecteur en un vecteur $x_k \in \mathbb R^n$ en lui rajoutant $n-k$ zéros.
On vérifie facilement que $z_k^T B_k z_k = x_k^T A x_k > 0$. 
Les valeurs propres de $B_k$ sont donc toutes strictement positives en vertu du théorème précédent. 
En se servant alors du théorème~\ref{thr:16}, on obtient facilement que $\det(B_k)$ est le produit des valeurs propres de $B_k$, alors $\det(B_k)>0$. 

Supposons maintenant que $\det(B_k)>0$ pour tout $k \in \{1,\dots,n\}$.
L'argument est par récurrence. Le cas $n=1$ est trivial.

Soit $n>1$. Les matrices $B_k$ $k=1,\dots,n-1$ sont définies positives par récurrence. Si $A$ elle même n'est pas définie positive, $\det(A)>0$ implique qu'il existe au moins deux valeurs propres négatives, disons $μ$ et $λ$ dans la factorisation donnée par le théorème spectral, et deux vecteurs propres orthogonaux $u,v ∈ ℝ^n$ correspondants.  Leurs dernières composantes sont pas égales à zéro, parce que $A_{n-1}$ est définie positive. Alors il existe $β ≠ 0$ tel que la dernière composante de $u+ β v$ est égale a zéro. Mais
\begin{displaymath}
  (u+ β v)^T A (u+ β v) = μ + β^2 λ <0,
\end{displaymath}
ce qui est une contradiction  au fait que $A_{n-1}$ est définie positive. 

\begin{example}
  \label{exe:18}
  Nous pouvons alors montrer qu'une matrice symétrique est définie positive de deux manières différentes d'après les deux théorèmes précédents : considérons la matrice suivante
  \begin{displaymath}
    A =
    \begin{pmatrix}
    2 & -1  & 0 \\
      -1 & 2 & -1 \\
      0 & -1 & 2 
    \end{pmatrix}. 
  \end{displaymath}
$\bullet$ Son polynôme caractéristique est égal à $\det(A-XI_n)=-X^3+6X^2-10X+4$ dont les racines sont $2$, $2+\sqrt{2}$ et $2-\sqrt{2}$. Les valeurs propres de $A$ sont toutes les trois strictement positives donc la matrice est définie positive.

$\bullet$ D'une autre façon, $\det(B_1)=2 >0$, $\det(B_2)=3 > 0$ et $det(B_3)=det(A)=4 >0$ donc la matrice est définie positive.

\end{example}
%Nous appliquons notre algorithme~\ref{alg:1}. Avant la $i$-ème itération, la matrice $A$ est de la forme 
% \begin{displaymath}
%   P^T A P = \begin{pmatrix}
%       c_1 \\
%       & c_2 \\
%       & & \ddots & &&\\
%       & & & c_{i-1} \\
%       & & & &  b_{i,i} & \dots & b_{i,n} \\
% %      & & & &  b_{i+1,i} & \dots & b_{i+1,n} \\
%       & & & &     \vdots       &  & \vdots \\
%       & & & &  b_{n,i} & \dots & b_{n,n} \\      
%     \end{pmatrix}. 
% \end{displaymath} 
%  On observe que $b_{ii} \neq 0$. Parce que, si $b_{ii} = 0$ pour la première fois, nous n'avons jamais échangé de colonnes ou de lignes avant et  $ 0 = c_1\cdots c_{i-1} \cdot b_{ii} = \det(B_i)$, et c'est une contradiction.  
%  Alors il n'est jamais nécessaire d'échanger des lignes et colonnes et notre algorithme trouve une matrice $R$, triangulaire supérieure, dont les éléments diagonaux sont tous égaux à $1$, et telle que 
% \begin{displaymath}
%   R^T \cdot A \cdot R =  
% \begin{pmatrix}
%       c_1 \\
%              & \ddots & \\
%        & & c_{n}
%     \end{pmatrix}
% \end{displaymath}
% On observe que $\det(B_k) = c_1\cdots c_k$, alors toutes les $c_i$ sont positives. Alors $A$ est définie positive. 
\end{proof}


\begin{theorem}
  \label{thr:18}
  Soit $A \in \R^{n\times n}$ une matrice symétrique et $f(x) = x^TAx$ la forme quadratique correspondante à $A$. On a
  \begin{equation}
    \label{eq:11}
    \max_{x \in S^{n-1}} f(x) = \lambda_1  \, \text{ et } \,  \min_{x \in S^{n-1}} f(x)  = \lambda_n
  \end{equation}
  où $\lambda_1$ et $\lambda_n$ sont les valeurs propres maximale et minimale de $A$ respectivement. 
\end{theorem}

\begin{proof}
  Nous utilisons la factorisation 
  \begin{displaymath}
    A = P
    \begin{pmatrix}
      \lambda_1 \\
      & \ddots \\
      && \lambda_n
    \end{pmatrix} P^T
  \end{displaymath}
où $P \in \R^{n\times n}$ est une matrice orthogonale dont les colonnes sont $p_1,\dots,p_n$. Si $x = \sum_i \alpha_i \,p_i$,   alors 
\begin{displaymath}
\|x\|^2 = \sum_i \alpha_i^2   \text{ et } x^T A x =  \sum_i (\lambda_i \alpha_i^2)
\end{displaymath}
et si $\|x\|^2 = 1$, 
\begin{displaymath}
\lambda_n =  \lambda_n  \sum_i  \alpha_i^2  \leq  \sum_i (\lambda_i \alpha_i^2) \leq \lambda_1  \sum_i  \alpha_i^2 = \lambda_1. 
\end{displaymath}
Ça démontre que $p_1$ et $p_n$ sont les solutions optimales des problèmes d'optimisation~\eqref{eq:11}. 

\end{proof}



\begin{definition}
  \label{def:23}
  Soit $A \in \R^{n \times n}$ une matrice symétrique. Pour $x \in \R^n\setminus \{0\}$ le  \emph{quotient Rayleigh–Ritz } est 
\begin{displaymath}
  R_A(x) = \frac{x^TAx}{x^Tx}. 
\end{displaymath}
\end{definition}
Pour $x \in \R^n \setminus \{0\}$ $ x / \|x\| \in S^{n-1}$ et $R_A(x) = (x / \|x\| )^T \cdot A  (x / \|x\|)$. 

\begin{theorem}[Théorème Min-Max]
\label{thr:19}
Soit $A \in \R^{n × n}$ une matrice symétrique avec les valeurs propres 
$\lambda_1 \geq \dots \geq \lambda_n$.   Si $U$ dénote un sous-espace de $\R^n$ alors 
\begin{equation}
  \label{eq:12} 
  \lambda_k = \max_{ \dim(U) = k } \, \min_{x \in U \setminus \{0\}}  R_A(x)  
\end{equation}
et
\begin{equation}
  \label{eq:13}
  \lambda_k = \min_{ \dim(U) = n-k+1 } \, \max_{x \in U \setminus \{0\}}  R_A(x)  
\end{equation}
\end{theorem}

\begin{proof}
Nous démontrons \eqref{eq:12}. La partie \eqref{eq:13} est un exercice. Soit $\{u_1,\dots,u_n\}$ une base orthonormale de vecteurs propres associés à $\lambda_1 ≥ \dots ≥ \lambda_n$ respectivement. On fixe un entier $k$, et un espace $U$ de dimension $k$. Clairement $\spa\{u_k,\dots,u_n\} \cap U \supsetneq \{0\}$. Alors il existe un vecteur $0 \neq x = \sum_{i = k}^n \alpha_i u_i \in U$. Clairement $R_A(x) \leq \lambda_k$. Pour $U = \spa\{u_1,\dots,u_k\}$, 
$
  \min_{x \in U \setminus \{0\}}  R_A(x) = \lambda_k.
$ Ensemble ça démontre \eqref{eq:12}. 
\end{proof}



\subsection*{Exercices}

\begin{enumerate}
\item Une matrice réelle  symétrique, différente de la matrice zéro, telle que toute composante sur la diagonale est zéro ne peut pas être semi définie positive, ni semi définie négative. 
\item Une matrice réelle symétrique, différente de la matrice zéro, dont la diagonale est égale à zéro, possède un $2 ×2$ mineur symétrique négatif. \label{item:17}
\item Soit $L \in ℝ^{n × n}$ une matrice %triangulaire inférieure 
de la forme 
  \begin{displaymath}
L = \left(\begin{array}{c|c}
H & 0  \\
\hline
C  & I_{n-i+1} \\
\end{array}\right)
\end{displaymath}
où  $H ∈ ℝ^{i×i}$ et $i ≥0$. Soit $Q ∈ ℝ^{n ×n}$ la matrice de la permutation (transposition)  qui échange $μ, ν > i$. Montrer 
\begin{displaymath}
  Q \cdot L = \left(\begin{array}{c|c}
H & 0  \\
\hline
C'  & I_{n-i+1} \\
\end{array}\right) \cdot Q
\end{displaymath}
où $C'$ provient de $C$ en échangeant les lignes $μ-i$ et $ν-i$. 
\label{item:15}
\item \label{item:16}
En s'appuyant sur l'exercice~\ref{item:15}) montrer l'assertion suivante. Si l'algorithme~\ref{alg:1} a exécuté $k$-itérations et dans chacune de ces $k$ itérations, il existe un $j ≥i$ tel que  $b_{jj} \neq 0$ (avec la notation de la $i$-ème itération), alors il existe une matrice de permutation $Q$ telle que le résultat de ces premières $k$ itérations s'écrit 
\begin{displaymath}
  R^T Q^T A Q R, 
\end{displaymath}
où $R$ est une matrice triangulaire supérieure dont les éléments diagonaux sont $1$.  En autres mots, il existe une permutation si appliquée aux lignes et colonnes, l'algorithme~\ref{alg:1} n'échange pas de lignes et colonnes pendant ces premiers $k$ itérations. 
\item Soient $A \in \Bbb R^{n \times n}$ réelle symétrique et $A'$ la matrice obtenue de $A$ en échangeant les lignes $i$ et $j$ ($A' = Q^T A Q$ pour une matrice de permutation (transposition) $Q$). Montrer que pour tout $K \subseteq \{1, ..., n\}$, il existe $K' \subseteq \{1, ..., n\}$ tel que $\det(A_K) = \det(A'_{K'})$ (mineurs symétriques) et inversément. En d'autres termes, montrer que tous les mineurs symétriques de $A$ se retrouvent dans $A'$ et vice versa.
\item Montrer la partie b) du théorème~\ref{thr:17}. 

\emph{Indication pour montrer $⟸$:  Il existe une matrice de permutation $Q$ telle que l'algorithme~\ref{alg:1} n'échange pas de colonnes et lignes si confronté avec  $Q^T \cdot A \cdot Q$ comme input et toutes itérations sont telles que $b_{ii} \neq 0$ jusqu'à  un point, où tout le reste de la matrice est $0$. Il faut s'appuyer sur les exercices \ref{item:17} et \ref{item:16}.}

\item Une matrice symétrique $A \in \R^{n \times n}$ est définie négative, si et seulement si $\det(B_k) \neq 0$ et $\det(B_k) = (-1)^k|\det(B_k)|$ pour tout $k$. 
\item Une matrice symétrique $A \in \R^{n \times n}$ est semi-définie négative, si et seulement si $\det(B_K) = (-1)^{|K|}|\det(B_K)|$ pour tout $K\subseteq \{1,.\dots,n\}$.  
\item Montrer la partie \eqref{eq:13} du théorème~\ref{thr:19}. 
\item 
Soit $A \in \R^{n \times n}$ une matrice symétrique avec les valeurs propres $\lambda_1 \geq \dots \geq \lambda_n$. 
Soit $B_K$ une matrice comme décrite en dessus o\`u $|K| = k$ avec les valeurs propres  $\mu_1 \geq \dots  \geq \mu_{n-k}$. Pour $1 \leq i \leq k$, alors 
\begin{displaymath}
  \lambda_i \geq  \mu_i \geq  \lambda_{i+k}.
\end{displaymath}
 
\end{enumerate}








\begin{definition}
  \label{def:22}
  Une matrice hermitienne $A \in \C^{n \times n}$ est 
  \begin{itemize}
  \item définie positive, si $x^TA\overline{x}>0$ pour tout $x \in \C^n \setminus  \{0\},$ 
  \item définie négative, si $x^TA\overline{x}<0$ pour tout $x \in \C^n \setminus  \{0\},$ 
  \item semi-définie positive, si $x^TA\overline{x}\geq 0$ pour tout $x \in \C^n$,
  \item semi-définie négative, si $x^TA\overline{x}\leq 0$ pour tout $x \in \C^n$.  
  \end{itemize}
\end{definition}


Le théorème~\ref{thr:15} trouve son analogue comme suivant. La démonstration est un exercice. 

\begin{theorem}
\label{thr:20}
  Une matrice hermitienne $A \in \C^{n \times n}$ est 
  \begin{enumerate}
  \item définie positive,
    si et seulement si toutes ses valeurs propres sont (strictement) positives. 
  \item définie négative,  si et seulement si toutes ses valeurs propres sont (strictement) négatives. 
  \item semi-définie positive, si et seulement si toutes ses valeurs propres sont non-négatives (donc positives ou zéro).  
  \item semi-définie négative, si et seulement si toutes ses valeurs propres sont non-positives (négatives ou zéro).  
  \end{enumerate}
\end{theorem}


\section{La décomposition en valeurs singulières}
\label{sec:la-decomposition-en}


On commence avec un théorème qui décrit la décomposition en valeurs singulières et montre qu'elle existe. 


\begin{theorem}
  \label{thr:21}
  Une matrice $A \in \C^{m \times n}$ peut être décomposée comme 
  \begin{displaymath}
    A = P\cdot D \cdot Q
  \end{displaymath}
où $P \in \C^{m \times m}$ et $Q \in \C^{n \times n}$ sont unitaires et $D \in \R_{\geq 0}^{m \times n}$ est une matrice diagonale. Si $A$ est réelle, $P$ et $Q$ sont réelles. 
\end{theorem}


\begin{proof}
  La matrice $A^* \cdot  A$ est hermitienne et semi-définie positive dès que
  \begin{displaymath}
    x^*A^*Ax = (Ax)^* (Ax) \geq 0. 
  \end{displaymath}
Alors les valeurs propres de $A^*A$ sont non-négatives ($\lambda_i\geq 0$). Soient $\sigma_1^2 \geq \sigma_2^2 \dots \geq \sigma_n^2\geq 0$ les valeurs propres et soit $\{u_1,\dots,u_n\}$ une base orthonormale  correspondante de vecteurs propres. La matrice $Q \in \C^{n \times n}$ est la matrice dont les lignes sont $u_1^*, \dots, u_n^*$. 

Soit $ r \in \N_0$ tel que $\sigma_r >0$ et $\sigma_{r+1} = 0$; on a $\sigma_1 \geq \ldots \geq \sigma_r \geq 0=\sigma_{r+1} = \ldots = 0=\sigma_n  $. Nous construisons les vecteurs 
\begin{displaymath}
  v_i = A \, u_i  / \sigma_i, \, 1 \leq i \leq r. 
\end{displaymath}
Les $v_i$ sont orthonormaux, parce que 
\begin{displaymath}
  \|v_i\|^2 = (A \, u_i)^* A u_i / \sigma_i^2 = 1
\end{displaymath}
et pour $1 \leq i\neq j \leq r$, 
\begin{displaymath}
  v_i^* v_j = u_i^* u_j = 0. 
\end{displaymath}
Avec le procédé de Gram-Schmidt, nous complétons les $v_i$ tels que $\{v_1,\dots,v_m\}$ est une base orthonormale de $\C^m$. Les colonnes de la matrices $P$ sont alors $v_1,\dots,v_m$ dans cet ordre. La matrice $D \in \C^{m\times n}$ est la matrice diagonale dont les $r$ premières composantes sur la diagonale sont $\sigma_1,\dots,\sigma_r$ dans cet ordre. Avec ces matrices  $P,D$ et $Q$ nous avons 
\begin{displaymath}
  A = P\cdot D \cdot Q 
\end{displaymath}
ou de manière équivalente 
\begin{displaymath}
  P^* \cdot A \cdot Q^* = D,
\end{displaymath}

Nous montrons  ça en détail. Nous avons 
\begin{equation}
  \label{eq:39}
     (P^* \cdot A \cdot Q^*)_{ij} = v_i^* A u_j 
\end{equation}
et c'est égal à zéro si $j > r$, parce que dans ce cas $A u_j =0$ dès que $u_j^*A^*Au_j = 0$.
Si $i>r$ \eqref{eq:39} pas égale  a zéro implique $A u_j \neq 0$ et alors $j≤r$. Mais dans ce cas, par construction, $v_i$ est orthogonal à  $A u_j / σ_j$ et \eqref{eq:39}  est néanmoins zéro. 


Et  si $1 ≤i,j \leq r$, alors 
\begin{displaymath}
  u_i^* A^* A u_j / \sigma_i = u_i^* u_j \,  \sigma_j^2 / \sigma_i=
  \begin{cases}
    \sigma_i & \text{si } i=j\\
    0 & \text{autrement}.  
  \end{cases}
\end{displaymath}
\end{proof}


\begin{definition}
  \label{def:24}
  En suivant la notation du théorème~\ref{thr:21}, 
  les nombres $\sigma_1,\dots,\sigma_r$ sont les \emph{valeurs singulières} de $A$. La factorisation $A = P\cdot D \cdot Q$ est une \emph{décomposition en valeurs singulières}. 
\end{definition}




\begin{example}
  \label{exe:19}
  Trouver une décomposition en valeurs singulières de 
  \begin{displaymath}
    A =
    \begin{pmatrix}
      0 & -1.6  & 0.6 \\
      0 & 1.2 & 0.8 \\
      0 & 0 & 0 \\
      0 & 0 & 0
    \end{pmatrix}. 
  \end{displaymath}
On commence avec 
\begin{displaymath}
  A^* \cdot A =
  \begin{pmatrix}
    0 & 0 & 0 \\
    0 & 4 & 0 \\
    0 & 0 & 1
  \end{pmatrix}
\end{displaymath}
On obtient $\sigma_1 = 2, \sigma_2 = 1$ et $\sigma_3 = 0$. Les valeurs singulières sont les $\sigma_i>0$, i.e., $\sigma_1=2$ et $\sigma_2=1$. On calcule les vecteurs propres correspondant à $\sigma_1 = 2, \sigma_2 = 1$ et $\sigma_3 = 0$. La matrice $Q$ est 
\begin{displaymath}
  Q =
  \begin{pmatrix}
    0 & 1 & 0 \\
    0 & 0 & 1  \\
    1 & 0 & 0  
  \end{pmatrix}
\end{displaymath}
et
\begin{displaymath}
v_1 = \frac{1}{2} \cdot   A 
  \begin{pmatrix}
    0\\1\\0
  \end{pmatrix} =
  \begin{pmatrix}
    -0.8 \\ 0.6 \\ 0 \\ 0
  \end{pmatrix}, 
v_2 =  A 
  \begin{pmatrix}
    0\\0\\1
  \end{pmatrix} =
  \begin{pmatrix}
    0.6 \\ 0.8 \\ 0 \\ 0
  \end{pmatrix}.
\end{displaymath}
On complète avec $v_3 = e_3$ et $v_4 = e_4$, alors 
\begin{displaymath}
P =   \begin{pmatrix}-0.8 & 0.6 & 0 & 0\\0.6 & 0.8 & 0 & 0\\0 & 0 & 1 & 0\\0 & 0 & 0 & 1\end{pmatrix}
\end{displaymath}
et 
\begin{displaymath}
  \begin{pmatrix}-0.8 & 0.6 & 0 & 0\\0.6 & 0.8 & 0 & 0\\0 & 0 & 1 & 0\\0 & 0 & 0 & 1\end{pmatrix} \cdot
  \begin{pmatrix}
    2 & 0 & 0 \\
    0 & 1 & 0 \\
    0 & 0 & 0 \\
    0 & 0 & 0
  \end{pmatrix}
\cdot
\begin{pmatrix}
  0 & 1 & 0\\0 & 0 & 1\\1 & 0 & 0 
\end{pmatrix} = A. 
\end{displaymath}

\end{example}

\begin{definition}
  La \emph{pseudo inverse} d'une matrice 
  \begin{displaymath}
    D =
    \begin{pmatrix}
      \sigma_1 \\
      & \sigma_2 \\
      & & \ddots \\
      & & & \sigma_r \\
      & & & & 0 \\
      & & & & & \ddots  \\
      & & & & & & 0  \\      
    \end{pmatrix}
    \in \R^{m \times n}
  \end{displaymath}
où $\sigma_i \in \R_{>0}$ est 
\begin{displaymath}
  D^+ =  \begin{pmatrix}
      \sigma_1^{-1} \\
      & \sigma_2^{-1} \\
      & & \ddots \\
      & & & \sigma_r^{-1} \\
      & & & & 0 \\
      & & & & & \ddots  \\
      & & & & & & 0  \\      
    \end{pmatrix}
    \in \R^{n \times m}
\end{displaymath}
Toutes les composantes qui ne sont pas décrites sont zéro. 
La \emph{pseudo inverse} d'une matrice $A \in \C^{m \times n}$ avec une décomposition en valeurs singulières $A = P \cdot D \cdot Q$ est 
\begin{displaymath}
  A^+ = Q^* D^+ P^*. 
\end{displaymath}

\end{definition}


\begin{example}
  La pseudo inverse de la matrice $A$ d'exemple~\ref{exe:19} est 
  \begin{displaymath}
    A^+ = 
\begin{pmatrix}0 & 0 & 1\\1 & 0 & 0\\0 & 1 & 0\end{pmatrix} \cdot 
\begin{pmatrix}0.5 & 0 & 0 & 0\\0 & 1 & 0 & 0\\0 & 0 & 0 & 0\end{pmatrix}
\cdot
\begin{pmatrix}-0.8 & 0.6 & 0 & 0\\0.6 & 0.8 & 0 & 0\\0 & 0 & 1 & 0\\0 & 0 & 0 & 1\end{pmatrix} 
  \end{displaymath}
\end{example}

Pourquoi est-ce que nous parlons de \emph{la} pseudo inverse? Parce qu'elle est unique. 

\begin{theorem}
  \label{thr:22}
  Soit  $A \in ℂ^{m \times n}$, alors il existe au plus une seule matrice $X \in ℂ^{n \times m}$  telle que les quatre conditions de \emph{Penrose} sont satisfaites: 
  \begin{enumerate}[i)] 
  \item $AXA = A$ \label{pen1}
  \item $(AX)^* = AX$ \label{pen2}
  \item $XAX = X$ \label{pen3}
  \item $(XA)^* = XA$. \label{pen4}
  \end{enumerate}
\end{theorem}
\begin{proof}
  Soient $X$ et $Y$ deux matrices satisfaisant \ref{pen1}-\ref{pen4}. Alors
  \begin{eqnarray*}
    X & = & XAX \\
     & = & XAYAX \\
     & = & XAYAYAYAX\\
    & = & (XA)^*(YA)^*Y(AY)^*(AX)^*\\
    & = & A^*X^*A^*Y^*YY^*A^*X^*A^* \\
    & = & (AXA)^*Y^*YY^*(AXA)^* \\
    & = & A^* Y^* Y Y^* A^* \\
    & = & (YA)^* Y (AY)^* \\
    & = & YAYAY \\
    & = & YAY \\
    & = & Y. 
  \end{eqnarray*}
\end{proof}

\begin{theorem}
  \label{thr:23}
  La pseudo inverse d'une matrice $A \in \C^{m \times n}$ satisfait les conditions \ref{pen1}-\ref{pen4}. 
\end{theorem}

\begin{proof}
  Soit $A = PDQ$ une décomposition en valeurs singulières et $A^+ = Q^*D^+P^*$.  Il est facile de voir que $D^+$ satisfait les conditions \ref{pen1}-\ref{pen4} relatives à $D$. Les conditions sont aussi vite montrées pour $A$ et $A^+$. Par exemple \ref{pen1} est montrée comme suit : 
  \begin{eqnarray*}
    A A^+A & = & PDQQ^*D^+P^*PDQ \\
           & = & PDD^+DQ \\
           & = & PDQ\\
           & = & A. 
  \end{eqnarray*}
Il est un exercice de vérifier les  conditions \ref{pen2}-\ref{pen4}.
\end{proof}


\section{Encore les systèmes d'équations }
\label{sec:encore-les-systemes}
Nous considérons encore une fois un système 
\begin{equation}
  \label{eq:14}
  Ax = b, 
\end{equation}
où $A \in \C^{m \times n}$ et $b \in \C^m$. 

\begin{definition}
  \label{def:25}
  La \emph{solution minimale} de \eqref{eq:14} est 
  la solution du problème des moindres carrés
  \begin{displaymath}
    \min_{x \in \C^n} \|Ax - b\|^2 
  \end{displaymath}
    correspondant avec norme $\|x\|$ minimale.
\end{definition}



\begin{theorem}
  \label{thr:24}
  La solution minimale de \eqref{eq:14} est 
  \begin{displaymath}
    x = A^+ b,
  \end{displaymath}
où $A^+$ est la pseudo inverse de $A$. 
\end{theorem}

\begin{proof}
Tout d'abord on remarque que pour $B\in \mathbb{C}^{n \times n}$ unitaire et $ y\in \mathbb{C}^n, \|y\|^2=y^T\overline {y}=y^TB^T\overline { B}\overline { y }=\|By\|^2$. Ainsi on a 
  \begin{eqnarray*}
    \min_{x \in \C^n} \|Ax - b\| & = &      \min_{x \in \C^n} \|PDQx - b\| =     \min_{x \in \C^n} \|P^* (PDQx - b)\|        \\
     & = &     \min_{x \in \C^n} \|DQx - P^*b\| =     \min_{y \in \C^n} \|Dy - P^*b \| \\
      & = &       \min_{y \in \C^n} \|Dy - c \|
  \end{eqnarray*}
où $c  = P^*b$. Dès lors on peut facilement vérifier que $y$ est une solution minimale de $Dy=c \Leftrightarrow  Q^*y$ est une solution minimale de $Ax=b$. Mais comme les solutions optimales de $Dy=c$  sont les $y \in \C^n$ tels que $y_i = c_i / \sigma_i $ pour $1 \leq i \leq r$ et $y_{r+1} \dots y_n$ sont arbitraires alors la solution où $y_{r+1} =\dots= y_n=0$ est celle de norme minimale. Elle est donnée par
\begin{displaymath}
  y = D^+ c. 
\end{displaymath}
La solution minimale de \eqref{eq:14} est alors 
\begin{displaymath}
  x = Q^*y = Q^* D^+ P^* b = A^+b. 
\end{displaymath}
\end{proof}

\begin{example}
  \label{exe:20}
  Trouver la solution minimale du système 
  \begin{displaymath}
     \begin{pmatrix}
      0 & -1.6  & 0.6 \\
      0 & 1.2 & 0.8 \\
      0 & 0 & 0 \\
      0 & 0 & 0
    \end{pmatrix} x =
    \begin{pmatrix}
      5\\7\\3\\-2
    \end{pmatrix}.
  \end{displaymath}

La pseudo-inverse  de la matrice ci-dessus est 
\begin{displaymath}
  A^+ = \begin{pmatrix}0 & 0 & 0 & 0\\-0.4 & 0.3 & 0 & 0\\0.6 & 0.8 & 0 & 0\end{pmatrix}
\end{displaymath}
et 
\begin{displaymath}
  \begin{pmatrix}0 & 0 & 0 & 0\\-0.4 & 0.3 & 0 & 0\\0.6 & 0.8 & 0 & 0\end{pmatrix} \cdot
  \begin{pmatrix}
    5\\7\\3\\-2
  \end{pmatrix}
 =
 \begin{pmatrix}
   0\\0.1\\8.6
 \end{pmatrix}. 
\end{displaymath}
\end{example}





\section{Le meilleur sous-espace approximatif} 
\label{sec:le-meilleur-sous}

Nous nous occupons du problème suivant. Soient $a_1,\dots,a_m \in \R^n$ des vecteurs et $1 \leq k \leq n$, trouver un sous-espace $H \subseteq \R^n$ de dimension $k$ tel que 
\begin{displaymath}
  \sum_i d(a_i,H)^2 
\end{displaymath}
soit minimale. Ici $d(a_i,H)$ est la \emph{distance} de $a_i$ a $H$. Si $H = \spa\{u_1,\dots,u_k\}$ où $\{u_1,\dots,u_k\}$  est une base orthonormale de $H$, alors $a_i = \sum_{j=1}^k \pscal{a_i,u_j} u_j + d_i$ où $d_i = a_i - \sum_{j=1}^k \pscal{a_i,u_j} u_j$ est orthogonal à $u_1,\dots,u_k$ et alors à $H$.  Avec le théorème de Pythagore (Proposition~\ref{prop:4}), on a
\begin{displaymath}
  d(a_i,H)^2 + \sum_{j=1}^k \pscal{a_i,u_j}^2 = \|a_i\|^2. 
\end{displaymath}
Le sous-espace $H$ de dimension $k$ qui minimise $\sum_i d(a_i,H)^2$ est alors celui qui maximise 
\begin{displaymath}
\sum_i  \sum_{j=1}^k \pscal{a_i,u_j}^2 = \sum_{j=1}^k \|A u_j\|^2 = \sum_{j=1}^k u_j^T A^TAu_j.
\end{displaymath}

Pour $k=1$, nous connaissons déjà une manière de résoudre ce problème. Il faut résoudre 
\begin{displaymath}
\max_{u \in S^{n-1}}   u^TA^TAu .
\end{displaymath}
La matrice $A^TA$ est symétrique, alors on peut la factoriser comme 
\begin{equation}
  \label{eq:17}
  A^TA = U
  \begin{pmatrix}
    \lambda_1 \\
    & \ddots \\
    & & \lambda_n
  \end{pmatrix} U^T
\end{equation}
où $U = \left(u_1,\dots,u_n\right) \in \R^{n \times n}$ est orthogonale. Nous pouvons supposer que les $\lambda_i$ sont ordonnés comme $\lambda_1 \geq \lambda_2 \geq \cdots \geq \lambda_n \geq 0$. Les valeurs propres sont non négatives dès que $A^TA$ est semi-définie positive. Selon Théorème~\ref{thr:18} la solution est $H = \spa\{u_1\}$. 

\medskip 
La généralisation suivante du Théorème~\ref{thr:18} est un exercice. 

\begin{theorem}
\label{thr:25}
  Soit $A \in \R^{n\times n}$ une matrice symétrique et $f(x) = x^TAx$ la forme quadratique correspondante à $A$.
Soit 
\begin{displaymath}
  A = U \cdot
  \begin{pmatrix}
    \lambda_1\\
    & \ddots \\
    & & \lambda_n
  \end{pmatrix}
U^T
\end{displaymath}
une factorisation de $A$ telle que $U = (u_1,\dots,u_n) \in \R^{n \times n }$ est orthogonale et $\lambda_1 \geq \cdots \geq \lambda_n$. 
Pour $1 \leq \ell <n$ on a
  \begin{equation}
\label{eq:15}
    \max_{\substack{x \in S^{n-1} \\ x \perp u_1, \dots, x\perp u_\ell}} f(x) = \lambda_{\ell +1}  = \min_{\substack{x \in S^{n-1} \\ x \perp u_{\ell+2}, \dots, x\perp u_n}} f(x)
  \end{equation}
et $u_{\ell+1}$ est une solution optimale. 
\end{theorem}



Maintenant, nous pouvons résoudre le problème central 
\begin{equation}
  \label{eq:16}
  \min_{\substack{H \trianglelefteq \R^n \\ \dim(H) = k}} \sum_{i=1}^m d(a_i,H)^2. 
\end{equation}

\begin{theorem}
  \label{thr:26}
  Soient $a_1,\dots,a_m \in \R^n$,
$
    A =\left(
      a_1,\cdots ,a_m\right)^T
 $
  et $u_1,\dots,u_k$  les premières colonnes de la matrice orthogonale $U \in \R^{n \times n}$ de la factorisation~\eqref{eq:17}. Le sous-espace $H = \spa\{u_1,\dots,u_k\}$ est une solution du problème~\eqref{eq:16}. 
\end{theorem}


\begin{proof}
Pour $k=1$ nous avons déjà montré l'assertion. Soit $k \geq 2$ et $W \trianglelefteq \R^n$ une solution optimale du problème~\eqref{eq:16} et soit $w_1,\dots,w_k$ une base orthonormale de $W$. 
Nous pouvons supposer $w_k \perp \spa\{u_1,\dots,u_{k-1}\}$, (voir Exercice~\ref{item:3}). 

Par induction, nous avons 
\begin{displaymath}
  \sum_{j=1}^{k-1} w_j^TA^TAw_j \leq  \sum_{j=1}^{k-1} u_j^TA^TAu_j.
\end{displaymath}
Dès que   
\begin{displaymath}
  \max_{\substack{x \in S^{n-1} \\ x \perp \spa\{u_1,\dots,u_{k-1}\}}} x^TA^TAx 
\end{displaymath}
est atteint à $u_k$ nous avons 
\begin{displaymath}
  w_k^TA^TAw_k \leq u_k^TA^TAu_k
\end{displaymath}
et alors
\begin{displaymath}
  \sum_{j=1}^{k} w_j^TA^TAw_j \leq  \sum_{j=1}^{k} u_j^TA^TAu_j.
\end{displaymath}

\end{proof}


\begin{definition}
  \label{def:26}
  Soit $A \in \R^{m \times n}$.
  \begin{enumerate}
 
  \item[1)] On définit la \emph{norme Frobenius} de $A$ comme le nombre 
  \begin{displaymath}
    \|A\|_F = \sqrt{\sum_{ij}a_{ij}^2}.
  \end{displaymath}
  
  \item[2)] Pour une norme vectorielle $\|\cdot\|$ définie positive sur $\Bbb R^n$, on définit la norme $\mnorm{\cdot}$ de $A$, appelée \emph{norme matricielle subordonnée} à $\|\cdot\|$, comme le nombre 
   \begin{displaymath}
    	\mnorm{A} = \sup_{v \neq 0} \frac{\|Av\|}{\|v\|} = \sup_{\|v\| = 1} \|Av\|.
  \end{displaymath}
  
   \end{enumerate}
\end{definition}

\begin{remark}
	La norme Frobenius n'est pas une norme matricielle subordonnée.
\end{remark}

\begin{example}
	Soit $A \in \R^{m \times n}$. On considère la norme euclidienne standard (aussi appelée norme 2) sur $\Bbb R^n$ : 
	$$\| v \|_2 = \sqrt{\sum\nolimits_{i=1}^n (v_i)^2}.$$
	Alors la norme matricielle subordonnée à $\|\cdot\|_2$ correspond à
	\begin{displaymath}
    	\mnorm{A}_2 = \sup_{v \neq 0} \frac{\|Av\|_2}{\|v\|_2} = \sup_{\|v\|_2 = 1} \|Av\|_2 = \sup_{\|v\|_2 = 1} \sqrt{v^T A^T A v} = \sqrt{\lambda_1},
  	\end{displaymath}
  	où $\lambda_1$ est la plus grande valeur propre de $A^T A$ (ou de manière équivalente, $\sqrt{\lambda_1}$ est la plus grande valeur singulière de $A$).
\end{example}

\begin{exercise}
	Soient $A \in \R^{m \times n}$, $\|\cdot\|$ une norme définie positive sur $\Bbb R^n$, $\mnorm{\cdot}$ la norme matricielle subordonnée à $\|\cdot\|$. Alors pour tout $v \in \Bbb R^n$, on a : 
	\begin{displaymath}
		\|Av\| \leq \mnorm{A} \cdot \|v\|.
	\end{displaymath}
\end{exercise}

\begin{definition}
  \label{def:27}
  Soit $A \in K^{n \times n}$. La \emph{trace} de $A$ est la somme de ses coefficients diagonaux, $\Tr(A) = \sum_{i=1}^n a_{ii}$. 
\end{definition}

\begin{lemma}
  \label{lem:11}
  Pour $A,B \in K^{n \times n}$ $\Tr(AB) = \Tr(BA)$. 
\end{lemma}


\begin{lemma}
  \label{lem:10}
  Pour $A \in \R^{m \times n}$, $\|A\|_F^2 = \sum_{i=1}^r \sigma_i^2$ où $\sigma_1, \dots, \sigma_r$ sont les valeurs singulières de $A$. 
\end{lemma}

\begin{proof}
  On a $\|A\|_F^2 = \Tr(A^TA) = \Tr(U \cdot \diag(\sigma_1^2,\dots,\sigma_n^2) \cdot  U^T)$ où $U \in \R^{n \times n}$ est orthogonale. Alors 
  \begin{displaymath}
    \|A\|_F^2 = \Tr(\diag(\sigma_1^2,\dots,\sigma_n^2)) =  \sum_{i=1}^r \sigma_i^2
  \end{displaymath}
\end{proof}




Maintenant nous allons résoudre le problème suivant. Étant donnés $A \in \R^{m \times n}$ et $k \in \N$, trouver une matrice $B \in \R^{m \times n}$ de $\rank(B) \leq  k$ tel que 
\begin{displaymath}
  \|A - B\|_F
\end{displaymath}
soit minimale. 

Si $A = P \cdot \diag(\sigma_1,\dots,\sigma_r,0,\dots 0) \cdot Q$ est une décomposition en valeurs singulières où les colonnes de $P$ sont $v_1,\dots,v_m$ et les lignes de $Q$ sont $u_1^T,\dots,u_n^T$ on peut écrire
\begin{equation}
  \label{eq:18}
  A = \sum_{i=1}^r \sigma_i v_iu_i^T
\end{equation}
et on dénote la somme des premiers $k$ termes comme 
\begin{displaymath}
  A_k = \sum_{i=1}^k \sigma_i v_iu_i^T
\end{displaymath}
Le rang de $A_k$ est au plus $k$. 

\begin{lemma}
  \label{lem:12}
  Les lignes de $A_k$ sont les projections des lignes de $A$ dans le sous-espace $V_k = \spa\{u_1,\dots,u_k\}$. 
\end{lemma}

\begin{proof}
  Soit $a^T$ une ligne de $A$. La projection de $a$ dans le sous-espace $\spa\{u_1,\dots,u_k\}$ est 
  \begin{displaymath}
    \sum_{i=1}^k a^Tu_i \cdot u^T_i.
  \end{displaymath}
Alors les projections des lignes de $A$ dans le sous-espace $V_k$ sont données par $\sum_{i=1}^k A u_i u_i^T = \sum_{i=1}^k \sigma_i v_i u_i^T = A_k$.  
\end{proof}

\begin{theorem}
  \label{thr:27}
  Pour une matrice $B \in \R^{m \times n}$ de rang plus petit ou égal à $k$, on a 
  \begin{displaymath}
    \|A - A_k\|_F \leq \|A - B\|_F.
  \end{displaymath}
\end{theorem}

\begin{proof}
On dénote les lignes de $A$ par $a_1^T,\dots,a_m^T$ et soit $B$ une matrice de rang au plus $k$. Les lignes de $B$ sont dénotées comme $b_1^T,\dots,b_m^T$. Soit $H = span\{b_1,\dots,b_m\}$. La dimension de $H$ est $\rank(B) \leq k$. On a 
\begin{displaymath}
  \|A - B\|_F^2 = \sum_{i=1}^m \|a_i - b_i\|^2 \geq \sum_{i=1}^m d(a_i,H)^2.
\end{displaymath}
Soit $\wt{H} = \spa\{u_1,\dots,u_k\}$. Nous avons démontré que 
\begin{enumerate}[i)]
\item $\wt{H}$ est le meilleur sous-espace approximatif des lignes de $A$ alors $\sum_{i=1}^m d(a_i,H)^2 \geq  \sum_{i=1}^m d(a_i,\wt{H})^2$ et 
\item Les lignes de $A_k$ sont les projections des lignes de $A$ dans $\wt{H}$. 
\end{enumerate}
En dénotant les lignes de $A_k$ par $\wt{a}_1^T,\dots,\wt{a_m}^T$, alors 
\begin{displaymath}
  \|A - B\|_F^2 \geq  \sum_{i=1}^m d(a_i,\wt{H})^2 = \sum_{i=1}^m \|a_i - \wt{a}_i\|^2 = \|A - A_k\|_F^2. 
\end{displaymath}
\end{proof}

\subsection*{Exercices}

\begin{enumerate}
\item Est-ce que la décomposition en valeurs singulières est unique? Est-ce que les valeurs singulières sont uniques?  \label{item:2}
\item  Dans la démonstration du théorème~\ref{thr:21}, montrer que $\rank(A) = r$. 
\item Démontrer que la pseudo-inverse satisfait les conditions \ref{pen2}-\ref{pen4}. 
\item Si $Ax = b$ a plusieurs solutions, il existe une solution unique avec une norme minimale. 
\item Montrer Théorème~\ref{thr:25}. 
\item Soient  $G,H \subseteq \R^n$ des sous-espaces de $\R^n$ 
et $k=\dim(G) > \dim(H)$. Montrer que $G$ possède une base orthonormale $w_1,\dots,w_k$ telle que $w_k \perp H$. \label{item:3}  
\end{enumerate}






%%% Local Variables:
%%% mode: latex
%%% TeX-master: "notes"
%%% End:

\chapter{Systèmes différentiels linéaires}
\label{cha:syst-diff-line}

On considère le système différentiel suivant
\begin{equation}
  \label{eq:19}
  \begin{array}{ccccc}
    \x'_1(t) & = &a_{11}\x_1(t) & + \cdots + & a_{1n}\x_n(t) \\
    \x'_2(t) & = &a_{21}\x_1(t) & + \cdots + & a_{2n}\x_n(t) \\
            & \vdots \\             
    \x'_n(t) & = &a_{n1}\x_1(t) & + \cdots + & a_{nn}\x_n(t)
  \end{array}
\end{equation}
où les $a_{ij} \in \R$.  
En notation matricielle, on peut écrire le système comme
\begin{displaymath}
  \x' = A\,\x
\end{displaymath}
où 
\begin{displaymath}
  A =
  \begin{pmatrix}
    a_{11} & \cdots & a_{1n}\\
          & \vdots & \\
          a_{n1} & \cdots & a_{nn}\\          
  \end{pmatrix}, \quad \x \in C^{1}(\R; \R^{n}).
\end{displaymath}

On cherche des fonctions dérivables $\x_i: \R \longrightarrow \R$  qui, ensemble, constituent $\x$ et qui  satisfont \eqref{eq:19}. Un tel $\x$ est une \emph{solution} du système~\eqref{eq:19}. 



\begin{example}
  \label{exe:49}
  Considérons l'équation différentielle $\x'(t) = \x(t)$. Une solution est $\x(t) = e^t$. Une autre solution est $\x(t) = 2\cdot e^t$. Si on spécifie la \emph{condition initiale} $\x(0) = 1$, alors $\x(t) = e^t$ est l'unique solution qui satisfait cette condition initiale. Généralement, si on spécifie $\x(0) = \alpha$, alors  $\x(t) = \alpha \cdot e^t$ est la solution qui satisfait la condition initiale. 

Considérons $\x'(t) = -\x(t)$, une solution est $\x(t) = e^{(-t)}$. %$\x(t) = \cos(t)$.
C'est aussi une solution qui respecte la condition initiale $\x(0) = 1$. 
\end{example}



Essayons d'abord de résoudre le système en mettant  $\x(t) = e^{\lambda t} v$ où $v \in \R^n$ est un vecteur constant. Dans ce cas $\x' = A\x$ se récrit comme   $\lambda e^{\lambda t}  v = e^{\lambda t} A v$. Nous avons démontré le lemme suivant. 

\begin{lemma}
  \label{thr:29}
  Si $\lambda \in \R$ est une valeur propre de $A$ et si $v \in \R^n \setminus \{0\}$ est un vecteur propre correspondant, alors $\x(t) = e^{\lambda t} v$ est  une solution du système~\eqref{eq:19} pour les conditions initiales $\x(0) = v$. 
\end{lemma}


Le théorème suivant est démontré en cours \emph{analyse 2}. 
\begin{theorem}[Cours d'analyse II] 
  \label{thr:28}
  Étant donné les \emph{conditions initiales} $\x(0)$
  il existe une unique solution $\x$ du système~\eqref{eq:19}. 
\end{theorem}
Nous sommes concernés par le problème de \emph{trouver} la solution $\x$ explicitement. On commence avec une observation qui est un exercice simple. 

\begin{lemma}
  \label{lem:13}
  L'ensemble $\X = \{ \x \colon \x \text{ est une solution du système \eqref{eq:19}}\}$ est un espace vectoriel sur $\R$.  
\end{lemma}

Est-ce que c'est possible de donner  une base de $\X$ explicitement? Dans le cas où $A$ est diagonalisable : 
\begin{displaymath}
  A = P \cdot \diag(\lambda_1,\dots,\lambda_n) \cdot P^{-1} 
\end{displaymath}
où $P \in \R^{n \times n}$ est inversible et les  $\lambda_i$ sont réels, le théorème suivant décrit une base intuitive de $\X$.

\begin{theorem}
  \label{thr:30}
  Si $\R^n$ possède une base $\{v_1,\dots,v_n\} \subseteq \R^n$ de vecteurs propres de $A$ telle que $A \, v_i = \lambda_i v_i$, alors  
  \begin{displaymath}
    \x^{(i)}(t) = e^{\lambda_i t} \cdot v_i, \, i=1,\dots,m
  \end{displaymath}
est une base de $\X$. 
\end{theorem}

\begin{proof}
  Montrons d'abord que les $\x^{(i)}$ sont linéairement indépendants. Supposons que $\sum_{i} \alpha_i \x^{(i)} = 0$. C'est-à-dire que les $n$ fonctions qui sont les composantes de $\sum_{i} \alpha_i \x^{(i)}$ sont toutes la fonction identiquement nulle. En évaluant en $t=0$, on trouve
  \begin{displaymath}
    0 = \sum_i \alpha_i v_i e^{\lambda_i 0} = \sum_i \alpha_i v_i.  
  \end{displaymath}
Or les $v_i$ sont linéairement indépendants. On a donc $\alpha_i = 0$ pour tout $i$ ce qui démontre que les $\x^{(i)}$ sont linéairement indépendants. 


Maintenant soit $\y \in \X$ et soient  $\alpha_i \in \R$  tels que 
\begin{displaymath}
  \y(0) = \sum_i \alpha_i v_i.  
\end{displaymath}
Alors $\x := \sum_i \alpha_i \x^{(i)} \in \X$  et comme $\x(0) = \y(0)$, le Théorème~\ref{thr:28} implique que $\x = \y$. Les $\x^{(i)}$ engendrent donc $\X$, et $\{\x^{(1)},\dots, \x^{(n)}\}$ est une base de $\X$. 
\end{proof}


Est-ce qu'on peut aussi trouver une solution dans la cas où $A$ est diagonalisable dans les nombres complexes, donc si 
\begin{displaymath}
A=P \cdot \diag(\lambda_1,\dots,\lambda_n) \cdot P^{-1} 
\end{displaymath}
où $P \in \C^{n \times n}$ est inversible et les  $\lambda_i \in \C$? Pour discuter de ça, il faut d'abord définir, ce qu'est une solution complexe du système~\eqref{eq:19}. Toute fonction $f: \R \longrightarrow \C$ s'écrit comme 

\begin{displaymath}
  f(x) = f_{\Re}(x) + i \cdot f_{\Im}(x) 
\end{displaymath}
où $f_{\Re}(x), f_{\Im}(x)$ sont des fonctions de $ \R \longrightarrow \R$.  Si $f_\Re$ et $f_\Im$ sont dérivables, on dit que $f(x)$ est dérivable et on définit 
\begin{displaymath}
  f'(x) = f_\Re'(x) + i \cdot f_\Im'(x). 
\end{displaymath}
Si $\x_1,\dots,\x_n\colon \R \longrightarrow \C$ sont dérivables, comme avant 
\begin{displaymath}
  \x =
  \begin{pmatrix}
    \x_1\\ \vdots \\ \x_n
  \end{pmatrix}
\end{displaymath}
est une \emph{solution complexe} du système~\eqref{eq:19} si $\x' = A\x$.   Et comme avant, on peut noter le lemme suivant, en se rappellant que $e^{a + ib} = e^a (\cos b + i \cdot \sin b)$. 


\begin{lemma}
  \label{lem:15}
  Si $\lambda \in \C$ est une valeur propre de $A$ et si $v \in \C^n \setminus \{0\}$ est un vecteur propre correspondant, alors $\x(t) = e^{\lambda t} v$ est  une solution du système~\eqref{eq:19} pour les conditions initiales $\x(0) = v$. 
\end{lemma}
\begin{proof}
  On écrit 
  \begin{displaymath}
    \x' = \lambda e^{\lambda t} v = e^{\lambda t} Av = A\x.
  \end{displaymath}
\end{proof}


\begin{lemma}
  \label{lem:14}
  Étant donné une solution complexe $\x = \x_\Re + i \x_\Im$ du système~\eqref{eq:19}, alors $\x_\Re$ et $\x_\Im$ sont des solutions réelles.  
\end{lemma}
\begin{proof}
  Dès que $\x_\Re + i \x_\Im $ est une solution, on a 
  \begin{displaymath}
   \x'_\Re+ i \x'_\Im = \x' = A \x = A\x_\Re + i A\x_\Im. 
  \end{displaymath}
Comme $A $ est réelle on a $\x_\Re' = A\x_\Re$ et $\x'_\Im = A \x_\Im$ en prenant les parties réelles et imaginaires des deux côtés. 
\end{proof}


Supposons alors que $A \in \R^{n \times n}$ est diagonalisable . Et soit $\{v_1,\dots,v_n\}$ une base de $\C^n$ de vecteurs propres associés à 
$\lambda_1,\dots,\lambda_n$ respectivement. Si $v_i = u_i + i \cdot w_i$  où $u_i,w_i \in \R^n$, les $u_1,\dots,u_n,w_1,\dots,w_n$ engendrent $\R^n$ (voir exercice~\ref{item:5}). Comme nous avons noté 
\begin{displaymath}
  \x^{(j)} = e^{\lambda_j t} v_j
\end{displaymath}
sont des solutions complexes du système~\eqref{eq:19}.   

Aussi, on peut supposer que la base et les valeurs propres sont tels que les vecteurs/valeurs propres complexes viennent en paires conjugées complexes. Plus précisément, si $2k$ valeurs propres sont complexes et le reste sont réelles, on pose
\begin{equation}
\label{eq:22}
  v_{2j-1} = \overline{v_{2j}}\, \text{ et } \, \lambda_{2j-1} = \overline{\lambda_{2j}} \, \text{ pour } \, 1 \leq j \leq k \leq n/2 
\end{equation}
et 
\begin{equation}
  \label{eq:23}  
  v_j \in \R^n, \lambda_j \in \R \text{ pour } j > 2k. 
\end{equation}
%
Considérons maintenant une solution donnée par $v = u+iw$  $\lambda= a+ib$. 
\begin{eqnarray*}
  \x & = & e^{a \, t} \left(\cos (b t)  + i \sin (b t ) \right)  (u + i w)  \\
   & = & e^{a \, t} \left(\cos (b t) u - \sin (bt)w \right)  + ie^{a \, t} \left(\sin (b t ) u + \cos(bt)w \right). 
\end{eqnarray*}
Ceci nous donne alors deux solutions réelles 
\begin{eqnarray*}
  \x^{(1)} & = & e^{a \, t} \left(\cos (b t) u - \sin (bt)w \right), \\
  \x^{(2)} & = &  e^{a \, t} \left(\sin (b t ) u + \cos(bt)w \right). 
\end{eqnarray*}
\begin{remark}
  \label{rem:2}
  Les solutions réelles données par $v$ et $\lambda$ engendrent le même espace que les solutions réelles données par $\overline{v}$ et $\overline{\lambda}$. 
\end{remark}

Nous pouvons alors noter une marche à suivre pour résoudre le système~\eqref{eq:19} étant donné $\x(0)$ si $A$ est diagonalisable.
\begin{enumerate}
\item Trouver une base de vecteurs propres $v_1,\dots,v_n$ de $A$ ordonnée comme dans \eqref{eq:22} et \eqref{eq:23}. 
\item Pour chaque paire $v_{2j},\lambda_{2j}$, $1 \leq j \leq k$ trouver les  deux solutions réelles dénotées par  $\x^{(2j-1)}$ et $\x^{(2j)}$. 
\item Pour chaque paire réelle $v_j, \lambda_j$ $n\geq j>2k$, trouver la solution $\x^{(j)}$. 
\item Trouver la combinaison linéaire 
  \begin{displaymath}
    \x(0) = \sum_{j} \alpha_j \x^{(j)}(0)
  \end{displaymath}
\item La solution est 
  \begin{displaymath}
    \x = \sum_{j} \alpha_j \x^{(j)} 
  \end{displaymath}
\end{enumerate}



\begin{example}
  \label{exe:21}
  Résoudre le système $\x' = A\x$ où 
  \begin{displaymath}
    A  =
    \begin{pmatrix}
      1 & 2 \\
      -2 & 1
    \end{pmatrix} \, \text{ et } \, \x(0) =
    \begin{pmatrix}
      1\\1
    \end{pmatrix}. 
  \end{displaymath}
 On trouve que $\lambda_1 = 1 + 2 i$ et $\lambda_2 = 1 - 2i$ sont les valeurs propres de $A$ et 
 \begin{displaymath}
   v_1 =
   \begin{pmatrix}
     1\\i
   \end{pmatrix} \text{ et } v_2 =
   \begin{pmatrix}
     1 \\ -i
   \end{pmatrix}
 \end{displaymath}
sont les vecteurs propres correspondants. 
Les deux solutions impliquées par $v_1$ sont 
\begin{eqnarray*}
  \x^{(1)} & = & e^{ t} \left(\cos ( 2t)
                 \begin{pmatrix}
                   1\\0
                 \end{pmatrix}
- \sin (2t)\
  \begin{pmatrix}
    0\\1
  \end{pmatrix}
\right) \\
  \x^{(2)} & = &  e^{t} \left(\sin ( 2t )
                 \begin{pmatrix}
                   1\\0
                 \end{pmatrix}
+ \cos(2t)
  \begin{pmatrix}
    0\\1
  \end{pmatrix}
\right). 
\end{eqnarray*} 
La solution qu'on cherche est 
\begin{displaymath}
  \x = \begin{pmatrix}
           e^{t} \sin (2t) + e^{t} \cos(2t) \\
           - e^{t} \sin(2t) + e^{t} \cos(2t)
         \end{pmatrix}.
\end{displaymath}

\end{example}






% Une situation très agréable est si $A$ est diagonalisable. Soit $P^{-1}AP = \diag(\lambda_1,\dots,\lambda_n)$ où $P = (v_1,\dots,v_n)$. 
% Avec le changement de variables $\x = P\cdot \y $ on écrit
% \begin{displaymath}
%   \y' = P^{-1} \x' = P^{-1} A\x = P^{-1}AP\y = \diag(\lambda_1,\dots,\lambda_n) \y. 
% \end{displaymath}
% Le système 
% \begin{equation}
%   \label{eq:20}  
%   \y' = \diag(\lambda_1,\dots,\lambda_n) \y 
% \end{equation}
% est découplé et les conditions initiales sont $\y(0) = P^{-1} \x(0)$. La solution est $\y_i(t) = \y_i(0) e^{\lambda_i \cdot t}$ et $\x = P \cdot \y$ est la solution du système~\eqref{eq:19} pour les conditions initiales $\x(0)$. 

% \begin{remark}
%   \label{rem:1}
%   Notez que, même si $A$
%   est diagonalisable, seulement dans les nombre complexes, les
%   fonctions $\x$ sont réelles.
% \end{remark}





\subsection*{Exercices} 

\begin{enumerate}
\item Montrer Lemme~\ref{lem:13}. 
\item Une fonction $f:\C \longrightarrow \C$ est \emph{holomorphe} en $z_0 \in \C$ si
  \begin{displaymath}
    f'(z_0) = \lim_{z \rightarrow z_0} \frac{f(z) - f(z_0)}{z - z_0} 
  \end{displaymath}
existe. Soit $f$ holomorphe sur $\C$ et $g = f_{|\R}$ la fonction $f$ réduite à $\R$. 
 Montrer 
\begin{enumerate}[i)]
\item  $g(x) = g_\Re(x) + i \cdot g_\Im(x)$ est dérivable au  sens de notre définition, particulièrement $g_\Re(x)$ et $ g_\Im(x)$ sont dérivables. 
\item $f'_{| \R} (x) = g'_\Re(x) + i \cdot g'_\Im(x)$. 
\end{enumerate}
\item Soit $\{u_1+ i \cdot w_1,\dots,u_n + i \cdot w_n\}$ une base de $\C^n$ où $u_i,w_i \in \R^n$  pour tout $i$. Montrer que $\spa\{u_i, w_i \colon 1 \leq i \leq n\} = \R^n$. \label{item:5}
\end{enumerate}






\section{L'exponentielle d'une matrice}
\label{sec:lexp-dune-matr}



\begin{definition}
  \label{def:28}
  Pour $A \in \C^{n \times n}$ on définit 
  \begin{displaymath}
    e^A = I + A + \frac{1}{2!} A^2 + \frac{1}{3!}A^3 + \cdots 
  \end{displaymath}
\end{definition}

\noindent On rappelle la définition d'une série intégrable 
\begin{displaymath}
  \sum_{j=0}^\infty a_j z^j,
\end{displaymath}
où les coefficients $a_j \in\C$, et
qui converge sur un \emph{disque} de rayon $\rho$. C'est à dire que, si $|z|< \rho$ la série converge et la fonction $f\colon \{x \in \C \colon |x| < \rho \}  \rightarrow \C$ définie par $f(x) = \sum_{j=0}^\infty a_j x^j $ est \emph{holomorphe} avec dérivée $f'(x) =  \sum_{j=0}^\infty j a_j x^{j-1}$. 
Une série intégrable importante est la série
\begin{displaymath}
  e^{x} = \sum_{j=0}^\infty \frac{1}{j!} x^j,
\end{displaymath}
qui définit la fonction holomorphe $\exp: \C \longrightarrow \C$ 
\begin{displaymath}
  e^{a+i\,b} = e^a (\cos b + i \sin b).  
\end{displaymath}

\noindent On va maintenant généraliser la définition de la \emph{norme Frobenius} pour les matrices complexes. Pour $A \in \C^{m\times n}$, 
\begin{displaymath}
  \|A\|_F = \sqrt{\sum_{ij} |a_{ij}|^2 }. 
\end{displaymath}

\begin{lemma}
  \label{lem:16}
  Pour $A \in \C^{n \times m}$ et $B \in \C^{m × n}$  on a 
  \begin{displaymath}
    \|A\cdot B\|_F \leq \|A\|_F\cdot \|B\|_F. 
  \end{displaymath}
\end{lemma}

  \begin{proof}Soient $a_1^T,\dots,a_n^T \in \C^m$ les lignes de $A$ et $\overline{b_1},\dots,\overline{b_n} \in \C^m$ les colonnes de $B$. Avec Cauchy-Schwarz 
    \begin{displaymath}
          |(AB)_{ij}|^2 = (a_i^T \overline{b_j})(\overline{a_i}^T b_j)  \leq \|a_i\|^2 \|b_j\|^2
    \end{displaymath}
et donc 
\begin{displaymath}
  \|AB\|_F^2 = \sum_{ij} |(AB)_{ij}|^2 \leq \sum_i\|a_i\|^2 \cdot \sum_i \|b_i\|^2 = \|A\|_F^2 \cdot \|B\|_F^2. 
\end{displaymath}
  \end{proof}


  \begin{lemma}
    \label{lem:17}
    La série $e^A$ converge. 
  \end{lemma}

  \begin{proof}
Il est facile de se convaincre que la convergence pour la norme de Frobenius revient à prouver que la suite est de Cauchy. 

Evidemment $\forall c \in \mathbb{C}$ la série $\sum _{ j=0 }^{ +\infty  }{ \frac { { x }^{ j } }{ j! } ={ e }^{ j } } $ converge. La suite des sommes partielles est donc de Cauchy.

Considérons maintenant la suite $(b_n)_{n\in \mathbb{N}} = { (\sum _{ j=0 }^{ n }{ \frac { { A }^{ j } }{ j! } ) }  }_{ n\in \mathbb{N} }$. Soient $m,n \in \mathbb{N}$ alors $\parallel b_{ m }-b_{ n }\parallel _{ F }=\parallel \sum _{ j=n+1 }^{ m }{ \frac { A^{ j } }{ j! }  } \parallel _{ F }\quad \le \quad \sum_{j=n+1}^{m}{\frac{\parallel A\parallel_{F}^{j}}{j!}} \le \quad \epsilon $ pour $n,m \ge N_{\epsilon}$ (qui existe car la suite des sommes partielles de la série $\sum _{ j=0 }^{ +\infty }{ \frac { \parallel A\parallel _{ F }^{ j } }{ j! }  } = e^{\parallel A\parallel _{ F }^{ j }} $ est de Cauchy.)
  \end{proof}


Nous avons montré que $e^{At} = \sum_{k=0}^\infty \frac{t^k}{k!} A^k$ converge pour tout $t \in \R$. Plus précisément chaque composante $\sum_{k=0}^\infty \frac{t^k}{k!} A^k$ est une série intégrable avec un rayon de convergence $\infty$. Nous pouvons donc dériver les termes de la somme pour obtenir 
\begin{equation}
  \label{eq:24}
  \frac{d}{dt} e^{At} = A e^{At}. 
\end{equation}


\begin{theorem}
  \label{thr:31}
  La solution du problème initial $\x' = A\x$, $\x(0) =v$ est 
  \begin{displaymath}
    \x(t) = e^{At} v.
  \end{displaymath}
\end{theorem}

\begin{proof}
  Soit $\x(t) = e^{At} v$. Alors $\x'(t) = A e^{At}v = A\x(t)$. Plutôt $\x(0) = v$. 
\end{proof}

\begin{definition}
  \label{def:29}
  Une matrice $N$ est \emph{nilpotente} s'il existe un $k \in \N$ tel que $N^k = 0$. 
\end{definition}

Nous allons montrer ce théorème dans le prochain cours. 
\begin{theorem}
  \label{thr:32}
  Chaque matrice $A \in \C^{n \times n}$ peut être factorisée comme 
  \begin{displaymath}
    A = P ( \diag(\lambda_1,\dots,\lambda_n) + N) P^{-1}
  \end{displaymath}
où $N \in \C^{n \times n}$ est nilpotente, $P \in \C^{n \times n}$ est inversible,  $\lambda_1,\dots,\lambda_n \in \C$ sont les valeurs propres de $A$ et $\diag(\lambda_1,\dots,\lambda_n)$ et $N$ commutent. 
\end{theorem}


\begin{lemma}
  \label{lem:18}
  Pour $A,B \in \C^{n \times n}$, si $A\cdot B = B \cdot A$ on a $e^{A+B} = e^A e^B$. 
\end{lemma}



Comment peut-on maintenant résoudre le problème initial $\x' = Ax, \, \x(0) = v$ explicitement? Nous savons que cette solution est $\x = e^{tA} \cdot v$ et nous savons que c'est une solution réelle pour $A \in \R^{m \times n}$. Mais les premiers termes s'écrivent comme 
\begin{displaymath}
  \sum_{i=0}^m t^i A^i = P \left(\sum_{i=0}^m t^i \diag(\lambda_1,\dots,\lambda_n)^i + t^i N \right) P^{-1} 
\end{displaymath}
où nous avons utilisé le théorème \ref{thr:32}. Dès que $N$ et $\diag(\lambda_1,\dots,\lambda_n)$ commutent, la solution \emph{réelle} que l'on cherche est  
\begin{eqnarray*}
  \x & = &  P e^{t\diag(\lambda_1,\dots,\lambda_n)} e^{tN} P^{-1} v \\
     & = & P \left( \diag(e^{\lambda_1 \, t},\dots,e^{\lambda_n \, t})\cdot  \sum_{j=0}^{k-1} t^j N^{j} / j!\right)P^{-1},
\end{eqnarray*}
où $k \in \N$ est tel que $N^k = 0$. 


\section{Décomposition selon le polynôme caractéristique}
\label{sec:decomp-selon-le}

Dans ce chapitre, on va démontrer le Théorème~\ref{thr:32}. Dans le chapitre suivant, on va donner une forme normale spécifique pour chaque matrice complexe, le \emph{forme normale de Jordan}. 

% \section{Polynômes}
% \label{sec:polyn-les-lalg}

% Soit $K$ un corps. 
% On dénote l'anneau des polynômes de $K$ par $K[x]$. 
% Un élément de $K[x]$ s'écrit comme 
% \begin{displaymath}
%   p(x) = a_0 + a_1 x + \cdots + a_n x^n 
% \end{displaymath}
% où les \emph{coefficients} $a_i \in K$. 

% La formule de multiplication de deux polynômes $f(x) = a_0+a_1x+ \cdots a_n x^n$ et $g(x) = b_0 + \cdots + b_m x^m$  est 
% \begin{equation}
% \label{eq:21}
%   f(x) \cdot g(x) = \sum_{i = 0}^{m+n} \left(\sum_{k+l = i}  a_{k} b_l\right) x^i
% \end{equation}


% \begin{definition}
%   \label{def:30}
%   Un polynôme $f(x) \in K[x]$
%   tel que $\deg(f) \geq 1$ est \emph{irréductible} si
%   \begin{displaymath}
%     f(x) = g(x) \cdot h(x) 
%   \end{displaymath}
% implique $\deg(g) \cdot \deg(h) = 0$, alors un des facteurs est une constante. 
% \end{definition}




% \begin{definition}
%   \label{def:33}
%   Un diviseur commun de $a(x) \in K[x]$
%   et $b(x) \in K[x]$
%   est un diviseur de $a(x)$
%   et $b(x)$.
%   Un diviseur commun le plus grand de $a(x)$
%   et $b(x)$
%   est un diviseur commun de $a(x)$
%   et $b(x)$
%   tel que tous les autres  diviseurs communs de $a(x)$ et $b(x)$ le divisent. On dénote les plus grands diviseurs communs de $a$ et $b$ par $\emph{pgdc}(a,b)$ (ou, en anglais, $\emph{gcd}(a,b)$, greatest common divisor).
% \end{definition}


% \begin{theorem}
%   \label{thr:36}
%   Soient $a(x),b(x)$ deux polynômes, tels que $\left\{ a,b \right\} \neq \left\{ 0 \right\}$. Un polynôme 
%   \begin{equation}
%     \label{eq:25}   
%     d(x) = g(x) a(x) + h(x) b(x) \neq 0
%   \end{equation}
%   de degré minimal, où $g,h \in K[x]$ , est un plus grand diviseur commun de $a$ et $b$. 
% \end{theorem}

% \begin{proof}
%   On montre qu'un tel $d(x)$ est un diviseur commun de $a$ et $b$ en procédant par l'absurde. Supposons que $d$ ne divise pas $a$. Alors il existe $q$ et $r$ tels que 
%   \begin{displaymath}
%     a = q\cdot d +r 
%   \end{displaymath}
% et $\deg(r) < \deg(d)$. Alors 
% \begin{displaymath}
%   r = a - q\cdot d = (1 - g\,q) a - h\,q\,b
% \end{displaymath}
% est un polynôme de la forme~\eqref{eq:25} avec un degré strictement plus petit que celui de $d$. 

% Il est clair que tous les diviseurs communs de $a$ et $b$ divisent $d$. 
% \end{proof}




% \begin{theorem}
%   \label{thr:39}
%   Soit $p(x)$ irréductible et supposons que $p(x) \mid f(x) \cdot  g(x)$, alors $p(x)\mid f(x)$ ou $p(x) \mid g(x)$. 
% \end{theorem}

% \begin{proof}
% Si $p(x)$ ne divise ni $f(x)$ ni $g(x)$ alors $1 = f(x) h_1(x) + p(x)h_2(x)$ et 
% $1 = g(x) h_3(x) + p(x) h_4(x)$ alors $\gcd(p(x), f(x)g(x))=1$. 
% \end{proof}


% \begin{theorem}
%   \label{thr:40}
%   Un polynôme  $f(x) \in K[x]$, $f(x) ≠ 0$  a une factorisation 
%   \begin{displaymath}
%     f(x) = a^* \prod_j p_j(x)
%   \end{displaymath}
%   où $a^* \in K$ et les $p_j(x)$ sont irréductibles avec coefficient dominant $1$. Cette factorisation est unique sauf pour des permutations des $p_j$. 
% \end{theorem}





\begin{definition}
  \label{def:34}
  Soient $V$ un espace vectoriel sur un corps $K$, $A: V \rightarrow V$ un endomorphisme et $f(x) = a_0+ \cdots + a_n x^n\in K[x]$. L'\emph{évaluation de $f$ sur $A$} est l'endomorphisme $f(A): V \rightarrow V$ 
  \begin{displaymath}
    f(A) = a_n A^n + a_{n-1}A^{n-1}+ \cdots + a_1 A + a_0 \mathrm{id},
  \end{displaymath}
  où $A^n = \underbrace{A \circ A \circ \dots \circ A}_{n \text{ fois}}$.
\end{definition}



\begin{definition}
  \label{def:35}
  Soient $A:V \rightarrow V$ un endomorphisme et $W \subseteq V$ un sous-espace de $V$. On dit que $W$ est \emph{invariant sous $A$} si $A(x) \in W$ pour tout $x \in W$. 
\end{definition}



\begin{lemma}
  \label{lem:20}
  Soient $f(x) ∈ K[x]$ et $A:V ⟶  V$ un endomorphisme,  alors $\ker(f(A))$ est invariant sous $A$. 
\end{lemma}


\begin{proof}
Si $v ∈ \ker(f(A))$ on trouve que $f(A)\,Av = Af(A)\,v = 0$. Alors, $Av ∈ \ker(f(A))$. 
\end{proof}


\begin{theorem}
  \label{thr:37}
  Soit $A: V \rightarrow V$ un endomorphisme et soit $f(x) = f_1(x) \cdot f_2(x)$ tel que
  \begin{enumerate}[i)]
  \item $\deg(f_1) \cdot \deg(f_2) \neq 0$,
  \item $\gcd(f_1,f_2) = 1$ 
  \end{enumerate}
  alors 
  $
      \ker(f(A)) = \ker(f_1(A)) \oplus \ker(f_2(A)) 
   $.   
\end{theorem}
\begin{proof}
  Dès que $\gcd(f_1,f_2)=1$ il existe $g_1(x),g_2(x)$ tels que 
  \begin{displaymath}
    1 = g_1(x) f_1(x) + g_2(x) f_2(x)
  \end{displaymath}
  et alors 
  \begin{equation}
    \label{eq:26}   
    g_1(A) \cdot f_1(A) +  g_2(A) f_2(A) = I. 
  \end{equation}
  Pour $v \in \ker(f(A))$, alors 
\begin{displaymath}
   g_1(A) \cdot f_1(A) \cdot v  + g_2(A) f_2(A) \cdot v  = v. 
\end{displaymath}
Mais $g_1(A) \cdot f_1(A) \cdot v \in \ker(f_2(A))$  dès que 
\begin{displaymath}
  f_2(A) \cdot g_1(A) \cdot f_1(A) \cdot v =   g_1(A) \cdot f_1(A) ⋅ f_2(A) \cdot v = g_1(A) f(A) v = 0
\end{displaymath}
et d'une manière similaire on voit que $g_2(A) f_2(A) \cdot v \in \ker(f_1(A))$. Il reste à démontrer que la somme est directe. 

Soit alors $v ∈  \ker(f_1(A))  ∩  \ker(f_2(A))$.  L'équation~\eqref{eq:26} montre 
\begin{displaymath}
  v =  g_1(A) \cdot f_1(A) \, v+  g_2(A) f_2(A) \, v = 0,
\end{displaymath}
qui démontre que la somme est directe. 
\end{proof}



% \begin{theorem}
% \label{thr:min-poly}
% Soient $V$ un espace vectoriel de dimension finie sur un corps $K$, et $A: \, V \rightarrow V$ un endomorphisme.
% Il y a un polyn{\^o}me $m_A(x) \in K[x]$ de degr{\'e} minimal tel que $m_A(A) = 0$ et le coefficient dominant de $m_A(x)$ est $1$. En plus,
% \begin{enumerate}
% \item $m_A(x)$ est unique,
% \item si $p(A) = 0$ pour un $p \in K[x]$, alors $m_A(x)$ divise $p$, et
% \item pour $\lambda \in K$, on a $m_A(\lambda) = 0$ si et seulement si $p_A(\lambda) = 0$, o{\`u} $p_A(x)$ est le polyn{\^o}me charact{\'e}ristique de $A$.
% \end{enumerate}
% \end{theorem}
% \begin{definition}
% On appelle $m_A(x)$ de Th{\'e}or{\`e}me~\ref{thr:min-poly} le \emph{polyn{\^o}me minimal de $A$}.
% \end{definition}
% \begin{proof}
% \underline{Existence:}
% Car $V$ est de dimension finie, l'espace vectoriel des endomorphismes $V \rightarrow V$ est de dimension finie.
% Alors, il existe un $k \in \N$ minimal tel que les endomorphismes
% \[
%  \mathrm{id}, A, A^2, \dots A^k
% \]
% sont lin{\'e}airement d{\'e}pendants,
% $
% 0 = \sum_{j=0}^k \alpha_j A^j$ pour quelques $\alpha_j \in K$, et $\alpha_k \neq 0$.
% On d{\'e}finit $m_A(x) = \sum_{j=0}^k (\alpha_j / \alpha_k) x^j$.
% Car on choisit $k$ minimal, $\{\mathrm{id},\dots,A^{k-1}\}$ est un ensemble libre et alors $m_A(x)$ est de degr{\'e} minimal.

% \underline{$i)$ et $ii)$:} Soit $p(x) \in K[x]$ polyn{\^o}me quelconque tel que $p(A) = 0$.
% Car il y a deux polyn{\^o}mes $g,h$ tel que 
% \[
% \gcd (m_A, p) (x) = g(x)m_A(x) + h(x) p(x) \neq 0,
% \]
% on a $\gcd (m_A, p) (A) = 0$.
% Car $m_A$ est de degr{\'e} minimal, on a $\deg (\gcd(m_A,p)) = \deg (m_A)$, et car le polyn{\^o}mes ont le coefficient dominant $1$, on a $m_A(x) = \gcd (m_A, p) (x)$, alors $ii)$ est vrai.
% De plus, si $\deg (p) = \deg (m_A)$, on a $p = \gcd (p, m_A)$ aussi, ce qui implique $i)$.

% \underline{$iii)$}
% Car $m_A \mid p_A$, si $m_A(\lambda) = 0$ alors $p_A(\lambda) = 0$.

% Si $p_A(\lambda) = 0$, il existe $v \in V$ tel que $Av = \lambda v$.
% Alors,
% \[
% 0 (v) = m_A(A) (v) = \left( \sum_{j=0}^k a_j A^j \right) (v) = \sum_{j=0}^k a_j A^j(v) = \left( \sum_{j=0}^k a_j \lambda^j \right) (v) = m_A (\lambda) v,
% \]
% et on a $iii)$.
% \end{proof}





\begin{lemma}
  \label{lem:19}
  Soit $V$ un espace vectoriel de dimension finie sur $\C$ et soit $T\colon V ⟶ V$ une application linéaire. Alors $V$ est la somme directe de sous-espaces   $V = V_1 ⊕ \cdots ⊕ V_K$  tels que 
  \begin{enumerate}[i)]
  \item $T(V_i) ⊆ V_i$ pour tout $i$ et \label{item:10}
  \item $T_{∣V_i} \colon V_i ⟶ V_i$
    est de la forme $N_i + λ \, I$ où $N_i$ est nilpotente. \label{item:11}
  \end{enumerate}
\end{lemma}

\begin{proof}
Soit $p(x) =$ le polyn{\^o}me caractéristique  de $T$, alors $p(T) = 0$.
Le coefficient dominant de $p(x)$ est $1$.   Le théorème fondamental de l'algèbre implique que 
\begin{displaymath}
  p(x) = ( x - λ_1)^{m_1} \cdots ( x - λ_k)^{m_k} 
\end{displaymath}
avec des $λ_i$ différents. 
Le diviseur le plus grand de $( x - λ_i)^{m_i}$ et $p(x) / ( x - λ_i)^{m_i}$ est $1$ pour $i ≠ j$. En utilisant théorème~\ref{thr:37} en $k-1$ étapes, alors 
\begin{displaymath} 
V =   \ker p(T) = \ker (T - λ_1I)^{m_1} ⊕  \cdots ⊕ \ker( T - λ_kI)^{m_k}
\end{displaymath}
et avec $V_i = \ker( T - λ_iI)^{m_i}$ on a $V = V_1 ⊕ \cdots ⊕ V_K$  et \ref{item:10}) avec lemme~\ref{lem:20}. \newline

De plus, $$T_{∣V_i} = (T - λ_iI)_{∣V_i} + λ_iI_{∣V_i} =\colon N_i + λ_iI$$ et $N_i = (T - λ_iI)_{∣V_i}$ est bien nilpotente, car $V_i = \ker( T - λ_iI)^{m_i}$ et donc $N_i^{m_i} = (T - λ_iI)^{m_i}_{∣V_i} = 0$.
\end{proof}


\begin{remark}
  \label{rem:3}
  Lemme~\ref{lem:19} démontre qu'il existe une base 
  \begin{displaymath}
  \mathscr{B} =   b_1^1,\dots,b_{\ell_1}^1,b_1^2,\dots,b_{\ell_2}^2,\dots,b_1^k,\dots,b_{\ell_k}^k
  \end{displaymath}
  où $b_1^i,\dots,b_{\ell_i}^i$ est une base de $V_i$ telle que la matrice $A^T_\mathscr{B}$ de $T$ par rapport à la base $\mathscr{B}$ est une matrice bloc diagonale 
  \begin{displaymath}
      A^T_{\mathscr{B} }  =
      \begin{pmatrix}
        B_1 \\
        & B_2 \\
        & & \ddots \\
        & &  & B_k
      \end{pmatrix}
  \end{displaymath}
  et les matrices $B_i \in \C^{\ell_i × \ell_i}$ sont de la forme $B_i = N_i + λ_i I$ où les $N_i$ sont  nilpotentes.


  
Des que  les $N_i$ et $ λ_i I$ commutent, ça démontre le Théorème~\ref{thr:32}. 


\begin{example}
  \label{exe:47}
  Soit
  \begin{displaymath}
    A = \left(\begin{array}{rrr}
25 & 34 & 18 \\
-14 & -19 & -10 \\
-4 & -6 & -1
\end{array}\right)
\end{displaymath}

Le polynôme caractéristique de $A$ est $p_A(x) = -  (x-1)^2 (x-3)$. Alors
\begin{displaymath}
  \ker(p_A(A))  = ℝ^3 = \ker\left((A - I)^2\right) ⊕  \ker(A - 3I). 
\end{displaymath}
On a
\begin{displaymath}
(A - I)^2 =   \left(\begin{array}{rrr}
28 & 28 & 56 \\
-16 & -16 & -32 \\
-4 & -4 & -8
\end{array}\right)
\end{displaymath}
et une base du $\ker\left((A - I)^2\right)$ est
\begin{displaymath}
  \left\{
    \begin{pmatrix}
      2\\0\\-1
    \end{pmatrix}, \,
    \begin{pmatrix}
      0\\2\\ -1
    \end{pmatrix}
\right\} 
\end{displaymath}
Et
\begin{displaymath}
A - 3I =   \left(\begin{array}{rrr}
22 & 34 & 18 \\
-14 & -22 & -10 \\
-4 & -6 & -4
                 \end{array}\right)               
\end{displaymath}
avec base de noyaux
\begin{displaymath}
  \left\{
    \begin{pmatrix}
      -7 \\ 4 \\ 1
    \end{pmatrix} \right\} 
\end{displaymath}
Avec matrice
\begin{displaymath}
  P = \left(\begin{array}{rrr}
2 & 0 & -7 \\
0 & 2 & 4 \\
-1 & -1 & 1
\end{array}\right)
\end{displaymath}
on a
\begin{displaymath}
  P^{-1}AP = \left(\begin{array}{rrr}
16 & 25 & 0 \\
-9 & -14 & 0 \\
0 & 0 & 3
                   \end{array}\right)                 
\end{displaymath}
et
\begin{displaymath}
\left(\begin{array}{rr}
15 & 25 \\
-9 & -15
\end{array}\right)^2 = 0 
\end{displaymath}

Alors
\begin{displaymath} 
P^{-1}AP =
\begin{pmatrix}
  N_1 +  1 ⋅I_2 & 0 \\
  0         & N_2 + 3  ⋅ I_1
\end{pmatrix}
\end{displaymath}
est en forme blocque diagonale, où $N_1$ et $N_2$ sont nilpotentes. En fait, $N_2=0$. 
\end{example}

\subsection*{Exercices}
\label{sec:exercices}
\begin{enumerate}
\item Montrer que $K[x]$ est un anneau avec $1_{K[x]} = 1_K$.  
\item Montrer que  $a(x) \in K[x]$ et $b(x) \in K[x]$ $\deg(a)+\deg(b)>0$  possèdent exactement un diviseur commun le plus grand avec coefficient principal égal à $1_K$.  
\item Soit $V$ un espace vectoriel de dimension fini sur $ℂ$,  $T : V ⟶V$ un endomorphisme et $f(x) = (x - λ)^m ∈ ℂ[x]$. Montrer que $\ker(f(T)) \neq \{0\}$ si et seulement si $λ$ est un valeur propre de $T$. 
\end{enumerate}






\section{La forme normale de Jordan}
\label{sec:la-forme-normale}



\begin{definition}
  Un \emph{bloc Jordan} est une matrice de la forme 
  \begin{displaymath}
    \begin{pmatrix}
      λ & 1 \\
        & λ & 1 \\
        &   & \ddots & \ddots \\ 
        &   &             & λ & 1 \\
        &   &         &  & λ  \\
    \end{pmatrix}
  \end{displaymath}
où les éléments non décrits sont zéro. 

Une matrice $A \in \C^{n \times n}$ est en \emph{forme normale de Jordan} si $A$ est en forme bloc diagonale, où tous les blocs sur la diagonale sont des blocs Jordan, i.e. $A$ est de la forme
\begin{displaymath}
  A =
  \begin{pmatrix}
    B_1 \\
        & B_2 \\
        &    & \ddots \\
        &    &       & B_k
  \end{pmatrix}
\end{displaymath}
où les matrices $B_j \in \C^{n_j\times n_j}$ sont des blocs de Jordan. 
\end{definition}


Notre but est de montrer le théorème suivant. 

\begin{theorem}
  \label{thr:41}
  Soit $A \in \C^{n \times n}$, alors il existe des matrices $P,J \in \C^{n \times n}$ telles que $J$ est en forme normale de Jordan, $P$ est inversible et 
  \begin{displaymath}
    A = P^{-1} \,J \,P. 
  \end{displaymath}
\end{theorem}

\begin{definition}
  \label{def:36}
  Soit $V  = \C^n$. Le \emph{décalage}  est l'application linéaire 
  \begin{displaymath}
    U
    \begin{pmatrix}
      x_1 \\ \vdots \\ x_n
    \end{pmatrix}
     = 
     \begin{pmatrix}
       x_2 \\ x_3 \\ \vdots \\ 0
     \end{pmatrix}. 
  \end{displaymath}
  Le décalage plus une constante est aussi une application linéaire 
  \begin{displaymath}
    U + \lambda \cdot I. 
  \end{displaymath}
\end{definition}

Il est facile de voir que la matrice représentant le décalage plus $λ$ est 
un seul bloc de Jordan 
\begin{displaymath}
 \begin{pmatrix}
      λ & 1 \\
        & λ & 1 \\
        &   & \ddots & \ddots \\ 
        &   &             & λ & 1 \\
        &   &         &  & λ  \\
    \end{pmatrix}.   
\end{displaymath}


%  Dès que l'espace des applications linéaires sur $V$ est un espace vectoriel sur $\C$ de dimension finie, il existe un $k \in \N$ tel que 
%  \begin{displaymath}
%    I, \, T, \, T^2, \dots, T^k
%  \end{displaymath}
%sont linéairement dépendants. De plus, il existe un polynôme $p(x) \in \C[x] \setminus \{ 0 \}$ tel que $p(T) = 0$. En effet, on peut choisir $p(x)$ le polynôme 
%caractéristique de $T$.

Rappel: Si  $\phi_{\mathscr{B}}$ est l'ismorphisme $\phi_{\mathscr{B}} \colon V \longrightarrow \C^n$, où $\phi_{\mathscr{B}}(x) = [x]_{\mathscr{B}}$ sont les coordonnées de $x$ par rapport à la base ${\mathscr{B}}$,  on a le diagramme suivant 
\begin{displaymath}
  {
  \begin{CD}
    V     @>T>>  V\\
    @VV \phi_{\mathscr{B}} V        @VV \phi_{\mathscr{B}} V\\ 
    \C^n     @>A^T_{\mathscr{B}} \cdot x>>  \C^n
  \end{CD}} 
\end{displaymath} 
\end{remark}

Il est clair, qu'il faut s'occuper maintenant des applications linéaires 
\begin{displaymath}
  T_{∣V_i} \colon V_i ⟶ V_i 
\end{displaymath}
qui sont de la forme $N + λ I$ pour une application nilpotente $N$. Le théorème suivant s'occupe des applications linéaires nilpotentes. La matrice de $λI$ est toujours $λ I$ pour chaque base. Il est alors clair que le théorème suivant démontre le théorème~\ref{thr:41}. 

\begin{theorem}
  \label{thr:38}
  Soit $V$ un espace vectoriel sur $\C$ de dimension finie et $N\colon V ⟶V$  une application linéaire nilpotente.  Alors $V$ possède une base $ℬ$ de la forme 
  \begin{displaymath}
    x_1,Nx_1, \dots, N^{m_1-1}x_1, x_2,Nx_2, \ldots , N^{m_2-1}x_2, \quad \dots \quad , x_k,Nx_k, \dots, N^{m_k-1}x_k
  \end{displaymath}
telle que $N^{m_i}x_i = 0$ pour tout $i$. 
\end{theorem}

\begin{remark}
  \label{rem:4}
  Si on inverse l'ordre de chaque orbite $x_i, Nx_i, \dots, N^{m_i-1} x_i$,
  %Si on inverse l'ordre de la base $ℬ$ et si on liste les éléments de droite à gauche
  on obtient une base $ℬ'$ et la matrice $A_{ℬ'}^{N }$ de l'application $N$  a la forme 
  \begin{displaymath}
    A_{ℬ'}^N =
    \begin{pmatrix}
      J_1 \\
      & J_2 \\
      & & \ddots \\
      & & & J_k
    \end{pmatrix}
  \end{displaymath}
en forme normale de Jordan, où 
\begin{displaymath}
  J_i =
  \begin{pmatrix}
    0 & 1 \\
    &  0 & 1 \\
    &    & \ddots & \ddots \\
    &    &        & 0 & 1 \\
    & & & & 0
  \end{pmatrix} \in \C^{m_i × m_i}. 
\end{displaymath}
Par conséquent, $N+ λI$ est représentée par 
\begin{displaymath}
 A_{ℬ'}^{N + λI} =    A_{ℬ'}^N + λ I_n
\end{displaymath}
en forme normale de Jordan. 
\end{remark}


\begin{proof}[Démonstration du Théorème~\ref{thr:38}] 
  Pour $x \in V \setminus \{0\}$ on appelle 
  \begin{displaymath}
    m_x = \min \{ i \colon N^ix = 0\}
  \end{displaymath}
  la \emph{durée de vie} de $x$. 
  La séquence 
  \begin{displaymath}
    x, Nx, \dots, N^{m_x-1} x
  \end{displaymath}
  est l'\emph{orbite} de $x$ sous $N$. 
  
  En concaténant les orbites des éléments d'une base et en travaillant sur cet ensemble, nous obtiendrons un ensemble de vecteurs qui engendrent $V$. 
  Supposons alors qu'au début de l'étape $q$, nous avons un ensemble $x_1,\dots,x_\ell$ avec $x_1,\dots,x_\ell \neq0$  dont les orbites 
  \begin{equation}
    \label{eq:27}
    x_1,Nx_1,\dots,N^{m_1-1}x_1, \,\dots \, ,  x_\ell,Nx_\ell,\dots,N^{m_\ell-1}x_\ell
  \end{equation}
  engendrent $V$ (pour la première étape, on prend $\ell = n$ avec des $x_i$ formant une base de $V$). Ici $m_i$ est la durée de vie de $x_i$. Si~\eqref{eq:27} est linéairement dépendant, nous allons soit supprimer un $x_i$ et son orbite (car superflus), soit remplacer un $x_i$ par un vecteur $y$ tel que 
  \begin{enumerate}[i)]
  \item Les orbites de $x_1,\dots, x_{i-1},y,x_{i+1},\dots,x_{\ell}$ engendrent aussi l'ensemble  $V$, 
  \item la somme des durées de vie de $x_1,\dots, x_{i-1},y,x_{i+1},\dots,x_{\ell}$  est strictement plus petite que la somme des durées de vie de $x_1,\dots,x_{\ell}$. 
  \end{enumerate}
Cela prouvera le théorème parce qu'un tel procédé doit se terminer. \\

Dès que l'ensemble~\eqref{eq:27} est linéairement dépendant, il existe une combinaison linéaire non triviale de~\eqref{eq:27} qui est égale a $0$ :
\begin{displaymath}
0 =   β_0^1 x_1 + β_1^1 Nx_1+ \dots+ β_{m_1-1}^1 N^{m_1-1}x_1 + \dots + 
β_0^\ell x_\ell + β_1^\ell Nx_\ell+ \dots+ β_{m_\ell-1}^\ell N^{m_\ell-1}x_\ell 
\end{displaymath}

\textbf{Cas 1 :} \\
Supposons que dans notre ensemble $x_1,Nx_1,\dots,N^{m_1-1}x_1, \,\dots \, ,  x_\ell,Nx_\ell,\dots,N^{m_\ell-1}x_\ell$, il existe $i$ tel que la durée de vie de $x_i$ est $1$ (i.e. $Nx_i = 0$ et l'orbite associée est seulement constituée de $x_i$) et supposons que ce $x_i$ apparaisse (avec un coefficient non nul) dans la combinaison linéaire ci-dessus. \\
En passant tous les termes sauf $x_i$ à gauche, on obtient  que $x_i$ est une combinaison linéaire non triviale des éléments de $\{ x_1,Nx_1,\dots,N^{m_1-1}x_1, \,\dots \, ,  x_\ell,Nx_\ell,\dots,N^{m_\ell-1}x_\ell \} \setminus \{x_i\}$. Donc on peut supprimer $x_i$ de cet ensemble et on obtient un nouvel ensemble de la même forme qu'en~\eqref{eq:27}, engendrant le même espace, mais avec une orbite en moins. \\

\textbf{Cas 2 :} \\
Supposons que nous avons la combinaison linéaire ci-dessus, mais que nous ne sommes pas dans le cas $1$. \\
Maintenant, nous allons appliquer l'application $N$ $k$-fois, où $k \geq 0$ est le plus grand entier tel que les termes 
\begin{displaymath}
  β_i^j N^{k+i}x_j 
\end{displaymath}
ne sont pas tous égaux à zéro. Ainsi, nous avons trouvé un sous-ensemble $J ⊆ \{1,\dots,\ell \}$ et des $γ_j ≠ 0$ tels que 
\begin{displaymath}
  \sum_{j \in J} γ_j N^{m_j-1}x_j = 0.
\end{displaymath}
Soit $m = \min_{j \in J} {m_j-1} \geq 1$ et soit $i \in J$ un index où le minimum est atteint. Alors 
\begin{displaymath}
 0 =  N^m  \sum_{j \in J} γ_j N^{m_j-1 - m}x_j  = N^m \left( γ_i x_i + \sum_{j \in J, j \neq i} γ_j N^{m_j-1 - m}x_j \right)
\end{displaymath}

Maintenant, en posant $$y = \sum_{j \in J} γ_j N^{m_j-1 - m}x_j = γ_i x_i + \sum_{j \in J, j \neq i} γ_j N^{m_j-1 - m}x_j$$ 
Si $y \neq 0$, on remplace $x_i$ par $y$.
Il est alors facile de voir que les orbites de 
\begin{displaymath}
  x_1,\dots, x_{i-1},y,x_{i+1},\dots,x_\ell
\end{displaymath}
engendrent encore $V$. Et la durée de vie de $y$ est au plus $m<m_i$. \\
Sinon, les orbites de
\begin{displaymath}
  x_1,\dots, x_{i-1},x_{i+1},\dots,x_\ell
\end{displaymath}
suffisent alors à engendrer $V$.

On a alors démontré le théorème.  
\end{proof}



\subsection*{Exercices} 

\begin{enumerate}
\item Montrer que les \textbf{orbites} de 
$ x_1,\dots, x_{i-1},y,x_{i+1},\dots,x_\ell $ engendrent encore $V$. (Voir démonstration du théorème~\ref{thr:38}). 

\item Le but de cet exercice est de faire la preuve du Théorème~\ref{thr:41} "à l'envers". \newline
Soit $T \colon V \rightarrow V$ un endomorphisme. Soit $\phi : V \rightarrow \Bbb C^n$ l'isomophisme associé à une base $B$ de $V$ et à la base canonique $E$ de $\Bbb C^n$. Supposons que $A_{B} = ([T(b_1)]_E, ..., [T(b_n)]_E)$, la matrice de $T$ relativement à la base $B$, admette une forme normale de Jordan $J$ avec matrice de passage $P = (p_1, ..., p_n)$. \newline
Montrer qu'il existe des sous-espaces $V_1, ..., V_k$  de $V$ tels que pour tout $i$ :
\begin{enumerate}
\item $V = V_1 \oplus \cdots \oplus V_k$;
\item $V_i = \phi(\text{span}(p_{k_i}, ..., p_{k_i + l_i}))$;
\item $T(V_i) \subset V_i$;
\item $T_{∣V_i} = N_i + \lambda_i I$, où $N_i \colon V_i \rightarrow V_i$ est nilpotente;
\item $\{\lambda_1, ..., \lambda_k\} = \{J_{11}, ..., J_{nn}\}$.
\end{enumerate} 

\item Le but de cet exercice est de montrer les propriétés des décompositions comme dans le Lemme~\ref{lem:19}. \newline
Soit $T \colon V \rightarrow V$ un endomorphisme et soit $V_1, ..., V_k$ une décomposition de $V$ tel que $V = V_1 ⊕ \cdots ⊕ V_k$, $T(V_i) \subset V_i$ et $T_{∣V_i} = N_i + \lambda_i I$, où $N_i : V_i \rightarrow V_i$ est nilpotente. Montrer que :
\begin{enumerate}
\item[a)] $V_i \subset \ker (T - \lambda_i I)^{a_i}$ pour un entier $a_i$ tel que $N_i^{a_i} = 0$.
\item[b)] Les $\lambda_1, ..., \lambda_k$ sont des valeurs propres (pas forcément distinctes) de $T$. (\textit{Indice} : Utiliser par exemple le premier point).
\item[c)] Le polynôme $f(x) = \prod_{i=1}^{k} (x - \lambda_i)^{a_i}$ annule $T$. (\textit{Indice} : Montrer que $f(T)v = 0$ pour tout $v \in V$ en utilisant la décomposition de $V$ et le premier point).
\item[d)] En déduire que l'ensemble $\{ \lambda_1, ..., \lambda_q \}$ contient toutes les valeurs propres de $T$ (\textit{Indice} : Si $v \neq 0$ est un vecteur propre de $T$ de valeur propres $\lambda$, exprimer $f(T)v$ en fonction de $f$, $\lambda$, et $v$). 
\item[e)] En déduire que les valeurs sur la diagonale de n'importe quelle forme normale de Jordan de $T$ constituent l'ensemble des valeurs propres de $T$. (\textit{Indice} : Utiliser l'exercice 2.)
\end{enumerate}

\item Comparer les polynômes caractéristiques de $J$ et $A$. En déduire que les éléments diagonaux de $J$ contiennent exactement l'ensemble des valeurs propres de $A$ et le nombre d'apparitions de chaque valeur propre sur la diagonale de $J$ est égale à la multiplicité algébrique de ladite valeur propre.

\item Soit $A \in \Bbb C^{n \times n}$ et soient $J$ une forme normale de Jordan de $A$, $P$ la matrice de passage associée ($A = PJP^{-1}$). \newline 
Le but de cet exercice est de montrer que le nombre de blocs de Jordan sur $J$ associé à une valeur propre $\lambda$ est exactement $\dim \ker(A - \lambda I)$.
\begin{enumerate}

\item[a)] Soit $S = \begin{pmatrix} S_1 & 0 \\ 0 & S_2 \end{pmatrix}$ une matrice blocs diagonale. Montrer que $$\text{rang}(S) = \text{rang}(S_1) + \text{rang}(S_2)$$ 
Généraliser pour $p$ blocs sur la diagonale. (\textit{Indice} : Considérer les lignes linéairement indépendantes de $S_1, S_2$).
\item[b)] Soit $B = U + \lambda I \in \Bbb C^{q \times q}$ un bloc de Jordan, où $U$ est l'application de décalage. Montrez que la seule valeur propre de $B$ est $\lambda$ et que l'espace propre associé est engendré par $e_1$. Déduisez $\dim \ker (B - \lambda I) = 1$ et $\dim \text{Im} (B - \lambda I) = q-1$.
\item[c)] Soient $B_1, ..., B_k$ l'ensemble des blocs de Jordan sur $J$ associé à une valeur propre $\lambda$. Déduire de a) et b) que $\dim \text{Im} (J - \lambda I) = n - k$.
\item[d)] En déduire que $\dim \ker (A - \lambda I) = k$ et que $\ker (A - \lambda I) = \text{span}(Pe_{i_1}, ..., Pe_{i_k})$, où les $i_j$ sont les indices des premières lignes/colonnes des $B_1, ..., B_k$ dans $J$.
\end{enumerate}

\item Déduire des exercices $4$ et $5$ que si $A$ est diagonalisable, la forme normale de Jordan $J$ de $A$ est diagonale.

\item Soit $A \in \Bbb C^{n \times n}$ une matrice diagonalisable. Montrer que $\ker(A - \lambda I) = \ker(A - \lambda I)^k$ pour toute valeur propre $\lambda$ de $A$ et pour tout $k > 0$. (\textit{Indice} : Diagonaliser d'abord $(A - \lambda I)$ et $(A - \lambda I)^k$ de manière simultanée).

\item Trouver deux matrices $A \in \Bbb C^{n \times n}$ et $B \in \Bbb C^{n \times n}$ qui ont le même polynôme caractéristique, mais qui ne sont pas similaires (\emph{Rappel} : $A$ et $B$ sont dites similaires s'il existe une matrice $P$ inversible telle que $B = P^{-1} A P$).

\item Soit $J \in \Bbb C^{n \times n}$ une matrice en forme normale de Jordan. Montrer que $J$ et $J^T$ sont similaires. En déduire que pour tout $A \in \Bbb C^{n \times n}$, les matrices $A$ et $A^T$ sont similaires.
\end{enumerate}
%%% Local Variables:
%%% mode: latex
%%% TeX-master: "notes"
%%% End:

\chapter{Algèbre linéaire sur les entiers}
\label{cha:algebre-lineaire-sur}
 

Quand est-ce qu'un système d'équations linéaires possède une solution en nombre entiers ? Étant donnés $A ∈ℤ^{m × n}$ et $b ∈ℤ^m$, on aimerait décider si 
\begin{equation}
  \label{eq:40}
  A x = b, \, x ∈ ℤ^n 
\end{equation}
est résoluble et trouver toutes les solutions.



Une condition nécessaire pour que le système \eqref{eq:40} soit soluble est qu'il existe une solution $x ∈ℚ^n$, et donc que $\rank(A) = \rank(A|b)$. Aussi, on peut supprimer une ligne de $(A|b)$ qui est dans le span des autres lignes. On peut alors supposer que $A$ est de rang ligne plein, c'est-à-dire que $\rank(A) = m$. 

\begin{definition}
  \label{def:44}
  Un nombre entier $d ∈ℤ$ \emph{divise} un nombre entier $a ∈ℤ$ s'il existe un nombre entier $x ∈ℤ$ tel que $d⋅x =b$. On note alors $d\mid b$, et si $d$ ne divise pas $b$ on écrit $d \nmid b$. 
\end{definition}
% Si $d \mid a$, alors s'il existe $x ∈ℤ$ tel que $d ⋅x =a$ et si $a \neq 0$ on peut conclure $d ≤ |a|$, parce que $|a| = |d| ⋅|x|$. 
\begin{definition}
  \label{def:45}
  Un nombre $d ∈ℤ$ est un diviseur commun de $a∈ℤ$ et $b ∈ℤ$, si $d \mid a$ et $d \mid b$. Si $\max\{|a|,|b|\} ≥1$, l'ensemble des diviseurs commun de $a$ et $b$ est un ensemble fini. Dans ce cas, on dénote le \emph{plus grand diviseur commun} de $a$ et $b$ par $\gcd(a,b)$. 
\end{definition}


\begin{theorem}
  \label{thr:48}
  Soient $a,b ∈ℤ$ et $\max\{|a|,|b|\} ≥1$. On a
  \begin{displaymath}
    \gcd(a,b) = \min \{ x ⋅a + y ⋅ b: x,y ∈ℤ,  x ⋅a + y ⋅ b≥1\}.
  \end{displaymath}
\end{theorem}

\begin{proof}
  Soit $d$ un diviseur commun de $a$ et $b$. Alors il existe $x^*, y^* ∈ℤ$ tel que $a = d⋅x^*$ et $b = d ⋅y^*$. Si $x ⋅a + y ⋅ b≥1$, où $x,y ∈ℤ$, alors
  \begin{displaymath}
    x ⋅a + y ⋅ b = (x ⋅x^* + y ⋅ y^*)  d ≥ |d|. 
  \end{displaymath}
  Par conséquent, on a  $d \le \min\{ x ⋅a + y ⋅ b: x,y ∈ℤ,  x ⋅a + y ⋅ b≥1\}$.
 Montrons que $\min\{ x ⋅a + y ⋅ b: x,y ∈ℤ,  x ⋅a + y ⋅ b≥1\}$ est un diviseur commun de $a$ et $b$. Supposons que $\min \nmid a$. Alors
 la  division avec reste implique l'existence de $q,r ∈ℤ$ tels que  
  \begin{displaymath}
    a = q ⋅ \min + r \quad \text{ et } \quad 1 \leq r < \min.
  \end{displaymath}
  Soient $x,y$ les entiers qui vérifient $\min = x ⋅a + y ⋅b$, alors
  \begin{displaymath}
    1 ≤ r = a - q ⋅ (x ⋅a + y ⋅b) = (1-q⋅x) a - qy ⋅b.
  \end{displaymath}
  Or $r$ est strictement plus petit que $\min$, ce qui est absurde. Il suit donc que $\min \mid a$ et, de façon analogue, $\min \mid b$.
  Aussi, on a montré que pour tout diviseur commun $d$ de $a$ et $b$, $\min\{ x ⋅a + y ⋅ b: x,y ∈ℤ,  x ⋅a + y ⋅ b≥1\}\geq d$ et donc que $\min\{ x ⋅a + y ⋅ b: x,y ∈ℤ,  x ⋅a + y ⋅ b≥1\}= \gcd(a,b)$
  
\end{proof}



\begin{corollary}
  \label{co:5}
  Soient $a,b ∈ℤ$ et $\max\{|a|,|b|\} ≥1$. Le $\gcd(a,b)$ est le diviseur commun  positif qui est divisé par chaque diviseur commun de $a$ et $b$. 
\end{corollary}

Pour calculer le plus grand diviseur commun de $a$ et $b$ on peut utiliser l'algorithme d'Euclide. Soient $a_0≥a_1 ∈ℤ$ pas tout les deux nuls. Si $a_1 = 0$, alors
\begin{displaymath}
\gcd(a_0,a_1) =   a_0. 
\end{displaymath}
Autrement, on applique la division avec reste
\begin{displaymath}
  a_0 = q_1 a_1 + a_2, 
\end{displaymath}
où $q_1,a_2 ∈ℤ$ et $0 ≤ a_2<a_1$. Un nombre entier $d ∈ℤ$ est un diviseur commun de $a_0$ et $a_1$ si et seulement si $d$ est un diviseur commun de $a_1$ et $a_2$. L'algorithme d'Euclide est le procédé de calculer la suite $a_0,a_1,a_2,\dots,a_{k-1},a_k ∈ℤ$  où $a_{k-1}>0$, $a_k=0$ et 
\begin{displaymath}
  a_{i-1} = q_i a_i + a_{i+1} 
\end{displaymath}
est le résultat de la division avec reste de $a_{i-1} $ par $a_i$. 

\begin{example}
  \label{exe:25}
  On calcule le plus grand diviseur commun de $a_0 = 52$ et $a_1=22$:
  \begin{displaymath}
    52 =   2 ⋅ 22 + 8, \quad 22 = 2 ⋅ 8  + 6, \quad 8 = 1 ⋅ 6 +2,\quad 6 = 3 ⋅2 + 0. 
  \end{displaymath}
  La suite est alors $a_{0}= 52, a_{1}= 22, a_{3} = 8, a_{4} = 6, a_{5} = 2, a_{6} = 0$. Ainsi $\gcd(52,22) = 2$. 
\end{example}

Le calcul des suites $a_i$ et $q_i$ donne aussi une représentation $\gcd(a_0,a_1) = x ⋅a_0 + y ⋅a_1$, $x,y ∈ℤ$. En effet
\begin{displaymath}
  \begin{pmatrix}
    a_{i} \\  a_{i+1} 
  \end{pmatrix}
  =
  \begin{pmatrix}
    0 & 1 \\
    1 & -q_i
  \end{pmatrix}
  \begin{pmatrix}
     a_{i-1} \\  a_{i}
   \end{pmatrix}
\end{displaymath}
et alors
\begin{displaymath}
  \begin{pmatrix}
    a_{k-1} \\ a_k
  \end{pmatrix} =
  \begin{pmatrix}
    0 & 1 \\
    1 & -q_{k-1}
  \end{pmatrix} \cdots
  \begin{pmatrix}
    0 & 1 \\
    1 & -q_{1}
  \end{pmatrix}
  \begin{pmatrix}
    a_0 \\ a_1
  \end{pmatrix}.
\end{displaymath}
\begin{example}
  \label{exe:26}
  On continue l'exemple~\ref{exe:25}.
  \begin{eqnarray*} 
    \begin{pmatrix}
      2 \\ 0
    \end{pmatrix} & = & 
    \begin{pmatrix}
      0 &1 \\
      1 & -3
    \end{pmatrix}
    \begin{pmatrix}
      0 &1 \\
      1 & -1
    \end{pmatrix}
    \begin{pmatrix}
      0 &1 \\
      1 & -2
    \end{pmatrix}
    \begin{pmatrix}
      0 &1 \\
      1 & -2
    \end{pmatrix}
    \begin{pmatrix}
     5 \\ 22
    \end{pmatrix} \\
   &  = & 
  \begin{pmatrix}
    3 & -7 \\
    -11 & 26 
  \end{pmatrix}
  \begin{pmatrix}
    52 \\22
  \end{pmatrix}
  \end{eqnarray*}
  Alors $2 = \gcd(52,22) = 3 ⋅52 -7 ⋅22$. 
\end{example}

Nous pouvons donc résoudre le problème~\eqref{eq:40} dans le cas où $m=1$ et $n=2$.

\begin{theorem}
  \label{thr:49}
  Soient $a,b ∈ℤ$ pas tous les deux nuls et $c ∈ℤ$. L'équation
  \begin{equation}
    \label{eq:41}
    x ⋅a + y ⋅b = c, \, x,y ∈ℤ
  \end{equation}
  possède une solution si et seulement si $\gcd(a,b) \mid c$. 
\end{theorem}
\begin{proof}
  Soient $x',y' ∈ℤ$ tel que $x' a + y'b = \gcd(a,b)$. Si $\gcd(a,b) \mid c$ alors il existe un $z ∈ℤ$ tel que $z ⋅\gcd(a,b) = c$ et $z x' a + z y'b = c$ est une solution en nombre entiers de~\eqref{eq:41}.

  S'il existe une solution de~\eqref{eq:41}, alors chaque diviseur commun de $a$ et $b$ est aussi un diviseur de $c$. 
\end{proof}

\section*{Exercices}


\begin{enumerate}
\item Démontrer le Corollaire~\ref{co:5}.
\item Soient $n≥2$ et $a_1,\dots,a_n∈ ℤ$ pas tous égaux à zéro. On définit $\gcd(a_1,\dots,a_n)$ comme étant le plus grand diviseur commun de $a_1,\dots,a_n$. Montrer: 
  \begin{enumerate}[i)]
    \item  $\gcd(a_1,\dots,a_n) =  \min\{x_1a_1+ \cdots + x_n a_n : x_1a_1+ \cdots + x_n a_n≥1, \, x_i ∈ℤ, \, i=1,\dots,n\}$. 
  \item $\gcd(a_1,\dots,a_n) = \gcd(\gcd(a_1,a_2), a_3, \dots, a_n)$ pour $n≥3$. 
  \end{enumerate}
\item Soit $a_0≥a_1≥\dots≥a_k=0$ la suite calculée par l'algorithme d'Euclide. Montrer $a_{i-1} ≥ 2⋅ a_{i+1}$ et conclure  que $k ≤2 ⋅ \log_2(a_0)+1$.  
\end{enumerate}



\section{La forme normale d'Hermite}
\label{sec:la-forme-normale-1}

Maintenant on s'occupe du problème~\eqref{eq:40} où $A ∈ ℤ^{m ×n}$ et $b ∈ℤ^m$ et $\rank(A) = m$.

\begin{lemma}
  \label{lem:23}
  Soit $A ∈ℤ^{n ×n}$ une matrice inversible (sur $ℚ$), alors $A^{-1} ∈ ℤ^{n ×n}$ si et seulement si $\det(A) = \pm 1$. 
\end{lemma}

\begin{proof}
  Supposons que $A^{-1} ∈ℤ^{n ×n}$. Alors $1 = \det(I_n)  = \det(A^{-1}) \det(A)$. Les deux facteurs $\det(A^{-1})$ et $ \det(A)$ sont des nombres entiers. Les seul diviseurs de $1$ en nombre entiers sont $1$ et $-1$.

  Réciproquement, si $\det(A) = \pm 1$, on a
  \begin{displaymath}
    A^{-1} = \mathrm{ad}(A) / \det(A) ∈ℤ^{n ×n}. 
  \end{displaymath}
  où $\mathrm{ad}(A) ∈ ℤ^{n ×n}$ est la matrice complémentaire de $A$. On se rappelle que $(\mathrm{ad}(A))_{ij} = (-1)^{i+j}\det(A_{ji})$ où $A_{ji}∈ℤ^{(n-1)×(n-1)}$ est la matrice qu'on obtient de $A$ en supprimant la $j$-ème ligne et $i$-ème colonne.
  \end{proof}

  \begin{definition}
    \label{def:46}
    Une matrice $U ∈ℤ^{n ×n}$ telle que $\det(U) = \pm 1$ est appelée  \emph{unimodulaire}.
  \end{definition}

  \begin{remark}
    Soit $U ∈ℤ^{n ×n}$  une matrice unimodulaire. 
    Un $x^* ∈ ℤ^n$ est une solution du problème~\eqref{eq:40} si et seulement $U^{-1} x^*∈ ℤ^n$ est une solution du problème
    \begin{equation}
      \label{eq:42}
      A U x = b, \, x ∈ℤ^n. 
    \end{equation}
  \end{remark}
  %
L'idée est maintenant de trouver une matrice unimodulaire $U ∈ℤ^{n ×n}$ telle que
\begin{equation}
  \label{eq:44}
  A ⋅ U = [H | 0]
\end{equation}
où
\begin{displaymath}
  H =
  \begin{pmatrix}
    h_{11} \\
    h_{21} & h_{22}\\
    &  & \ddots \\
    h_{m1} & \hdots & \hdots & h_{mm}
  \end{pmatrix}
\end{displaymath}
est une matrice triangulaire. Le problème~\eqref{eq:40} est alors soluble si et seulement si
$H^{-1} b ∈ℤ^n$.

\begin{definition}
  \label{def:47}
  Soit $A ∈ ℤ^{m ×n}$ une matrice. Une \emph{opération élémentaire unimodulaire} est l'une des trois opérations suivantes
  \begin{enumerate}[i)]
    \item Multiplier une colonne par $-1$.  \label{item:21}
    \item Échanger deux colonnes de $A$.  \label{item:22}
    \item Additionner un multiple entier d'une colonne de $A$ à une autre colonne de $A$. \label{item:23}
  \end{enumerate}
\end{definition}



\begin{example}
  \label{exe:27}
  Une suite d'opérations élémentaire unimodulaire sur $A$ correspond à la multiplication $A⋅U$ où $U ∈ℤ^{n ×n}$ est une matrice unimodulaire.
Soit 
  \begin{displaymath}
    A =
    \begin{pmatrix}
      3 & 6 & 2 \\
      11 & 5 & 7
    \end{pmatrix}. 
  \end{displaymath}
  Échanger les colonnes $1$ et $2$ correspond à la multiplication à droite de $A$ avec la matrice unimodulaire 
  \begin{displaymath}
    \begin{pmatrix}
    0& 1 & 0 \\
    1 & 0 & 0 \\
    0 & 0 & 1
      
    \end{pmatrix}.
  \end{displaymath}
  \begin{displaymath}
     \begin{pmatrix}
      6  & 3& 2 \\
       5 & 11 & 7
     \end{pmatrix} = A ⋅
     \begin{pmatrix}
       0& 1 & 0 \\
       1 & 0 & 0 \\
       0 & 0 & 1
     \end{pmatrix}
   \end{displaymath}
   Additionner $-3$ fois la colonne $3$ sur la colonne $1$ est la multiplication à droite de $A$ avec la matrice unimodulaire 
   \begin{displaymath}
     \begin{pmatrix}       
     0 & 1 & 0\\ 
     1 & 0 & 0 \\
     -3 & 0 & 1
   \end{pmatrix}
 \end{displaymath} 
   \begin{displaymath}
     \begin{pmatrix}
      0  & 3& 2 \\
      -16 & 11 & 7
    \end{pmatrix} = A ⋅  \begin{pmatrix}
      
     0 & 1 & 0\\
     1 & 0 & 0 \\ 
     -3 & 0 & 1
   \end{pmatrix}
 \end{displaymath}
\end{example}


\begin{example}
  \label{exe:28}
  Est-ce que le système
  \begin{equation}
    \label{eq:43}
     \begin{pmatrix}
      3 & 6 & 2 \\
      11 & 5 & 10
    \end{pmatrix} x =
    \begin{pmatrix}
      2 \\ 2
    \end{pmatrix}, x ∈ℤ^3
      \end{equation}
  possède une solution ?
  Soustrayons la colonne $3$ à la colonne $1$:
  \begin{displaymath}
     \begin{pmatrix}
      1 & 6 & 2 \\
      1 & 5 & 10
    \end{pmatrix}
  \end{displaymath}
  Ensuite, on soustrait six fois la colonne $1$ à la colonne $2$ et deux fois la colonne $1$ à la colonne $3$ :
 \begin{displaymath}
     \begin{pmatrix}
      1 & 0 & 0 \\
      1 & -1 & 8
    \end{pmatrix}
  \end{displaymath}
  Puis, on additionne huit fois la colonne $2$ à la colonne $3$:
  \begin{displaymath}
  \begin{pmatrix}
      1 & 0 & 0 \\
      1 & -1 & 0
    \end{pmatrix}. 
  \end{displaymath}      
  Le système \eqref{eq:43} se réduit finalement à résoudre \[   \begin{pmatrix}
      1 & 0 & 0 \\
      1 & -1 & 0
    \end{pmatrix} y = b, y \in \Z^3 \]
  
  où $y = U^{-1}x$ pour une matrice unimodulaire $U$ adéquate.
   On conclut donc qu'il possède une solution entière. En fait
    \begin{displaymath}
     \begin{pmatrix}
      3 & 6 & 2 \\
      11 & 5 & 10
    \end{pmatrix} x =
    b, x ∈ℤ^3
  \end{displaymath} est soluble pour tout $b ∈ℤ^2$.
\end{example}


\begin{lemma}
  \label{lem:24}
  Soit  $A ∈ℤ^{ m ×n}$ une matrice à coefficients entiers, alors il existe une matrice unimudulaire $U ∈ℤ^{ n ×n}$ telle que la première ligne de $AU$ est de la forme $(d,0,\cdots,0)$, où $d ∈ℤ$.
\end{lemma}

\begin{proof}

    Si la première ligne n'est pas de cette forme, et si elle possède seulement une
  composante qui n'est pas égale à zéro, on échange les colonnes de sorte que
  la matrice résultante soit de la forme souhaitée.
  
  Autrement, il existe deux indices de colonne $j_1\neq j_2$, tels
  que $a_{1j_1}\neq 0$ et $a_{1j_2} \neq 0$. On peut supposer, quitte à permutter les colonnes $j_1$ et $j_2$, que
  $|a_{1j_1}| ≥ |a_{1j_2}|$. La division avec reste nous donne des entiers $q ∈ℤ$ et
  $0 ≤r < |a_{1j_2}|$ tels que
  \begin{displaymath}
    a_{1j_1} = q ⋅ a_{1j_2} + r. 
  \end{displaymath}
  On applique l'opération unimodulaire: \emph{Soustraire $q$ fois la
    colonne $j_2$ à la colonne $j_1$}, ce qui a pour effet de remplacer $a_{1j_1}$ par $r$ et
   de laisser les autres composantes de la première ligne intactes. Comme
    \begin{displaymath}
      0 < |r|+ |a_{1j_2}|<  |a_{1j_{1}}|+ |a_{1j_2}|
    \end{displaymath}
    ce procédé ne peut être répété infiniment. Il existe alors une
    matrice unimodulaire qui transforme $A$ en une matrice dont la première
    ligne possède une seule composante non nulle. Un échange de
    colonnes adéquat donne la forme désirée.
  \end{proof}

  \begin{corollary}
    \label{co:9}
    Soit  $A ∈ℤ^{ m ×n}$ une matrice en nombre entiers. Alors il  existe une matrice unimudulaire $U ∈ℤ^{ n ×n}$ telle que $A ⋅U$ est de la forme~\eqref{eq:44}.
  \end{corollary}
  \begin{proof}
    On raisonne par récurrence sur $m$. Le cas $m=1$ suit directement du Lemme~\ref{lem:24}. Soit $m>1$. 
    Le Lemme~\ref{lem:24} implique qu'il existe une matrice unimodulaire $U_1 ∈ℤ^{n ×n}$ telle que
    \begin{displaymath}
      A ⋅ U_1 =
      \begin{pmatrix}
        d & 0 \cdots 0 \\
        a  & A'
      \end{pmatrix}
    \end{displaymath}
    où $d ∈ℤ$, $a ∈ ℤ^{m-1}$ et $A' ∈ ℤ^{(m-1) × (n-1)}$. Par l'hypothèse de récurrence, il existe une matrice unimudulaire $U_2 ∈ℤ^{(n-1) ×(n-1)}$ telle que $A' U_2$ est de la forme désirée. Clairement
      \begin{displaymath}
         \begin{pmatrix}
          1 & 0^T \\
          0 & U_2
        \end{pmatrix} ∈ℤ^{n ×n}
      \end{displaymath}
      est une matrice unimodulaire et 
      \begin{displaymath}
        A U_1
        \begin{pmatrix}
          1 & 0^T \\
          0 & U_2
        \end{pmatrix}
      \end{displaymath}
      est de la  forme~\eqref{eq:44}. 
  \end{proof}



  \begin{definition}
    \label{def:49}
    Une matrice $A ∈ℤ^{m ×n}$  est en \emph{forme normale d'Hermite}, si elle est de la forme \eqref{eq:44}, où $h_{ii}>0$ pour tout $1 ≤i≤ m$ et $0≤ h_{ij} < h_{ii}$ pour tout $1≤j<i≤m$. 
  \end{definition}
	
\begin{theorem}
  \label{thr:28}
    Soit  $A ∈ℤ^{ m ×n}$ une matrice en nombre entiers. Alors il  existe une matrice unimudulaire $U ∈ℤ^{ n ×n}$ telle que $A ⋅U$ est en \emph{forme normale d'Hermite}.
  \end{theorem}
  
  \begin{definition}
    \label{def:50}
    Soit $A \in \mathbb{Z}^{m\times n}$ et  $rang(A)=m$. L'ensemble  $\Lambda(A):=\left\{ Ax,x\in \mathbb{Z}^{n} \right\}$ est  un \emph{réseau entier généré}  de $A$. Une matrice  $B \in \mathbb{Z}^{m\times m}$ telle que  $\Lambda(A)=\Lambda(B)$ est appelée base de $\Lambda(A)$. 
  \end{definition}

  \begin{remark}
    Une  \emph{base} $B ∈ ℤ^{m ×m}$  de $Λ(A)$ est inversible. 
  \end{remark}
	
 \begin{corollary}
    \label{co:9}
    Chaque réseau entier possède une base.
  \end{corollary}


  % \begin{lemma}
  %   \label{lem:25}
  %   Soient $B_1,B_2 ∈ℤ^{m×m}$ deux matrices inversibles. Alors $Λ(A) = Λ(B)$ si et seulement s'il existe une matrice unimodulaire $U ∈ ℤ^{m ×m}$ telle que $B_1 ⋅U = B_2$. 
  % \end{lemma}

  % \begin{proof}
  %   On a $U ∈ℤ^{ n ×n}$ unimodulaire tq. $A ⋅ U = [H | 0]$ et $H$ est
  %   en forme normale de Hermite. Il est alors simple de remarquer que
  %   $\Lambda(A)=\Lambda(H)$. Dès lors $H$ est une base du réseau
  %   entier $Λ(A)$.
  % \end{proof}

  \begin{theorem}
    \label{thr:28}
    Soient $A,B \in \mathbb{Z}^{m\times m}$ en forme normale d'Hermite.    Alors $Λ(A) = Λ(B)$ si et seulement si $A = B$. 
  \end{theorem}

 \begin{proof}
 $\\$
 $\boxed { \Leftarrow  }$  Trivial.$\\$
 $\\$ 
$ \boxed { \Rightarrow  }$ On montre qui si $A\neq B$ alors $\Lambda(A) \neq \Lambda(B)$.$\\$
Supposons alors que $\Lambda(A) =\Lambda(B)$.$\\$
On note $A=\begin{pmatrix} a_{ 11 } & \quad  & \quad  \\ \quad  & \ddots  & \quad  \\ a_{m1}  & \quad  & a_{ mm } \end{pmatrix}$ et $B=\begin{pmatrix} b_{ 11 } & \quad  & \quad  \\ \quad  & \ddots  & \quad  \\ b_{m1}  & \quad  & b_{ mm } \end{pmatrix}$. $\\$
Soit $i$ l'indice minimal tel que la $i$-ème ligne de $A$ et celle de $B$ soient différentes. Alors $\exists j \in \left\{ 1,\dots ,i  \right\}$  tel que $a_{ij} \neq b_{ij}$ et sans perte de généralité on a $a_{ij}>b_{ij}$ . Clairement en notant $A_j$ ( resp. $B_j$) la $j$-ème colonne de $A$ (resp. $B$), on a $A_j - B_j = \begin{pmatrix} 0 \\ \vdots  \\ 0\\ a_{ ij }-b_{ ij } \\ \vdots  \end{pmatrix} \in \Lambda (A)$ avec $a_{ii}>a_{ij}-b_{ij}>0$. 

Il existe donc forcément un vecteur entier $z$ tel que $A_j - B_j = Az$. On montre aisément que les $i-1$ premières coordonées de $z$ doivent être nulles car aucun élément diagonal de $A$ ne l'est (ou, de manière équivalente, car $A$ et de rang plein). Or, en comparant la $i$-ème coordonnée, on trouve que $a_{ij} - b_{ij} = a_{ii} z_i$. Dès lors, $a_{ii}|a_{ij}-b_{ij}$ ce qui est une contradiction.
 \end{proof}
  \begin{remark}
    \label{rem:5}
    Le Théorème~\ref{thr:28} nous permet de vérifier, pour $A ∈ℤ^{m × n_1}$ et $B ∈ℤ^{m × n_2}$ de rang ligne pleins, si $Λ(A) = Λ(B)$.  On calcule $(H_A|0)$ et $(H_B|0)$ les formes normales d'Hermite de $A$ et $B$. Comme $Λ(A) =Λ(H_A)$ et $Λ(B) = Λ(H_B)$, on a que $Λ(A) = Λ(B)$ si et seulement si $H_A = H_B$. 
  \end{remark} 
  
  \begin{example}
    \label{exe:29}
    On va transformer $A = \left(\begin{matrix}4 & 6 & 10\\6 & 12 & 9\end{matrix}\right)$ en forme normale d'Hermite afin de trouver toutes les solutions entières de
    \begin{equation}
      \label{eq:46}
      \left(\begin{matrix}4 & 6 & 10\\6 & 12 & 9\end{matrix}\right) x =
      \begin{pmatrix}
        6 \\ 3
      \end{pmatrix}, \, x ∈ ℤ^3. 
    \end{equation}
    
    \begin{equation}
      \label{eq:45}
      \begin{array}{cc}
      \left(\begin{matrix}4 & 6 & 10\\6 & 12 & 9\end{matrix}\right) &  
\left(\begin{matrix}1 & 0 & 0\\0 & 1 & 0\\0 & 0 & 1\end{matrix}\right) \\

\left(\begin{matrix}4 & 2 & 2\\6 & 6 & -3\end{matrix}\right) &  
\left(\begin{matrix}1 & -1 & -2\\0 & 1 & 0\\0 & 0 & 1\end{matrix}\right) \\

\left(\begin{matrix}2 & 4 & 2\\6 & 6 & -3\end{matrix}\right) &
\left(\begin{matrix}-1 & 1 & -2\\1 & 0 & 0\\0 & 0 & 1\end{matrix}\right)\\

\left(\begin{matrix}2 & 0 & 0\\6 & -6 & -9\end{matrix}\right)&
\left(\begin{matrix}-1 & 3 & -1\\1 & -2 & -1\\0 & 0 & 1\end{matrix}\right)\\

\left(\begin{matrix}2 & 0 & 0\\6 & 3 & -9\end{matrix}\right) &
\left(\begin{matrix}-1 & 4 & -1\\1 & -1 & -1\\0 & -1 & 1\end{matrix}\right)\\

\left(\begin{matrix}2 & 0 & 0\\6 & 3 & 0\end{matrix}\right) &
\left(\begin{matrix}-1 & 4 & 11\\1 & -1 & -4\\0 & -1 & -2\end{matrix}\right)\\


\left(\begin{matrix}2 & 0 & 0\\0 & 3 & 0\end{matrix}\right) &
\left(\begin{matrix}-9 & 4 & 11\\3 & -1 & -4\\2 & -1 & -2\end{matrix}\right)\\
      \end{array}      
    \end{equation}

    L'ensemble des  solutions entières de~\eqref{eq:46} est 
    \begin{displaymath}
      \left\{ \left(\begin{matrix}-23\\8\\5\end{matrix}\right) + \left(\begin{matrix}11\\-4\\-2\end{matrix}\right) ⋅z : z ∈ℤ\right\}
    \end{displaymath}
    
  \end{example}
  
  \begin{definition}
   \label{def:51}
   Soit $A \in \mathbb{Z}^{m \times n}$. Le \emph{noyau} de $A$ sur $\Z$ est défini par $ker_{\mathbb{Z}}(A) := \left\{ y \in \mathbb{Z}^n, Ay=0 \right\}  $.
  \end{definition}
  
    \begin{theorem}
    \label{thr:29}
    Soient $A,B \in \mathbb{Z}^{m \times m}$ en forme normale de Hermite avec $rang(A)=rang(B)=m$. Alors $\Lambda(A)=\Lambda(B) \Leftrightarrow \exists U\in \mathbb{Z}^{m \times m}$ unimodulaire telle que $B=AU$.
    \end{theorem}
    
    \begin{proof}
    $\\$
    $\boxed { \Rightarrow  }$ Si $\Lambda(A)=\Lambda(B)$ alors $\\$ $A=BP, P\in \mathbb{Z}^{m \times m}$ $\\$ $B=AQ, Q\in \mathbb{Z}^{m \times m}$ $\\$ alors on obtient $A=AQP$ ce qui implique $QP=I_m$ et donc on a bien $P,Q$ unimodulaires.
    $\\$
    $\boxed { \Leftarrow  }$ Si $B=AU, U\in \mathbb{Z}^{m \times m}$ unimodulaire alors comme $U \mathbb{Z}^{m}= \mathbb{Z}^{m}$ on obtient immédiatement  $\Lambda(A)=\Lambda(B)$.
    \end{proof}
    \begin{definition}
   	\label{def:48}
	Soient $A\in \mathbb{Z}^{m \times n}$ avec $rang(A)=m$ et $\Lambda(A)$ le réseau entier de $A$. Le \emph{déterminant} du réseau $\Lambda(A)$ est donné par $det(\Lambda(A)):=|det(B)|$ où $B \in \mathbb{Z}^{m \times m}$ est une base de $\Lambda(A)$
   
   \end{definition}
   
   \begin{remark}
    \label{rem:5}
    Le théorème~\ref{thr:29} assure que $|det(B)|$ ne dépend pas du choix de $B$ et donc que $det(\Lambda(A))$ a du sens.
    \end{remark}
  
\section*{Exercices}
\begin{enumerate}
\item Soit $A ∈ℤ^{m ×n}$ avec rang ligne plein. Le noyaux entier de $A$ est l'ensemble
  \begin{displaymath}
    \ker_ℤ(A) = \{x ∈ℤ^n : Ax = 0\}. 
  \end{displaymath}
  Soit  $U ∈ℤ^{n ×n}$ une matrice unimodulaire telle que
  \begin{displaymath}
    A ⋅ U = (H | 0) 
  \end{displaymath}
  est la forme normale d'Hermite.
  Montrer que $\ker_ℤ(A) = \{ y_1 u_1 + \cdots + y_{n-m} u_{n-m} : y_i ∈ℤ\}$ où $u_1,\dots,u_{n-m}$ sont les dernières $n-m$ colonnes de $U$. 
\end{enumerate}

\section{La forme normale de Smith}
\label{sec:la-forme-normale-1}

À partir de maintenant, on se donne une matrice $A\in \mathbb{Z}^{m\times n}$ avec $rang(A)=k$ et $k$ n'est pas forcément $m$.

\begin{definition} \label{def:52}
	Soit  $A\in \mathbb{Z}^{m\times n}$ alors $\Lambda(A)=\left\{ Ax,x\in \mathbb{R}^{n} \right\} $ est le \emph{réseau entier général} de $A$.
\end{definition}



\begin{theorem}
    \label{thr:28}
    Soit  $A\in \mathbb{Z}^{m\times n}$ avec $rang(A)=k$ alors $\exists B \in \mathbb{Z}^{m \times k}$ tq. $\Lambda(A)=\Lambda(B)$. $B$ est alors appelée une base générale de $\Lambda(A)$.
    \end{theorem}
    
    \begin{remark}
      \label{rem:5}
      $rang(A)=rang(B)=k$.
    \end{remark}
    
    \begin{proof}
      Supposons que les   $k$ premières lignes de $A$ sont linéairement indépendantes. 
    Dès lors on a $A=\begin{pmatrix} A' \\ A'' \end{pmatrix}$ avec $A'\in \mathbb{Z}^{k \times n}$ et $rang(A')=k$ soit alors $U\in \mathbb{Z}^{n \times n}$ unimodulaire tq. $A'U=[H|0]$ ou $H \in \mathbb{Z}^{k \times k}$ est en forme normale de Hermite. Alors on peut se convaincre en raisonnant sur les rangs que $AU=\begin{pmatrix} H & 0 \\ B & 0 \end{pmatrix}$. Dés lors on a $\Lambda(A)=A\mathbb{Z}^n=AU\mathbb{Z}^n=\begin{pmatrix} H \\ B \end{pmatrix}\mathbb{Z}^k=\Lambda(\begin{pmatrix} H \\ B \end{pmatrix})$.
    
    \end{proof}
    
    \begin{theorem}
    \label{thr:28}
    Soit $G$ un sous groupe de $\mathbb{Z}^n$ alors $\exists B \in \mathbb{Z}^{n \times k}$ avec $rang(B)=k$ et tq. $\Lambda(B)=G$ 
    \end{theorem}
    
    \begin{proof}
    Soit $(v_1,\dots,v_k) \in G^k$ tq.  $(v_1,\dots,v_k)$ est une base de $span(G)$ (qui existe car on peut toujours extraire une base d'un espace de dimension fini d'une partie génératrice même infinie). Alors posons $B=(v_1\dots v_k)$$\\$
    \underline { cas\quad 1: } $\Lambda(B)=G$ et c'est terminé. $\\$
     \underline { cas\quad 2: }  $\Lambda(B)\subsetneq G$ alors $\exists v^* \in G-\Lambda(B)$. Soit alors $B^* \in \mathbb{Z}^{n \times k}$ une base générale du réseau général $G\supseteq \Lambda(v_1\dots v_k v^*)\supsetneq  \Lambda(B)$. Alors $\exists U\in \mathbb{Z}^{k \times k}$ tq. $B=B^*U$ et on a nécessairement $|det(U)| \in \mathbb{N} _{\ge 2}$. Dés lors on peut remarquer que $|det({B^TB})|=det(U)^2\times|det(B^{*^{T}}B^*)|$ et donc $|det({B^TB})|\le \frac { 1 }{ 4 } |det({B^TB})|$ mais comme $|det(B^{*^{T}}B^*)|\ge 1$ et on peut répéter ces opérations sur $B^*$ mais ce procédé ne peut pas continuer indéfiniment.
    \end{proof}
    
     \begin{theorem}
    \label{thr:28}
    Soit $A \in \mathbb{Z}^{m\times n} - \left\{ 0 \right\} $ alors $\exists U \in \mathbb{Z}^{m\times m}$ et $V \in \mathbb{Z}^{n\times n}$ unimodulaires tq. $UAV = \begin{pmatrix} \delta _{ 1 } & \quad & \quad  & \quad & \quad \\ \quad & \ddots  & \quad & \quad & \quad \\  \quad&  \quad& \delta _{ k } & \quad & \quad \\ \quad & \quad & \quad & \quad &\quad  \end{pmatrix}$  avec $\delta_i \in \mathbb{N}_{\ge1}$ et $\delta_1|...|\delta_k$ et les coefficients non spécifiés sont 0.
    \end{theorem}
\begin{proof}
On raisonne par récurrence sur m.$\\$
 \underline { m=1: } alors $A=(a_1 \dots a_n)$ mais alors on sait que $\exists V \in \mathbb{Z}^{n\times n}$ tq. $A\dashrightarrow (d \quad 0\dots 0)$. On prenant $U=I_m$ on termine l'initialisation.$\\$
  \underline { m $>$ 1: } Si nécessaire on échange les lignes de $A$ de sorte à ce que la première ligne soit non nulle. Alors $A=\begin{pmatrix}  a_1& \dots &a_n \\ \quad &  A'&\quad \end{pmatrix}$ et on sait que $\exists V \in \mathbb{Z}^{n\times n}$ tq. $A\dashrightarrow \begin{pmatrix} d & 0 & \dots  & 0 \\ \alpha _{ 1 } & \quad  & \quad  & \quad  \\ \vdots  & \quad  & A''  & \quad \\ \alpha _{ m } & \quad  & \quad  & \quad  \end{pmatrix}$ $\\$
  Si $d|\alpha_i \quad \forall i\in \left\{ 2,\dots, m \right\} $ alors on effectue $C_i \leftarrow  C_i - \frac {\alpha_i}{ d } C_1$ et on obtient $A\dashrightarrow \begin{pmatrix} d & 0 & \dots  & 0 \\ 0 & \quad  & \quad  & \quad  \\ \vdots  & \quad  & A''  & \quad \\ 0 & \quad  & \quad  & \quad  \end{pmatrix}$ $\\$
  Sinon $d\nmid  \alpha_j$ pour $j\in \left\{ 2,\dots, m \right\}$ et donc on a $pgcd(d,\alpha_2,\dots,\alpha_n)<d$. Ainsi $\exists U \in \mathbb{Z}^{m\times m}$ tq. $A\dashrightarrow \begin{pmatrix} \widetilde { d }  & \beta_2 & \dots  & \beta_n \\ 0 & \quad  & \quad  & \quad  \\ \vdots  & \quad  & A'''  & \quad \\ 0 & \quad  & \quad  & \quad  \end{pmatrix}$ avec $\widetilde { d } <d$. On construit alors une suite strictement décroissante de $\widetilde { d }$ mais ce dernier doit rester strictement positif et donc ce procédé se termine forcément. On obtient donc finalement $A\dashrightarrow \begin{pmatrix} f & 0 & \dots  & 0 \\ 0 & \quad  & \quad  & \quad  \\ \vdots  & \quad  & A^*  & \quad \\ 0 & \quad  & \quad  & \quad  \end{pmatrix}$. Mais par hypotèse de récurrence on sait que $\exists U^*\in \mathbb{Z}^{m-1 \times m-1}, V^*\in \mathbb{Z}^{n-1 \times n-1}$ unimodulaires tqs. $U^*A^*V^*=\begin{pmatrix} d_{ 2 } & \quad  & \quad  & \quad  \\ \quad  & \ddots & \quad  & \quad  \\ \quad  & \quad  & d_k & \quad  \\ \quad  & \quad  & \quad  & \quad  \end{pmatrix}$ avec $d_2|\dots |d_k$ dès lors on remarque que $\begin{pmatrix} 1 & \quad  \\ \quad  & U ^*\end{pmatrix}A\begin{pmatrix} 1 & \quad  \\ \quad  & V^{ * } \end{pmatrix}=\begin{pmatrix} f & \quad  & \quad  & \quad  \\ \quad  & \ddots & \quad  & \quad  \\ \quad  & \quad  & d_k & \quad  \\ \quad  & \quad  & \quad  & \quad  \end{pmatrix}$. Dès lors si $f|d_2$ c'est terminé sinon on peut effectuer l'opération $L_1\leftarrow L_1+L_2$ et transformer  $A\dashrightarrow \begin{pmatrix} f & d_2  & \quad  & \quad  \\ \quad  & \ddots & \quad  & \quad  \\ \quad  & \quad  & d_k & \quad  \\ \quad  & \quad  & \quad  & \quad  \end{pmatrix}$ puis on répète l'argument déjà utilisé pour transformer cette nouvelle matrice en $A\dashrightarrow \begin{pmatrix} \widetilde { f }& \quad  & \quad  & \quad  \\ \quad  & \ddots & \quad  & \quad  \\ \quad  & \quad  & d'_{k'} & \quad  \\ \quad  & \quad  & \quad  & \quad  \end{pmatrix}$ avec $\widetilde { f }<f$. Une fois de plus ce procédé crée une suite strictement décroissante de $\widetilde { f }$, ce dernier restant strictement positif le processus doit se terminer et on finit par obtenir  $A\dashrightarrow \begin{pmatrix} d_1& \quad  & \quad  & \quad  \\ \quad  & \ddots & \quad  & \quad  \\ \quad  & \quad  & d_{k} & \quad  \\ \quad  & \quad  & \quad  & \quad  \end{pmatrix}$ avec $d_1|\dots|d_k$.
\end{proof} 
    
    
%%% Local Variables:
%%% mode: latex
%%% TeX-master: "notes"
%%% End:

\chapter{Groupes}
\label{cha:groupes}

Dans ce chapitre on s'intéresse à l'étude des groupes.

\section{Groupes abéliens engendrés finis}

 \begin{definition}
    \label{def:49}
    Soit $(G,+)$ un groupe alors $H\subset G$ est un sous groupe si $(H,+|_{ H })$ est un groupe.
 \end{definition}
 
 \begin{lemma}
    \label{lem:23}
    $(H,\times|_H)$ est un sous groupe de $(G,\times) \Leftrightarrow \forall a,b \in H\quad a\times b^{-1} \in H$
   \end{lemma}
   
    \begin{definition}
    \label{def:49}
    $H\le G$ est un sous groupe normal de $G$ ssi $\forall g \in G\quad Hg=gH$ On note $H\unlhd G$.
     \end{definition}
     
      \begin{definition}
    \label{def:49}
    Soient $g,g' \in G$ alors on pose $Hg \circ Hg' := H(gg')$.
     \end{definition}
     
     \begin{theorem}
    \label{thr:28}
    $\circ$  munit $G/\!\raisebox{-.65ex}{\ensuremath{{H}}}  := \left\{ gH,\quad g\in G \right\}$ d'une structure de groupe ssi $H\unlhd G$.
    \end{theorem}
    
       \begin{definition}
    \label{def:49}
    Soient $(G,+),(G',\times)$ deux groupes et $\phi :G\rightarrow G'$ une application alors $\phi$ est un homomorphisme ssi 
    $\forall a,b \in G, \phi(a+b)=\phi(a) \times \phi(b)$.
    \end{definition}
    
    \begin{remark}
    $ker(\phi) \unlhd G$.
    \end{remark}
    
     \begin{theorem}
    \label{thr:28}
    Si $\phi$ est un morphisme alors $G/\!\raisebox{-.65ex}{\ensuremath{{ker(\phi)}}} \cong Im(\phi)$.
     \end{theorem}
     
         \begin{definition}
    \label{def:49}
    Soit $(G,+)$ un groupe abélien. On dit que $(G,+)$ est engendré fini si $\exists g_1,\dots,g_n \in G\ $ tqs. $G= \left\{ x_1g_1+\dots+x_ng_n, x_i \in \mathbb{Z} \right\} $.
        \end{definition}
        
        
          \begin{definition}
    \label{def:49}
    Soient $(G_1,+), (G_2,\times)$ deux groupes alors $G_1\otimes G_2 :=(G_1\times G_2, \bullet )$ avec $\forall (g_1,g_2), (g'_1,g'_2) \in G_1\times G_2, (g_1,g_2)\bullet (g'_1,g'_2)= (g_1+g_2, g'_1 \times g'_2)$.
    \end{definition}
    
        \begin{remark}
        \label{rem:2}
         $G_1\otimes G_2$ est un groupe.
        \end{remark}
        
        
         \begin{lemma}
    \label{lem:23}
    Soit $\phi:G\rightarrow G$ un automorphisme et $H\unlhd G$ alors $ G/\!\raisebox{-.65ex}{\ensuremath{{H}}} \cong G/\!\raisebox{-.65ex}{\ensuremath{{ \phi (H)}}}$.
    \end{lemma}
        
          \begin{theorem}
    \label{thr:28}
    Soit $G$ un groupe abélien engendré fini alors $\exists d_1,\dots,d_k \in \mathbb{N}_{ \ge 1}$ et $l\in \mathbb{N}$ tq $G  \cong \mathbb{Z}_{d_1} \otimes \dots \otimes \mathbb{Z}_{d_k} \otimes \mathbb{Z}^l$ avec $\mathbb{Z}^l:=\mathbb{Z} \otimes \dots 
   \otimes \mathbb{Z}$ $l$ fois. De plus on a $d_1|\dots\ |d_k$.
    \end{theorem}
    
    \begin{proof}
    Soient $g_1,\dots, g_n \in G$ tqs. $G= \left\{ x_1g_1+\dots+x_ng_n, x_i \in \mathbb{Z} \right\} $ alors on peut vérifier que $\phi :\mathbb{ Z }^{ n }\rightarrow G, (x_ 1,\dots ,x_n)\mapsto x_1g_1+\dots +x_ng_n$ est un homomorphisme surjectif. Mais alors $ker(\phi) \le \mathbb{Z}^n$ et donc $\exists B\in \mathbb{Z}^{n\times k}$ avec $rang(B)=k$ et $\Lambda(B)=ker(\phi)$. En calculant la forme normale de Smith de B on obtient que $\exists U\in \mathbb{Z}^{n\times n}, V\in \mathbb{Z}^{k\times k}$ unimodulaires tqs.
$B=U\begin{pmatrix} d_{ 1 } & \quad  & \quad  \\ \quad  & \ddots & \quad  \\ \quad  & \quad  &d_k  \\  \quad & 0 &\quad   \end{pmatrix}V$ avec $1\le d_1|\dots|d_k$ et donc on a $ker(\phi)=\left\{ U\begin{pmatrix} d_{ 1 } & \quad  & \quad  \\ \quad  & \ddots & \quad  \\ \quad  & \quad  &d_k  \\  \quad & 0 &\quad   \end{pmatrix}y, y\in \mathbb{Z}^k \right\}$. Alors par les théorèmes explicités plus haut on obtient $G \cong {\mathbb{Z}^n}/\!\raisebox{-.65ex}{\ensuremath{{\left\{ U\begin{pmatrix} d_{ 1 } & \quad  & \quad  \\ \quad  & \ddots & \quad  \\ \quad  & \quad  &d_k  \\  \quad & 0 &\quad   \end{pmatrix}y, y\in \mathbb{Z}^k \right\}}}}$. 
Puis comme $U: \mathbb{Z}^n \rightarrow  \mathbb{Z}^n$ est un automorphisme on arrive finalement à$\\$ $G \cong {\mathbb{Z}^n}/\!\raisebox{-.65ex}{\ensuremath{{\left\{ (d_1y_1,\dots,d_ky_k,0,\dots,0), y_i \in \mathbb{Z} \right\}}}}$. Il suffit maintenant de constater que$\\$ ${\mathbb{Z}^n}/\!\raisebox{-.65ex}{\ensuremath{{\left\{ (d_1y_1,\dots,d_ky_k,0,\dots,0), y_i \in \mathbb{Z} \right\}}}} \cong \mathbb{Z}_{d_1} \otimes \dots \otimes \mathbb{Z}_{d_k} \otimes \mathbb{Z}^{n-k}$ .
    \end{proof} 
    
         \begin{lemma}
    \label{lem:23}
    Soient $L,K$ deux groupes abéliens, $M$ un sous groupe normal de $K$ $f:K\rightarrow L$ un isomorphisme. Alors $K/\!\raisebox{-.65ex}{\ensuremath{{M}}} \cong L/\!\raisebox{-.65ex}{\ensuremath{{\phi(M)}}}$.
    \end{lemma}
    
    
            \begin{lemma}
    \label{lem:23}
    Soient $G, H_1, H_2$ trois groupes abéliens avec $|H_i|<\infty$ pour $i=1,2$ alors si $G \cong H_1 \otimes  \mathbb{Z}^{n_1}$ et $G \cong H_2 \otimes  \mathbb{Z}^{n_2}$ alors $H_1 \cong H_2$ et $n_1=n_2$.
      \end{lemma}
      
      \begin {proof}
      Soit $\phi: H_1 \otimes  \mathbb{Z}^{n_1} \rightarrow H_2 \otimes  \mathbb{Z}^{n_2}$ un isomorphisme.
      Soit $h \in H_1$ alors $\phi(h,0)=(h',x) \in H_2 \times \mathbb{Z}^{n_2}$ mais comme $\phi$ préserve l'ordre on a nécessairement $x=0$ on a ainsi $\phi(H_1,0) \subset (H_2,0)$ et de même $\phi^{-1}(H_2,0) \subset (H_1,0)$ et donc finalement $\phi(H_1,0)=(H_2,0)$ ce qui permet de conclure que $H_1 \cong H_2$. $\\$
      Maintenant on a d'après le lemme ci dessus:
      $\mathbb{Z}^{n_2}  \cong \\
      {H_2 \otimes  \mathbb{Z}^{n_2}}/\!\raisebox{-.65ex}{\ensuremath{{H_2 \otimes \left\{ 0_{\mathbb{Z}^{n_2}} \right\}  }}}
      \cong {H_1 \otimes  \mathbb{Z}^{n_1}}/\!\raisebox{-.65ex}{\ensuremath{{H_1 \otimes \left\{ 0_{\mathbb{Z}^{n_1}} \right\}  }}}
      \cong \mathbb{Z}^{n_1}$. Mais alors on a nécessairement $n_1 = n_2$. En effet et sans perte de généralité supposons $n_1>n_2$ alors soient $x_1,\dots,x_{n_1} \in \mathbb{Z}^{n_1}$ des vecteurs $\mathbb{Q}$ linéairement indépendants alors on a forcement $\phi(x_1),\dots, \phi(x_{n1}) \in \mathbb{Z}^{n_2} \mathbb{Q}$ linéairement dépendants. Ainsi $\exists \alpha_1,\dots, \alpha_{n_1} \in \mathbb{Z}$ non tous nuls avec $\alpha_1\phi(x_1)+\dots+\alpha_{n_1}\phi(x_{n1})=0$ mais comme $\phi$ est un isomorphisme on a $\phi(\alpha_1x_1+\dots+\alpha_{n_1}x_{n_1})=0$. Cela implique par injectivité que $\alpha_1x_1+\dots+\alpha_{n_1}x_{n_1}=0$. Ceci est absurde.
      \end {proof}
      
    
                  \begin{theorem}
    \label{thr:28}
    Soient $m_1,m_2 \in \mathbb{N}_{\ge }$ alors $\mathbb{Z}_m \cong \mathbb{Z}_{m_1} \otimes \mathbb{Z}_{m_2}$ ssi $m=m_1 \times m_2$ et $pgcd(m_1,m_2)=1$. 
    \end{theorem}

    
    \begin {proof}
    $\boxed { \Leftarrow  } $ Tout d'abord il s'agit de remarquer que $\phi : \mathbb{Z}_m \rightarrow \mathbb{Z}_{m_1} \otimes \mathbb{Z}_{m_2}, a \mapsto (a,a)$ est bien définie et est un morphisme de groupe. Ensuite on a $|\mathbb{Z}_m| = |\mathbb{Z}_{m_1} \times \mathbb{Z}_{m_2}|$ donc il suffit de montrer que $\phi$ est surjective. En effet soit $(a,b) \in \mathbb{Z}_{m_1} \otimes \mathbb{Z}_{m_2}$ alors on a $1=pgcd(m_1,m_2)=rm_1 + sm_2, r,s \in \mathbb{Z}$. Dès lors on vérifie que $\phi(rm_1b+sm_2a)=(a,b)$. $\\$
    $\boxed { \Rightarrow  } $ En égalisant les cardinaux on a $m=m_1m_2$ de plus en posant $d=pgcd(m_1,m_2)$ on a $\frac { m }{ d } = ordre (d) = ordre(\phi(d)) \le \frac { m }{ d^2 }$ et donc $1\le d\le1$ et donc $d=1$.
     \end{proof}
     
    \begin{remark}
    \label{rem:2}
    Soit $n=p_1^{a_1}\dots p_k^{a_k}, p_i \in \mathbb{P} a_i \in \mathbb{N}_{\ge 1}$ alors $\mathbb  \mathbb{Z}_n \cong \mathbb{Z}_{p_1^{a_1}} \otimes \dots \otimes \mathbb{Z}_{p_k^{a_k}}$.
    \end{remark}
    
               \begin{lemma}
    \label{lem:23}
    Soient $p\in \mathbb{P}$ et  $\alpha_1 \le \dots \le  \alpha_k, \beta_1 \le \dots \le \beta_l \in \mathbb{N}_{\ge 1}$ alors 
    $\mathbb{Z}_{p^{\alpha_1}} \otimes \dots \otimes \mathbb{Z}_{p^{\alpha_k}} \cong \mathbb{Z}_{p^{\beta_1}} \otimes \dots \otimes \mathbb{Z}_{p^{\beta_l}}$ ssi $k=l$ et $\alpha_i=\beta_i \forall i \in \left\{ 1,\dots, k \right\} $.
    
    \end{lemma}

\begin{proof}
$\boxed { \Leftarrow  } $ Trivial. $\\$
$\boxed { \Rightarrow  }$ Soit $\phi$ l'isomorphisme entre ces deux groupes. On a  $p^{\alpha_k} = ordre((0,\dots,1)) = ordre(\phi((0,\dots,1)))\le p^{\beta_l}$. De même on obtient l'inégalité inverse puis finalement on trouve donc $\alpha_k=\beta_l$. Ainsi on a $\mathbb{Z}_{p^{\alpha_1}} \otimes \dots \otimes \mathbb{Z}_{p^{\alpha_{k-1}}} \cong \mathbb{Z}_{p^{\beta_1}} \otimes \dots \otimes \mathbb{Z}_{p^{\beta_{l-1}}}$ puis on prouve le théorème par induction.
\end{proof}

  \begin{remark}
    \label{rem:2}
    Soit $G$ abélien engendré fini. Alors $G \cong \mathbb{Z}_{d_1} \otimes \dots \otimes \mathbb{Z}_{d_k} \otimes \mathbb{Z}^l$ avec $d_1, \dots, d_k \in \mathbb{N}_{ \ge 1}$ tqs $d_1|\dots|d_k$ et $l\in \mathbb{N}$. Alors $d_k=p_1^{a_1}\dots p_n^{a_n}, p_i \in \mathbb{P} ,a_i \in \mathbb{N}_{\ge 1}$. On obtient ainsi $d_i=p_1^{e^i_1}\dots p_{n}^{e^i_n}$ avec $0\le e^i_j \le a_j$. Dès lors on a que $G \cong \mathbb{Z}_{p_1^{e_1^1}} \otimes \dots \otimes \mathbb{Z}_{p_n^{e_n^1}} \otimes \dots \otimes \mathbb{Z}_{p_1^{a_1}} \otimes \dots \otimes \mathbb{Z}_{p_n^{a_n}} \otimes  \mathbb{Z}^l$.
        \end{remark}

        
        
        
        
        

\bibliographystyle{alpha}
\bibliography{books}
\end{document}


%%% Local Variables:
%%% mode: latex
%%% TeX-master: t
%%% End:


