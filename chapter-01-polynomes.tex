\chapter{Polynômes}
\label{cha:polynomes}

\section{Notions fondamentales}
\label{sec:noti-fond}


Soit $R$ un anneau. On se souvient que cela signifie que  $R$ est un ensemble,  muni des opérations binaires  $+ : R × R → R$  et $⋅: R × R → R$ telles que: 
\begin{enumerate}[(R1)]
\item $a+ b  = b+a$ pour tout     $a,b ∈ R$. \label{R1}
\item $a + (b+c) = (a + b) +c$, pour tout $a,b,c ∈ R$. \label{R2}
\item Il existe un élément $0 ∈R$ tel que $0+a =a$ pour chaque $a ∈R$. \label{R3}
\item Pour chaque $a ∈R$ il existe un élément $-a ∈R$ tel que $a + (-a) = 0$. \label{R4}
\item $a(bc) = (ab) c$ , pour tout $a,b,c ∈ R$. \label{R5}
\item Il existe un élément $1 ∈R$ tel que $a ⋅ 1 = 1 ⋅a = a$ pour chaque $a ∈R$. \label{R6}
\item $a (b+c) = ab + ac$ et $(b+c) a =ba +ca$  pour tout $a,b,c ∈R$.\label{R7} 
\end{enumerate}
Si, de plus, on a 
\begin{enumerate}[(R1)]
  \setcounter{enumi}{7}
\item  $a b = ba$ pour tous $a,b ∈R$ \label{R8}. 
\end{enumerate}
alors l'anneau $R$ est appelé \emph{anneau commutatif}. Le \emph{centre} de $R$ est l'ensemble $Z(R) = \{ r ∈ R : ra = ar \text{ pour tout } a ∈ R\}$. 
Un élément  $a ≠0$ de $R$ est un \emph{diviseur de zéro} s'il existe un $b≠0$ tel que $ab = 0$ ou $ba = 0$. Un anneau commutatif sans diviseurs de zéro est appelé \emph{anneau intègre}.  



\begin{exercise}
  Soit $R$ un anneau, alors l'élément $1$ est unique. 
\end{exercise}



\begin{example}
  \label{exe:32}
  \begin{enumerate}[i)]
  \item Les nombres entiers $ℤ$ avec l'addition et la multiplication standard forment un anneau commutatif sans diviseurs de zéro. $ℤ$ est donc un anneau intègre. 
  \item $5 ⋅ℤ = \{ 5 z :z ∈ ℤ\}$ avec l'addition et la multiplication standard n'est pas un anneau. L'axiome (R\ref{R6}) n'est pas satisfait.
  \item L'ensemble des matrices  $ℤ^{2 ×2}$ avec l'addition et multiplication des matrices est un anneau non commutatif avec des diviseurs de zéro. 
    \begin{eqnarray*}
      \begin{pmatrix}
        1 & 0 \\
        1 & 1 
      \end{pmatrix}
       \begin{pmatrix}
        1 & 1 \\
         0& 1 
       \end{pmatrix}   & = &
                            \begin{pmatrix}
                               1 & 1 \\
                               1 & 2 \\
                             \end{pmatrix}  \\      
      \begin{pmatrix}
        1 & 1 \\
        0 & 1 
      \end{pmatrix}
       \begin{pmatrix}
        1 & 0 \\
         1 & 1 
       \end{pmatrix}   & = &
                            \begin{pmatrix}
                               2 & 1 \\
                               1 & 1 \\
                             \end{pmatrix}       
    \end{eqnarray*}
     \begin{eqnarray*}
      \begin{pmatrix}
        0 & 0 \\
        0 & 1 
      \end{pmatrix}
       \begin{pmatrix}
        1 & 0 \\
         0& 0 
       \end{pmatrix}   & = &
                            \begin{pmatrix}
                               0 & 0 \\
                               0 & 0 \\
                             \end{pmatrix}  
     \end{eqnarray*}
   \end{enumerate}
\end{example}



La démonstration du théorème suivant est un exercice.  
\begin{theorem}
  \label{thr:58}
  Soit $R$ un anneau et $S ⊆R$. Si les deux conditions suivantes sont vérifiées 
  \begin{enumerate}[a)]
  \item $1 ∈ S$ \label{item:27}
  \item Pour $s,t ∈S$ alors  $s⋅t∈S$   et $s-t ∈S$. \label{item:28}
  \end{enumerate} 
  alors $S$ est aussi un anneau. 
\end{theorem}

Un élément $r ∈R$ est \emph{inversible} s'il existe un élément $r^{-1} ∈ R$ tel que
\begin{displaymath}
  r ⋅ r^{-1} = r^{-1} ⋅r = 1. 
\end{displaymath}
On dénote l'ensemble des éléments inversibles par $R^*$. On se rappelle que $(R^*, ⋅)$ est un groupe, le \emph{groupe des éléments inversibles}. Un anneau intègre tel que $R \setminus \{0\} = R^*$ est un \emph{corps}. 

\begin{exercise}
  \label{exe:39}
  Soit $R$ un anneau et $r ∈ R^*$. Alors $r$ n'est pas un diviseur de zéro. 
\end{exercise}

\begin{exercise}
  \label{exe:42}
  Soit $R$ un anneau et $R^{n ×n}$ l'anneau des matrices $n ×n$ sur $R$. Montrer que le centre de $R^{n ×n}$ est $Z(R^{n ×n}) = \{ a I_n :a ∈Z(R)\}$. 
\end{exercise}

\begin{theorem}
  \label{thr:50}
  Soit $R$ un anneau, alors il existe un anneau $S ⊇R$ ($R$ est un sous-anneau de $S$) et un élément $x ∈ S \setminus R$ tel que
  \begin{enumerate}[(i)]
  \item $a x = x a $ pour chaque $a ∈ R$.
  \item Si 
    $  a_0+ a_1x + \cdots + a_n x^n =0$  et $a_i ∈R$ pour tout $i$,  
    alors  $a_i = 0$ pour tout $i$.
  \end{enumerate}
\end{theorem}
Ce théorème est démontré dans le cours \emph{anneaux et corps}.   Il nous donne une manière formelle d'introduire le concept d'une \emph{indéterminée} (ou  \emph{variable}) qui n'est autre que ledit élément $x ∈ S \setminus R$.  

\begin{definition}
  \label{def:51}
  Un polynôme sur $R$ est une expression de la forme
$p(x) = a_0 + a_1 x + \cdots + a_n x^n$, où $n ∈ℕ$,  $a_i ∈R$ pour tout $i$. L'élément $a_i$ est le $i$-ème \emph{coefficient} de $p(x)$. L'ensemble des polynômes sur $R$ est dénoté par $R[x]$. 
\end{definition}


\begin{example}
  \begin{enumerate}[i)]
  \item  $p(x) = 3 + x^2 + 5x^4 ∈ ℤ[x]$.
  \item $p(x) = \smat{1 &0 \\ 1 & 1} +   \smat{3 &3 \\ 2 & 1} x^3 ∈ ℤ^{2×2}[x]$. 
  \end{enumerate}
\end{example}



\begin{proposition}
  \label{prop:8}
  Deux polynômes 
\begin{equation}
  \label{eq:34}
  p(x) = a_0 + a_1x + a_2x^2 + \cdots \,\,\text{ et }  \,\, q(x) = b_0 + b_1x + b_2x^2 + \cdots
\end{equation}
sont \emph{égaux} si et seulement si $a_i =b_i$ pour tout $i ∈ℕ$. Dans ce
cas, on écrit $p(x) = q(x)$.
\end{proposition}

\begin{exercise}
  \label{exe:34}
  Démontrez  la proposition~\ref{prop:8}. 
\end{exercise}


\begin{theorem}
  \label{thr:51}
  $R[x]$ est un anneau. Si $R$ est commutatif, alors $R[x]$ est commutatif. Si $R$ est anneau sans diviseur de zéro    alors $R[x]$ est un  anneau sans diviseur de zéro. 
\end{theorem}
\begin{proof}
  Pour montrer que $R[x]$ est un anneau on  invoque  Theorème~\ref{thr:58}. Clairement $1 ∈ R[x]$ ce qui implique  \ref{item:27}). Pour \ref{item:28})
  il faut montrer que
  pour deux polynômes $f(x) = a_0 + a_1 x + \cdots + a_n x^n$ et $g(x) = b_0 + b_1x + \cdots + b_m x^m$ sur $R$, on a : 
  \begin{enumerate}[i)]
  \item $f- g ∈ R[x]$ et
  \item $f ⋅ g ∈R[x]$. 
  \end{enumerate}
  La somme s'écrit comme
  \begin{equation}
    \label{eq:47}
    f(x) + g(x) = ∑_{i=0}^{\max\{m,n\}} (a_i +  b_i) \, x^i  ∈R[x], 
  \end{equation}
  où $a_i = 0$ pour $i>n$ et $b_i = 0$ pour $i>m$. 
  % Alors   $R[x]$ est stable pour $+$. 
  % L'inverse de $g(x)$ est $-g(x) = -b_0 - b_1 x  \cdots - b_m x^m ∈ R[x]$.
  Alors
  on a $f-g ∈ R[x]$. 
  
  Le    produit de $f$ et $g$ s'écrit comme

  \begin{equation}
  \label{ceq:21}
  p(x) ⋅q(x) = a_0 b_0 + (a_0b_1 +a_1b_0) x + (a_0b_2+ a_1b_1 + a_2b_0)x^2 + \cdots .
\end{equation}
Bref, on a la formule 
  \begin{equation}
    \label{eq:8}
    f(x) ⋅g(x) = ∑_{i=0}^{m+n} ( ∑_{j+k = i} a_j b_k) \, x^i ∈ R[x]. 
  \end{equation}
  C'est à dire que $R[x]$ est stable pour l'opération $⋅$ de $S$.  Alors, $R[x]$ est un sous-anneau de $S$ et 
  donc $R[x]$ est un anneau. La formule~\eqref{eq:8} implique que
  $R[x]$ est commutatif, si $R$ est commutatif.

  Finalement, si $f(x) = a_0 + a_1 x + \cdots ≠ 0$ et
  $g(x) = b_0 + b_1 x + \cdots ≠ 0 $ sont deux polynômes non-nuls et si $R$ est un 
  anneau sans diviseur de zéro, il faut montrer que $f(x) ⋅ g(x) ≠
  0$. Soit $n = \max\{ i :a_i ≠0 \}$ et $m = \max\{ i :b_i ≠0 \}$. Le coefficient de $x^{m+n}$ du polynôme $f⋅g$ est $a_n ⋅b_m$. Ce coefficient n'est pas nul dès que $R$  est un 
  anneau sans diviseur de zéro. 
\end{proof}

\begin{example}
  \label{exe:33}
  \begin{eqnarray*}
    f(x) & = &  3 \, x^{3} + x + 2 \\
    g(x) & = & 2 \, x^{4} + 2 \, x^{2} + 1 \\
    f(x) ⋅g(x) & = & 6 \, x^{7} + 8 \, x^{5} + 4 \, x^{4} + 5 \, x^{3} + 4 \,  x^{2} + x + 2 \\ 
  \end{eqnarray*}
\end{example}



% \begin{theorem}
%   \label{thr:43}
%   L'ensemble des polynômes $K[x]$ sur un corps $K$ est un anneau intègre.
% \end{theorem}

\begin{definition}
  \label{def:52}
Le \emph{degré} de $p(x) = a_0 + a_1x + a_2x^2 + \cdots \neq 0$ est 
\begin{displaymath}
  \deg(p) = \max\{i \colon  a_i \neq 0\}
\end{displaymath}
et $\deg(0) = -\infty$. 
Si $p \neq 0$, le coefficient $a_{\deg(p)}$ est le \emph{coefficient dominant} de $p$. 
Un polynôme de degré zéro est une \emph{constante}.
\end{definition}

\begin{example}
  \label{exe:35}
  Soit $f(x) = 2 + 3x + 5x^3 ∈ ℤ[x]$, alors $\deg(f) =3$  et le coefficient dominant de $f$ est $5$. 
\end{example}

\begin{theorem}
  \label{thr:34} Soit $R$ un anneau,   $f,g \in R[x] \setminus \{0\}$ tel que le coefficient dominant de $f$ ou de $g$ n'est pas un diviseur de zéro.  Alors, $\deg(f \cdot g) = \deg(f) + \deg(g)$. 
\end{theorem}
\begin{proof}
  
  Soient $f(x) = a_0 + \cdots + a_n x^n$ et $g(x) = b_0+ \cdots b_m x^m$ tels que $a_n, b_m  \neq 0$. Le coefficient de $x^{n+m}$  est $a_n \cdot  b_m \neq 0$. Les coefficients de $x^k$, $k> n+m$ sont tous nuls.  
\end{proof}

Un polynôme $p(x) = a_0 + a_1 x + \cdots + a_n x^n ∈ R[x]$ induit une application $f_p:  R ⟶ R$, $f_p(r) = a_0+ a_1 r+ \cdots + a_n r^n$. Nous écrivons aussi $p(r)$ pour $f_p(r)$ et on parle de l'\emph{évaluation} de $p$ sur $r$. 

\begin{example}
  \label{exe:43}
  Soit $p(x) = x ∈ R[x]$ et $q(x) = (x + a) ∈ R[x]$, alors $p(x) q(x) = x^2 + ax$. Quand est-ce que l'on a $p(r) q(r) = (p ⋅q)(r)$? C'est le cas si et seulement si
  \begin{displaymath}
     r^2 + r a = r^2 + a r  
   \end{displaymath}
   c'est-à-dire si et seulement si $ra = ar$. 
   On attire l'attention sur le fait que cet exemple traite de l'évaluation en $r \in R$ de polynômes. 
   Ici, $p(r)q(r)$ dénote l'évaluation de $p$ et $q$ en $r$ \emph{puis} d'en multiplier les résultats.
   Tandis que $(p⋅q)(r)$ représente la multiplication de $p$ et $q$ \emph{puis} on évalue ce produit en $r$.
\end{example}


\begin{theorem}
  \label{thr:53}
  Soit $R$ un anneau et $\alpha \in Z(R)$ un élément du centre de $R$. L'application 
  \begin{displaymath}   
    \begin{array}{rcl}
    \Phi \colon R[x] &\rightarrow & R \\
           f(x) &\mapsto &f(\alpha)          
    \end{array}
  \end{displaymath}
est un morphisme d'anneaux surjectif. 
\end{theorem}


\begin{exercise}
  \label{exe:44}
  Démontrer le théorème~\ref{thr:53}. 
\end{exercise}


\section{Interpolation}
\label{sec:interpolation}


Deux polynômes différents $p$ et $q$ peuvent induire la même application $f_p$ et $f_q$, même s'ils sont des polynômes sur un corps $K$.

\begin{example}
  \label{exe:36}
  Soit $K = ℤ_2$, $p(x) = x + x^2 $ et $q(x) = x^2 + x^3$ deux
  polynômes sur $K$. Il est clair que $p ≠ q$. Mais $f_p:\, K →K$ et
  $f_p:\, K →K$ sont les mêmes applications car pour tout
  $x ∈ℤ_2$ on a $p(x) = q(x) = 0$.
\end{example}

Par contre, si $K$ est un corps infini, deux polynômes différents induisent deux applications différentes. Nous allons voir les détails de ce fait maintenant.

\begin{theorem}
  \label{thr:52}
  Soit $K$ un corps, $r_0,\dots,r_n ∈K$ des éléments distincts (c.-à-d. $r_i ≠ r_j$ pour $i≠j$) et $y_0,\dots,y_n ∈K$. Il existe  exactement un seul polynôme $f(x) ∈ K[x]$ de degré au plus $n$ tel que
  \begin{displaymath}
    f(r_i) = y_i \text{ pour tout } i ∈ \{0,\dots,n\}. 
  \end{displaymath}
\end{theorem}

\begin{proof}
  Un polynôme  $f(x) = a_0 + a_1x + \cdots + a_n x^n$ satisfait $f(r_i) = y_i$ pour tout $i$ si et seulement si le coefficients $a_0,\dots,a_n$ satisfont
  \begin{equation}
    \label{eq:48}
    \begin{pmatrix}
      1 & r_0& \cdots & r_0^n \\
      1 & r_1& \cdots & r_1^n \\
       & &  \cdots  &  \\
       1 & r_n& \cdots & r_n^n \\
     \end{pmatrix}
     \begin{pmatrix}
       a_0\\ a_1 \\ \vdots \\ a_n
     \end{pmatrix} =
     \begin{pmatrix}
       y_0\\ y_1 \\ \vdots \\ y_n
     \end{pmatrix}
   \end{equation}

   La matrice $V_{r_0,\dots,r_n} ∈K^{(n+1)×(n+1)}$  à gauche de~\eqref{eq:48} est connue sous le nom de \emph{matrice de Vandermonde}
   des éléments $r_0,\dots,r_n$. Le théorème sera prouvé une fois que nous aurons démontré que $\det(V_{r_0,\dots,r_n}) ≠0$.

   On démontre  $\det(V_{r_0,\dots,r_n}) ≠0$ par récurrence sur $n$. Pour $n = 0$, le déterminant est $1$. Pour $n>0$, on soustrait $r_0$ fois la colonne $n$ de la colonne $n+1$. Après, on soustrait $r_0$ fois la colonne $n-1$ de la colonne $n$ etc. Le résultat est la matrice
   \begin{equation}
     \label{eq:49}
     \begin{pmatrix}
       1 & 0  &      \hdotsfor{2} &  0 \\
       1 & r_1- r_0&  r_1 (r_1-r_0)  & \cdots & r_1^{n-1} (r_1 - r_0) \\
       & & &  & \cdots  &  \\
       1 & r_n -r_0&  r_1 (r_n-r_0) & \cdots & r_n^{n-1} (r_n - r_0) \\
     \end{pmatrix}
   \end{equation}
   Le déterminant de la matrice~\eqref{eq:49} est celle de
   $V_{r_0,\dots,r_n}$. Développement du déterminant le long de la
   première ligne donne le déterminant de la matrice
 \begin{equation}
\label{eq:50}
    \begin{pmatrix}
       r_1- r_0&  r_1 (r_1-r_0) \cdots & r_1^{n-1} (r_1 - r_0) \\
        & &  \cdots  &  \\
        r_n -r_0&  r_1 (r_n-r_0) \cdots & r_n^{n-1} (r_n - r_0) \\
     \end{pmatrix}
   \end{equation}
   Alors
   \begin{displaymath}
     \det(V_{r_0,\dots,r_n})  = (r_n-r_0) \cdots (r_1-r_0)  \det(V_{r_1,\dots,r_n})
   \end{displaymath}      
Comme les $r_i$ sont distincts, le produit   $(r_n-r_0) \cdots (r_1-r_0) $ n'est pas zéro. Par l'hypothèse de récurrence, $\det(V_{r_1,\dots,r_n}) ≠0$ et donc $\det(V_{r_0,r_1,\dots,r_n}) ≠0$.  
\end{proof}

Le théorème~\ref{thr:52} ne contredit pas l'exemple~\ref{exe:36} car le corps $ℤ_2$ n'admet que $2$ éléments. Il existe bel et bien un unique polynôme $p \in  ℤ_2[x]$ de degré inférieur ou égal à 1 tel que $p(0) = p(1) = 0$. Une énumération des polynômes possibles ou la résolution du système linéaire donne $p=0$.

\begin{exercise}
  \label{exe:37}
  Montrer que le déterminant de $V_{r_0,\dots,r_n}$ est
  \begin{displaymath}
    \det (V_{r_0,\dots,r_n}) = ∏_{0 ≤ i<j ≤n} (r_j - r_i). 
  \end{displaymath}
\end{exercise}



\begin{example}
  \label{exe:38}
  On cherche le polynôme $f(x) ∈ ℤ_5[x]$ de degré au plus $3$ tel que
  $f(0) = 2$, $f(1)=2$, $f(2) = 2$ et $f(3) = 3$.

  En trouvant l'unique solution du système
  \begin{displaymath}
    \left(\begin{array}{rrrr}
1 & 0 & 0 & 0 \\
1 & 1 & 1 & 1 \\
1 & 2 & 4 & 3 \\
1 & 3 & 4 & 2
\end{array}\right)
\begin{pmatrix}
  a_0 \\ a_1 \\ a_2 \\ a_3
\end{pmatrix}
=
\begin{pmatrix}
  2 \\ 2 \\ 2 \\3
\end{pmatrix},
\end{displaymath}on obtient
\begin{displaymath}
  \begin{pmatrix}
  a_0 \\ a_1 \\ a_2 \\ a_3
\end{pmatrix} =
\begin{pmatrix}
  2 \\ 2 \\ 2 \\ 1
\end{pmatrix}.
\end{displaymath}
Ainsi, $f(x) = 2 + 2x + 2x^2 + x^3$ est le polynôme recherché. 

\end{example}


\begin{corollary}
  \label{thr:42}
  Soit $K$ un corps infini. Deux polynômes $p(x),q(x) ∈ K[x]$ sont égaux, si et seulement si les applications $f_p$ et $f_q$ sont les mêmes.
\end{corollary}
\begin{proof}
  Si $p = q$, alors $p(r) = q(r)$ pour tout $r ∈K$ et donc $f_p = f_q$. D'autre part, si $f_p = f_q$, on peut supposer qu'un des deux polynômes n'est pas le polynôme $0$ (s'ils sont tous les deux nuls, alors $p=q$).  Comme $p(r) = q(r)$ pour tout $r ∈K$ et le corps $K$ est infini, les polynômes prennent les mêmes valeurs sur $\max\{\deg(p),\deg(q)\}+1$ éléments de $K$. Il en résulte que $p = q$ par le Theorème~\ref{thr:52}. 
\end{proof}


\section{Divisibilité et racines} 
\label{sec:divisibilite}
Soit $R$ un anneau. La \emph{division avec reste} est l'opération suivante. 

\begin{theorem}
  \label{thr:33}
  Soient $f,g \in R[x]$, $\deg(g) >0$ et le coefficient dominant de $g$ un élément  inversible de $R$.   Il existe $q,r \in R[x]$ unique  tels que 
  \begin{displaymath}
    f(x) = q(x) g(x) + r(x) 
  \end{displaymath}
  et $\deg(r) < \deg(g)$. 
\end{theorem}


\begin{proof}
  La preuve se fait par récurrence sur $\deg(f)$. Si $\deg(f) < \deg(g)$, alors on pose $q = 0$ et $r = f$.

Soit alors $\deg(f) = n \geq \deg(g)=m$ et 
\begin{displaymath}
  f(x) = a_0+ \cdots +a_n x^n \text{ et } g(x) = b_0 + \cdots + b_m x^m 
\end{displaymath}
où $a_n$ et $b_m$ sont les coefficients dominants de $f$ et $g$ respectivement. 
Clairement 
\begin{displaymath}
  \deg\left( f(x) - \frac{a_n}{ b_m } x^{n-m} g(x) \right) < \deg(f(x))
\end{displaymath}
et par hypothèse de récurrence 
\begin{displaymath}
  f(x) - \frac{a_n}{ b_m } x^{n-m} g(x)  = q(x) g(x) + r(x) 
\end{displaymath}
tel que $\deg(r(x)) < \deg(g(x))$. On  obtient alors
\begin{displaymath}
  f(x) = \left(q(x) + \frac{a_n}{ b_m } x^{n-m} \right) g(x) + r(x). 
\end{displaymath}
%
Supposons maintenant qu'il existe deux autres polynômes $q'(x)$ et $r'(x)$ tels que 
\begin{displaymath}
    f(x) = q'(x) g(x) + r'(x) 
  \end{displaymath}
  et $\deg(r') < \deg(g)$. 
Alors 
\begin{displaymath}
   r(x) - r'(x) = (q(x) - q'(x)) ⋅ g(x). 
\end{displaymath}
Par le théorème~\ref{thr:34}
\begin{displaymath}
 \deg( r - r')  = \deg(q - q') + \deg(g) \geq \deg(g), 
\end{displaymath}
ce qui contredit le fait que $\deg(r) < \deg(g)$ et $\deg(r') < \deg(g)$. 
\end{proof}


% \begin{definition}
%   Pour $f(x)  = a_0 + \cdots + a_n x^n \in K[x]$ et $\alpha \in K$, l'évaluation $f(\alpha)$ est $ a_0 + a_1 \alpha + \cdots + a_n \alpha^n \in K$. 
% \end{definition}


\begin{example}
  \label{exe:24}
  La division avec reste du polynôme $x^5+2x^2+1$ par $2x^3+x+1$ de $ℤ_3[x]$ donne

  \begin{displaymath}
    x^5+2x^2+1 = (2x^2 +2) (2x^3 + x +1) + (x+2). 
  \end{displaymath}
  
\end{example}

%\begin{proposition}
%  \label{prop:5}
%  Pour $\alpha \in K$, $\phi_\alpha: \, K[x] \longrightarrow K$, %$\phi_\alpha(f(x)) = f(\alpha)$ est un homomorphisme.  
%\end{proposition}





\begin{definition}
  \label{def:32}
  Un polynôme  $q(x)$ \emph{divise} un  polynôme $f(x)$ s'il existe un polynôme $g(x)$ tel que $f(x) = g(x) \cdot q(x)$. On dit que $q(x)$ est un \emph{diviseur} de $f(x)$ et on écrit $q(x) \mid f(x)$. 
\end{definition}


\begin{example}
  \label{exe:41}
  Soient $q(x) = x^2 +1 ∈ℤ_2[x]$ et $ f(x) = x^3 + x^2 - x+1 ∈ ℤ_2[x]$. On a
  $f(x) = q(x) (x+1)$ et donc 
   $q(x) \mid f(x)$. 
\end{example}


\begin{definition}
  \label{def:31}
  Soit $p(x) \in \K[x] \setminus\{0\}$. Un $\alpha \in K$ tel que $f_p(\alpha) = 0$ est une  \emph{racine} de $f(x)$.  
\end{definition}


\begin{example}
  \label{exe:40}
  Soit $p(x) = x^4 + x^3 + x^2 + x + 1 ∈ ℤ_5(x)$, alors $α = 1$ est une racine de $p(x)$. 
\end{example}f


\begin{theorem}[Théorème fondamental de l'algèbre]
  \label{thr:44}
  Tout polynôme $p(x) ∈ℂ[x] ⧹\{0\}$ non constant admet au moins une racine complexe.
\end{theorem}




\begin{theorem}
  \label{thr:35}
  Soit $f(x)∈ K[x] \setminus \{0\}$ un polynôme  et $\alpha \in K$, alors $\alpha$ est une racine de $f$ si et seulement si $(x- \alpha)  \mid f(x)$. 
\end{theorem}

\begin{proof}
  Si $f(x) = q(x) \cdot (x - \alpha)$, alors $f(\alpha) = 0$. 

%Autrement, si $f$ est une constante, $f = 0$ et $(x - \alpha)$ divise $f$.
Dans l'autre sens, si $f$ est une constante, $f(\alpha) = 0$ implique que $f = 0$ et $(x - \alpha)$ divise $f$. 

Si $f$ n'est pas une constante, il existe $q(x)$ et $r(x)$ tels que
\begin{displaymath}
  f(x) = q(x) \cdot (x - \alpha) + r(x)
\end{displaymath}
%$\deg(r) = 0$. Alors $f(\alpha) = 0$ implique $r=0$. 
avec $\deg(r) \leq 0$. Alors $f(\alpha) = 0$ implique $r=0$. 
\end{proof}


\begin{definition}
  \label{def:41}
  La \emph{multiplicité} d'une racine $α$ de $p(x) ∈ K[x] ⧹\{0\}$ est le plus grand entier $i≥1$ tel que $ (x-α)^i \mid p(x)$. Si $p(x)$ est le polynôme caractéristique d'un endomorphisme d'un espace vectoriel, on appelle la multiplicité de $α$ la \emph{multiplicité algébrique}. 
\end{definition} 


\begin{example}[Suite de l'exemple 1.10]  
  \label{exe:45}
  Le polynôme  $p(x) = x^4 + x^3 + x^2 + x + 1  ∈ℤ_5$  est divisible par $x-1$, et
  \begin{displaymath}
    p(x)  = (x^3 + 2*x^2 + 3*x + 4)  (x-1). 
  \end{displaymath}
  De plus, $1$ est aussi une racine de $x^3 + 2*x^2 + 3*x + 4$. En fait 
  \begin{displaymath}
    x^4 + x^3 + x^2 + x + 1 = (x-1)^4,
  \end{displaymath}
  alors la multiplicité de la racine $1$ de $p(x)$ est $4$ parce que $(x-1)^4$ divise $p$ mais pas $(x-1)^5$, en tant que polynôme de degré 5 (théorème~\ref{thr:34}).   
\end{example}


\section{Factorisation}
\label{sec:fact-de-polyn}

Dans cette section, on fixe un corps K.

\begin{definition}
  \label{def:53}
  Un polynôme $p(x) ∈ K[x] \setminus \{0\}$ est \emph{irréductible}, si
  \begin{enumerate}[i)]
  \item $\deg(p) ≥1$ et
  \item  si $p(x) = f(x) g(x)$ alors $\deg(f) = 0$ ou $\deg(g) = 0$. 
  \end{enumerate}
\end{definition}

\begin{example}
  \label{exe:46}
  \begin{enumerate}
  \item   Chaque polynôme linéaire $p(x) = ax + b ∈ K[x]$, $a ∈ K \setminus\{0\}$ est irréductible. En effet, si $p(x) = f(x) g(x)$ et $\deg(f)>0$ et $\deg(g)>0$, alors le théorème~\ref{thr:34} implique que $\deg(p) >1$.
  \item $x^2 +1 ∈ℝ[x]$ est irréductible. Autrement, il existe un polynôme linéaire $f(x) =  x - α ∈ ℝ[x]$ qui divise $f$ ce qui implique que $α$ est une racine de $x^2 +1$. Cependant, aucun nombre $α$ réel ne satisfait $α^2 = -1$.
  \end{enumerate}
\end{example}

\begin{theorem}
  \label{thr:54}
  Soient $f(x)$ et $g(x)$ deux polynômes sur $K$ non tous deux nuls et soit 
  \begin{displaymath}
    I = \{ u ⋅ f + v ⋅ g : u,v ∈ K[x]\}.
  \end{displaymath}
  Il existe un  polynôme $d(x)∈K[x]$ tel que
  \begin{equation}
    \label{eq:51}
    I = \{ h ⋅ d : h ∈ K[x]\}. 
  \end{equation}
\end{theorem}

\begin{proof}
  Remarquons que $I$ contient des polynômes non nuls (notamment $f$ ou $g$). Soit $d ∈ I \setminus \{0\}$ de degré minimal et $u',v' ∈ K[x]$ tels que
  \begin{displaymath}
    u' ⋅ f + v' ⋅g = d. 
  \end{displaymath}
  Soit $u⋅f + v⋅ g ∈ I$. La division avec reste donne $u⋅f + v⋅ g = qd +r$, avec $\deg(r) < \deg(d)$. Alors
  \begin{displaymath}
    r = (u - qu') ⋅ f + (v - qv') ⋅ g ∈ I
  \end{displaymath}
  et par minimalité de $d ∈ I \setminus\{0\}$, $r=0$. Ainsi il existe un $h∈K[x]$ tel que $h⋅d =  u⋅f + v⋅ g$. Il est clair que $h⋅d ∈I$ pour tous les $h ∈K[x]$ et l'assertion est démontrée.   
\end{proof}


\begin{definition}
  \label{def:54}
  Un polynôme $f(x) ∈ K[x] \setminus \{0\}$  dont le coefficient dominant est $1$ est appelé \emph{polynôme unitaire}. 
\end{definition}


\begin{definition}
  \label{def:55}
  Soient $f,g ∈ K[x]$ non tous deux nuls. Un \emph{diviseur commun} de $f$ et $g$  est un diviseur de $f$ et $g$. 
\end{definition}

\begin{theorem}
  \label{thr:55}
  Soient $f, g$ et $d$  comme dans le  théorème~\ref{thr:54}.
  \begin{enumerate}[i)]
  \item $d$ est un diviseur commun de $f$ et $g$. \label{item:24}
  \item Chaque diviseur commun de $f$ et $g$ est un diviseur de $d$. \label{item:25}
  \item Si $d$ est unitaire, alors $d$ est unique.  \label{item:26}
  \end{enumerate}
\end{theorem}

\begin{proof}
  L'assertion \ref{item:24}) suit du fait que $f,g ∈ I$ et de~\eqref{eq:51}. Soient $u,v ∈K[x]$  tels que
  $    d = u⋅f  + v ⋅ g $
  et soit $w$ un diviseur commun de $f$ et $g$. Alors il existe $f',g' ∈K[x]$ tels que  $f = f' w$ et $g = g' w$. Par conséquent
  \begin{displaymath}
    d = (u⋅f'  + v ⋅ g') w,
  \end{displaymath}
  ce que montre que $w \mid d$ et \ref{item:25}). 
  Soient $d$ et $d'$ deux polynômes unitaires satisfaisant~\eqref{eq:51}. \ref{item:24}) et \ref{item:25}) impliquent que $d \mid d'$ et $d' \mid d$.
  Alors il existe $z,z' ∈K[x]$ tel que $d = d' z'$ et $d' = dz$. Par suite, $d = d ⋅z ⋅ z'$. Le theorème~\ref{thr:34} implique que $z,z' ∈ K$. Et comme $d$ et $d'$ sont unitaires, $z=z'=1$, ce que démontre \ref{item:26}). 
\end{proof}

\begin{definition}
  \label{def:56}
  L'unique polynôme unitaire $d ∈ K[x]$ satisfaisant \eqref{eq:51} est appelé  le \emph{plus grand commun diviseur} de $f$ et $g$. Il est noté $\gcd(f,g)$ ou pgcd$(f,g)$.
\end{definition}

\subsection{L'algorithme d'Euclide}
\label{sec:lalg-de-eucl}



Pour calculer le plus grand diviseur commun de $f(x)$ et $g(x)$  on peut utiliser l'algorithme d'Euclide. Soient $f_0,f_1 ∈K[x]$ pas tous les deux nuls et $\deg(f_0) ≥ \deg(f_1)$. Si $f_1 = 0$, alors
\begin{displaymath}
\gcd(f_0,f_1) =   f_0. 
\end{displaymath}
Autrement, on applique la division avec reste
\begin{displaymath}
  f_0 = q_1 f_1 + f_2, 
\end{displaymath}
où $q_1,f_2 ∈K[x]$ et $ \deg(f_2)< \deg(f_1)$. Un polynôme  $d ∈K[x]$ est un diviseur commun de $f_0$ et $f_1$ si et seulement si $d$ est un diviseur commun de $f_1$ et $f_2$. L'algorithme d'Euclide est le procédé de calculer la suite $f_0,f_1,f_2,\dots,f_{k-1},f_k ∈K[x]$  où $\deg(f_{k-1})≥0$, $f_k=0$ et 
\begin{displaymath}
  f_{i-1} = q_i f_i + f_{i+1} 
\end{displaymath}
est le résultat de la division avec reste de $f_{i-1} $ par $f_i$. Le procédé se termine toujours car la suite des degrés $\{\deg(f_i)\}$ est entière et strictement décroissante (méthode de descente infinie de Fermat). Le dernier reste non nul $f_{k-1}$ est un multiple constant de $\gcd(f_0,f_1)$ : il suffit de diviser par le coefficient dominant pour le rendre unitaire.

%% maybe prove this fact: f_{k-1} is in I and divides both f_0 and f_1, so it divides d and d divides it. The unicity of {thr:55}iii) gives the result.

\begin{example}
  \label{exe:25-b}
  On calcule le plus grand diviseur commun de
$f_0 =  4 x^{6} + x^{4} + 2 x^{2} + 2 ∈ℤ_5[x]$  et 
$f_1 =  3 x^{4} + x^{3} + 2 x^{2} + 2 x + 2 ∈ℤ_5[x]$. 
  \begin{displaymath}
   q_1 =  3 x^{2} + 4 x + 2, \, 
   f_2 =  4 x^{3} + 4 x^{2} + 3 x + 3, 
 \end{displaymath}

 \begin{displaymath}
   q_2 =  2 x + 2, \, 
f_3 =  3 x^{2} + 1, 
\end{displaymath}

\begin{displaymath}
  q_3 =  3 x + 3, \, 
  f_4 =  0. 
\end{displaymath}
Alors tout diviseur commun de $f_0$ et $f_1$ divise $f_3 =  3 x^{2} + 1$ et $f_3$ est aussi un diviseur commun. On divise par $3$ (ou multiplie par $2$) et on obtient  $\gcd(f_0,f_1) = x^2 +2$. 

\end{example}

Le calcul des suites $f_i$ et $q_i$ donne aussi une représentation $\gcd(f_0,f_1) = u ⋅f_0 + v ⋅f_1$, $u,v ∈K[x]$. En effet


\begin{displaymath}
  \begin{pmatrix}
    f_{i} \\  f_{i+1} 
  \end{pmatrix}
  =
  \begin{pmatrix}
    0 & 1 \\
    1 & -q_i
  \end{pmatrix}
  \begin{pmatrix}
     f_{i-1} \\  f_{i}
   \end{pmatrix}
\end{displaymath}
et alors
\begin{displaymath}
  \begin{pmatrix}
    f_{k-1} \\ f_k
  \end{pmatrix} =
  \begin{pmatrix}
    0 & 1 \\
    1 & -q_{k-1}
  \end{pmatrix} \cdots
  \begin{pmatrix}
    0 & 1 \\
    1 & -q_{1}
  \end{pmatrix}
  \begin{pmatrix}
    f_0 \\ f_1
  \end{pmatrix}.
\end{displaymath}
\begin{example}
  \label{exe:26}
  On continue l'exemple~\ref{exe:25-b}.


  \begin{displaymath}
    \left(\begin{array}{rr}
3 x + 3 & x^{3} + 4 x^{2} + 2 x \\
x^{2} + 2 x + 2 & 2 x^{4} + 4 x^{2} + 3
\end{array}\right) = 
    \left(\begin{array}{rr}
0 & 1 \\
1 & 2 x + 2
\end{array}\right) \left(\begin{array}{rr}
0 & 1 \\
1 & 3 x + 3
\end{array}\right) \left(\begin{array}{rr}
0 & 1 \\
1 & 2 x^{2} + x + 3
                         \end{array}\right)                                 
                     \end{displaymath}
                     et
\begin{displaymath}
\left(\begin{array}{r}
3 x^{2} + 1 \\
0
\end{array}\right) = \left(\begin{array}{rr}
3 x + 3 & x^{3} + 4 x^{2} + 2 x \\
x^{2} + 2 x + 2 & 2 x^{4} + 4 x^{2} + 3
\end{array}\right)
\left(\begin{array}{r}
4 x^{6} + x^{4} + 2 x^{2} + 2 \\
3 x^{4} + x^{3} + 2 x^{2} + 2 x + 2
\end{array}\right).   
\end{displaymath}

\begin{displaymath}
 \left(\begin{array}{r}
x^{2} + 2 \\
0
\end{array}\right) =
\left(\begin{array}{rr}
x + 1 & 2 x^{3} + 3 x^{2} + 4 x \\
2 x^{2} + 4 x + 4 & 4 x^{4} + 3 x^{2} + 1
\end{array}\right)
\left(\begin{array}{r}
4 x^{6} + x^{4} + 2 x^{2} + 2 \\
3 x^{4} + x^{3} + 2 x^{2} + 2 x + 2
\end{array}\right)
\end{displaymath}

Alors

\begin{displaymath}
  x^{2} + 2 = \gcd(f_0,f_1) = (x + 1)f_0 + (2 x^{3} + 3 x^{2} + 4 x) f_1. 
\end{displaymath}
\end{example}



\subsection{Factorisation en irréductibles}
\label{sec:fact-en-irred}
Clairement, tout polynôme $f(x) ∈ K[x] \setminus \{0\}$    peut être factorisé comme
\begin{equation}
  \label{eq:33}
  f(x) = a ⋅ ∏_i p_i(x),
\end{equation}
dont les $p_i$ sont irréductibles est unitaires et $a ∈ K$. On va voir maintenant que cette factorisation est unique. 

\begin{theorem}
\label{thr:56}
  Soit $p(x) ∈K[x]$ irréductible et supposons que $p(x)$ divise un produit $f_1(x) \cdots f_k(x)$ de polynômes non nul. Alors $p(x)$ divise un polynôme $f_i(x)$. 
\end{theorem}


\begin{proof}
  Par récurrence il suffit de démontrer l'assertion pour $k=2$. Ainsi, supposons $p \mid fg$, $f,g ∈K[x] \setminus \{0\}$. Si $p$ ne divise pas $f$, alors $\gcd(p,f) = 1$ car les seuls diviseurs de $p$ sont des multiples constants de $1$ et lui-même. Soient donc $u,v ∈ K[x]$ t.q.  $up + vf = 1$. Alors $upg + vfg = g$, et donc $p \mid g$. 
\end{proof}


\begin{theorem}
  \label{thr:57}
  La factorisation~\eqref{eq:33} est unique à l'ordre près des $p_i$. 
\end{theorem}

\begin{proof}
  Pour une  factorisation $f(x) = a ∏_i q_i(x)$, où les $q_i$ sont irréductibles et unitaires on utilise le théorème~\ref{thr:56} pour déduire qu'il  existe $j$ tel que  $p_1 \mid q_j$. Comme $p_1$ et $q_j$ sont  irréductibles et unitaires, $p_1 = q_j$. En divisant par $p_1$, l'assertion suit par récurrence. 
\end{proof}


\begin{corollary}
  \label{co:11}
  Soient $f(x) ∈ K[x] \setminus \{0\}$ et $α_1,\dots, α_ℓ$ des racines de $f$ de multiplicité $k_1,\dots,k_ℓ$ respectivement. Alors il existe $g(x)∈ K[x]$ tel que  
  \begin{displaymath}
    f(x) = g(x) ⋅ ∏_{i=1}^ℓ (x - α_i)^{k_i} .
  \end{displaymath}
\end{corollary}

\begin{exercise}
  Démontre le Corollaire~\ref{co:11}. 
\end{exercise}


%%% Local Variables:
%%% mode: latex
%%% TeX-master: "notes"
%%% End:
