\RequirePackage[2020-02-02]{latexrelease}
\makeatletter\let\ifGm@compatii\relax\makeatother

\documentclass[t,aspectratio=149,mathserif]{beamer}     

\usepackage[utf8]{inputenc} 
\usepackage{utf8math}

%Uncomment again! 

\usepackage{kerkis}
\usepackage{kmath}
%\renewcommand{\rmdefault}{ibh}
%\usepackage{helvet}
%\usepackage{concmath}



%%%%%%% Beamer Style 
\setbeamertemplate{footline}{}
\beamertemplatenavigationsymbolsempty

\usetheme{default}
\usecolortheme{beaver}

%%%%%%%

\usepackage{mathrsfs}
\providecommand{\mB}{\mathscr{B}}
\providecommand{\mN}{\mathscr{N}}


\usepackage{tikz}

%%% IPE

\usetikzlibrary{arrows.meta,patterns}
\usetikzlibrary{ipe} % ipe compatibility library

%%% Fadings 
\usetikzlibrary{fadings}
\tikzfading[name=fade out,
            inner color=transparent!0,
            outer color=transparent!100]


\usepackage{enumerate}

\usepackage{tkz-graph}

\usetikzlibrary{calc}
\usetikzlibrary{arrows}

\usetikzlibrary{decorations.pathmorphing}
\usetikzlibrary{decorations.markings}






\usepackage{pifont}

\usepackage{xspace} 



%%%% Defining colors

\usepackage{color}
\definecolor{navy}{RGB}{0,0,205}
\definecolor{NavyBlue}{RGB}{0,0,205}
\definecolor{darkred}{RGB}{178,34,34}
\definecolor{darkblue}{RGB}{0,10,230}
\definecolor{green}{RGB}{20,180,20}
\definecolor{titleblue}{rgb}{0,0,0.3}
\definecolor{mediumblue}{rgb}{0.5,0.5,1}
\definecolor{lightblue}{cmyk}{0.2,0.1,0,0}
\definecolor{darkgreen}{rgb}{0.0,0.5,0}
\definecolor{orange}{rgb}{1,0.5,0}





%%%% My epmphasis envorinments 
\newcommand{\myemph}[1]{\em \textcolor{darkred}{#1}}
\newcommand{\mybf}[1]{\em \textcolor{darkred}{#1}}
\newcommand{\mycite}[1]{{\color{NavyBlue}{#1}}}


\newcommand{\mnred}[1]{\mathbf{\color{red}#1}}
\newcommand{\mred}[1]{\mathbf{\color{darkred}#1}}
\newcommand{\mgreen}[1]{\mathbf{\color{green}#1}}


\newcommand{\myred}[1]{\emph{\color{darkred}#1}}
\newcommand{\myblue}[1]{{\color{darkblue}#1}}
\newcommand{\boldred}[1]{{\bf\color{red}#1}}
\newcommand{\btt}{\tt\color{darkblue}}


%%%%% To make some vertical space 
\newcommand{\smallplatz}{\vspace{.2cm}}
\newcommand{\platz}{\vspace{.5cm}}
\newcommand{\Platz}{\vspace{1cm}}



%%%%%%%% mathtools needed for \DeclareMathOperator
\usepackage{mathtools}
\DeclareMathOperator\rank{rank}
\DeclarePairedDelimiter{\abs}{\lvert}{\rvert}
\DeclarePairedDelimiter{\norm}{\lVert}{\rVert}


\newcommand{\R}{\mathbb{R}}
\newcommand{\Z}{\mathbb{Z}}
\newcommand{\N}{\mathbb{N}}
\newcommand{\Q}{\mathbb{Q}}
\newcommand{\adj}{\mathrm{adj}}

\newcommand{\vb}{\mathbf{b}}
\newcommand{\vx}{\mathbf{x}}
\newcommand{\vy}{\mathbf{y}}
\newcommand{\vz}{\mathbf{z}}
\newcommand{\vc}{\mathbf{c}}
\newcommand{\vu}{\mathbf{u}}
\newcommand{\vw}{\mathbf{w}}
\newcommand{\vv}{\mathbf{v}}
\newcommand{\proj}{\mathbf{proj}}
\newcommand{\vol}{\mathrm{vol}}
\newcommand{\CVP}{\mathrm{CVP}}
\newcommand{\IP}{\mathrm{IP}}
\newcommand{\MSD}{\mathrm{MSD}}
\newcommand{\A}{\mathscr{A}}
\newcommand{\OPT}{\mathrm{OPT}}

 

\newcommand{\diam}{\mathrm{diam}}
\newcommand{\Span}{\mathrm{Span}}
\newcommand{\Col}{\mathrm{Col}}
\newcommand{\Row}{\mathrm{Row}}
\newcommand{\Rang}{\mathrm{Rang}}

\newcommand{\rang}{\mathrm{Rang}}
\newcommand{\dist}{\mathrm{dist}}
\renewcommand{\L}{\mathcal{L}}


\newcommand{\height}{\mathrm{height}}
\newcommand{\loss}{\mathrm{loss}}

\newcommand{\col}{\mathrm{col}}
\newcommand{\row}{\mathrm{row}}




\begin{document}

\begin{frame}{Integer Programming} 


     \begin{alertblock}{Algorithms, lower bounds, open problems}
     %  
     \end{alertblock}

\platz
 \begin{center}
    \begin{tikzpicture}[scale=0.6]
      %%%%%%%%% Polygon{
      
      \clip (-4.2, -3.2) rectangle (4.2, 2.2);
      \draw  (-3.5, -1) -- (-1.5, 1) -- (0.5, 1.5) -- (1.5, 0) -- ( 1.5 , -1) -- (0.5, -2.5) ; 
      
      \fill [color = red,opacity=.5] (-3.5, -1) -- (-1.5, 1) -- (0.5, 1.5) -- (1.5, 0) -- ( 1.5 , -1) -- (0.5, -2.5) ;    
      %%%%%%%%%%%% }  
      
      
      
      \def\bAA{1};
      \def\bAB{0};
      \def\bBA{0};
      \def\bBB{1};
      


      \foreach \x in { -4, -3, ..., 5  } {
        \foreach \y in { -6, -5, ..., 6 } {
          \fill[color=darkgray] (\x* \bAA + \y * \bAB, \x* \bBA + \y * \bBB) circle (2pt);
        }
      }
      
      \draw [color = green , ->, line width = 2pt] (1.5, 0.5) -- (2.5,1.3) node[below]  {$c$}; 
      
      \fill[color=blue] (1,0) circle (3pt);

    \end{tikzpicture}
  \end{center}    

 \platz 


   {\small     \hfill    
     \begin{minipage}{.3\linewidth}
       \centering 
        Friedrich Eisenbrand\\
    \today
     \end{minipage}  
}
\end{frame}
 

\begin{frame}{Test} 


  \begin{displaymath}
a+    \alpha + \frac{2}{β} + γ
  \end{displaymath}
\end{frame}
%%% Local Variables:
%%% mode: latex
%%% TeX-master: "Slides"
%%% End:


\begin{frame}

  \begin{center}
       \tikzstyle{ipe stylesheet} = [
  ipe import,
  even odd rule,
  line join=round,
  line cap=butt,
  ipe pen normal/.style={line width=0.4},
  ipe pen heavier/.style={line width=0.8},
  ipe pen fat/.style={line width=1.2},
  ipe pen ultrafat/.style={line width=2},
  ipe pen normal,
  ipe mark normal/.style={ipe mark scale=3},
  ipe mark large/.style={ipe mark scale=5},
  ipe mark small/.style={ipe mark scale=2},
  ipe mark tiny/.style={ipe mark scale=1.1},
  ipe mark normal,
  /pgf/arrow keys/.cd,
  ipe arrow normal/.style={scale=7},
  ipe arrow large/.style={scale=10},
  ipe arrow small/.style={scale=5},
  ipe arrow tiny/.style={scale=3},
  ipe arrow normal,
  /tikz/.cd,
  ipe arrows, % update arrows
  <->/.tip = ipe normal,
  ipe dash normal/.style={dash pattern=},
  ipe dash dashed/.style={dash pattern=on 4bp off 4bp},
  ipe dash dotted/.style={dash pattern=on 1bp off 3bp},
  ipe dash dash dotted/.style={dash pattern=on 4bp off 2bp on 1bp off 2bp},
  ipe dash dash dot dotted/.style={dash pattern=on 4bp off 2bp on 1bp off 2bp on 1bp off 2bp},
  ipe dash normal,
  ipe node/.append style={font=\normalsize},
  ipe stretch normal/.style={ipe node stretch=1},
  ipe stretch normal,
  ipe opacity 10/.style={opacity=0.1},
  ipe opacity 30/.style={opacity=0.3},
  ipe opacity 50/.style={opacity=0.5},
  ipe opacity 75/.style={opacity=0.75},
  ipe opacity opaque/.style={opacity=1},
  ipe opacity opaque,
]
\definecolor{red}{rgb}{1,0,0}
\definecolor{green}{rgb}{0,1,0}
\definecolor{blue}{rgb}{0,0,1}
\definecolor{yellow}{rgb}{1,1,0}
\definecolor{orange}{rgb}{1,0.647,0}
\definecolor{gold}{rgb}{1,0.843,0}
\definecolor{purple}{rgb}{0.627,0.125,0.941}
\definecolor{gray}{rgb}{0.745,0.745,0.745}
\definecolor{brown}{rgb}{0.647,0.165,0.165}
\definecolor{navy}{rgb}{0,0,0.502}
\definecolor{pink}{rgb}{1,0.753,0.796}
\definecolor{seagreen}{rgb}{0.18,0.545,0.341}
\definecolor{turquoise}{rgb}{0.251,0.878,0.816}
\definecolor{violet}{rgb}{0.933,0.51,0.933}
\definecolor{darkblue}{rgb}{0,0,0.545}
\definecolor{darkcyan}{rgb}{0,0.545,0.545}
\definecolor{darkgray}{rgb}{0.663,0.663,0.663}
\definecolor{darkgreen}{rgb}{0,0.392,0}
\definecolor{darkmagenta}{rgb}{0.545,0,0.545}
\definecolor{darkorange}{rgb}{1,0.549,0}
\definecolor{darkred}{rgb}{0.545,0,0}
\definecolor{lightblue}{rgb}{0.678,0.847,0.902}
\definecolor{lightcyan}{rgb}{0.878,1,1}
\definecolor{lightgray}{rgb}{0.827,0.827,0.827}
\definecolor{lightgreen}{rgb}{0.565,0.933,0.565}
\definecolor{lightyellow}{rgb}{1,1,0.878}
\definecolor{black}{rgb}{0,0,0}
\definecolor{white}{rgb}{1,1,1}
\begin{tikzpicture}[ipe stylesheet]
  \fill[lightgray]
    (128, 640)
     -- (192, 704)
     -- (384, 704)
     -- (320, 640)
     -- cycle;
  \draw[->]
    (256, 672)
     -- (288, 656);
  \draw[->]
    (256, 672)
     -- (336, 736);
  \draw[ipe dash dotted]
    (336, 736)
     -- (336, 672);
  \draw[blue, ->]
    (256, 672)
     -- (336, 672);
  \draw[->]
    (256, 672)
     -- (240, 656);
\end{tikzpicture}
  \end{center}

\end{frame}

\end{document} 


%%% Local Variables: 
%%% mode: latex
%%% TeX-master: t
%%% End: 



