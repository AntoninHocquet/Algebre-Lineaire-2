\RequirePackage[2020-02-02]{latexrelease}
\makeatletter\let\ifGm@compatii\relax\makeatother

% Otherwise some colors may look a bit different in ipe and TeX
\PassOptionsToPackage{rgb}{xcolor}

\documentclass[t,aspectratio=149,mathserif]{beamer}     
  
    
\usepackage[utf8]{inputenc}   
\usepackage{utf8math}

%Uncomment again! 

%\usepackage{kerkis}
%\usepackage{kmath}
%\renewcommand{\rmdefault}{ibh}
\usepackage{helvet}
\usepackage{concmath}



%%%%%%% Beamer Style 
\setbeamertemplate{footline}{}
\beamertemplatenavigationsymbolsempty

\usetheme{default}
\usecolortheme{beaver}

%%%%%%%
 
\usepackage{mathrsfs}
\providecommand{\mB}{\mathscr{B}}
\providecommand{\mN}{\mathscr{N}}


\usepackage{tikz}

%%% IPE

\usetikzlibrary{arrows.meta,patterns}
\usetikzlibrary{ipe} % ipe compatibility library

%%% Fadings 
\usetikzlibrary{fadings}
\tikzfading[name=fade out,
            inner color=transparent!0,
            outer color=transparent!100]


\usepackage{enumerate}

\usepackage{tkz-graph}

\usetikzlibrary{calc}
\usetikzlibrary{arrows}

\usetikzlibrary{decorations.pathmorphing}
\usetikzlibrary{decorations.markings}






\usepackage{pifont}

\usepackage{xspace} 



%%%% Defining colors

\usepackage{color}
\definecolor{navy}{RGB}{0,0,205}
\definecolor{NavyBlue}{RGB}{0,0,205}
\definecolor{darkred}{RGB}{178,34,34}
\definecolor{darkblue}{RGB}{0,10,230}
\definecolor{green}{RGB}{20,180,20}
\definecolor{titleblue}{rgb}{0,0,0.3}
\definecolor{mediumblue}{rgb}{0.5,0.5,1}
\definecolor{lightblue}{cmyk}{0.2,0.1,0,0}
\definecolor{darkgreen}{rgb}{0.0,0.5,0}
\definecolor{orange}{rgb}{1,0.5,0}





%%%% My epmphasis envorinments 
\newcommand{\myemph}[1]{\em \textcolor{darkred}{#1}}
\newcommand{\mybf}[1]{\em \textcolor{darkred}{#1}}
\newcommand{\mycite}[1]{{\color{NavyBlue}{#1}}}


\newcommand{\mnred}[1]{\mathbf{\color{red}#1}}
\newcommand{\mred}[1]{\mathbf{\color{darkred}#1}}
\newcommand{\mgreen}[1]{\mathbf{\color{green}#1}}


\newcommand{\myred}[1]{\emph{\color{darkred}#1}}
\newcommand{\myblue}[1]{{\color{darkblue}#1}}
\newcommand{\boldred}[1]{{\bf\color{red}#1}}
\newcommand{\btt}{\tt\color{darkblue}}


%%%%% To make some vertical space 
\newcommand{\smallplatz}{\vspace{.2cm}}
\newcommand{\platz}{\vspace{.5cm}}
\newcommand{\Platz}{\vspace{1cm}}



%%%%%%%% mathtools needed for \DeclareMathOperator
\usepackage{mathtools}
\DeclareMathOperator\rank{rank}
\DeclarePairedDelimiter{\abs}{\lvert}{\rvert}
\DeclarePairedDelimiter{\norm}{\lVert}{\rVert}


\newcommand{\R}{\mathbb{R}}
\newcommand{\Z}{\mathbb{Z}}
\newcommand{\N}{\mathbb{N}}
\newcommand{\Q}{\mathbb{Q}}
\newcommand{\adj}{\mathrm{adj}}

\newcommand{\vb}{\mathbf{b}}
\newcommand{\vx}{\mathbf{x}}
\newcommand{\vy}{\mathbf{y}}
\newcommand{\vz}{\mathbf{z}}
\newcommand{\vc}{\mathbf{c}}
\newcommand{\vu}{\mathbf{u}}
\newcommand{\vw}{\mathbf{w}}
\newcommand{\vv}{\mathbf{v}}
\newcommand{\proj}{\mathbf{proj}}
\newcommand{\vol}{\mathrm{vol}}
\newcommand{\CVP}{\mathrm{CVP}}
\newcommand{\IP}{\mathrm{IP}}
\newcommand{\MSD}{\mathrm{MSD}}
\newcommand{\A}{\mathscr{A}}
\newcommand{\OPT}{\mathrm{OPT}}

 

\newcommand{\diam}{\mathrm{diam}}
\newcommand{\Span}{\mathrm{Span}}
\newcommand{\Col}{\mathrm{Col}}
\newcommand{\Row}{\mathrm{Row}}
\newcommand{\Rang}{\mathrm{Rang}}

\newcommand{\rang}{\mathrm{Rang}}
\newcommand{\dist}{\mathrm{dist}}
\renewcommand{\L}{\mathcal{L}}


\newcommand{\height}{\mathrm{height}}
\newcommand{\loss}{\mathrm{loss}}

\newcommand{\col}{\mathrm{col}}
\newcommand{\row}{\mathrm{row}}




\begin{document}

\begin{frame}{Algèbre Linéaire Avancée II} 

  \begin{alertblock}{Informations sur}

    \begin{columns}
      \begin{column}{.5\textwidth}
        \begin{itemize}
        \item Contenue du cours
         \item Exercices et examen 
         \item Contact et forum
         \end{itemize}
       \end{column}
       \begin{column}{.5\textwidth}
         %  \begin{itemize}
         % \item Contenue du cours
         % \item Exercices et examen 
         % \item Contact et forum
         % \end{itemize}
       \end{column}
       
    \end{columns}
      
     \end{alertblock}
          \begin{center}
       \tikzstyle{ipe stylesheet} = [
  ipe import,
  even odd rule,
  line join=round,
  line cap=butt,
  ipe pen normal/.style={line width=0.4},
  ipe pen heavier/.style={line width=0.8},
  ipe pen fat/.style={line width=1.2},
  ipe pen ultrafat/.style={line width=2},
  ipe pen normal,
  ipe mark normal/.style={ipe mark scale=3},
  ipe mark large/.style={ipe mark scale=5},
  ipe mark small/.style={ipe mark scale=2},
  ipe mark tiny/.style={ipe mark scale=1.1},
  ipe mark normal,
  /pgf/arrow keys/.cd,
  ipe arrow normal/.style={scale=7},
  ipe arrow large/.style={scale=10},
  ipe arrow small/.style={scale=5},
  ipe arrow tiny/.style={scale=3},
  ipe arrow normal,
  /tikz/.cd,
  ipe arrows, % update arrows
  <->/.tip = ipe normal,
  ipe dash normal/.style={dash pattern=},
  ipe dash dotted/.style={dash pattern=on 1bp off 3bp},
  ipe dash dashed/.style={dash pattern=on 4bp off 4bp},
  ipe dash dash dotted/.style={dash pattern=on 4bp off 2bp on 1bp off 2bp},
  ipe dash dash dot dotted/.style={dash pattern=on 4bp off 2bp on 1bp off 2bp on 1bp off 2bp},
  ipe dash normal,
  ipe node/.append style={font=\normalsize},
  ipe stretch normal/.style={ipe node stretch=1},
  ipe stretch normal,
  ipe opacity 10/.style={opacity=0.1},
  ipe opacity 30/.style={opacity=0.3},
  ipe opacity 50/.style={opacity=0.5},
  ipe opacity 75/.style={opacity=0.75},
  ipe opacity opaque/.style={opacity=1},
  ipe opacity opaque,
]
\definecolor{red}{rgb}{1,0,0}
\definecolor{blue}{rgb}{0,0,1}
\definecolor{green}{rgb}{0,1,0}
\definecolor{yellow}{rgb}{1,1,0}
\definecolor{orange}{rgb}{1,0.647,0}
\definecolor{gold}{rgb}{1,0.843,0}
\definecolor{purple}{rgb}{0.627,0.125,0.941}
\definecolor{gray}{rgb}{0.745,0.745,0.745}
\definecolor{brown}{rgb}{0.647,0.165,0.165}
\definecolor{navy}{rgb}{0,0,0.502}
\definecolor{pink}{rgb}{1,0.753,0.796}
\definecolor{seagreen}{rgb}{0.18,0.545,0.341}
\definecolor{turquoise}{rgb}{0.251,0.878,0.816}
\definecolor{violet}{rgb}{0.933,0.51,0.933}
\definecolor{darkblue}{rgb}{0,0,0.545}
\definecolor{darkcyan}{rgb}{0,0.545,0.545}
\definecolor{darkgray}{rgb}{0.663,0.663,0.663}
\definecolor{darkgreen}{rgb}{0,0.392,0}
\definecolor{darkmagenta}{rgb}{0.545,0,0.545}
\definecolor{darkorange}{rgb}{1,0.549,0}
\definecolor{darkred}{rgb}{0.545,0,0}
\definecolor{lightblue}{rgb}{0.678,0.847,0.902}
\definecolor{lightcyan}{rgb}{0.878,1,1}
\definecolor{lightgray}{rgb}{0.827,0.827,0.827}
\definecolor{lightgreen}{rgb}{0.565,0.933,0.565}
\definecolor{lightyellow}{rgb}{1,1,0.878}
\definecolor{black}{rgb}{0,0,0}
\definecolor{white}{rgb}{1,1,1}
\begin{tikzpicture}[ipe stylesheet]
  \filldraw[ipe pen heavier, fill=lightgray]
    (128, 640)
     -- (192, 704)
     -- (384, 704)
     -- (320, 640)
     -- cycle;
  \filldraw[ipe pen heavier, ->, fill=lightgray]
    (256, 672)
     -- (304, 720);
  \filldraw[->, fill=lightgray]
    (256, 672)
     -- (240, 656);
  \filldraw[->, fill=lightgray]
    (256, 672)
     -- (272, 656);
  \filldraw[ipe dash dashed, fill=lightgray]
    (304, 720)
     -- (304, 672);
  \filldraw[draw=blue, ipe pen heavier, ->, fill=lightgray]
    (256, 672)
     -- (304, 672);
  \node[ipe node]
     at (312, 720) {$v$};
  \node[ipe node, text=blue]
     at (304, 664) {$\mathrm{proj}(v)$};
\end{tikzpicture}
       
     \end{center}    
     
     \platz
          
     {\small     \hfill    
       \begin{minipage}{.25\linewidth} \scriptsize         
         Friedrich Eisenbrand\\
         20. Février 2024 
       \end{minipage}  
     }
  
   \end{frame}
   
   

   \begin{frame}{Contenue}
   
     \begin{enumerate}
     \item Polynômes: {\small Racines, division avec reste,
         reconstruction, algorithme d'Euclide, Factorisation en
         irréductibles }
     \item Algorithmes et leur analyse:  {\small Déterminant, équations linéaires, polynôme caractéristique, diagonalisation }
     \item  Formes bilinéaires: {\small théorème de Sylvester, produits scalaires, norme, orthogonalisation, moindre carrées }
     \item Théorème spectral: {\small  valeurs singulières, pseudo-inverses}
     \item Systèmes différentiels linéaires: {\small exponentiel d'une matrice}
     \item La forme normale de Jordan
     \item  Algèbre linéaire sur les entiers{ \small structure de groupes abélien  engendrés finis }
   \end{enumerate}


   \uncover<2->{
   \myemph{Notes e cours}: Sur moodle.    \myemph{Régulièrement}  mis à jours.

   \platz
   \myemph{Fin du semestre}: Reflètent le contenu  relevant  
   }
   \end{frame}



   \begin{frame}
     \frametitle{Exercices}

     \begin{itemize}
     \item Une série chaque semaine sur moodle.
     \item Première à télécharger  aujourd'hui (20.02.24)
     \item Exercice $+$ et $*$: \myemph{Pas} de corrigé. Seront \myemph{discutés} lors des sessions du mardi.
     \item Série de la semaine $n$
       \begin{itemize}
       \item Ex. $+$: Exercice préparatoire discuté Mardi semaine $n$
       \item Ex. $*$: Exercice type question ouverte examen   discuté Mardi semaine $n+1$
       \item Aujourd'hui: On discute exercice $+$ semaine $1$
       \item Autres exercices: Travaille pendant séance vendredi, semaine $n$.
       \item Ce vendredi: On discute exercices semaine $1$
       \end{itemize}
     \item Séance vendredi:
       \begin{itemize}
       \item Répartition des salles affichée sur moodle (selon nom)
       \item 14 assistants (7 ass. ét. + 7 ass. doc.) à disposition pour questions, aide, discussions. 
       \end{itemize}
       
     \end{itemize}
     
   \end{frame}

   \begin{frame}
     \frametitle{Examen}

     \begin{itemize}
     \item 3 questions \myemph{ouvertes} (type ex. $*$). Une des trois est exercice $*$ original des séries
     \item $∼$ 7 questions  \myemph{choix multiples} 
     \item  $∼$ 14 questions \myemph{vrai/faux}     
     \end{itemize}

     \Platz

     Examens des années précédentes  \myemph{sur moodle}
   \end{frame}

    \begin{frame}
     \frametitle{Take home exam} 
     \begin{itemize}
     \item 3 examens \myemph{take home} pendent le semestre
     \item Chaque représente $∼$ 1/3 d'un examen (une question ouverte, quelque CM + V/F)
     \item V/F et CM sont corrigés automatiquement
     \item Question ouverte à rendre pour feedback (en latex)  \myemph{fortement conseillé} 
     \item Dates:
       \begin{itemize}
       \item 18.03 - 24.03
       \item 22.04 - 28.04
       \item 20.05 - 26.05
       \end{itemize}
     \item Ne \myemph{comptent pas} vers la note finale 
     \end{itemize}
   \end{frame}

   \begin{frame}{Contact et communication}

     \begin{itemize}
     \item Questions, commentaires: \myemph{Utilisez forum!}
     \item Assistants sont présent sur forum 
     \item \myemph{Formatez les maths en latex!} 
     \item Ne m'envoyez pas de  e-mail! 
     \item Office hours: Mardi 16:00 - 17:00.

       
      R.d.v. impératif: e-mail  {\tt pauline.bataillard@epfl.ch  }
     \end{itemize}
     
   \end{frame}
 \end{document} 


%%% Local Variables: 
%%% mode: latex
%%% TeX-master: t
%%% End: 



